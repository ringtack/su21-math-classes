\documentclass[math1010-lecture-notes]{subfiles}
\begin{document}

\chapter{Real Numbers}
In this class, we operate on the set of real numbers. It is important to rigorously define it, as
well as all underlying sets of numbers (natural numbers, integers, rationals, etc).

\section{Natural Numbers}
\begin{definition}[Natural Numbers]{}
  The set $\N = {1, 2, \ldots}$ is the set of natural numbers. Each integer $n$ has a
  \textbf{successor}, $succ(n) = n + 1$. $1$ is not the successor of any number. \\
  The following properties constitute the Peano Axioms of $\N$:
  \begin{enumerate}
    \item $1\in \N$.
    \item $n\in \N \Rightarrow succ(n)=n+1 \in \N$
    \item $\not\exists n ~\text{s.t.}~ succ(n) = 1$
    \item If $n, m\in \N, succ(n) = succ(m)$, then $n=m$.
    \item A subset $A\subset \N$ which contains $1$, and which contains  $n+1$ whenever it contains
       $n$, must equal $\N$.
  \end{enumerate}
  We accept only these 5 axioms to prove all other properties of $\N$.
\end{definition}
(5) is the basis for the principle of induction.

\begin{theorem}[Principle of Mathematical Induction]{}
  Let $ P_1, P_2, \ldots$ be a list of statements. Assume the following:
  \begin{enumerate}
    \item $P_1$ is true. [ Basis of induction ] 
    \item $\forall n\in N, n\ge 1$, if $P_n$ is true, then  $P_{n+1}$ is true. [ Inductive step ]
  \end{enumerate}
  Then all the statements $P_1, \ldots$ are true.
\end{theorem}
\begin{proof}[Proof]
  Let $A$ be the set of integers $n$ for which  $P_n$ is true. We want to prove $A=\N$. We use (5)
  to prove this. \\
  Indeed, $1\in A$ by assumption 1. Assuming that $n\in A$ for some $n$, we prove that  $n+1\in A$.
  This is true by assumption 2: if $n\in A$, then $P_n$ is true, hence  $P_{n+1}$ is true, hence
  $n+1\in A$.
  Thus, $A=\N$.
\end{proof}

\begin{example}
  Prove that $2^{1}+2^2+\ldots+2^{n}=2^{n+1}-2$.
\end{example}

\begin{proof}
  Let $P_n$: "$2^{1}+2^2+\ldots+2^{n}=2^{n+1}-2$".\\
  $P_1$ is true because $2^{1}=2^2-2$. \\
  For the induction step, we assume $P_n$ is true for some  $n$ and prove $P_{n+1}$ is true. Since
   $P_n$ is true,  \[
   2^{1}+2^2+\ldots+2^{n} = 2^{n+1}-2
 .\]  $P_{n+1}$ states that  \[
 2^{1}+\ldots+2^{n}+2^{n+1} = 2^{\left( n+1 \right) +1}-2
 .\] Using $P_n$ we have  \[
 2^1+\ldots+2^{n + 2^{n+1}} = \left( 2^{n+1}-2 \right) +2^{n+1} = 2\cdot 2^{n+1}-2=2^{\left( n+1
 \right) +1}-2
 ,\] Thus $P_{n+1}$ is true. 
 By the principle of induction, $P_n$ is true for all  $n\ge 1$.
\end{proof}


\begin{example}
  Let $x_1=1$ and define \[
    x_{n+1}=\frac{1}{2}x_n+1
  .\] 
  Prove that $\forall x, x_n\le x_{n+1}$ (or, $x_n$ is inreasing).
\end{example}

\begin{proof}[Proof]
  Let $P_n$: "$x_n\le x_{n+1}$". \\
  $P_1$ is true because $x_1=1\le \frac{3}{2}=x_2$.\\
  For the induction step, we assume $P_n$ is true for some  $n$ and prove $P_{n+1}$ is true. Since
  $P_n$ is true,  \[
    x_n \le x_{n+1}
  .\] $P_{n+1}$ states that  \[
  x_{n+1} \le x_{n+2}
.\] Using $P_n$ we have  \[
x_{n+1} = \frac{1}{2}x_n+1\le \frac{1}{2}x_{n+1}+1=x_{n+2} \hfill~\text{[by $P_n$, we know $x_n\le x_{n+1}$]}~
\] Thus $P_{n+1}$ is true.\\
By the principle of induction, $P_n$ is true for all $n\ge 1$.
\end{proof}

The principle of induction can be extended by allowing the first statement to begin at $P_m$ instead
of  $P_1$ for some fixed integer $m$.

 \begin{theorem}[Generalized Principle of Induction]{}
   Let $m$ be an integer, and consider a list of statements $P_m,P_{m+1},\ldots$. Then all the
   statements are true if the following two properties are true:
   \begin{enumerate}
     \item $P_m$ is true
     \item  $\forall n\ge m$, if $P_n$ is true, then  $P_{n+1}$ is true.
   \end{enumerate}
\end{theorem}
Theorem $\S$ 1.1.2 follows from
\begin{proposition}[Specific Case of (5)]{}
  Let $m\in N$. Assume that a subset  $A\subset \{m, m+1, \ldots\} $ contains $m$ and $n+1$ whenever
  it contains $n$. Then $A=\{ m,m+1,\ldots \}$.
\end{proposition}
\begin{proof}[Proof]
  In Peano Axiom 5, we start from $ 1$; here, we start from $m$. Let $B=\{p=n-(m-1)\mid n\in A\} $.
  Since $A\subset \{ m,m+1,\ldots \}$, $B\subset \{1, 2, \ldots\}=\N$. \\
  We observe that $1\in B$ (because $1=m-(m-1)$, and $m\in A$ by definition). Assuming $p\in B$, we
  have $p=n-(m-1)$ for some $n\in A$. Then $p+1=(n+1)-(m-1)\in B$, as we state that $n+1\in A$
  whenever $n\in A$. Thus, using Peano Axiom 5, we see that $A=B$. \\
  Now, by definition of $B$, we have
  \begin{align*}
    A &= \{n = p + \left( m-1 \right) \mid p \in B \}  \\
      &=  \{n=p+\left( m-1 \right) \mid p \in \N\} \\
      &=  \{m, m+1, \ldots\}  \\
  \end{align*}
\end{proof}




\begin{example}
  Prove that \[
  n! > n^2
  .\] for all $n\ge 4$.
\end{example}
\begin{proof}[Proof]
  Recall $n! = 1\cdot 2\cdot \ldots\cdot n$. Let $P_n$: "$n! > n^2$". We prove $P_n$ is true
  $\forall n\ge 4$.\\
  $P_4$ is true because  \[
  4! = 24 > 16 = 4^2
  .\] Assuming $n! > n^2$, we prove \[
  \left( n+1 \right)! > \left( n+1 \right) ^2
  .\] Using $P_n$ we have  \[
  \left( n+1 \right)! = n!\left( n+1 \right) = \left( 1\cdot 2\cdot \ldots\cdot n \right) \cdot \left( n+1 \right) > n^2(n+1) > \left( n+1 \right) ^2
.\] Thus $P_{n+1}$ is true. \\
By the principle of induction, $P_n$ is true for all  $n\ge 4$.
\end{proof} 


\section{Rational Numbers}
\begin{definition}[Integers]{}
  The set of integers is denoted by \[
    \Z=\{ \ldots -2,-1,0,1,2,\ldots \}
  .\] 
  A rigorous construction is omitted, and left as an exercise for the reader.
\end{definition}

\begin{definition}[Rational Numbers]{}
  The set $\Q$ of rational numbers is \[
    \Q = \{\frac{m}{n}\mid m,n\in \Z,n\neq 0\} 
  .\] We say $\frac{m}{n}$ and $\frac{p}{q}$ are equal if \[
  mq=np
  .\]  We can define addition and multiplication of rational numbers in the usual way:
  \begin{itemize}
    \item (addition) $\frac{m}{n}+\frac{p}{q}=\frac{mq+np}{nq}$ 
    \item (multiplication) $\frac{m}{n}\cdot \frac{p}{q} =  \frac{mp}{nq}$
  \end{itemize}
\end{definition}
$\Q$ is a nice algebraic system, until we solve systems like $x^2=2$. It turns out that this
equation has no rational roots; but we know that by the Pythagorean theorem, this equation has 
positive roots.

\begin{example}
  Prove that $\sqrt{2} $ is not a rational number.
\end{example}
\begin{proof}[Proof]
  Suppose that $\sqrt{2} \in \Q$; then $\sqrt{2} =\frac{p}{q}$ for some $p,q\in \Z$, and $p,q$ have
  no common divisors other than $1$. Then
  \begin{gather*}
    2 = \frac{p^2}{q^2}\\
    2q^2=p^2\\
  \end{gather*}
  This implies $p^2$ is even; but $p^2$ is even if and only if $p$ is even. Then $p=2n$ for some  $
  n \in N$; $p^2=\left( 2n \right) ^2=4n^2$. From this, we get $2q^2=4n^2$, and so $q^2=2n^2$ and
  $q$ is even as well. Since both $p,q$ are even, they have a common divisor
   $2\neq 1$, a contradition. Thus $ \sqrt{2} $ is not rational.
\end{proof}
\begin{proposition}[Algebraic Numbers]{}
  A number is an \textbf{algebraic number} if it satisfies the polynomial equation \[
    c_nx^{n}+c_{n-1}x^{n-1}+\ldots+c_1x+c_0=0
  ,\] where $ c_0,c_1,\ldots, c_n$ are integers, $c_n \neq 0, n\ge 1$. We say that the polynomial
  equation has  \textbf{degree $n$}.
\end{proposition}

\begin{example}
  $ \sqrt{2} $ is an algebraic number because it satisfies \[
  x^2-2=0
  .\] 
\end{example}

\begin{example}
  Every rational number $x=\frac{p}{q}$ is an algebraic number because $x$ satisfies \[ qx-p=0 .\] 
\end{example}

\begin{example}
  $x=\sqrt{2+\sqrt[3]{5} } $ is an algebraic number because
  \begin{align*}
    x^2 &= 2+\sqrt[3]{5}  \\
    x^2-2 &= \sqrt[3]{5}  \\
    \left( x^2-2 \right) ^3&=5
  ,\end{align*} hence $x$ satisfies the polynomial \[
    \left( x^2-2 \right) ^{3}-5=0
  .\] 
\end{example}

From this, we see that algebraic numbers do not necessarily need to be rational numbers. The
question now is: \textit{when does a polynomial of order $n$ have rational roots?} 

\begin{theorem}[Rational Zeros Theorem]{rzt}
  Let $ c_0,c_1,\ldots,c_n\in \Z, c_n\neq 0,c_0\neq 0$. Suppose $r$ is a rational root of
  \begin{gather}
    P(x) = c_nx^{n}+c_{n-1}x^{n-1}+\ldots+c_1x+c_0=0 \label{eq1}
  \end{gather}
  Let $r=\frac{p}{q}, p,q\in \Z,q\neq 0, \gcd(p,q)=1$. Then $p$ divides $ c_0$ and $q$ divides $c_n$.
\end{theorem}
\begin{proof}[Proof]
  Suppose $r\in \Q$ is a rational root of \[
    P(x) = c_nx^{n}+\ldots+c_1x+c_0
    .\] Note that $r=\frac{p}{q}, p,q\in \Z, q\neq 0, \gcd(p,q)=1$. Then \[
  P(r) = P(\frac{p}{q}) = c_n\left( \frac{p}{q} \right) ^{n} + \ldots + c_1\left( \frac{p}{q}
  \right) + c_0 = 0
  .\] Multiplying both sides by $q^{n}$, we have \[
  c_np^{n} + c_{n-1}p^{n-1}q + \ldots + c_1pq^{n-1}+c_0q^{n} = 0
  .\] Then
  \begin{align*}
    c_np^{n} + \ldots + c_1pq^{n-1} &= -c_0q^{n}\\
    p\left( c_np^{n-1}+c_{n-1}p^{n-2}q+\ldots+c_1q^{n-1} \right) &= -c_0q^{n}
  .\end{align*} Thus, $-c_0q^{n}$ is divisible by $p$. Since $\gcd{\left(p,q\right)}=1$, it follows
  that $c_0$ is divisible by $p$. \\
  Similarly, if we move $c_np^{n}$ to the other side, and factor out $q$, then we get  \[
    q\left( c_{n-1}p^{n-1} + c_{n-2}p^{n-2}q+\ldots+c_1pq^{n-2} c_0q^{n-1}\right) = -c_np^{n}
  ,\] and since $c_np^{n}$ is divisible by $q$ and $\gcd{\left(p,q\right)}=1$, $c_n$ is divisible by
  $q$.

  Thus, if $r=\frac{p}{q}$ is a rational root of $P(x)$, then $p \mid c_0$ and $q \mid c_n$.
\end{proof}

\begin{remark}
  Theorem \ref{rzt} allows us to find all possible rational roots of \ref{eq1}. Specifically, given
  a \[
    P(x) = c_nx^{n}+c_{n-1}x^{n-1}+\ldots+c_1x+c_0
  ,\] if \[
    a\in \{p \in \Z \mid p ~\text{divides}~ c_0\}, b\in \{q\in \Z\mid q ~\text{divides}~c_n\},
  \]
  then any rational root must have the form $\frac{a}{b}$.\\

  For example, given \[
    P(x) = 3x^3+x^2-8x+4
  ,\] the only possible rational roots are \[
    r = \pm 1, \pm \frac{1}{3}, \pm 2, \pm \frac{2}{3}, \pm 4, ~\text{and}~ \pm \frac{4}{3}
  .\] 
\end{remark}

\begin{corollary}[]{}
  Consider the polynomial equation \[
    x^{n}+c_{n-1}x^{n-1}+\ldots+c_1x+c_0=0
  ,\] with integer coefficients and $ c_0\neq 0$. Any rational solution of this equation must be an
  integer that divides $c_0$.
\end{corollary}
This corollary allows us to determine whether a number is a rational number; if its polynomial has
no rational roots, then the number cannot be rational.
\begin{example}
  $\sqrt[]{17} $ is not a rational number.
\end{example}
\begin{proof}[Proof]
  The only possible rational solutions of $x^2-17=0$ are $\pm 1 ~\text{and}~\pm 17$, and none of
  these numbers are solutions. Hence $\sqrt[]{17} $ is not a rational number.
\end{proof}


\section{Real Numbers}

As nice as the rationals are, $\Q$ is inadequate to describe all systems. For instance, $x^2=2$
clearly has a solution of $\sqrt[]{2} $, but this is impossible to describe using rationals. We get
the intuition that $\Q$ has ``gaps''. We skip a rigorous development of $\Q$ based on $\Z$, but
observe that $\Q$ satisfies the properties of a  \textbf{field} .

\begin{definition}[Fields]{}
  A \textbf{field} $\F$ is an algebraic system (a set along with operations on elements of the set) that
  satisfies nine properties, for all $a,b,c\in \F$:
  \begin{enumerate}
    \item Additive associativity: $a+(b+c)=(a+b)+c$.
    \item Additive commutativity: $a+b=b+a$.
    \item Additive identity: $a+0=0+a=a$.
    \item Additive inverse: For each  $a$, there is an element $-a$ such that  $a+(-a)=(-a)+a=0$. 
    \item Multiplicative associativity: $a(bc)=(ab)c$.
    \item Multiplicative commutativity: $ab=ba$.
    \item Multiplicative identity:  $a\cdot 1=1\cdot a=a$.
    \item Multiplicative inverse: For each  $a\neq 0$, there is an element $a^{-1}$ such that
      $aa^{-1}=1$.
    \item Distributive law: $a(b+c)=ab+ac$.
  \end{enumerate}
\end{definition}

We can also define an order structure $\le $ for a field $\F$:
\begin{definition}[Order]{}
  A \textbf{ordered field} $\F$ satisfies five properties for all $a,b\in \F$:
  \begin{enumerate}
    \item Given $a,b$, either $a\le b$ or $b\le a$.
    \item If $a\le b$ and $b\le a$, then $a=b$.
    \item (Transitive Law) If $a\le b$ and $b\le c$, then $a\le c$.
    \item If $a\le b$, then $a+c\le b+c$.
    \item If $a\le b$ and $0\le c$, then $ac\le bc$.
  \end{enumerate}
\end{definition}

\begin{theorem}[Properties of Fields]{}
  The following properties are consequences of the field axioms for  $a,b,c\in \F$:
  \begin{enumerate}
    \item If $a+c=b+c$, then  $a=b$.
    \item  $a\cdot 0=0$ for all $a$.
    \item  $(-a)b=-ab$ for all  $a,b$.
    \item  $(-a)(-b)=ab$ for all  $a,b$.
    \item  If $ac=bc$ and  $c\neq 0$, then $a=b$.
    \item  If $ab=0$, then either  $a=0$ or  $b=0$.
  \end{enumerate}
\end{theorem}
\begin{proof}[Proof]
  We provide a proof for some properties (the rest are left as an exercise for the reader).
  \begin{enumerate}
    \item $a+c=b+c$ implies  $a+c+(-c)=b+c+(-c)$, which becomes $a+0=b+0 \implies a=b$.
    \item $a\cdot 0=a\cdot (0+0)=a\cdot 0+a\cdot 0$, which implies $0+a\cdot 0=a\cdot 0+a\cdot 0$.
      By property 1, we get $0=0\cdot a$.
    \item $(-a)b + ab = (a+(-a))b = 0\cdot b = 0$, which implies $(-a)b$ is the inverse of  $ab$,
      and so  $(-a)b=-ab$. 
    \item Exercise
    \item Exercise
    \item If $ab=0$, and WLOG $b\neq 0$, we have $0=0\cdot b^{-1}=(ab)\cdot b^{-1}=a(bb^{-1})=a\cdot
      1=a$, which implies $a=0$.
  \end{enumerate}
\end{proof}

\begin{theorem}[Properties of Ordered Fields]{}
  The following properties are consequences of the ordered field properties for $a,b,c\in \F$:
  \begin{enumerate}
    \item If $a\le b$, then $-b\le -a$.
    \item If $a\le b$ and $c\le 0$, then $bc\le ac$.
    \item If $0\le a$ and $0\le b$, then $0\le ab$.
    \item $0\le a^2$ for all $a$.
    \item $0<1$.
    \item If $0<a$, then  $0<a^{-1}$.
    \item If $0<a<b$, then $0<b^{-1}<a^{-1}$.
  \end{enumerate}
\end{theorem}
\begin{proof}[Proof]
  \begin{enumerate}
    \item From property 4 of ordered fields, we choose $c=(-a) + (-b)$.
  \end{enumerate}
\end{proof}



The algebraic system of interest in analysis will be the \textbf{real numbers}, or $\R$, which will
include all rational numbers and irrational numbers; it has no ``gaps''. It turns out that $\R$ can
be defined entirely in terms of $\Q$. [TODO]\\

$\R$ satisfies the same ordered field axioms as $\Q$ (since $\R$ is also a field).





\end{document}
