\documentclass[math1010-lecture-notes]{subfiles}
\begin{document}

\chapter{Real Numbers}
In this class, we operate on the set of real numbers. It is important to rigorously define it, as
well as all underlying sets of numbers (natural numbers, integers, rationals, etc).

\section{Natural Numbers}
\begin{definition}[Natural Numbers]{}
  The set $\N = {1, 2, \ldots}$ is the set of natural numbers. Each integer $n$ has a
  \textbf{successor}, $succ(n) = n + 1$. $1$ is not the successor of any number. \\
  The following properties constitute the Peano Axioms of $\N$:
  \begin{enumerate}
    \item $1\in \N$.
    \item $n\in \N \Rightarrow succ(n)=n+1 \in \N$
    \item $\not\exists n ~\text{s.t.}~ succ(n) = 1$
    \item If $n, m\in \N, succ(n) = succ(m)$, then $n=m$.
    \item A subset $A\subset \N$ which contains $1$, and which contains  $n+1$ whenever it contains
       $n$, must equal $\N$.
  \end{enumerate}
  We accept only these 5 axioms to prove all other properties of $\N$.
\end{definition}
(5) is the basis for the principle of induction.

\begin{theorem}[Principle of Mathematical Induction]{}
  Let $ P_1, P_2, \ldots$ be a list of statements. Assume the following:
  \begin{enumerate}
    \item $P_1$ is true. [ Basis of induction ] 
    \item $\forall n\in N, n\ge 1$, if $P_n$ is true, then  $P_{n+1}$ is true. [ Inductive step ]
  \end{enumerate}
  Then all the statements $P_1, \ldots$ are true.
\end{theorem}
\begin{proof}[Proof]
  Let $A$ be the set of integers $n$ for which  $P_n$ is true. We want to prove $A=\N$. We use (5)
  to prove this. \\
  Indeed, $1\in A$ by assumption 1. Assuming that $n\in A$ for some $n$, we prove that  $n+1\in A$.
  This is true by assumption 2: if $n\in A$, then $P_n$ is true, hence  $P_{n+1}$ is true, hence
  $n+1\in A$.
  Thus, $A=\N$.
\end{proof}

\begin{example}
  Prove that $2^{1}+2^2+\ldots+2^{n}=2^{n+1}-2$.
\end{example}

\begin{proof}
  Let $P_n$: "$2^{1}+2^2+\ldots+2^{n}=2^{n+1}-2$".\\
  $P_1$ is true because $2^{1}=2^2-2$. \\
  For the induction step, we assume $P_n$ is true for some  $n$ and prove $P_{n+1}$ is true. Since
   $P_n$ is true,  \[
   2^{1}+2^2+\ldots+2^{n} = 2^{n+1}-2
 .\]  $P_{n+1}$ states that  \[
 2^{1}+\ldots+2^{n}+2^{n+1} = 2^{\left( n+1 \right) +1}-2
 .\] Using $P_n$ we have  \[
 2^1+\ldots+2^{n + 2^{n+1}} = \left( 2^{n+1}-2 \right) +2^{n+1} = 2\cdot 2^{n+1}-2=2^{\left( n+1
 \right) +1}-2
 ,\] Thus $P_{n+1}$ is true. 
 By the principle of induction, $P_n$ is true for all  $n\ge 1$.
\end{proof}


\begin{example}
  Let $x_1=1$ and define \[
    x_{n+1}=\frac{1}{2}x_n+1
  .\] 
  Prove that $\forall x, x_n\le x_{n+1}$ (or, $x_n$ is inreasing).
\end{example}

\begin{proof}[Proof]
  Let $P_n$: "$x_n\le x_{n+1}$". \\
  $P_1$ is true because $x_1=1\le \frac{3}{2}=x_2$.\\
  For the induction step, we assume $P_n$ is true for some  $n$ and prove $P_{n+1}$ is true. Since
  $P_n$ is true,  \[
    x_n \le x_{n+1}
  .\] $P_{n+1}$ states that  \[
  x_{n+1} \le x_{n+2}
.\] Using $P_n$ we have  \[
x_{n+1} = \frac{1}{2}x_n+1\le \frac{1}{2}x_{n+1}+1=x_{n+2} \hfill~\text{[by $P_n$, we know $x_n\le x_{n+1}$]}~
\] Thus $P_{n+1}$ is true.\\
By the principle of induction, $P_n$ is true for all $n\ge 1$.
\end{proof}

The principle of induction can be extended by allowing the first statement to begin at $P_m$ instead
of  $P_1$ for some fixed integer $m$.

 \begin{theorem}[Generalized Principle of Induction]{}
   Let $m$ be an integer, and consider a list of statements $P_m,P_{m+1},\ldots$. Then all the
   statements are true if the following two properties are true:
   \begin{enumerate}
     \item $P_m$ is true
     \item  $\forall n\ge m$, if $P_n$ is true, then  $P_{n+1}$ is true.
   \end{enumerate}
\end{theorem}

\begin{example}
  Prove that \[
  n! > n^2
  .\] for all $n\ge 4$.
\end{example}
\begin{proof}[Proof]
  Recall $n! = 1\cdot 2\cdot \ldots\cdot n$. Let $P_n$: "$n! > n^2$". We prove $P_n$ is true
  $\forall n\ge 4$.\\
  $P_4$ is true because  \[
  4! = 24 > 16 = 4^2
  .\] Assuming $n! > n^2$, we prove \[
  \left( n+1 \right)! > \left( n+1 \right) ^2
  .\] Using $P_n$ we have  \[
  \left( n+1 \right)! = n!\left( n+1 \right) = \left( 1\cdot 2\cdot \ldots\cdot n \right) \cdot \left( n+1 \right) > n^2(n+1) > \left( n+1
  \right) ^2
.\] Thus $P_{n+1}$ is true. \\
By the principle of induction, $P_n$ is true for all  $n\ge 4$.
\end{proof}




\end{document}
