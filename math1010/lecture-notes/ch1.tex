\documentclass[math1010-lecture-notes]{subfiles}
\begin{document}

\chapter{Real Numbers}
In this class, we operate on the set of real numbers. It is important to rigorously define it, as
well as all underlying sets of numbers (natural numbers, integers, rationals, etc).

\section{Natural Numbers}
\begin{definition}[Natural Numbers]{}
  The set $\N = {1, 2, \ldots}$ is the set of natural numbers. Each integer $n$ has a
  \textbf{successor}, $succ(n) = n + 1$. $1$ is not the successor of any number. \\
  The following properties constitute the Peano Axioms of $\N$:
  \begin{enumerate}
    \item $1\in \N$.
    \item $n\in \N \Rightarrow succ(n)=n+1 \in \N$
    \item $\not\exists n ~\text{s.t.}~ succ(n) = 1$
    \item If $n, m\in \N, succ(n) = succ(m)$, then $n=m$.
    \item A subset $A\subset \N$ which contains $1$, and which contains  $n+1$ whenever it contains
       $n$, must equal $\N$.
  \end{enumerate}
  We accept only these 5 axioms to prove all other properties of $\N$.
\end{definition}
(5) is the basis for the principle of induction.

\begin{theorem}[Principle of Mathematical Induction]{}
  Let $ P_1, P_2, \ldots$ be a list of statements. Assume the following:
  \begin{enumerate}
    \item $P_1$ is true. [ Basis of induction ] 
    \item $\forall n\in N, n\ge 1$, if $P_n$ is true, then  $P_{n+1}$ is true. [ Inductive step ]
  \end{enumerate}
  Then all the statements $P_1, \ldots$ are true.
\end{theorem}
\begin{proof}[Proof]
  Let $A$ be the set of integers $n$ for which  $P_n$ is true. We want to prove $A=\N$. We use (5)
  to prove this. \\
  Indeed, $1\in A$ by assumption 1. Assuming that $n\in A$ for some $n$, we prove that  $n+1\in A$.
  This is true by assumption 2: if $n\in A$, then $P_n$ is true, hence  $P_{n+1}$ is true, hence
  $n+1\in A$.
  Thus, $A=\N$.
\end{proof}

\begin{example}
  Prove that $2^{1}+2^2+\ldots+2^{n}=2^{n+1}-2$.
\end{example}

\begin{proof}
  Let $P_n$: "$2^{1}+2^2+\ldots+2^{n}=2^{n+1}-2$".\\
  $P_1$ is true because $2^{1}=2^2-2$. \\
  For the induction step, we assume $P_n$ is true for some  $n$ and prove $P_{n+1}$ is true. Since
   $P_n$ is true,  \[
   2^{1}+2^2+\ldots+2^{n} = 2^{n+1}-2
 .\]  $P_{n+1}$ states that  \[
 2^{1}+\ldots+2^{n}+2^{n+1} = 2^{\left( n+1 \right) +1}-2
 .\] Using $P_n$ we have  \[
 2^1+\ldots+2^{n + 2^{n+1}} = \left( 2^{n+1}-2 \right) +2^{n+1} = 2\cdot 2^{n+1}-2=2^{\left( n+1
 \right) +1}-2
 ,\] Thus $P_{n+1}$ is true. 
 By the principle of induction, $P_n$ is true for all  $n\ge 1$.
\end{proof}


\begin{example}
  Let $x_1=1$ and define \[
    x_{n+1}=\frac{1}{2}x_n+1
  .\] 
  Prove that $\forall x, x_n\le x_{n+1}$ (or, $x_n$ is inreasing).
\end{example}

\begin{proof}[Proof]
  Let $P_n$: "$x_n\le x_{n+1}$". \\
  $P_1$ is true because $x_1=1\le \frac{3}{2}=x_2$.\\
  For the induction step, we assume $P_n$ is true for some  $n$ and prove $P_{n+1}$ is true. Since
  $P_n$ is true,  \[
    x_n \le x_{n+1}
  .\] $P_{n+1}$ states that  \[
  x_{n+1} \le x_{n+2}
.\] Using $P_n$ we have  \[
x_{n+1} = \frac{1}{2}x_n+1\le \frac{1}{2}x_{n+1}+1=x_{n+2} \hfill~\text{[by $P_n$, we know $x_n\le x_{n+1}$]}~
\] Thus $P_{n+1}$ is true.\\
By the principle of induction, $P_n$ is true for all $n\ge 1$.
\end{proof}

The principle of induction can be extended by allowing the first statement to begin at $P_m$ instead
of  $P_1$ for some fixed integer $m$.

 \begin{theorem}[Generalized Principle of Induction]{}
   Let $m$ be an integer, and consider a list of statements $P_m,P_{m+1},\ldots$. Then all the
   statements are true if the following two properties are true:
   \begin{enumerate}
     \item $P_m$ is true
     \item  $\forall n\ge m$, if $P_n$ is true, then  $P_{n+1}$ is true.
   \end{enumerate}
\end{theorem}
Theorem $\S$ 1.1.2 follows from
\begin{proposition}[Specific Case of (5)]{}
  Let $m\in N$. Assume that a subset  $A\subset \{m, m+1, \ldots\} $ contains $m$ and $n+1$ whenever
  it contains $n$. Then $A=\{ m,m+1,\ldots \}$.
\end{proposition}
\begin{proof}[Proof]
  In Peano Axiom 5, we start from $ 1$; here, we start from $m$. Let $B=\{p=n-(m-1)\mid n\in A\} $.
  Since $A\subset \{ m,m+1,\ldots \}$, $B\subset \{1, 2, \ldots\}=\N$. \\
  We observe that $1\in B$ (because $1=m-(m-1)$, and $m\in A$ by definition). Assuming $p\in B$, we
  have $p=n-(m-1)$ for some $n\in A$. Then $p+1=(n+1)-(m-1)\in B$, as we state that $n+1\in A$
  whenever $n\in A$. Thus, using Peano Axiom 5, we see that $A=B$. \\
  Now, by definition of $B$, we have
  \begin{align*}
    A &= \{n = p + \left( m-1 \right) \mid p \in B \}  \\
      &=  \{n=p+\left( m-1 \right) \mid p \in \N\} \\
      &=  \{m, m+1, \ldots\}  \\
  \end{align*}
\end{proof}




\begin{example}
  Prove that \[
  n! > n^2
  .\] for all $n\ge 4$.
\end{example}
\begin{proof}[Proof]
  Recall $n! = 1\cdot 2\cdot \ldots\cdot n$. Let $P_n$: "$n! > n^2$". We prove $P_n$ is true
  $\forall n\ge 4$.\\
  $P_4$ is true because  \[
  4! = 24 > 16 = 4^2
  .\] Assuming $n! > n^2$, we prove \[
  \left( n+1 \right)! > \left( n+1 \right) ^2
  .\] Using $P_n$ we have  \[
  \left( n+1 \right)! = n!\left( n+1 \right) = \left( 1\cdot 2\cdot \ldots\cdot n \right) \cdot \left( n+1 \right) > n^2(n+1) > \left( n+1 \right) ^2
.\] Thus $P_{n+1}$ is true. \\
By the principle of induction, $P_n$ is true for all  $n\ge 4$.
\end{proof} 


\section{Rational Numbers}
\begin{definition}[Integers]{}
  The set of integers is denoted by \[
    \Z=\{ \ldots -2,-1,0,1,2,\ldots \}
  .\] 
  A rigorous construction is omitted, and left as an exercise for the reader.
\end{definition}

\begin{definition}[Rational Numbers]{}
  The set $\Q$ of rational numbers is \[
    \Q = \{\frac{m}{n}\mid m,n\in \Z,n\neq 0\} 
  .\] We say $\frac{m}{n}$ and $\frac{p}{q}$ are equal if \[
  mq=np
  .\]  We can define addition and multiplication of rational numbers in the usual way:
  \begin{itemize}
    \item (addition) $\frac{m}{n}+\frac{p}{q}=\frac{mq+np}{nq}$ 
    \item (multiplication) $\frac{m}{n}\cdot \frac{p}{q} =  \frac{mp}{nq}$
  \end{itemize}
\end{definition}
$\Q$ is a nice algebraic system, until we solve systems like $x^2=2$. It turns out that this
equation has no rational roots; but we know that by the Pythagorean theorem, this equation has 
positive roots.

\begin{example}
  Prove that $\sqrt{2} $ is not a rational number.
\end{example}
\begin{proof}[Proof]
  Suppose that $\sqrt{2} \in \Q$; then $\sqrt{2} =\frac{p}{q}$ for some $p,q\in \Z$, and $p,q$ have
  no common divisors other than $1$. Then
  \begin{gather*}
    2 = \frac{p^2}{q^2}\\
    2q^2=p^2\\
  \end{gather*}
  This implies $p^2$ is even; but $p^2$ is even if and only if $p$ is even. Then $p=2n$ for some  $
  n \in N$; $p^2=\left( 2n \right) ^2=4n^2$. From this, we get $2q^2=4n^2$, and so $q^2=2n^2$ and
  $q$ is even as well. Since both $p,q$ are even, they have a common divisor
   $2\neq 1$, a contradition. Thus $ \sqrt{2} $ is not rational.
\end{proof}
\begin{proposition}[Algebraic Numbers]{}
  A number is an \textbf{algebraic number} if it satisfies the polynomial equation \[
    c_nx^{n}+c_{n-1}x^{n-1}+\ldots+c_1x+c_0=0
  ,\] where $ c_0,c_1,\ldots, c_n$ are integers, $c_n \neq 0, n\ge 1$. We say that the polynomial
  equation has  \textbf{degree $n$}.
\end{proposition}

\begin{example}
  $ \sqrt{2} $ is an algebraic number because it satisfies \[
  x^2-2=0
  .\] 
\end{example}

\begin{example}
  Every rational number $x=\frac{p}{q}$ is an algebraic number because $x$ satisfies \[ qx-p=0 .\] 
\end{example}

\begin{example}
  $x=\sqrt{2+\sqrt[3]{5} } $ is an algebraic number because
  \begin{align*}
    x^2 &= 2+\sqrt[3]{5}  \\
    x^2-2 &= \sqrt[3]{5}  \\
    \left( x^2-2 \right) ^3&=5
  ,\end{align*} hence $x$ satisfies the polynomial \[
    \left( x^2-2 \right) ^{3}-5=0
  .\] 
\end{example}

From this, we see that algebraic numbers do not necessarily need to be rational numbers. The
question now is: \textit{when does a polynomial of order $n$ have rational roots?} 

\begin{theorem}[Rational Zeros Theorem]{rzt}
  Let $ c_0,c_1,\ldots,c_n\in \Z, c_n\neq 0,c_0\neq 0$. Suppose $r$ is a rational root of
  \begin{gather}
    c_nx^{n}+c_{n-1}x^{n-1}+\ldots+c_1x+c_0=0 \label{eq1}
  \end{gather}
  Let $r=\frac{p}{q}, p,q\in \Z,q\neq 0, gdc(p,q)=1$. Then $p$ divides $ c_0$ and $q$ divides $c_n$.
\end{theorem}
\begin{remark}
  Theorem \ref{rzt} allows us to find all possible rational roots of \ref{eq1}.
\end{remark}






\end{document}
