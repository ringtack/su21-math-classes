\documentclass{homework}
\homework{2}

\begin{document}
\begin{problem}{\S 1} (1.6)
   Prove that $7$ divides $11^{n}-4^{n}$ for all $n\in \Z^{+}$.
\end{problem}
\begin{proof}[Proof]
  Let $P_n$: ``$7$ divides $11^{n}-4^{n}$ for some $n\in \Z^{+}$ ''. \\
  $ P_1$ is true because $11-4=7$.\\
  Now, assume  $P_n$ is true. Then  \[
  11^{n}-4^{n} = 7k,\ k\in \Z^{+}
.\] To prove $P_{n+1}$ from $P_n$, we have
 \begin{align*}
  11^{n+1}-4^{n+1} &= 11^{n}\cdot 11-4^{n}\cdot 4 \\
  &= 11^{n}\cdot 11-4^{n}\cdot 11+4^{n}\cdot 11-4^{n}\cdot 4 \\
  &= \left( 11^{n}\cdot 11-4^{n}\cdot 11 \right) +\left( 4^{n}\cdot 11-4^{n}\cdot 11 \right)  \\
  &= 11\cdot 7k+7\cdot 4^{n} \\
  &= 7\left( 11k+4^{n} \right)
.\end{align*} Hence $P_{n+1}$ is true whenever $P_n$ is true, and thus by mathematical induction the
statement is true.
\end{proof}

\begin{problem}{\S 2}
  (1.11) Let $P_n$: `` $n^2+5n+1$ is even.''
  \begin{enumerate}[label=(\alph*)]
    \item Show that $P_{n+1}$ is true whenever  $P_n$ is true.
    \item For which  $P_n$ is the statement actually true? What is the moral of this exercise?
  \end{enumerate}
\end{problem}
\begin{solution}
  \begin{enumerate}[label=(\alph*)]
    \item Assume $P_n$ is true. Then  \[
    n^2+5n+1=2k
  .\] To prove $P_{n+1}$ from  $P_n$, we have
  \begin{align*}
    (n+1)^2+5(n+1)+1 &= n^2+2n+1+5n+5+1 \\
                     &= (n^2+5n+1)+2n+6 \\
                     &= 2k+2(n+3) \\
                     &= 2(k+n+6)
                   .\end{align*}
     Hence $P_{n+1}$ is true whenever  $P_n$ is true, and thus by
     mathematical induction the statement is true. 
   \item None! $ P_1=7,\ P_2=15,\ldots$ In general, $P_n$ is always odd:
      \begin{itemize}
        \item If $n=2k$ (even): $4k^2+10k+1 \implies$ odd
        \item If $n=2k+1$ (odd):  $4k^2+4k+1+10k+5+1=2(2k^2+7k+3)+1 \implies$ odd
     \end{itemize}
    Thus, in order for mathematical induction to be valid, there must be a base case.
  \end{enumerate}
\end{solution}

\begin{problem}{\S 3}
  Prove that \[
    \left( 1-\frac{1}{\sqrt{2} } \right) \ldots\left( 1-\frac{1}{\sqrt{n} } \right) < \frac{2}{n^2}
  .\] for all $n\ge 2$.
\end{problem}
\begin{proof}[Proof]
  Let the above statement denote $P_n$.\\
   $P_2$ is true:
    \begin{align*}
      1-\frac{1}{\sqrt{2}} = 1 - \frac{\sqrt{2}}{2} &< \frac{1}{2}\\
      \frac{1}{2}&<\frac{\sqrt{2} }{2}\\
      \frac{1}{4}&<\frac{2}{4}
   .\end{align*}
   Now, assume $P_n$ is true. Then \[
    \left( 1-\frac{1}{\sqrt{2} } \right) \ldots\left( 1-\frac{1}{\sqrt{n} } \right) < \frac{2}{n^2} 
   .\] 
   Let $a_n = 1-\frac{1}{\sqrt{n} }$. Then $a_{n+1} = 1-\frac{1}{\sqrt{n+1} }=\frac{\sqrt{n+1}
   -1}{\sqrt{n+1} }=\frac{n}{n+1+\sqrt{n+1} }$. We can rewrite $P_n$ using logarithms:  \[
     \log{\left(a_1\right)}+\ldots+\log{\left(a_n\right)} < \log{\left(2\right)}-2\log{\left(n\right)}
   .\] To prove $P_{n+1}$ from  $P_n$, we have 
    \begin{align*}
      \log{\left(a_1\right)}+\ldots+\log{\left(a_n\right)}+\log{\left(a_{n+1}\right)} &<
        \log{\left(2\right)}-\log{\left(n\right)}+\log{\left(a_{n+1}\right)} \\
              &= \log{\left(2\right)}-\log{\left(n\right)}+\log{\left(n\right)}-\log{\left((n+1+\sqrt{n+1} )\right)} \\
              &= \log{\left(2\right)}-\log{\left(n+1+\sqrt{n+1} \right)}
   .\end{align*} We now show that $\log{\left(n+1+\sqrt{n+1} \right)}$ is greater than
   $2\log{\left(n+1\right)}$ (and so its reciprocal is less), which completes the proof.
   \begin{align*}
     \log{\left(n+1+\sqrt{n+1} \right)} &> 2\log{\left(n+1\right)}\\
     n+1+\sqrt{n+1} &> (n+1)^2=n^2+2n+1\\
     n\sqrt{n+1} &> n+1\\
     n^2(n+1) &> n^2+2n+1\\
     n^3+n^2 &> n^2+2n+1\\
     n^3&>2n+1\\
     n^3-2n&>1
   ,\end{align*} which is clearly true for all $n\ge 2$, and so $ n+1+\sqrt{n+1} >
   (n+1)^2=n^2+2n+1$. From this we get (after removing logs) \[
     \frac{2}{n+1+\sqrt{n+1} } < \frac{2}{(n+1)^2}
   ,\] completing the proof. \\
   Hence $P_{n+1}$ is true whenever  $P_n$ is true, and thus by mathematical induction the statement
   is true.
\end{proof}

\begin{problem}{\S 4}
  Prove that for all $n\ge 3$, there exist different natural numbers $a1,a_2,\ldots, a_n$ such that
  \[
    1 = \frac{1}{a_1} + \ldots + \frac{1}{a_n}
  .\] 
\end{problem}
\begin{proof}[Proof]
  We begin by observing $n=3,4,5$.
   \begin{itemize}
    \item For $n=3$:  $ a_1=2,a_2=3,a_3=6$
    \item For $n=4$:  $ a_1=2,a_2=4,a_3=6,a_4=12$
    \item For $n=5$:  $ a_1=2,a_2=4,a_3=8,a_4=12,a_5=24$
  \end{itemize}
  From this, we get a pattern: for an $a_{n-1},a_n$, we have  $a_n = 2a_{n-1}$. Moreover, when we
  add a new number,  $a_{n-1}$ is updated, and $a_{n+1}=2a_n$. Formally, we define  \[
    P_n: ~\text{There exists different natural numbers}~a_1,\ldots, a_n,
    ~\text{with}~a_n=2a_{n-1},~\text{such that}~
    1=\frac{1}{a_1}+\frac{1}{a_2}+\ldots+\frac{1}{a_{n-1}}+\frac{1}{a_n}
  .\] 
  From above, we see that $P_3$ is true.\\
  Suppose $P_n$ is true. Then  \[
    1 = \frac{1}{a_1}  +\ldots+\frac{1}{a_{n-1}}+\frac{1}{a_n}
  .\] To prove $P_{n+1}$ from  $P_n$, we have  \[
  1 = \frac{1}{b_1} + \ldots+\frac{1}{b_{n-1}} + \frac{1}{b_n} + \frac{1}{b_{n+1}}
.\] Choose $a_i=b_i$ for all  $1\le i < n-1$ and $i=n$. Let  $b_{n-1}=\frac{2a_n}{3},
  b_{n+1}=2a_n$. Then
  \begin{align*}
    1 &= \frac{1}{b_1}+\ldots+\frac{1}{b_{n-1}}+\frac{1}{b_n}+\frac{1}{b_{n+1}} \\
    1 &= \frac{1}{a_1}+\ldots+\frac{3}{2a_n}+\frac{1}{a_n}+\frac{1}{2a_n} \\
    1 &= \frac{1}{a_1}+\ldots+\frac{2}{a_n}+\frac{1}{a_n} \\
    1 &= \frac{1}{a_1}+\ldots+\frac{2}{2a_{n-1}}+\frac{1}{a_n} \\
    1 &= \frac{1}{a_1}+\ldots+\frac{1}{a_{n-1}}+\frac{1}{a_n}
  .\end{align*}
  Hence $P_{n+1}$ is true whenever  $P_n$ is true, and thus by mathematical induction the statement
  is true.
\end{proof}

\begin{problem}{\S 5}
  (2.4) Show that $\sqrt[3]{5-\sqrt{3} } \not\in \Q$.
\end{problem}
\begin{solution}
  Let $a=\sqrt[3]{5-\sqrt{3} }$. Then
  \begin{align*}
    a^3&= 5-\sqrt{3}  \\
    \sqrt{3} &= 5-a^3 \\
    3 &= 25-10a^3+a^{6} \\
    0 &= a^{6}-10a^3+22
  .\end{align*}
  By the Rational Roots Theorem, we see that the only possible rational solutions are $\pm 1,\pm
  2,\pm 11,\pm 22$. Simple inspection by plugging in each possible rational solution indicates that
  none of them work, and so $\sqrt[3]{5-\sqrt{3} }$ is not rational.
\end{solution}

\begin{problem}{\S 6}
  (2.7) Show that
  \begin{enumerate}[label=(\alph*)]
    \item $\sqrt{4+2\sqrt{3} } - \sqrt{3}$
    \item $\sqrt{6+4\sqrt{2} } -\sqrt{2} $
  \end{enumerate}
  are actually rational.
\end{problem}
\begin{solution}
  We observe that the insides of the large square roots are actually perfect squares.
  \begin{enumerate}[label=(\alph*)]
    \item 
      \begin{align*}
        \sqrt{4+2\sqrt{3}}  &= \sqrt{3+2\sqrt{3}\cdot 1+1}  \\
                            &= \sqrt{\left( \sqrt{3}+1  \right)^2 }  \\
                            &= \sqrt{3} +1
      .\end{align*}
      From this, we get $\sqrt{4+2\sqrt{3} } - \sqrt{3} =\sqrt{3} +1-\sqrt{3} =1\in \Q$.
    \item
      \begin{align*}
        \sqrt[]{6+4\sqrt{2} } &= \sqrt{4+2\cdot 2\cdot \sqrt{2} +2}  \\
                              &= \sqrt{\left( 2+\sqrt{2}  \right) ^2}  \\
                              &= 2+\sqrt{2} 
      .\end{align*}
      From this, we get $\sqrt{6+4\sqrt{2} } -\sqrt{2} =2+\sqrt{2} -\sqrt{2} =2\in \Q$.
  \end{enumerate}
  Thus both are actually rational.
\end{solution}

\begin{problem}{\S 7}
  Find all rational solutions of the equation $3x^3+x^2-8x+4=0$.
  
\end{problem}

\begin{solution}
  By the Rational Root Theorem, the only possible rational solutions are of the form $\pm 1,\pm
  \frac{1}{3},\pm 2,\pm \frac{2}{3},\\\pm 4,~\text{and}~\pm \frac{4}{3}$. By plugging in each possible
  value, we observe that  \textbf{$1,-2,\frac{2}{3}$}  satisfy the above equation, and thus are the
  three rational roots of $3x^3+x^2-8x+4=0$.
\end{solution}







\end{document}
