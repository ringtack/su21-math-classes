\documentclass{homework}
\homework{3}

\begin{document}

\begin{problem}{\S 1}
  (3.3) Prove, from Theorem 3.1, that for all $a,b,c \in \F$,
  \begin{enumerate}[label=(\alph*)]
    \item $(-a)(-b)=ab$ 
    \item If $ac=bc$, and $c\neq 0$, then $a=b$.
  \end{enumerate}
\end{problem}

\begin{solution}
  \begin{enumerate}[label=(\alph*)]
    \item We have
      \begin{align*}
        -ab+(-a)(-b) &= (-a)b+(-a)(-b) && [~\text{Theorem 3.1.iii}~] \\
                     &= (-a)(b+ -b) && [~\text{DL}~] \\
                     &= (-a)0 \\
                     &= 0 && [~\text{Theorem 3.1.ii}~]
                   .\end{align*} Hence $(-a)(-b)$ is an inverse of $-ab$, and so $(-a)(-b)=ab$.
    \item If $c\neq 0$, then by M4,  $c^{-1}\in \F$ exists. Then
      \begin{align*}
        ac&= bc \\
        acc^{-1}&=bc c^{-1}\\
        a\cdot 1&= b\cdot 1 \\
        a&=b
      ,\end{align*} as required.
  \end{enumerate}
\end{solution}

\begin{problem}{\S 2}
  (3.4) Prove, from Theorem 3.2, that for all $a,b$ in an ordered field $\F$, 
  \begin{enumerate}[label=(\alph*)]
    \item $0 < 1$.
    \item If $0<a<b$, then $0<b^{-1}<a^{-1}$.
  \end{enumerate}
\end{problem}

\begin{solution}
  \begin{enumerate}[label=(\alph*)]
    \item We start with a lemma.
      \begin{lemma}[$0\neq 1$]{}
        Suppose $0=1$, and let $a\in \F$ be a non-zero element in an arbitrary field. Then
        \begin{align*}
          a &= a\cdot 1 && [~\text{M3}~] \\
            &= a\cdot 0 &&[~\text{by assumption}~] \\
            &= 0 && [~\text{Theorem 3.1.ii}~]
        ,\end{align*}
        a contradiction of $a$ non-zero. Hence $0\neq 1$.
      \end{lemma}
      From Theorem 3.2.iv, we have  $0\le a^2$ for all $a$. Then $ 0\le 1$; and by the lemma, $0<1$.
      
    \item By Theorem 3.2.vi, if $0<a,\ 0<b$, then $0<a^{-1},\ 0<b^{-1}$; and by Theorem 3.2.iii, we
      have $0<a^{-1}b^{-1}$. Let $c=a^{-1}b^{-1}$. Then from O5, we have 
      \begin{align*}
        ac &< bc\\
        aa^{-1}b&<a^{-1}b^{-1}b\\
        b^{-1}<a^{-1}
      .\end{align*} Since both are greater than zero, we have $0<b^{-1}<a^{-1}$.
  \end{enumerate}
\end{solution}

\begin{problem}{\S 3}
  (4.7) Let $S,T$ be bounded subsets of $\R$.
  \begin{enumerate}[label=(\alph*)]
    \item Prove that $S\subseteq T$, then  $\inf T \le \inf S \le \sup S \le \sup T$.
    \item Prove that $\sup(S\cup T)=\max\{ \sup S,\sup T \}$.
  \end{enumerate}
\end{problem}

\begin{solution}
  \begin{enumerate}[label=(\alph*)]
    \item We know, by definition, that $\inf{S}\le \sup{S}$.

      For any $s \in S$, we have $s \in T$; and by definition, we have $ \inf{T}\le s$. Thus
      $\inf{T}$ is a lower bound for $S$, and so $\inf{T}\le \inf{S}$.

      Similarly, by definition we have $s\le \sup{T}$. Thus $\sup{T}$ is an upper bound for $S$, and
      so $\sup{S}\le \sup{T}$.

    \item Suppose, without loss of generality, that $\sup{S}\ge \sup{T}$. We know that for any $a\in
      S\cup T$ that $a\in S$ or $a\in T$. $a\in S$ implies that $a\le \sup{S}$ by definition, so let
      $a\in T$. Then $a\le \sup{T}$; but $\sup{T}\le \sup{S}$ by our assumption. Thus $\sup{S}$ is
      an upper bound for $S\cup T$. Additionally, since $\sup{S}$ is by definition the least upper
      bound of $S$, and any $t\in T$ is bounded above by $\sup{T}\le \sup{S}$, any $a\in S\cup T$
      has $\sup{S}$ as its least upper bound; and so $\sup{(S\cup T)}=\sup{S}=\max\{ \sup{S},\sup{T}
      \}$. An analogous argument follows if $\sup{T}\ge \sup{S}$.
  \end{enumerate}
\end{solution}

\begin{problem}{\S 4}
  (4.8) Let $S,T$ be non-empty subsets of $\R$, and for all $s \in S, t\in T$, $s\le t$.
  \begin{enumerate}[label=(\alph*)]
    \item Observe that $S$ is bounded above, and $T$ is bounded below.
    \item Prove that $\sup{S}\le \inf{T}$.
    \item Give an example of such $S,T$ where $S\cap T$ is non-empty.
    \item Give an example of such $S,T$ where $S\cap T=\varnothing$.
  \end{enumerate}
\end{problem}

\begin{solution}
  \begin{enumerate}[label=(\alph*)]
    \item Any $t\in T$ bounds $S$ from above, and any $s \in S$ bounds $T$ from above.
    \item For any $s \in S$, $s\le \sup{S}$, and for any $a\in \R$ that satisfies $s\le a$,
      $\sup{S}\le a$. But since any $t\in T$ satisfies $s\le t$, we have $\sup{S}\le t$.

      Let $\textbf{m}$ be the set of all $m\in \R$ such that $m\le t$. By definition, for any $m\in
      \textbf{m}$, $m\le \inf{T}$; and since $\sup{S}\in \textbf{m} $, we have $\sup{S}\le \inf{T}$.

    \item Let $S=(-1,0],\ T=[0,1)$. Then  $\sup{S}=0\le 0=\inf{T}$, and $S\cap T=\{ 0 \}\neq
      \varnothing$.
    \item Let $S=(-1,0),\ T=(0,1)$. Then  $\sup{S}=0=\inf{T}$, and $S\cap T=\varnothing$.
  \end{enumerate}
\end{solution}

\begin{problem}{\S 5}
  (4.12) Let $\I$ be the set of irrational numbers. Prove that if $a<b$, then there exists an $x\in
  \I$ such that $a<x<b$.
\end{problem}
\begin{solution}
  We start with two lemmas.
  \begin{lemma}[]{}
    If $a\in \I$ and $b\in \Q$, then $a+b\in \I$ (in other words, irrational + rational =
    irrational).
  \end{lemma}
  \begin{proof}[Proof]
    Let $a\in \I$, $b\in \Q$. Then $b=\frac{p}{q}$ for some $p,q\in \Z$. Let $a+b=c$, and suppose
    $c$ is rational. Then $c=\frac{m}{n}$ for some $m,n\in \Z$. We have
    \begin{align*}
      a+b &= c \\
      a+\frac{p}{q}&= \frac{m}{n} \\
      a&= \frac{m}{n}-\frac{p}{q} \\
      a&= \frac{mq-pn}{nq}
    ;\end{align*} but $a$ is irrational, a contradiction. Hence $c=a+b$ is irrational.
  \end{proof}
  \begin{lemma}[]{}
    $\sqrt[]{2} $ is irrational.
  \end{lemma}
  \begin{proof}[Proof]
    Suppose $\sqrt[]{2} =\frac{p}{q}$, where $p,q\in \Z $ and $\gcd{(p,q)}=1$. Then
    \begin{align*}
      2&=\frac{p^2}{q^2}\\
      2q^2=p^2
    .\end{align*}
    Thus $p^2$ is even, and so $p$ is even, so let $p=2k$. Then
    \begin{align*}
      2q^2&= 4k^2 \\
      q^2=2k^2
    .\end{align*}
    Thus $q^2$ is even, and so $q$ is even as well. Thus $\gcd{(p,q)}\ge 2$.

    Bit $\gcd{(p,q)}=1$, a contradiction. Hence $\sqrt{2} $ is irrational.
  \end{proof}
  From these two lemmas, we get that for any $r\in \Q$, $r+\sqrt[]{2} \in \I$. Hence $\{r+\sqrt[]{2}
  \mid r\in \Q\} \subseteq \I$.\\
  Now, suppose $a,b\in \R$, with $a<b$. Then $a-\sqrt[]{2} <b-\sqrt[]{2} $ by O4. Since
  $a-\sqrt[]{2} ,b-\sqrt[]{2} \in \R$, by the denseness of $ \Q$, there exists some $r\in \Q$ such
  that \[
    a-\sqrt[]{2} <r<b-\sqrt[]{2} 
  .\] 
  Adding $\sqrt[]{2} $ to each side, we get \[
    a < r+\sqrt[]{2} <b
  ,\]  and since $r+\sqrt[]{2} \in \{r+\sqrt[]{2} \mid r\in \Q\} \subseteq \I$, there exists some
  $x\in \I$ such that $a<x<b$.
  
\end{solution}

\begin{problem}{\S 6}
  (4.14) Let $A,B$ be nonempty bounded subsets of $ \R$, and let $A+B$ be the set of all sums $a+b$ 
  where $a\in A$, $b\in B$.
  \begin{enumerate}[label=(\alph*)]
    \item Prove that $\sup{(A+B)}=\sup{A}+\sup{B}$.
    \item Prove that $\inf{(A+B)}=\inf{A}+\inf{B}$.
  \end{enumerate}
\end{problem}

\begin{solution}
  \begin{enumerate}[label=(\alph*)]
    \item 
      We have, by definition, $a+b\le \sup{(A+B)}$ for any $a\in A$, $b\in B$. Then
      \begin{align*}
        a+b-b&\le \sup{(A+B)}-b\\
        a&\le \sup{(A+B)}-b
      ,\end{align*} and so $\sup{(A+B)}-b$ is an upper bound for $A$ ; thus $\sup{A}\le \sup{(A+B)}-b$.
      From this, we have
      \begin{align*}
        \sup{A}&\le \sup{(A+B)}-b\\
        \sup{A}-\sup{A}+b&\le \sup{(A+B)}-b+b-supA\\
        b&\le \sup{(A+B)}-\sup{A}
      ,\end{align*} and so $\sup(A+B)-\sup{A}$ is an upper bound for $B$; thus $\sup{B}\le
      \sup{(A+B)}-\sup{A}$.

      Then $\sup{A}+\sup{B}\le \sup{(A+B)}$. But since \[
        a+b\le \sup{A}+\sup{B}
      ,\] we have that $\sup{A}+\sup{B}$ is an upper bound for $A+B$ as well; but since $\sup{(A+B)}$ is
      the least upper bound, and $\sup{A}+\sup{B}\le \sup{(A+B)}$, we necessarily have equality:
      $\sup{A}+\sup{B}=\sup{(A+B)}$.

    \item We have, by definition, $\inf{(A+B)}\le a+b$ for any $a\in A$, $b\in B$. Then \[
        \inf{(A+B)}-b\le a
    ,\] and so $\inf{(A+B)}-b$ is a lower bound for  $A$; thus $\inf{(A+B)}-b\le \inf{A}$. From
    this, we have
    \begin{align*}
      \inf{(A+B)}-b &\le \inf{A}\\
      \inf{(A+B)}-\inf{A}&\le b
    ,\end{align*}
    and so $\inf{(A+B)}-\inf{A}$ is a lower bound for $B$; thus $\inf{(A+B)}-\inf{A}\le \inf{B}$.
    Then $ \inf{(A+B)}\le \inf{A}+\inf{B}$. But since \[
      \inf{A}+\inf{B}\le a+b
    ,\] we have that $\inf{A}+\inf{B}$ is a lower bound for $A+B$ as well; but since $\inf{(A+B)}$
    is the least lower bound, and $\inf{(A+B)}\le \inf{A}+\inf{B}$, we necessarily have equality:
    $\inf{A}+\inf{B}=\inf{(A+B)}$.
  \end{enumerate}
\end{solution}







\end{document}
