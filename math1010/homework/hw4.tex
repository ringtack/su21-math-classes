\documentclass{homework}
\homework{4}

\begin{document}

\begin{problem}{\S 1}
  \begin{enumerate}[label=(\alph*)]
    \item 
      Prove that if $\alpha$ is a cut, then \[
        -\alpha := \{c-b\mid c\in \Q,c<0,b\in \Q\setminus \alpha\} 
      \] is a cut.
    \item Prove that for all cuts $ \alpha$, $ \alpha\ge 0^*$ if and only if $-\alpha\le 0^*$.
  \end{enumerate}
\end{problem}

\begin{solution}
  \begin{enumerate}[label=(\alph*)]
    \item Clearly, $-\alpha\neq \varnothing$. Choose any $c\in \Q$, $c<0$; since $\alpha$ is a cut,
      there exists a $b\in \Q\setminus \alpha$, and so $ c-b\in \alpha\neq \varnothing$.
      Additionally, $-\alpha\neq \Q$, as one can choose any $a\in \alpha$; $c-\alpha\not\in
      -\alpha$. Thus property (i) holds.

      Observe that $c\in 0^*$. Suppose $r\in -\alpha$; then $r=c_1-b_1$ for some $ c_1\in 0^*$, $
      b_1\in \Q\setminus \alpha$. For any $s \in \Q$, if $s<r$, then either
      \begin{itemize}
        \item $s=c_2-b_1,\ c_2<c_1$. By property (ii) for $0^*$, for any $c\in 0^*$, if $c'\in \Q$ 
          and $c'<c$, then $c'\in 0^*$. Hence $s=c_2-b_1\in -\alpha$, as required.
        \item $s=c_1-b_2,\ b_2>b_1$. By property (ii), if $ b_1\in \Q\setminus \alpha$, and $
          b_2>b_1$, then $ b_2\in \Q\setminus \alpha$ as well. Hence $s=c_1-b_2\in -\alpha$, as
          required.
      \end{itemize}
      Thus property (ii) holds.

      For any $r\in -\alpha$, where $r=c_1-b_1$, since $ c_1\in 0^*$, we can choose $ c_2\in 0^*$
      such that $ c_2>c_1$ (by property (iii) of $0^*$), and so there exists an $s \in -\alpha$,
      where $s=c_2-b_1>c_1-b_1=r$. Thus property (iii) holds.

      Therefore all three properties hold, and so $-\alpha$ is a cut.

    \item Suppose $a\ge 0^*$. Then $0^*\subset \alpha$. Let $r\in -\alpha$. Then $r=c-b$ for some
      $c\in 0^*$, $b\in \Q\setminus \alpha$. Since $b\in \Q\setminus \alpha$, and $0\in \alpha$, we
      have $0<b$ by property (ii) (and so $-b<0$); moreover, $c<0$ for any $c\in 0^*$. Thus, we have \[
        r = c - b < 0 - b = -b < 0
      ,\] and so $r<0$; hence $r\in -\alpha$ implies that $r\in 0^*$, and so $-\alpha\subset 0^*$.
      Therefore $-\alpha\le 0^*$.\\

      Now, suppose that $-\alpha\le 0^*$. Then $-\alpha\subset 0^*$; that is, for any $r\in
      -\alpha$, where $r=c-b$ for some $c\in 0^*$, $b\in \Q\setminus \alpha$, we have $r=c-b<0$.

      Since $b\in \Q\setminus \alpha$, any $b$ satisfies $a<b$ for any $a\in \alpha$. Moreover,
      $r=c-b<0$ implies that $c<b$ for all $c\in 0^*$. In other words, $b\ge 0$; and by the
      denseness of $Q^*$, there exists an $a\in \Q$ such that $a\le b$ and $0\le a$. Since $a<b$, we
      have $a\in \alpha$; hence $0^*\subset \alpha$, and so $0^*\le \alpha$.
  \end{enumerate}
\end{solution}

\begin{problem}{\S 2}
  Let $ \alpha$ be a cut, $\alpha>0^*$. Prove that \[
    \alpha^{-1} := \{r\in \Q\mid r<0\} \cup \{r\in \Q\mid 0\le r<t~\text{for some $t\in \Q$ such that
    $\frac{1}{t}\not\in \alpha$}~\} 
  \] is a cut and $\alpha^{-1}>0^*$.
\end{problem}
\begin{solution}
  Clearly, $ \alpha^{-1}\neq \varnothing$ (since $-1\in \alpha^{-1}$) and $ \alpha^{-1}\neq \Q$
  (choose any $s>t$; $s\not\in \alpha^{-1}$). Thus property (i) holds.

  Suppose $r\in \alpha^{-1}$, and let $s \in \Q$ such that $s<r$. If $s<0$, clearly $s \in
  \alpha^{-1}$, so choose $s\ge 0$. Since $0^*<\alpha$, there exists an $a\in \alpha$ that satisfies
  $a>0$; and by properties of ordered fields, we know $a^{-1}=\frac{1}{a}>0$. Hence if
  $\frac{1}{t}\not\in \alpha$, then $\frac{1}{t}>a>0$ and so $t>0$. From this, we get $0\le s<r<t$,
  and so $s \in \alpha^{-1}$. Thus property (ii) holds.

  Suppose $r\in alpha^{-1}$, and choose $s=\frac{r+t}{2}$. Since $s=\frac{r+t}{2}<\frac{t+t}{2}=t$,
  $s<t$; additionally, $s=\frac{r+t}{2}>\frac{r+r}{2}=r$, so $r<s$. From this, we get that $0\le
  r<s<t$, and so $s \in \alpha^{-1}$. Hence for any $r\in\alpha^{-1}$, there exists an $s \in
  \alpha^{-1}$ such that $r<s$.

  Since $\alpha>0$, any $a\in \alpha$ satisfies $\alpha>0$; thus any $t\not\in \alpha$ satisfies
  $t>\alpha>0$. Choose any $0\le s<t$; clearly $s\not\in 0^*$, so we have $0^*\neq alpha^{-1}$.
  Since we also have $0^*\le  \alpha^{-1}$ (one can easily see $0^*=\{r\in \Q\mid r<0\}\subset
  \alpha^{-1} $), we necessarily have $0^* < \alpha^{-1}$.
\end{solution}


\begin{problem}{\S 3}
  (8.2.a,e) Determine the limits of the following sequences, and then prove your claims:
  \begin{enumerate}[label={}]
    \item a. $a_n=\frac{n}{n^2+1}$
    \item e. $s_n=\frac{1}{n}\sin{n}$
  \end{enumerate}
  (8.7.a) Show that $\cos{\left( \frac{n\pi}{3} \right) }$ does not converge.
\end{problem}

\begin{solution}
  (8.2)
  \begin{enumerate}[label={}]
    \item a. Intuitively, the denominator increases faster than the numerator, so we hypothesize
      that $\lim{a_n}=0$. For $n\ge 1$, we can drop the absolute value. Since
      $\frac{n}{n^2+1}<\frac{n}{n^2}=\frac{1}{n}<\varepsilon$, we have $n>\frac{1}{\varepsilon}$.
      \begin{proof}[Proof]
        Let $\varepsilon>0$ and set $N=\frac{1}{\varepsilon}$. Then for any $n>N$, we have
        $n>\frac{1}{\varepsilon}$, hence $\varepsilon>\frac{1}{n}=\frac{n}{n^2}>\frac{n}{n^2+1}$,
        and so $\left| \frac{n}{n^2+1}-0 \right| <\varepsilon$, as desired.
      \end{proof}
      
    \item e. We propose $\lim{s_n}=0$. We know that for any $n$, $\left| \sin{n} \right| \le 1$, so
      $\left| \frac{1}{n}\sin{n} \right| \le \left| \frac{1}{n} \right| <\varepsilon$. Dropping the
      absolute value for positive $n$, we get $n>\frac{1}{\varepsilon}$.
      \begin{proof}[Proof]
        Let $ \varepsilon>0$, $N=\frac{1}{\varepsilon}$. Then $n>N$ implies
        $n>\frac{1}{\varepsilon} $, hence $\varepsilon>\frac{1}{n}=\left| \frac{1}{n} \right| \ge
        \left| \frac{1}{n}\sin{n}-0 \right| $, as desired.
      \end{proof}
      
  \end{enumerate}

  (8.7) Assume that $\lim{\cos{(\frac{n\pi}{3})}}=a$ for some $a$. Setting $\varepsilon=1$, \[
    \left| \cos{\frac{n\pi}{3}}-a \right| <1
  .\]  Considering multiples of $3$, we see both \[
    \left| \cos{\frac{3\pi}{3}}-a \right| =\left| -1-a \right| <1
  \], and \[
    \left| \cos{\frac{6\pi}{3}}-a \right| =\left| 1-a \right| <1
  .\] 
  By the Triangle Inequality, we have \[
    2=\left| 1-(-1) \right| = \left| (1-a+a-(-1)) \right| \le \left| 1-a \right| +\left| a-(-1)
    \right| <1+1=2
  ,\] a contradiction. Hence $\lim{\cos{(\frac{n\pi}{3})}}$ does not converge.
\end{solution}

\begin{problem}{\S 4}
  (8.4) Let $(t_n)$ be a bounded sequence (i.e. there exists $M$ such that for all $n$, $t_n\le M$),
  and let $(s_n)$ be a sequence such that $\lim{s_n}=0$. Prove that $\lim{s_nt_n}=0$.
\end{problem}
\begin{solution}
  Let $\varepsilon>0$. Since $\lim{s_n}=0$, there exists an $N$ such that $n>N$ implies $\left| s_n
  \right| <\varepsilon_1$ for any $\varepsilon_1>0$. Moreover, since $\left| t_n \right| \le M$,
  $n>N$ implies
  \begin{align}
    \left| s_nt_n \right| < \left| \varepsilon_1t_n \right| \le \left| \varepsilon_1M \right|
    =\varepsilon_1\left| M \right| && [~\text{since $\varepsilon_1>0$}~] \label{8.4eq}
  \end{align}
  By the Archimedean property, since both $ \varepsilon_1\left| M \right| $ and $ \varepsilon$ are
  positive, there exists a $k$ such that $k\varepsilon>\varepsilon_1\left| M \right| $. Since
  Equation (\ref{8.4eq}) holds for any $\varepsilon_1>0$, set $\varepsilon_1=\frac{k}{\left| M
  \right| }\left( \frac{\varepsilon}{1+\varepsilon}\right) $. Then
  \begin{align*}
    k\varepsilon&>\varepsilon_1\left| M \right| = \frac{k\left| M \right| }{\left| M \right| }\left(
  \frac{\varepsilon}{1+\varepsilon}\right) \\
      k\varepsilon &>k\left( \frac{\varepsilon}{1+\varepsilon} \right) \\
      \varepsilon&>\frac{\varepsilon}{1+\varepsilon}
  ,\end{align*} which is true for any $\varepsilon>0$. Hence \[
    \left| s_nt_n-0 \right| <\varepsilon
  ,\] as required.
\end{solution}

\begin{problem}{\S 5}
  (8.6) Let $(s_n)$ be a sequence in $ \R$.
  \begin{enumerate}[label=(\alph*)]
    \item Prove $lim{s_n}=0$ if and only if $lim{\left| s_n \right| }=0$.
    \item Observe that if $s_n=(-1)^{n}$, then $\lim{\left| s_n \right| }$ exists, but $\lim{s_n}$
      does not exist.
  \end{enumerate}
\end{problem}

\begin{solution}
  \begin{enumerate}[label=(\alph*)]
    \item Suppose $lim{s_n}=0$. Then $\left| s_n-0 \right| =\left| s_n \right| <\varepsilon$; thus
      $\left| \left| s_n \right| -0 \right| =\left| s_n \right| <\varepsilon$ as well.

      Conversely, suppose $\lim{\left| s_n \right|}=0$. Then $\left| \left| s_n \right| -0
      \right|<\varepsilon $; but $\left| \left| s_n \right|  \right| =\left| s_n \right|$
      (repeatedly applying absolute values has the same effect as applying only once); hence $\left|
      s_n \right| =\left| s_n-0 \right| <\varepsilon$, so $\lim{s_n}=0$.
    \item Observe that $ \left| s_n \right| =\left| (-1)^{n} \right| =1$. Clearly, $\left| 1-1
      \right| =0<\varepsilon$ for any $\varepsilon>0$, so $lim{\left| s_n \right| }$ exists, and
      equals $1$. From Example 4, however, one can clearly see that $\lim{s_n}$ does not exist.
  \end{enumerate}
\end{solution}

\begin{problem}{\S 6}
  (8.10) Let $(s_n)$ be a convergent sequence, and suppose $\lim{s_n}>a$. Prove there exists a
  number $N$ such that $n>N$ implies $s_n >a$.
\end{problem}

\begin{solution}
  Since $\lim{s_n}$ exists, let $s=\lim{s_n}$. Then for some $n>N$, we have \[
    \left| s_n-s \right| <\varepsilon
  \] for all $\varepsilon>0$. Moreover, since $s>a$, there exists some $\delta>0$ such that
  $s-\delta>a$. 

  Choose an $N'$ such that $n>N'$ implies
  \begin{align*}
    && \left| s_n-s \right| <\varepsilon&&\\
    s_n-s<\varepsilon \implies s_n<s+\varepsilon&& &&s_n-s>-\varepsilon\implies s_n>s-\varepsilon
  .\end{align*}
  Thus for any $\varepsilon>0$, we have $s-\varepsilon<s_n<s+\varepsilon$. Set $\varepsilon=\delta$.
  Then we have $s_n>s-\varepsilon>s-\delta>a$, and so $s_n>a$, as required.
\end{solution}

\begin{problem}{\S 7}
  (9.1) Use limit Theorems 9.2-9.7 to prove:
  \begin{enumerate}[label=(\alph*)]
    \item $\lim{\frac{n+1}{n}}=1$
    \item $\lim{\frac{3n+7}{6n-5}}=\frac{1}{2}$
    \item $\lim{\frac{17n^{5}+73n^{4}-18n^2+3}{23n^{5}+13n^3}}=\frac{17}{23}$
  \end{enumerate}
\end{problem}
\begin{solution}
  \begin{enumerate}[label=(\alph*)]
    \item 
      Multiplying by $\frac{\frac{1}{n}}{\frac{1}{n}}$, we get $\frac{1+\frac{1}{n}}{1}$. Trivially,
      $lim{1}=1$. By Theorem 9.7(a), we get $lim{\frac{1}{n}}=0$; by Theorem 9.3 we get
      $\lim{1+\frac{1}{n}}=\lim{1}+\lim{\frac{1}{n}}=1+0=1$; and by Theorem 9.6, we have
      $\lim{\frac{n+1}{n}}=\frac{1}{1}=1$, as desired.
    \item
      Multiplying by $\frac{\frac{1}{n}}{\frac{1}{n}}$, we get
      $\frac{3+\frac{7}{n}}{6-\frac{5}{n}}$. Trivially, $\lim{3}=3,\ \lim{6}=6$, and by Theorems 9.2
      and 9.7(a), we get $\lim{\frac{7}{n}}=\lim{\frac{5}{n}}=0$. By Theorem 9.3 we get
      $\lim{3+\frac{7}{n}}=3,\ \lim{6-\frac{5}{n}}=6$, and so by Theorem 9.6 we get 
      $\lim{\frac{3n+7}{6n-5}}=\frac{3}{6}=\frac{1}{2}$.
    \item Multiplying by $\frac{1}{n^{5}}$, we get
      $\frac{17+\frac{73}{n}-\frac{18}{n^3}+\frac{3}{n^{5}}}{23+\frac{13}{n^2}}$. By Theorems 9.2
      and 9.7(a), we get any $\frac{a}{n^{p}}=0$, for all $a\in \Z$ and $p>0$ (e.g. all the ones
      with some form of $\frac{1}{n^{p}}$). Trivially, $\lim{\alpha}=\alpha$ for $\alpha\in \Z$, so
      we get
      $\lim{\frac{17n^{5}+73n^{4}-18n^2+3}{23n^{5}+13n^3}}=\frac{17+0-0+0}{23+0}=\frac{17}{23}$.
  \end{enumerate}
\end{solution}

\begin{problem}{\S 8}
    (9.6) Let $x_1=1$, $x_{n+1}=3x_n^2$.
    \begin{enumerate}[label=(\alph*)]
      \item Show that if $a=\lim{x_n}$, then $a=\frac{1}{3}$ or $a=0$.
      \item Does $\lim{x_n}$ exist? Explain.
      \item Discuss the apparent contradiction between (a) and (b).
    \end{enumerate}
\end{problem}

\begin{solution}
  \begin{enumerate}[label=(\alph*)]
    \item Suppose that for all $n>N$, we have $\lim{x_n}=a$. If $a\neq 0$, then
      $lim{x_{n+1}}=a=\lim{3x^2}=3a^2$, and so $\frac{a^2}{a}=a=\frac{1}{3}$. However, if $a=0$,
      then the above equality also holds (indeed, if $a=0$, we cannot do $\frac{a^2}{a}$): $0=3\cdot
      0^2=0$. Hence $a=\frac{1}{3}~\text{or}~0$.
    \item The limit does not actually exist; Since $x_1=1\ge 1$, then $x_n^2\ge 1$ for any $n\ge 1$,
      and so $3x_n^2$ will always increase (and thus never converse to a value).
    \item Part (a) relies on the assumption that $\lim{x_n}$ exists in the first place; it assumes
      $\lim{x_n}=a$ for some $a$, and \textit{then} proceeds with finding the value of $a$. If the
      limit did not exist in the first place, then such calculations would be meaningless; any
      result could be returned. That is also why we see that $\lim{x_n}$ approaches \textbf{2}
      values, a clear contradiction of what a limit is.
  \end{enumerate}
\end{solution}





\end{document}
