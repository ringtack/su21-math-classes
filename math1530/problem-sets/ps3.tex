\documentclass{homework}
\homework{3}

\begin{document}

\begin{problem}{\S 1}
  (2.22) Let $\mc{C}_n$ denote a cyclic group of order $n$, $\D_n$ denote the $n^{th}$ dihedral group,
  and $\mc{S}_n$ the $n^{th}$ symmetric group.
  \begin{enumerate}[label=(\alph*)]
    \item Prove that $\mc{C}_2$ and $\mc{S}_2$ are isomorphic.
    \item Prove that $\D_3$ and $\mc{S}_3$ are isomorphic.
    \item Let $m\ge 3$. Prove that for every $n$, $\mc{C}_m$ and $ \mc{S}_n$ are not isomorphic.
    \item Prove that for every $n\ge 4$, $\D_n$ and $ \mc{S}_n$ are not isomorphic.
    \item More generally, let $m\ge 4$ and $n\ge 4$. Prove that $\D_m$ and $\mc{S}_n$ are not
      isomorphic.
    \item Prove that $ \D_4$ and $\mc{Q}$ are not isomorphic.
  \end{enumerate}
\end{problem}

\begin{solution}
  \begin{enumerate}[label=(\alph*)]
    \item $\mc{C}_2=\{ e,g \}$, $ \mc{S}_2=\{ e,\pi \}$. Define a mapping $\phi:\mc{C}_2\to \S_2$,
      where $\phi(e)=e,\ \phi(g)=\pi$. $\phi$ is clearly a bijective homomorphism; thus $\mc{C}_2$ 
      is isomorphic to $ \mc{S}_2$.
    \item Let $\phi_3:\D_3\to \mc{S}_3$ be the mapping that sends every $\sigma\in \D_3$ to the
      $\pi\in \mc{S}_3$ such that $\sigma(i)=\pi(i),\ i\in \{ 1,2,3 \}$. Problem 1 from last week's
      problem set shows that a map $\phi_n:\D_n\to \mc{S}_n$, as defined above, is a homomorphism,
      is injective for all $n\in \Z^+$, and surjective for $1\le n\le 3$. Hence $\phi_3$ is
      bijective, and so $\D_3$ is surjective to $\mc{S}_3$. (Alternatively, one could simply list
      all permutations in $\mc{S}_3$ and all transformations in $\D_3$, and observe that such a
      $\phi_3$ is isomorphic. The reader is spared the work here.)
    \item We begin with two lemmas.
      \begin{lemma}[]
        Let $G,H$ be groups, and let $G$ be cyclic. If $G$ is isomorphic to $H$, then $H$ is cyclic.
      \end{lemma}
      \begin{proof}[Proof]
        Given groups $G,H$, suppose $G$ is cyclic and let $f:G\to H$ be an isomorphism. 

        Let $ g_0\in G$ be a generator for $G$, and let $f(g)=h\in H$ for some $g\in G$. Since $G$ 
        is cyclic, $g=g_0^{m}$ for some $m\in \Z$. Then 
        \begin{align*}
          h=f(g)&=f(g_0^{m})\\
                &= f(g_0\cdot \ldots\cdot g_0) \\
                &= f(g_0)^{m}=h_0^{m}~\text{for some $h_0\in H$}~
        .\end{align*}
        Hence for any $h\in H$, $h=h_0^{m}$ for some $ h_0\in H$. Thus any  $h\in H$ is in $\left<
        h_0\right>$, and so $H$ is cyclic as well.
      \end{proof}

      \begin{lemma}[]
        Let $G$ be a group. If $G$ is cyclic, then any subgroup $H<G$ is cyclic.
      \end{lemma}
      \begin{proof}[Proof]
        Let $G$ be a group, and let $H<G$. Suppose $G$ is cyclic. Then for any $g\in G$,
        $g=g_0^{m}$, where $ g_0$ is a generator of $G$.

        Let $h\in H$. Since $G$ is cyclic and $h\in G$, $h=g_0^{m}$ for some $m\in \Z$. Let $k\in
        \Z$ be the smallest $k$ such that $ g_0^{k}\in H$. Then for any $h=g_0^{m}\in H$, we have
        $m=kq+r$ for some $q,r\in \Z,\ 0\le r<k$. Thus
        \begin{align*}
          g_0^{m}&=g_0^{kq+r}\\
          &= g_0^{kq}g_0^{r}
        .\end{align*} Since $H$ is a subgroup, any $h\in H$ has $h^{-1}\in H$. Thus
        \begin{align*}
          g_0^{m}&=g_0^{kq}g^{r}\\
          g_0^{-kq}g_0^{m}&=g_0^{r}\\
          g_0^{m-kq}&=g_0^{r}
        ,\end{align*} and by closure, $ g_0^{r}\in H$ as well. But $k$ is the smallest integer such
        that $g_0^{k}\in H$, and $0\le r<k$; thus $r=0$ (otherwise, we have a contradiction).

        Thus for any $h\in H$, $h=(g_0^{k})^{q}$, and so $H$ is a cyclic group generated by $g_0^{k}$.
      \end{proof}

        From Lemma 2, we get its contrapositive: \textit{if a subgroup $H$ of a group $G$ is not
      cyclic, then $G$ is not cyclic}, and we make one observation: \textbf{$\mc{S}_3$ is not
      cyclic} (one can easily see that any $\pi\in \mc{S}_3$ does not generate $\mc{S}_3$). From the
      contrapositive to Lemma 2, since $\mc{S}_3$ is a subgroup of $\mc{S}_n$, and $\mc{S}_3$ 
      is not cyclic, $\mc{S}_n$ is not cyclic. Taking the contrapositive of Lemma 1, (if $G$ is cyclic
      and $H$ is not cyclic, $H$ is not isomorphic to $G$), since $ \mc{S}_n$ is not cyclic and
      $\mc{C}_m$ is cyclic, they are not isomorphic.

    \item Recall that $ \D_n$ has order $2n$, while $\mc{S}_n$ has order $n!$. Since for any $n>3$,
      $2n\neq n!$, $\D_n$  is not isomorphic to $\mc{S}_n$.
    \item We start with another lemma:
      \begin{lemma}[]{}
        Let $G,\ H$ be groups. If $G$ is isomorphic to $H$, then for any $g\in G$, the corresponding
        (unique) $f(g)=h\in H$ has the same order as $g$.
      \end{lemma}
      \begin{proof}[Proof]
        Let $f:G\to H$ be an isomorphism, let $g\in G$ have order $n$, and let $f(g)=h\in H$. Recall
        that for a homomorphism, $f(e)=e'$, where $e'\in H$ is the identity. Then
        \begin{align*}
          f(e)=f(g^{n})&= f(g)\cdot \ldots\cdot f(g) \\
                       &= f(g)^{n} \\
                       &= h^{n}=e'
        .\end{align*} Since $f$ is isomorphic, and any $g^{m}\neq e$ when $m\in \Z$ and $m<n$, $n$
        is the smallest positive integer such that $h^{n}=e'$; that is, $h\in  H$ has order $n$ as
        well.
      \end{proof}

      Now, consider the dihedral group $\D_m$. We observe that all flips have order $2$: if we flip
      an $n$-gon twice, we get back to the original shape (formally, if we define a flip
      $f_j(i)=m-j+i$, then $f_j(f_j(i))=f_j(m-i+j)=m-(m-i+j)+j=m-m-j+j+i=i$ for all $0\le j<m$.
      Refer back to problem set 2 for a more complete definition of the dihedral group.)
      Additionally, we observe that there are only two rotations with order 3: given a rotation \[
        r_j(i)=i+j,\ j\in \{0, \ldots, m-1\}
      ,\] $r_j^3(i)=i$ only when $i+3j\mod{m}=i$; that is, $3j\mod{m}\equiv 0$. Since $j\in
      \{0,\ldots,m-1\}$, this is only the case when $j=\frac{m}{3}$ or $\frac{2m}{3}$. Thus $\D_m$ 
      only has two elements of order $3$.

      On the other hand, $\mc{S}_n$ clearly has more than $2$ elements with order $3$: one can
      easily choose permutations $\pi_1=(123),\ \pi_2=(124),\ \pi_3=(234)$ for any $ \mc{S}_n$ when
      $n\ge 4$.

      By the Lemma, if $\D_m$ and $\mc{S}_n$ were isomorphic, then any $\pi\in \mc{S}_n$ with
      order $k$ would correspond with a unique $\sigma\in \D_m$, also with order $k$; specifically,
      elements with order 3 in $ \mc{S}_n$ would have to map to unique elements of order 3 in
      $\D_m$. However, there are more elements with order $3$ in $\mc{S}_n$ than there are in
      $\D_m$; hence no such isomorphism exists between the two sets.
      
    \item In $\mc{Q}$, there are $6$ elements with order $4$: $\pm i,\ \pm j,~\text{and}~\pm k$; and
      $1$ element with order $2$: $-1$. However, in $\D_4$, there are only $2$ elements with order
      $4$: $r_1 ~\text{and}~r_3$; and $5$ with order $2$: all flips, and $r_2$. Thus, since the
      number of elements with order $2$ and order $4$ are different, by Lemma 3 they cannot be
      isomorphic.
  \end{enumerate}
\end{solution}


\begin{problem}{\S 2}
  (2.28) Consider the dihedral group $\D_4=\{ e,\rho_1,\rho_2,\rho_3,\phi_1,\phi_2,\phi_3,\phi_4\}$
  and the quaternion group $\mc{Q}=\{ \pm 1,\pm i,\pm j,\pm k \}$. For each of the following groups
  and subgroups, explicitly write down the cosets.
  \begin{enumerate}[label=(\alph*)]
    \item $G=\D_4$, $H=\{ e,\phi_1 \}$
    \item $G=\D_4$, $H=\{e, \phi_1,\phi_2,phi_3\}$
    \item $G=\D_4$, $H=\{ e, \phi_2 \}$
    \item $G=\mc{Q}$, $H=\{ \pm 1 \}$
    \item $G=\mc{Q}$, $H=\{ \pm 1, \pm i \}$
  \end{enumerate}
\end{problem}

\begin{solution}
  \begin{enumerate}[label=(\alph*)]
    \item
      \begin{align*}
        eH=\{ e,\phi_1 \}&&\rho_1H=\{ \rho_1,\phi_2 \}&&\rho_2H=\{ \rho_2,\phi_3 \}&&\rho_3H=\{
        \rho_3,\phi_4 \}\\
          \phi_1H=\{ \phi_1,e \}&&\phi_2H=\{ \phi_2,\rho_1 \}&&\phi_3H=\{ \phi_3,\rho_2
          \}&&\phi_4H=\{ \phi4,\rho_3 \}
      .\end{align*}
    \item
      \begin{align*}
        eH=\{ e,\rho_1,\rho_2,\rho_3 \}&&\rho_1H=\{ \rho_1,\rho_2,\rho_3,e \}&&\rho_2H=\{
        \rho_2,\rho_3,e,\rho_1 \}&&\rho_3H=\{ \rho_3,e,\rho_1,\rho_2 \}\\
        \phi_1H=\{ \phi_1,\phi_4,\phi_3,\phi_2 \}&&\phi_2H=\{ \phi_2,\phi_1,\phi_4,\phi_3
        \}&&\phi_3H=\{ \phi_3,\phi_2,\phi_1,\phi_4 \}&&\phi_4H=\{ \phi_4,\phi_3,\phi_2,\phi_1 \}
      .\end{align*}
    \item
      \begin{align*}
        eH=\{ e,\rho_2 \}&&\rho_1H=\{ \rho_1,\rho_3 \}&&\rho_2H=\{ \rho_2,e \}&&\rho_3H=\{
        \rho_3,\rho_1 \}\\
          \phi_1H=\{ \phi_1,\phi_3\}&&\phi_2H=\{ \phi_2,\phi_4 \}&&\phi_3H=\{ \phi_3,\phi_1 \}
                                    &&\phi_4H=\{ \phi_4,\phi_2 \}
      .\end{align*}
    \item
      \begin{align*}
        1H=\{ \pm 1 \}&&-1H=\{ \pm 1 \}&&iH=\{ \pm i \}&&-iH=\{ \pm i \}\\
        jH=\{ \pm j \}&&-jH=\{ \pm j \}&&kH=\{ \pm k \}&&-kH=\{ \pm k \}
      .\end{align*}
    \item
      \begin{align*}
        1H=\{ \pm 1,\pm i \}&&-1H=\{ \pm 1,\pm i \}&&iH=\{ \pm i,\pm 1 \}&&-iH=\{ \pm i,\pm 1 \}\\
        jH=\{ \pm j,\pm k \}&&-jH=\{ \pm j,\pm k \}&&kH=\{ \pm k,\pm j \}&&-kH=\{ \pm k,\pm j \}
      .\end{align*}
  \end{enumerate}
\end{solution}

\begin{problem}{\S 3}\\
  (2.31) Let $G$ be a group. The \textbf{center} of $G$ is defined \[
    Z(G) = \{g\in G\mid gg'=g'g ~\text{for every}~g'\in G\} 
  .\] 
  \begin{enumerate}[label=(\alph*)]
    \item Prove that $Z(G)$ is a subgroup of $G$.
    \item When does $Z(G)$ equal $G$?
    \item Compute the center of the symmetric group $\mc{S}_n$.
    \item Compute the center of the dihedral group $\D_n$.
    \item Compute the center of the quaternion group $\mc{Q}$.
  \end{enumerate}

  (2.34) Let $G$ be a finite group whose only subgroups are $\{ e \}$ and $G$. Prove that either
  $G=\{ e \}$, or $G$ is a cyclic group whose order is prime.
\end{problem}

\begin{solution}\\

  (2.31)
  \begin{enumerate}[label=(\alph*)]
    \item Let $g_1,g_2\in Z(G)$. Then for any $g'\in G$, we have \[
        g'(g_1g_2)=(g'g_1)g_2=(g_1g')g_2=g_1(g'g_2)=g_1(g_2g')=g_1g_2g'
      .\] Hence $ g_1g_2$ commutes with every $g'\in G$, and so $g_1g_2\in Z(G)$.\\
      By definition, $e\in Z(G)$.\\
      Let $g\in Z(G)$. Then $gg'=g'g$ for any $g'\in G$. From this, we get
      \begin{align*}
        gg'&= g'g \\
        g^{-1}gg'&=g^{-1}g'g\\
        g'&= g^{-1}g'g \\
        g'g^{-1}&= g^{-1}g'gg^{-1} \\
        g'g^{-1}&=g^{-1}g'
      .\end{align*} Hence for any $g\in Z(G)$, $g^{-1}\in Z(G)$.

      Therefore $Z(G)$ is a subgroup of $G$.
    \item Suppose $Z(G)=G$. Then for any $g\in Z(G)$, $g\in G$. Additionally, for every $g\in Z(G)$,
      $gg'=g'g$ for any $g'\in G$. Thus if $Z(G)=G$, by definition $G$ is an Abelian group. (If $G$ 
      is cyclic, $Z(G)=G$ as well; but all cyclic groups are Abelian).
    \item $Z(\mc{S}_n)=\{ e \}$; in other words, $\mc{S}_n$ has a trivial center.
      \begin{proof}[Proof]
        Consider the set of bijective permutations $\prod$, where for some $\pi_j\in \prod$, \[
          \pi_j(i)=\left\{ \begin{array}{lr} i&i=j\\k~(\text{for some}~k\neq i)&i\neq
          j\end{array}, i,j,k\in \{ 1,2,\ldots,n \}\right.
            \] Let $\pi_i \in \prod$, $i\in \{ 1,\ldots,n \}$, and consider any permutation $\pi$
            not in $\prod$ (except $e$). Suppose $\pi(i)=j$ for some $j\in \{ 1,\ldots,n \}$, and
            let $k$ be some number in $\{ 1,\ldots,n \}$ such that $\pi_i(j)=k$. Then \[
              \pi_i\circ \pi(i) = \pi_i(j) = k
            ,\] while \[
              \pi\circ \pi_i(i) = \pi(i)=j
            .\] Hence $\pi_i\pi\neq \pi\pi_i$, and so any $\pi\not\in \prod$ (except obviously
            $e$) does not commute with any $\pi_i\in \prod$.

            Since for any $n\ge 3$, there exists some non-trivial $\pi_i\in \prod$, and some
            non-trivial $\pi\not\in \prod$, the only element that commutes with every $\pi\in
            \mc{S}_n$ is $e$; therefore $Z(\mc{S}_n)=\{ e \}$.
      \end{proof}
    \item For odd $n\ge 3$, $Z(\D_n)=\{ e \}$; for even $n\ge 3$, $Z(\D_n)=\{e,r_{\frac{n}{2}}\}$,
      where $r_{\frac{n}{2}}$ is the half ($180^{\circ}$) rotation.
      \begin{proof}[Proof]
        Recall (from my problem set 2) that $V_n=\{ 0,\ldots,n-1 \}$, arithmetic is defined modulo
        $n$, a rotation $r_j\in \D_n$ is defined \[
          r_j(i)=i+j,\ j\in V_n
        ,\] a flip is defined \[
          f_j(i)=n-i+j,\ j\in V_n
        ,\] and $\D_n$ is composed entirely of rotations and flips; that is, \[
        \D_n=\{ \sigma \mid \sigma=r_1^{j}f_0^{k},\ j\in V_n,\ k\in \{ 0,1 \}\}
        .\] From this, we see three things:
        \begin{itemize}
          \item $f_j$ has order $2$ (and so $f_j=f_j^{-1}$): $f_j(f_j(i))=f_j(n-i+j)=n-(n-i+j)+j=i$.
          \item $r_j^{-1}=r_{n-j}$: $r_{n-j}(r_j(i))=r_{n-j}(i+j)=i+j+n-j=i+n=i$, and
            $r_j(r_{n-j})(i)=r_j(i+n-j)=i+n-j+j=i+n=i$.
          \item  $f_jr_if_j=r_i^{-1}$:
            $f_j(r_i(f_j(k)))=f_j(r_i(n-k+j))=f_j(n-k+j+i)=n-(n-k+j+i)+j=n-n+k-i-j+j=k-i=k+(n-i)=r_i^{-1}(k)$.
        \end{itemize}
        Now, observe that any rotation commutes with another rotation, and not every flip commutes
        with every other flip. Thus, any center must be a rotation that commutes with every flip
        (because then the rotation commutes with all rotations and all flips); in other words, for
        $r_j\in \D_n$, we must have $r_jf=fr_j$ for some flip $f$. From above, we have
        \begin{align*}
          fr_jf&=r_j^{-1}\\
          ffr_jf&= fr_j^{-1} \\
          r_jf&=fr_j^{-1}
        .\end{align*} Thus $r_j$ commutes with any $f$ if $r_j=r_j^{-1}$. We know that
        $r_j^{-1}=r_{n-j}$; thus $r_j=r_{n-j}$ requires  $j=n-j$, or equivalently  $j=\frac{n}{2}$.
        For odd $n$, this is not closed in $\Z$; thus $r_{\frac{n}{2}}\in \D_n$ only if $n$ is
        even.\\

        By definition, $e$ commutes with every element in $ \D_n$. Therefore, $Z(\D_n)=\{ e \}$ when
        $n$ is odd, and $Z(\D_n)=\{ e,r_{\frac{n}{2}} \}$ when $n$ is even.
      \end{proof}
      
    \item From the definition of the quaternion group $\mc{Q}$, one can clearly see that none of
      $i,j,k$ commute ($ij=k\neq -k=ij$, $jk=i\neq-i=kj$, etc.), while $\pm 1$ do commute ($1$ is
      the identity, and for any $a\in \{ i,j,k \},\ -1\cdot a=-a=a\cdot -1$); hence $Z(\mc{Q})=\{
      \pm 1 \}$.
      
  \end{enumerate}

  \hrule
  \vspace{1ex}
  (2.34) We start with two lemmas:
  \begin{lemma}[]{}
    If a group $G$ only has trivial subgroups, then for any $g\in G$, $\left| g \right| =\left| G
    \right| $.
  \end{lemma}
  \begin{proof}[Proof]
    By Corollary 2.42, any $g\in G$ has order $m$, where $m|n$; thus there exists a $k\in \Z$ such
    that $km=n$. Let $\left| g \right| =m$, $\left| G \right| =n$, and suppose $m<n$; then $k>1$.
    But if $k\ge 2$, then $\left<g \right>$ would form a cyclic subgroup of of order $\frac{n}{k}<n$,
    a contradiction of $G$ having only trivial subgroups. Hence $\left| g \right|=\left| G \right|$.
  \end{proof}
  \begin{lemma}[]{}
    Every group $G$ with prime order is cyclic.
  \end{lemma}
  \begin{proof}[Proof]
    Let $\left| G \right| =n$. By Corollary 2.42, every $g\in G$ has order $m|n$; but since $n$ is
    prime, either $m=1$ ($g=e$) or $m=n$. Then $\left<g \right>=\{ e,\ldots,g^{n-1} \}$, so $\left|
    \left<g \right> \right| =n=\left| G \right| $. Thus $G=\left<g \right>$, so $G$ is a cyclic
    group.
  \end{proof}

  Now, suppose $G$ only has trivial subgroups $\{ e \}$ and $G$. From the first lemma, we see that
  every $g\in G$ has order $n$. Trivially, $G=\{ e \}$ is a valid group; so consider only $G\neq \{
  e\}$.

  Suppose $\left| G \right| $ is composite. Then for some $a,b\in \Z,\ a>1,\ b>1$, $n=ab$. Thus
  $g^{n}=g^{ab}=e$. By closure of a group, if $g\in G$, then $g^{a}\in G$ as well. Then
  $g^{n}=g^{ab}=(g^{a})^{b}=e$, so $g^{a}$ has order $b<n$; but that contradicts every $g\in G$
  having order $n$. Thus $\left| G \right| $ is prime. By the second lemma, $G$ is cyclic.

  Therefore, if $G$ only has trivial subgroups $\{ e \}$ and $G$, then $G$ is either $\{ e \}$, or a
  cyclic group with prime order.
\end{solution}

\begin{problem}{\S 4}
  (2.32) Let $G$ be a group, and let $g\in G$. The \textbf{centralizer} of $G$ is defined \[
    Z_G(g) = \{ g'\in G \mid gg'=g'g \}
  .\] 
  \begin{enumerate}[label=(\alph*)]
    \item Prove that $Z_G(g)$ is a subgroup of $G$.
    \item Compute $Z_G(g)$ for the following groups and elements:
      \begin{enumerate}
        \item $G=\D_4$, $g=\rho_1$ ($90^{\circ}$ rotation).
        \item $G=\D_4$, $g=f$ a flip fixing two vertices of a square.
        \item $G=\GL_2(\R)$, $g=\begin{pmatrix} a&0\\0&d \end{pmatrix} $.
      \end{enumerate}
    \item Prove that $Z_G(g) = G$ iff $g\in Z(G)$ (the center of $G$).
    \item More generally, if $S\subseteq G$ is any subset, then \[
      Z_G(S) = \{g\in G\mid sg=gs ~\text{for all $s \in S$}~\} 
    .\] Prove that $Z_G(S)$ is a subgroup of $G$.
  \end{enumerate}
\end{problem}
\begin{solution}
  \begin{enumerate}[label=(\alph*)]
    \item Let $ g_1,g_2\in Z_G(g)$. Then
      \begin{align*}
        gg_1g_2=(g_1g)g_2=g_1(g_2g)=g_1g_2g 
      .\end{align*} Hence $g_1g_2$ commutes with $g$, and so $g_1g_2\in Z_G(g)$.

      By definition, the identity $e$ commutes with any $g\in G$; thus $e\in Z_G(g)$.

      Let $g'\in Z_G(g)$. Then
      \begin{align*}
        gg'&= g'g \\
        g'^{-1}gg'&= g'^{-1}g'g \\
        g'^{-1}gg'g'^{-1}&= gg'^{-1} \\
        g'^{-1}g&=gg'^{-1}
      .\end{align*} Hence $g'^{-1}\in Z_G(g)$ as well, and so $Z_G(g)$ is a subgroup of $G$.
    \item
      \begin{enumerate}
        \item $Z_{\D_4}(\rho_1)=\{ e, \rho_1,\rho_2,\rho_3 \}$
        \item $Z_{\D_4}(f)=\{ e, \phi_1, \phi_3, \rho_2 \}$ 
        \item 
          \[
            \begin{pmatrix} a&0\\0&d \end{pmatrix} \begin{pmatrix} a_1&b_1\\c_1&d_1 \end{pmatrix}
            =\begin{pmatrix} aa_1&ab_1\\c_1d&dd_1 \end{pmatrix} 
          .\] 
          \[
            \begin{pmatrix} a_1&b_1\\c_1&d_1 \end{pmatrix} \begin{pmatrix} a&0\\0&d \end{pmatrix}
            =\begin{pmatrix} aa_1&b_1d\\ac_1&dd_1 \end{pmatrix} 
          .\] 
            Thus, if $a=d$, then $Z_{\GL_2(\R)}\begin{pmatrix} a&0\\0&d \end{pmatrix} =\GL_2(\R)$.
            Otherwise, \[Z_{\GL_2(\R)}\begin{pmatrix} a&0\\0&d \end{pmatrix} =\left\{ \begin{pmatrix}
          a_1&0\\0&d_1\end{pmatrix}\mid a_1\neq d\lor d_1\neq a  \right\}\]
      \end{enumerate}
    \item Suppose $Z_G(g)=G$. Then for any $g'\in G$, we have $gg'=g'g$. By definition, this means
      that $g\in Z(G)$.
    
      Conversely, suppose $g\in Z(G)$. Then for any $g'\in G$, we have $gg'=g'g$; but this means
      that every $g'\in G$ is also in $Z_G(g)$. Hence $Z_G(g)=G$.
    \item Let $S\subseteq G$, and consider $g_1,g_2\in Z_G(S)$. We have
      \[
        sg_1g_2=g_1sg_2=g_1g_2s
      ;\] hence $g_1g_2\in Z_G(S)$.

      $e\in Z_G(S)$, as for any $s\in S\subseteq G$, we have $es=se$.

      Let $g\in Z_G(S)$. We have
      \begin{align*}
        sg&= gs \\
        sgg^{-1}&=gsg^{-1}\\
        g^{-1}s&=g^{-1}gs\\
        g^{-1}s&=sg^{-1}
      .\end{align*} Hence $g^{-1}\in Z_G(S)$, and so $Z_G(S)$ is a subgroup of $G$.
  \end{enumerate}
\end{solution}


\begin{problem}{\S 5}
  Calculate the order of $(1,2)\in \Z / 3\Z \times \Z / 4\Z$, then complete (2.38)
  Let $G$ be a group, and $A,B$ subgroups of $G$, and consider the map \[
    \phi: A\times B\to G,\ \phi(a,b)=ab
  .\] 
  \begin{enumerate}[label=(\alph*)]
    \item If $G$ is Abelian, prove that $ \phi$ is a homomorphism.
    \item If $G$ is Abelian, prove that \[
          \ker(\phi)=\{(c,c^{-1})\mid c\in A\cap B\} 
      .\] 
    \item Suppose that there are elements $a\in A,\ b\in B$ with $ab\neq ba$. Prove that $\phi$ is
      \textbf{not} a homomorphism.
  \end{enumerate}
\end{problem}

\begin{solution}
  The order of $(1,2)$ is $6$: $1\in \Z / 3\Z$ has order $3$, while $2\in \Z / 4\Z$ has order $2$;
  from this, we can easily determine the order to be $6$.

  \begin{enumerate}[label=(\alph*)]
    \item Let $a_1,a_2\in A,\ b_1,b_2\in B $. We have \[
        \phi(a_1a_2,b_1b_2) =
        (a_1a_2)(b_1b_2)=a_1(a_2b_1)b_2=a_1(b_1a_2)b_2=(a_1b_1)(a_2b_2)=\phi(a_1b_1)\phi(a_2b_2)
      .\] Hence $\phi$ is a homomorphism.
    \item Suppose $\phi(a,b)=ab=e$. This is only the case when $a$ and $b$ are inverses; in other
      words, $a=b^{-1}$ and $b=a^{-1}$. We have $b\in B$; but if $b=a^{-1}$, since (by definition of a
      subgroup) $a^{-1}\in A$, we have $b\in A$ has well. In other words, $b\in A\cap B$. Similarly,
      since $a=b^{-1}$, and $b^{-1}\in B$, we have $a\in B$, and so $a\in A\cap B$.

      Therefore $a,b\in A\cap B$, and $b=a^{-1}$, $a=b^{-1}$. In other words, if $\phi(a,b)=e$,
      then $a=c,\ b=c^{-1},\ c\in A\cap B$. Hence $\ker(\phi)=\{ (c,c^{-1})\mid c\in A\cap B \}$.
    \item Suppose there exists $a\in A,\ b\in B$ with $ab\neq ba$. Let $a_1\in A,\ b_1\in
      B$. Then \[
        \phi(a_1a,bb_1)=a_1abb_1\neq a_1bab_1=\phi(a_1,b)\phi(a,b_1)
      .\] Hence $\phi$ is not a homomorphism.
  \end{enumerate}
\end{solution}





\end{document}
