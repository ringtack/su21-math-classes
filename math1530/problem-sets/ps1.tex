\documentclass{homework}
\homework{1}

\begin{document}

\begin{problem}{\S 1}
  Choose any three problems from 1.2(a-i) as a warmup. Then complete Exercise 1.3. \\
  \begin{itemize}
    \item 1.2: Use truth tables to prove:
    \begin{itemize}
      \item 1.2.a: $P \iff \neg(\neg P)$.
      \item 1.2.c: $\left( P\Rightarrow Q \right) \iff\left( \neg Q\Rightarrow \neg P
        \right)$.
      \item 1.2.d: $\left( P\Rightarrow Q \right) \iff\left( \neg P \right) \lor Q $.
    \end{itemize}
    \item 1.3: Let P and Q be statements.
      \begin{enumerate}[label=(\alph*)]
        \item Prove that \[
        P \veebar \neg P
      \] is true, and explain why this justifies the Law of the Excluded Middle (which states that
      exactly one of $P$ and $\neg P$ is true).
        \item Prove that \[
            \left( \neg Q\Rightarrow \neg P \right) \Rightarrow \left( P\Rightarrow Q \right)
          \] is true, and explain why this justifies the method of Proof by Contradiction (which
          states that in order to prove that $P$ is true, it suffices to show that $\neg P$ is
          false).
      \end{enumerate}
  \end{itemize}
\end{problem}

\begin{solution}
  \begin{itemize}
    \item 
      \begin{table}[htpb]
        \centering
        \caption{1.2.a}
        \begin{tabular}{c|c|c}
          P & $\neg P$ & $\neg(\neg P)$ \\
          \toprule
          \bottomrule
          T & F & T \\
          F & T & F
        \end{tabular}
      \end{table}
    \item
      \begin{table}[htpb]
        \centering
        \caption{1.2.c}
        \begin{tabular}{c|c|c|c}
          $P$ & $Q$ & $P\Rightarrow Q$ & $\neg Q \Rightarrow \neg P$ \\
          \toprule
          \bottomrule
          T & T & T & T \\
          T & F & F & F \\
          F & T & T & T \\
          F & F & T & T
        \end{tabular}
      \end{table}
    \item
      \begin{table}[htpb]
        \centering
        \caption{1.2.d}
        \begin{tabular}{c|c|c|c|c}
          $P$ & $Q$ & $\neg P$& $P\Rightarrow Q$ & $\neg P \lor Q$ \\
          \toprule
          \bottomrule
          T & T & F & T & T \\
          T & F & F & F & F \\
          F & T & T & T & T \\
          F & F & T & T & T
        \end{tabular}
      \end{table}
      \clearpage
    \item
      \begin{table}[htpb]
        \centering
        \caption{1.3.a}
        \begin{tabular}{c|c|c}
          $P$ & $\neg P$ & $P\veebar \neg P$ \\
          \toprule
          \bottomrule
          T & F & T \\
          F & T & T 
        \end{tabular}
      \end{table}
      Since the statement is true regardless of $P$, $P \veebar \neg P$ is true. This also
      justifies the Law of the Excluded Middle, as an XOR statement is true only when one, but
      not both, of the statements are true; hence only one of $P$ and $\neg P$ may be true at
      once in order for $P \veebar \neg P$ to be true.
    \item 
      \begin{table}[htpb]
        \centering
        \caption{1.3.b}
        \begin{tabular}{c|c|c|c|c|c|c}
          $P$ & $Q$ & $\neg P$ &  $\neg Q$ & $P\Rightarrow Q$ & $\neg Q \Rightarrow \neg P$ & 
          $\left( \neg Q \Rightarrow \neg P \right) \Rightarrow \left( P\Rightarrow Q \right) $\\
          \toprule \bottomrule
          T & T & F & F & T & T & T\\
          T & F & F & T & F & F & T\\
          F & T & T & F & T & T & T\\
          F & F & T & T & T & T & T
        \end{tabular}
      \end{table}
      Since the statement is true regardless of $P$ or $Q$, $\left( \neg Q \Rightarrow \neg P
      \right) \Rightarrow \left( P\Rightarrow Q \right)$ is true. This also justifies Proof by
      Contradiction: if $\neg P$ is false, then $P$ is necessarily true in order for
      the above statement to be true as well (alternatively, from the Law of the Excluded Middle,
      $\neg P$ being false necessarily implies that $P$ is true).
  \end{itemize}
\end{solution}

\begin{problem}{\S 2}
  Complete Exercise 1.7: \textit{Prove each of the following formulas:} 
  \begin{enumerate}[label=(\alph*)]
    \item $S\cap \left( T \cup U \right) = \left( S\cap T \right) \cup \left( S\cap U \right)$
    \item $S\cup \left( T\cap U \right) = \left( S\cup T \right) \cap \left( S\cup U \right)$
    \item Suppose $S, T\subset U$. Then \[
        \left( S\cup T \right) ^{c} = S^{c}\cap T^{c} ~\text{and}~ \left( S\cap T \right) ^{c}
        =S^{c}\cup T^{c}
      .\] 
    \item $S\Delta T = \left( S\cup T \right) \setminus \left( S\cap T \right) = \left( S\setminus
      T\right) \cup \left( T\setminus S \right) $
  \end{enumerate}
\end{problem}


\begin{solution}
\begin{enumerate}[label=(\alph*)]
  \item
    \begin{proof}[Proof]
      Let $e\in S\cap \left( T\cup U \right) $. Then
      \begin{align*}
        &\left( e\in S \right) \land \left( e\in T \lor e\in U \right) \\
        = &\left( e\in S\land e\in T \right) \lor \left( e\in S\land e\in U \right)
      .\end{align*}
      Thus $e\in \left( S\cap T \right) \cup \left( S\cap U \right)$, and so $S\cap \left( T\cup U 
        \right) \subset \left( S\cap T \right) \cup \left( S\cap U \right) $.\\
      Conversely, let $e\in \left( S\cap T \right) \cup \left( S\cap U \right)$. Then
      \begin{align*}
        &\left( e\in S\land e\in T \right) \lor \left( e\in S\land e\in U \right) \\
        = &~e\in S\land \left(e\in T\lor e\in U\right)
      .\end{align*}
      Thus, $e\in S\cap \left( T\cup U \right)$, and so $\left( S\cap T
      \right) \cup \left( S\cap U \right) \subset S\cap \left( T\cup U \right) $. \\
      Since both are subsets of each other, $S\cap \left( T\cup U \right) = \left( S\cap T
      \right)\cup \left( S\cap U \right) $.
    \end{proof}
    
  \item
    \begin{proof}[Proof]
      Let $e\in S\cup \left( T\cap U \right)$. Then
      \begin{align*}
        &~e\in S\lor \left( e\in T\land e\in U \right) \\
        = &\left( e\in S\lor e\in T \right) \land \left( e\in S\lor e\in U \right)
      .\end{align*}
      Thus $e\in \left( S\cup T \right) \cap \left( S\cup U \right)$, and so $S\cup \left( T\cap U
        \right) \subset \left( S\cup T \right) \cap \left( S\cup U \right) $. \\
        Conversely, let $e\in \left( S\cup T \right) \cap \left( S\cup U \right)$. Then 
        \begin{align*}
          &\left( e\in S\lor e\in T \right) \land \left( e\in S\lor e\in U \right) \\
          &~e\in S\lor \left( e\in T\land e\in U \right)
        .\end{align*}
      Thus, $e\in S\cup \left( T\cap U \right)$, and so $\left( S\cup T \right) \cap \left( S\cup U 
      \right) \subset S\cup \left( T\cap U \right) $.\\
      Since both are subsets of each other, $s\cup \left( t\cap u \right) = \left( s\cup t \right)
      \cap \left( s\cup u \right).$
    \end{proof}

  \item 
    \begin{proof}[Proof]
        Let $e\in \left( S\cup T \right) ^{c}$. Then 
        \begin{align*}
          &~e\in U\land \neg \left( e \in S\lor e\in T\right) \\
          = &~e\in U\land \left( e\not\in S\land  e\not\in T \right) \\
          = &\left( e\in U\land e\not\in S\right) \land \left( e\in U\land e\not\in T \right) 
        .\end{align*}
        Thus, $e\in S^{c}\cap T^{c}$, and so $\left( S\cup T \right) ^{c}\subset S^{c}\cap T^{c}$.\\
        Conversely, let $e\in S^{c}\cap T^{c}$. Then
        \begin{align*}
          &\left( e\in U\land e\not\in S \right) \land \left(  e\in U\land e\not\in T \right)\\
          =&~e\in U\land \left( e\not\in S\land e\not\in T \right) \\
          =&~e\in U\land \neg \left(  e\in S\lor e\in T\right) 
        .\end{align*}
      Thus, $e\in \left( S\cup T \right) ^{c}$, and so $S^{c}\cap T^{c}\subset \left( S\cup T 
      \right) ^{c}$.\\
      Since both subsets are equal, $\left( S\cup T \right) ^{c}=S^{c}\cap T^{c}$. \\ \\
      Now, let $e\in \left( S\cap T \right) ^{c}$. Then 
      \begin{align*}
        &~e\in U\land \neg \left( e\in S\land e\in T \right) \\
        &~e\in U\land \left( e\not\in S\lor e\not\in T \right) \\
        &\left( e\in U\land e\not\in S \right) \lor \left( e\in U\land e\not\in T \right) 
      .\end{align*}
      Thus, $e\in S^{c}\cup T^{c}$, and so $\left( S\cap T\right) ^{c} \subset S^{c}\cup T^{c}$.\\
      Conversely, let $e\in S^{c}\cup T^{c}$. Then 
      \begin{align*}
        &\left( e\in U\land e\not\in S \right) \lor \left( e\in U\land e\not\in T \right)\\
        &~e\in U \land \left( e\not\in S\lor e\not\in T \right) \\
        &~e\in U\land \neg \left( e\in S\land e\in T \right)
      .\end{align*}
      Thus, $e\in \left( S\cap T \right) ^{c}$, and so $S^{c}\cup T^{c}\subset \left( S\cap 
      T \right) ^{c}$.\\
      Since both are subsets of each other, $\left( S\cap T \right) ^{c}=S^{c}\cup T^{c}$.
    \end{proof}


  \item 
    \begin{proof}[Proof]
      Let $e\in \left( S\cup T \right) \setminus \left( S\cap T \right) $. Then
      \begin{align*}
        &\left( e\in S\lor e\in T \right) \land \neg \left( e\in S\land e\in T \right) \\
        &\left( e\in S\lor e\in T \right) \land \left( e\not\in S\lor e\not\in T \right) \\
        &\left( \left( e\in S\lor e\in T \right) \land e\not\in S \right) \lor \left( 
        \left( e\in S\lor e\in T \right)\land e\not\in T\right) \\
        &\left( \left( e\in S\land e\not\in S \right) \lor \left( e\in T\land e\not\in S \right)
        \right) \lor \left( \left(  e\in S\land e\not\in T \right) \lor \left( e\in T\land e\not\in
      T\right) \right) \\
        &\left( e\in T\land e\not\in S \right) \lor \left( e\in S\land e\not\in T \right) 
      .\end{align*}
      Thus, $e\in \left( S\setminus T \right) \cup \left( T\setminus S \right) $, and so $\left( S\cup T \right)
      \setminus \left( S\cap T \right) \subset \left( S\setminus T \right) \cup \left( T\setminus S \right) $.\\
      Conversely, let $e\in \left( S\setminus T \right) \cup \left( T\setminus S \right) $. Then
      \begin{align*}
        &\left( e\in T\land e\not\in S \right) \lor \left( e\in S\land e\not\in T \right) \\
        &\left( e\in S\land e\not\in T \right) \lor \left( e\in T\land e\not\in S \right) \\
        &\left( \left( e\in S\land e\not\in S \right) \lor \left( e\in T\land e\not\in S \right)
        \right) \lor \left( \left(  e\in S\land e\not\in T \right) \lor \left( e\in T\land e\not\in
      T\right) \right) \\
        &\left( \left( e\in S\lor e\in T \right) \land e\not\in S \right) \lor \left( 
        \left( e\in S\lor e\in T \right)\land e\not\in T\right) \\
        &\left( e\in S\lor e\in T \right) \land \left( e\not\in S\lor e\not\in T \right) \\
        &\left( e\in S\lor e\in T \right) \land \neg \left( e\in S\land e\in T \right)
      .\end{align*}
      Thus, we get the statement 
      $e\in \left( S\cup T \right) \setminus \left( S\cap T\right) $, and so $\left( S\setminus 
      T \right) \cup \left( T\setminus S \right) \subset \left( S\cup T \right) \setminus \left(
    S\cap T \right) $.\\
      Since both are subsets of each other, $\left( S\cup T \right) \setminus \left( S\cap T \right)
      = \left( S\setminus T \right) \cup \left( T\setminus S \right)$.
    \end{proof}
\end{enumerate}

From these problems, we observe that sets and logical statements are quite similar. A set is
analogous to a logical statement, and the operators union and intersection resemble the logical
``or'' and ``and'' respectively (specifically, given sets $S,T$, $e\in S\cup T$ is equivalent to
$e\in S\lor e\in T$, and $e\in S\cap T$ is equivalent to $e\in S\land e\in T$). 
Given a well defined complement of $S$, the complement $S^{c}$ is analogous
to the logical "not" (just as only one of $P$ and  $\neg P$ may be true, only one of  $e\in S^{c}$
and $e\in S$ may be true). The symmetric difference is analogous to the ``xor'' operator in the sense
that an element $e$ being in $S\Delta T$ meaning $e$ is in $S$ or $T$, but not both, is similar in
structure to notion that $P\veebar Q$ means that in order to be true, either P or Q could be true,
but not both.
\end{solution}


\begin{problem}{\S 3}
  Complete Exercise 1.16:
  \begin{itemize}
    \item Let $S, T$ be finite sets with $\left| S \right| =\left| T \right| $, and let $f: S \to
      T$ be a function from $S$ to $T$. Prove the following are equivalent:
      \begin{itemize}
        \item $f$ is injective.
        \item $f$ is surjective.
        \item $f$ is bijective.
      \end{itemize}
  \end{itemize}
  and Exercise 1.17:
  \begin{itemize}
    \item Give an example of a function $f: \N \to \N$ that is injective, but not surjective.
    \item Give an example of a function $f: \N \to \N$ that is surjective, but not injective.
  \end{itemize}
\end{problem}

\begin{solution}
  \begin{itemize}
    \item (1.16)
      \begin{proof}[Proof]
        Let $n=\left| S \right| =\left| T \right| $. We start by showing $f$ injective implies $f$
        surjective. Let $f$ be an injective function, and suppose that $f$ is not surjective. Then
        $\exists t\in T$ such that $\forall s \in S, f(s)\neq t$; and so $\left|\image S\right|<n$.
        By the definition of a function, every $s \in S$ is mapped to some element $f(s)\in T$; and
        since $\left| S \right| = n$ and $\left| \image S\right|<n$, at least one $e\in \image S$ is
        mapped to by at least two distinct elements $s,s'\in S$ (analogously, imagine each $e\in \image
        S$ represents a ``hole'', and each $s \in S$ a pigeon; since there are at most $n-1$ holes,
        and $n$ pigeons, by the PHP, at least one hole must have at least two distinct pigeons). \\
        But this implies that $e=f(s)=f(s'), s\neq s'$, a contradiction to injectivity. Thus, if $f$
        is injective, then $f$ must be surjective as well.
        \\ \\
        Now, we show that $f$ surjective implies $f$ injective. Let $f$ be a surjective function, 
        and suppose that $f$ is not injective. Then $ \exists s,s'\in S$ such that $f(s)=f(s'),
        s\neq s'$. By definition of a function, each $s \in S$ is mapped to one and only one
        $f(s)\in \image S$. But since $f$ is not injective, at least one $f(s) \in \image S$ is
        mapped to by at least two distinct $s,s'\in S$ (i.e. $\exists s,s'\in S, \exists
        f(s),f(s')\in \image S, f(s)=f(s'), s\neq s'$), which implies that $\left| \image S \right|
        < n$ (equivalently, at least one $t\in T$ is not mapped to by any $s \in S$), a contradiction
        to surjectivity. Thus, if $f$ is surjective, then $f$ must be injective as well.
        \\
        Since $f$ injective implies $f$ surjective, and $f$ surjective implies $f$ injective, if
        $f$ is either injective or surjective, it is bijective as well; and trivially, $f$ bijective
        implies both injective and surjective. Thus the three statements are equivalent.
      \end{proof}
    \item (1.17)
      \begin{itemize}
        \item Let \begin{align*}
            f: \N &\longrightarrow \N \\
            n &\longmapsto f(n) = n + 1
          .\end{align*}
          $f$ is injective, as no two $n_1,n_2\in \N$ share a $succ(n)$ unless  $n_1=n_2$
          (equivalently, $n_1+1 = n_2 + 1$ implies  $n_1=n_2$). $f$ is also not surjective, as $1\not\in
          \image f$.
        \item Let \begin{align*}
          f: \N &\longrightarrow \N \\
          n &\longmapsto f(n) = \left\lceil \frac{n}{2} \right\rceil 
        .\end{align*}
        $f$ is surjective, as for any $k \in \N$, take $n = 2k \in \N$; then we get $f(n) =
        \left\lceil \frac{2k}{2}\right\rceil = k$. On the other hand,  $f$ is not injective. Let
        $n_1, n_2\in \N, n_1 = 1, n_2 =2$. Then $f(n_1)=f(n_2)=1$, but $n_1\neq n_2$.
      \end{itemize}
  \end{itemize}
\end{solution}




\end{document}
