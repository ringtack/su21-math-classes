\documentclass{homework}
\homework{--- Capstone 2}

\begin{document}

\begin{problem}{\S 1}
  An integral domain $R$ is an \textbf{Euclidean domain} if there is a function $\delta: R\setminus
  \{ 0 \}\to \{ 0,1,\ldots \}$ such that
  \begin{itemize}
    \item if $a,b\in R$ non-zero, then $\delta(a)\le \delta(ab)$.
    \item if $a,b\in R$ and $b\neq 0$, then there exist $q,r\in R$ such that $a=bq+r$ and either
      $r=0$ or $\delta(r)<\delta(b)$.
  \end{itemize}
  Explain why $\Z$ and $F[x]$ are Euclidean domains where $F$ is a field. Prove that any ideal in a
  Euclidean domain is principal.
\end{problem}

\begin{solution}
  For $\Z$, consider the absolute value function $\left| \cdot  \right| : \Z\setminus \{ 0 \}\to \{
  0,1,\ldots\}$. Clearly, $\left| \cdot  \right| $ satisfies the two requirements, and so $\Z$ is a
  Euclidean domain.

  For $F[x]$, consider the degree map $\deg: F[x]\setminus \{ 0 \} \to \{ 0,1,\ldots \}$. We know
  that for any two polynomials $f,g\in F[x]$, we have \[
    \deg{(fg)}=\deg{(f)}+\deg{(g)}\ge \deg{(f)}
  ;\] and by Proposition 5.20 (Division Algorithm for polynomials), the second condition follows.
  Thus $F[x]$ is a Euclidean domain.

  Let $E$ be a Euclidean domain, and let $I_E$ be an ideal of $E$. Let $a\in I_E$ be the value with
  the smallest $\delta(a)$ (note that $a$ is not unique; multiple such $a$s could exist. For
  example, in the principal ideal $(x^2+2)F[x]$, all polynomials with degree $2$ would work for
  $a$). We claim that $(a)=aE$ generates $I_E$.

  Let $b\in I_E$ be any element in the ideal. Since $E$ is a Euclidean domain, we have \[
    b = aq+r
  ,\] and either $r=0$ or $\delta(r)<\delta(a)$. If $r=0$, $b$ is clearly a multiple of $a$ (and
  hence $b\in I$), so suppose $r\neq 0$. By the closure of ideals, $r=b-aq\in I_E$; but then
  $r\in I$, and $\delta(r)<\delta(a)$, a contradiction (since by construction, we assume that $a$
  has the smallest delta). Thus $r$ must be $0$, and hence for any $b\in I_E$, $b$ is divisible by
  $a$. Therefore $(a)$ generates $I_E$, and so $I_E$ is a principal ideal. Since the choice of $a$
  and $I_E$ was arbitrary, this holds for any ideal of $E$, and so every ideal of $E$ is principal.
\end{solution}

\begin{problem}{\S 2}
  \begin{enumerate}[label=(\alph*)]
    \item Let $p$ be a prime integer and $J$ be the set of polynomials in $\Z[x]$ whose constant
      terms are divisible by $p$; show that $J$ is a maximal ideal.
    \item Show that $(x-1)\Z[x]$ is a prime ideal in $\Z[x]$ that is not maximal.
    \item Find an ideal in $\Z\times\Z$ that is prime, but not maximal.
  \end{enumerate}
\end{problem}

\begin{solution}
  \begin{enumerate}[label=(\alph*)]
    \item 
      First, we observe that $J=p\Z[x]+x\Z[x]$; in other words, all polynomials in $f\in J$ are
      of the form \[ f(x)=np+a_1x+\ldots, ~\text{where}~ n\in \Z.\]
      Consider the map $\phi:\Z[x]\to \Z / p\Z$, given by \[
        \phi(f)=\phi(a_0+a_1x+\ldots)=a_0\mod{p}
      .\] In other words, $\phi$ maps a polynomial $f\in \Z[x]$ into the equivalence class of its
      constant term, modulo $p$. $\phi$ is clearly surjective, since $a_0\in \Z$, and every $a\in \Z
      / p\Z$ is also in $\Z$. We observe that $\ker{(\phi)}$ consists of all polynomials of the form
      \[
        f(x)\in p\Z[x]+x\Z[x]
      ;\] equivalently, $\ker{(\phi)}=\{ \text{all polynomials with constant terms}
      \}=\{f(x)\mid f(x)\in p\Z[x]+x\Z[x]\}=J$ (intuitively, only the constant term matters, since
      any of the $x^n$ terms do not affect the result; only the equivalence class of $a_0$ matters).
      By Proposition 3.31(b)iii, since $\phi: \Z[x]\to \Z / p\Z$ is surjective and $\ker{(\phi)}=J$,
      the map \[
        \overline{\phi}: \Z[x] / J \to \Z / p\Z
      \] is an isomorphism. Proposition 3.17 additionally tells us that $\Z / p\Z$ is a field; hence
      $\Z[x] / J \cong \Z / p\Z$ is a field as well. Thus by Proposition 3.40, $J$ is thus a maximal
      ideal.

    \item Let $I=(x-1)\Z[x]$ (the ideal generated by $(x-1)$). Consider the evaluation map at 1: \[
        E_1: \Z[x] \to \Z,\ E_1(f(x))=f(1)
        .\] The kernel of $E_1$ is the set of all polynomials with $1$ as a root; equivalently, \[
        \ker{(E_1)}=\{f(x)\mid f(x)\in (x-1)a(x),\ a(x)\in \Z[x]\} = (x-1)\Z[x]=I
      \] (the ideal $(x-1)\Z[x]$ consists of all polynomials with $(x-1)$ as a factor; in other words,
      all polynomials that have $1$ as a root are in $(x-1)\Z[x]=I$). Clearly, $E_1$ is surjective as
      well; for any $a\in \Z$, the polynomial $ax\in \Z[x]$ yields $E_1(ax)=a$. Hence by Proposition
      3.31(b)iii, the map \[
        \overline{E_1}:\Z[x] / I \to \Z
      \] is an isomorphism. But $\Z$ is an integral domain; hence $\Z[x] / I \cong \Z$ is an
      integral domain as well. Thus by Proposition 3.40, $I$ is a prime ideal. However, $I$
      is not maximal, since $\Z$ is not a field (thus by isomorphism, the quotient ring $\Z[x]
      / I$ is not a field either, and so $I$ is not a maximal ideal).

    \item Let $I=(0,b)\Z\times \Z$ (the ideal generated by $(0,b)$). Consider the map \[
        \phi: \Z\times \Z,\ \phi(a,b)=a ~\text{for}~a,b\in \Z
      .\] The kernel of $\phi$ consists of all elements in $\Z\times \Z$ of the form $(0,b)$ ($b$
      can be anything, as long as $a$ is $0$); equivalently, $\ker{(\phi)}=I$. Clearly, $\phi$ is
      surjective as well; for any $a\in \Z$, the pair $(a,0)\in \Z\times \Z$ yields $\phi(a,0)=a$.
      Hence by Proposition 3.31(b)iii, the map \[
        \overline{\phi}: \Z\times \Z / I \to \Z
      \] is an isomorphism. But $\Z$ is an integral domain; hence $\Z\times \Z / I$
      is an integral domain as well. Thus by Proposition 3.40, $I$ is a prime ideal. However, $I$ is
      not maximal, for the same reasons listed above; $\Z$ is not a field, so the quotient ring
      $\Z\times \Z / I$ is not a field either, and so $I$ is not a maximal ideal.
  \end{enumerate}
\end{solution}

\begin{problem}{\S 3}
  (5.7) 
  \begin{enumerate}[label=(\alph*)]
    \item Let $F$ be a finite field. Prove that \[
        \prod_{\alpha\in F^*} \alpha=-1
      .\] Let $p$ be a prime, and apply this formula to the field $\F_p$ to deduce Wilson's formula:
      \[
        (p-1)!\equiv -1(\mod{p})
      .\] 
    \item As a follow-up, let $m\ge 2$ be an integer that need not be prime. Prove that \[
        \prod_{\alpha\in \left( \Z / m\Z \right)^*} \alpha=\pm 1 
    .\] 
  \end{enumerate}
\end{problem}

\begin{solution}
  \begin{enumerate}[label=(\alph*)]
    \item We begin with one lemma:
      \begin{lemma}
        Any finite field $F$ with more than $2$ elements has only two self-inverses: $\pm 1$.
      \end{lemma}
      \begin{proof}[Proof]
        Let $a\in F^*$. $a$ is a self-inverse if \[
          x^2=1\iff x^2-1=0\iff (x+1)(x-1)=0
        .\] Since $F$ is an integral domain (and so has the cancellation property), either $x=1$ or
        $x=-1$. Thus $x=\pm 1$ are the only self-inverses in $F$. [Note: if $F=\F_2$ only has two
        elements, and $1=-1$, so $\F_2$ only has one self-inverse].
      \end{proof}
      
      Now, consider $\prod_{\alpha\in F^*}$. For any $\alpha$ where $x^2\neq 1$, $\alpha$ has a
      unique inverse $\beta$ such that $\alpha\beta=\beta\alpha=1$; this shows that $\beta\in F^*$
      as well. Thus, for any non-self-inverse $\alpha\in F^*$, its unique inverse $\beta\in F^*$ as
      well, so all non-self-inverses cancel out.

      Hence we are left with only self-inverses; if $F$ only has two elements, then
      $\prod_{\alpha\in F^*} \alpha=1=-1$, so consider $F$ with more than two elements. By the
      lemma, the only self-inverses are $\pm 1$; thus \[
        \prod_{\alpha\in F^*} \alpha=1\cdot -1=-1 
      ,\] as required. 

      Moreover, for a finite field $\F_p$, \[
        \prod_{\alpha\in \F_p^*} \alpha=(p-1)(p-2)\ldots(2)(1)=(p-1)!\equiv -1(\mod{p})  
      ,\] thus proving Wilson's Formula.

    \item For $m=2$, this is clearly satisfied, so consider all $m>2$. Consider any $\alpha\in
      \left( \Z / m\Z \right)^*$. Like above, any $\alpha\in \left( \Z / m\Z \right)^*$ has a unique
      multiplicative inverse $\beta\in \left( \Z / m\Z \right)^*$ such that
      $\alpha\beta=\beta\alpha=1$; moreover, $\beta\in \left( \Z / m\Z \right)^*$. Thus
      $\alpha\beta=1$ for all non-self-inverses, and like before, both are cancelled out in the
      product.

      Thus again we are left with only self-inverses; i.e. all $\alpha\in \left( \Z / m\Z \right)^*$
      that satisfy $\alpha^2=1$. Note that if $\alpha^2\equiv 1(\mod{m})$, then \[
        (m-\alpha)^2=m^2-2m\alpha-\alpha^2\equiv-\alpha^2\equiv-1(\mod{m})
      .\] That is, if $\alpha$ is a self-inverse, then $m-\alpha$ is also a self-inverse; moreover,
      $m-\alpha\in \left( \Z / m\Z \right)^*$ (since $0<m-\alpha<m$, so $m-\alpha\in \Z / m\Z$).
      Additionally, $m-\alpha\neq \alpha$; otherwise, if $m-\alpha\equiv\alpha$, then $m\equiv
      2\alpha\equiv 0(\mod{m})$, so $\alpha\equiv 0(\mod{m})$. But $\gcd{(\alpha,m)}=1$ (recall by
      Proposition 3.17 that $\alpha\in \left( \Z / m\Z \right)^*$ if and only if
      $\gcd{(\alpha,m)}=1$), a contradiction.

      Consider \[
        \alpha(m-\alpha)=\alpha m-\alpha^2\equiv-\alpha^2\equiv-1(\mod{m})
      .\] In other words, for every self-inverse $\alpha\in \left( \Z / m\Z \right)^*$, there exists
      some $m-\alpha\neq \alpha\in \left( \Z / m\Z \right)^*$ such that
      $\alpha(m-\alpha)\equiv-1(\mod{m})$. Thus, letting \[
        \varphi=\{ ~\text{all self-inverses $\alpha\in \left( \Z / m\Z \right)^*$}~ \}
      ,\] we have (since all non-self-inverses cancel out) \[
      \prod_{\alpha\in \left( \Z / m\Z \right)^*} \alpha=\prod_{\alpha\in \varphi}
      \alpha=(-1)^{\frac{\left| \varphi \right|}{2}}=\pm 1
    .\] (Note that $\left| \varphi \right| $ is even, since every self-inverse $\alpha\in \varphi$
    has a distinct and different complement self-inverse $m-\alpha\in \varphi$).

    Therefore $\prod_{\alpha\in \left( \Z / m\Z \right)^*} \alpha=\pm 1$, as required. (It seems
    that whether the product is $1$ or $-1$ depends on the number of self-inverses are in $\Z /
    m\Z$; if there are an odd number of self-inverse---complement-self-inverse pairs, then it is
    $-1$; otherwise, it is $1$. However, I'm not entirely sure how to characterize the number of
    pairs are in a given $\Z / m\Z$).
  \end{enumerate}
\end{solution}

\begin{problem}{\S 4}
  (5.15) Let $F$ be a field, and suppose that the polynomial $X^2+X+1$ is irreducible in $F[X]$. Let
  \[
    K = F[X] / (X^2+X+1)F[X]
  \] be the quotient ring, so we know from Theorem 5.26 that $K$ is a field. We will put bars over
  polynomials to indicate they represent elements in $K$.
  \begin{enumerate}[label=(\alph*)]
    \item Find a polynomial $p(X)\in F[X]$ of degree at most $1$ satisfying \[
        \overline{p(X)}=(\overline{X+3})\cdot (\overline{2X+1})
      .\] 
    \item Find a polynomial $q(X)\in F[X]$ of degree at most $1$ satisfying \[
        \overline{q(X)}\cdot (\overline{X+1})=\overline{1}
      .\] 
    \item Find a polynomial $r(X)\in F[X]$ of degree at most $1$ satisfying \[
        \overline{r(X)}^2=-\overline{3}
    .\] 
  \end{enumerate}
\end{problem}

\begin{solution}
  \begin{enumerate}[label=(\alph*)]
    \item First, note that \[
        (X+3)\cdot (2X+1)=2X^2+7X+3=5X+1+(2X^2+2X+2)
      .\] Thus $\overline{p(X)}=(\overline{X+3})\cdot (\overline{2X+1})=\overline{5X+1}$, so
      $p(X)=5X+1$.
    \item First, observe that \[
        \overline{-X^2-X}=\overline{1}
      ,\] since \[
        \overline{-X^2-X}\equiv \overline{-X^2-X}+\overline{0}\equiv-X^2-X+(X^2+X+1) \equiv \overline{1}
      .\] Thus, if we take $q(X)=-X$ ($-X$ exists in $F[X]$ since $F[X]$ is a ring), then \[
        \overline{q(X)}(\overline{X+1})=\overline{-X^2-X}=\overline{1}
      .\] 
    \item Observe that \[
        -\overline{3}\equiv-\overline{3}+\overline{0}\equiv-\overline{3}+\overline{4X^2+4X+4}\equiv
        \overline{4X^2+4X+1}
      .\] Thus, if we take $r(X)=2X+1$, then \[
        \overline{r(X)}^2=(\overline{2X+1})^2=\overline{4X^2+4X+1}=-\overline{3}
      .\] 
  \end{enumerate}
\end{solution}

\begin{problem}{\S 5}
  Given a field extension $L / K$, a $K$-automorphism of $L$ is an isomorphism $\phi:L\to L$ for
  which $\phi(a)=a$ for all $a\in K$; that is, $\phi$ fixes all elements of $K$. The set of all
  $K$-automorphisms of $L$ is called the \textbf{Galois group} of $L / K$, and is denoted by
  $\Gal_KL$.
  \begin{enumerate}[label=(\alph*)]
    \item Prove that $\Gal_KL$ is in fact a group.
    \item Suppose that $f(x)\in K[x]$ is a polynomial and $\alpha\in L$ is a root. If $\sigma\in
      \Gal_KL$, show that $\sigma(\alpha)$ is also a root of $f(x)$.
    \item Let $L=\Q(\sqrt{D})$ for some square-free integer $D$. Write down all elements in
      $\Gal_{\Q}L$, justifying your reasoning.
  \end{enumerate}
\end{problem}

\begin{solution}
  \begin{enumerate}[label=(\alph*)]
    \item First, we show closure (to ensure that two $K$-automorphisms in $\Gal_KL$ is still a
      $K$-automorphism in $\Gal_KL$). Let $phi_1,\phi_2\in \Gal_KL$. If $a\in K$, then \[
        \phi_1\circ \phi_2(a)=\phi_1(a)=a=\phi_2(a)=\phi_2\circ \phi_1(a)
      .\] Moreover, recall that the composition of two isomorphisms is itself an isomorphism. Thus
      $\phi_1\circ \phi_2$ is a $K$-automorphism, and so $\Gal_KL$ is closed.

      Associativity naturally follows from the associativity of function composition: for
      $\phi_1,\phi_2,\phi_3\in \Gal_KL$, $a\in L$, \[
        (\phi_1\circ (\phi_2\circ \phi3))(a)=\phi_1\circ
        (\phi_2(\phi_3(a)))=\phi_1(\phi_2(\phi_3(a)))=(\phi_1\circ \phi_2)\circ \phi_3(a)
      .\] 

      Next, consider $\phi_I:L\to L$, $a\mapsto a$. $\phi_I(a)=a$; moreover, this is clearly an
      isomorphism. Thus $\phi_I\in \Gal_KL$. For any $\phi\in \Gal_KL$, $a\in L$, \[
        \phi\circ \phi_I(a)=\phi(a)=\phi_I\circ \phi(a)
      ,\] and so $\phi_I$ is the identity element of $\Gal_KL$.

      Finally, let $\phi\in \Gal_KL$. For $b\in L\setminus K$, let $b'=\phi(b)$. Note that
      isomorphism properties force $b'\in L\setminus K$, and $b'$ unique. Define a function \[
        \phi':L\to L,\ a\mapsto a \forall a\in K,\ b'\mapsto b \forall b'\in L\setminus K
      .\] In other words, $\phi'$ maps all $a\in K$ to itself, and for any $b'\in L\setminus K$,
      $\phi'$ maps $b'$ to the unique $b$ that satisfies $phi(b)=b'$ (again, we know the existence
      and uniqueness of $b$ because $\phi$ is isomorphic). By definition, we have \[
        \phi\circ \phi'(a)=\phi(a)=a=\phi'(a)=\phi\circ \phi'(a)
      \] for all $a\in K$, and \[
      \phi'\circ \phi(b)=\phi'(b')=b=\phi\circ \phi'(b)
      \] for all $b\in L\setminus K$. Thus every $\phi\in \Gal_KL$ has an inverse, and so $\Gal_KL$
      is a group.
    \item Suppose $f(x)\in K[x]$ and $\alpha\in L$ is a root of $f(x)$. Note that $f(x)$ can be
      represented as \[
        f(x)=a_0+a_1x+\ldots,~\text{where}~a_i\in K
      ;\] thus \[
        f(\alpha)=a_0+a_1\alpha+a_2\alpha^2+\ldots=0
      .\] But homomorphisms preserve $0$, so \[
      f(\alpha)=0=\sigma(0)=\sigma(a_0+a_1\alpha+a_2\alpha^2+\ldots)=\sigma(a_0)+\sigma(a_1)(\sigma(\alpha))+\ldots
      .\] But every $a_i\in K$, so they remain unchanged by $\sigma$: \[
        0=\sigma(f(\alpha))=\sigma(a_0)+\sigma(a_1)(\sigma(\alpha))+\ldots=a_0+a_1\sigma(\alpha)+a_2\sigma(\alpha)^2+\ldots=f(\sigma(\alpha)) 
      .\] Hence if $\alpha$ is a root of $f$, then $\sigma(\alpha)$ is also a root.
    \item Any element of $\Gal_{\Q}L$ must be a $\Q$-automorphism; in other words, for any
      $a+b\sqrt{D}\in L$, if $\phi\in \Gal_{\Q}L$, then \[
        \phi(a+b\sqrt{D})=\phi(a)+\phi(b)\phi(\sqrt{D})=a+b\cdot \phi(\sqrt{D}) [~\text{since}~
        a,b\in \Q]
      .\] But since $D\in \Z\subseteq \Q$ and $\phi$ fixes elements of $\Q$,
      $D=\phi(D)=\phi(\sqrt{D}^2)=\phi(\sqrt{D})^2$; thus $\phi(\sqrt{D})=\pm D$. Hence the Galois
      group $\Gal_{\Q}L$ has two elements: $\phi_1(a+b\sqrt{D})=a+b\sqrt{D}$ (the identity) and
      $\phi_2(a+b\sqrt{D})=a-b\sqrt{D}$.
  \end{enumerate}
\end{solution}

\begin{problem}{\S 6}
  (5.19) Let $F$ be a finite field with $q$ elements.
  \begin{enumerate}[label=(\alph*)]
    \item Prove that every non-zero element of $F$ is a root of the polynomial $x^{q-1}-1$.
    \item Prove that every element of $F$ is a root of the polynomial $x^q-x$.
    \item Prove the formula \[
        \prod_{\alpha\in F}(x-\alpha)=x^q-x  
    .\] 
  \end{enumerate}
\end{problem}

\begin{solution}
  \begin{enumerate}[label=(\alph*)]
    \item By definition, every non-zero element $a\in F$ has a multiplicative inverse; that is, \[
        F^*=F\setminus \{ 0 \}, ~\text{and}~ \left| F \right| =q-1
      .\] Let $\alpha\in F^*$, and note that $F^*$ is a group with $q-1$ elements. By Corollary 2.42,
      the order of $\alpha$ divides the order of $F^*$; equivalently, if $\left| \alpha \right| =n$,
      then $a-q=kn$ for some $k\in \Z$. Thus \[
        \alpha ^{q-1}=\alpha ^{kn}=(\alpha^n)^k=1^k=1
      .\] Thus for any non-zero $a\in F$, $a\in F^*$, and so given a polynomial \[
        f(x)=x^{q-1}-1
      ,\] we have $f(a)=a^{q-1}-1=1-1=0$. Hence every non-zero $a\in F$ is a root of $f(x)=x^{q-1}-1$.
    \item Consider $g(x)=xf(x)=x(x^{q-1}-1)=x^q-x$. From above, any non-zero $\alpha\in F$ is a root
      of $x^{q-1}-1$, and so is a root of $g(x)$ as well: \[
        g(x)=\alpha^q-\alpha=\alpha(\alpha ^{q-1}-1)=\alpha(1-1)=0
      .\] Moreover, clearly $g(0)=0^q-0=0$. Thus every element in $F$ is a root of $g(x)=x^q-x$.
    \item (b) tells us that $g(x)=x^{q}-x$ has $q$ distinct roots; specifically, every element of
      $F$ is a root of $g(x)$. Theorem 5.34 tells us that $g(x)$ has at most $\deg{(g)}=q$ roots.
      Thus the elements of $F$ are only \textbf{and} all of $g(x)$'s roots. That is, since
      polynomials uniquely factor into their roots (up to ordering), we have \[
        g(x)=x^q-x=\prod_{\alpha\in F}(x-\alpha)  
      ,\] as required.
  \end{enumerate}
\end{solution}






\end{document}
