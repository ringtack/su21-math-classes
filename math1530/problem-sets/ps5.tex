\documentclass{homework}
\homework{Week 7}

\begin{document}



\begin{problem}{\S 1}
  (3.40) Let $R$ be a commutative ring.
  \begin{enumerate}[label=(\alph*)]
    \item Let $I$ be an ideal of $R$. Prove that the map \[
        \pi: R \longrightarrow R / I, a\longmapsto a+I
      \] is a surjective ring homomorphism.
    \item Let $I,J$ be ideals of $R$. Prove that the map \[
        \phi: R \longrightarrow R / I \times  R / J, a\longmapsto (a+I,a+J)
      \] is a homomorphism. What is its kernel? Give an example where it is surjective, and give an
      example where it is not.
  \end{enumerate}
\end{problem}

\begin{solution}
  \begin{enumerate}[label=(\alph*)]
    \item 
      Let $I$ be an ideal of $R$, with $\pi$ as defined above.
      \begin{itemize}
        \item $\pi(1_R)=1+I=1_{R / I}$ (since for any $a+I\in R / I$, we have $(a+I)(1+I)=(a\cdot
          1+I)=a+I$).
        \item Let $a,b\in R$. Then $\pi(a+b)=(a+b)+I=(a+I)+(b+I)=\pi(a)+\pi(b)$, by the definition of
          coset addition.
        \item $\pi(ab)=ab+I=(a+I)(b+I)=\pi(a)\pi(b)$, by the definition of coset multiplication.
      \end{itemize}
      Hence $\pi$ is a ring homomorphism. Surjectivity is almost trivial: let $a+I \in R / I$. Since
      $a\in R$, we have $\pi(a)=a+I$.
    \item Let $I,J$ be ideals of $R$, with $\phi$ as defined above.
      \begin{itemize}
        \item $\phi(1_R)=(1+I,1+J)=1_{R/I \times  R / J}$ (since $(a+I,a+J)(1+I,1+J)=(a\cdot
          1+I,a\cdot 1+J)=(a+I,a+J)$).
        \item
          $\phi(a+b)=((a+b)+I,(a+b)+J)=((a+I)+(b+I),(a+J)+(b+J))=(a+I,a+J)+(b+I,b+J)=\phi(a)+\phi(b)$
          by the definition of coset addition and addition in product rings.
        \item $\phi(ab)=(ab+I,ab+J)=((a+I)(b+I),(a+J)(b+J))=(a+I,a+J)(b+I,b+J)=\phi(a)\phi(b)$ by
          the definition of coset multiplication and multiplication in product rings.
      \end{itemize}
      Hence $\phi$ is a ring homomorphism. Suppose $\phi(a)=0_{R / I \times R / J}=(0+I,0+J)$.
      $a+I=0+I$ whenever $a\in I$, and $a+J=0+J$ whenever $a\in J$. Thus $a$ must be in both $I$ and
      $J$, and so $\ker(\phi)=I\cap J$ (and thus $I\cap J$ is also an ideal of $R$, since the kernel
      of any ring homomorphism $\phi: R\to R'$ is an ideal of $R$).

      Consider $\phi: \Z\to \Z / 2\Z \times  \Z / 3\Z$. Then \[
        \phi(0)=(0,0),\ \phi(1)=(1,1),\ \phi(2)=(0,2),\ \phi(3)=(1,0),\ \phi(4)=(0,1),\ \phi(5)=(1,2)
      ,\] so $\phi$ is surjective. Now consider $\phi: \Z \to \Z / 2\Z \times \Z/4\Z$. Clearly,
      $\phi(n)\neq (1,2)$ for any $n\in \Z$; similarly with $(1,0)$ (since no number is both odd and
      even); thus $\phi$ is not surjective.
  \end{enumerate}
\end{solution}

\begin{problem}{\S 2}
  (3.41) Let $I$ be the principal ideal $(x^2+1)\R[x]$ of $\R[x]$. Prove that the map \[
    \phi: \R[x] / I \longrightarrow \C,\ \phi(f(x)+I)=f(i)
  \] is a well-defined isomorphism.
\end{problem}
\begin{solution}
  Consider the evaluation map $E_i:\R[x]\to \C,\ E_i(f(x))=f(i)$. We first show that $E_i$ is a ring
  homomorphism:
  \begin{itemize}
    \item $E_i(1)=1$
    \item $E_i(a(x)+b(x))=a(i)+b(i)=E_i(a(x))+E_i(b(x))$
    \item $E_i(a(x)b(x))=a(i)b(i)=E_i(a(x))E_i(b(x))$
  \end{itemize}
  Next, consider $E_i(a(x))=0$. For $a(i)=0$, $a(x)$ must have $i$ as a root; in other words,
  $a(x)=(x^2+1)b(x)$ for some $b(x)\in \R[x]$. Thus $\ker(E_i)=(x^2+1)\R[x]=I$, and so by Proposition
  3.31b, $\overline{E_i}:\R[x] / I \to \C$ is a well-defined, injective ring homomorphism. It
  remains to show that $E_i$ is surjective; but that's easy, since for any $a+bi\in \C$, we can
  construct a polynomial $f(x)=a+bx\in \R[x]$ such that $f(i)=a+bi$. Hence $E_i$ is surjective, and
  so $\overline{E_i}=\phi:\R[x] / I \to \C$ is a well-defined isomorphism.
\end{solution}

\begin{problem}{\S 3}
  (3.42) Let $R$ be a commutative ring and let $I,J$ be ideals of $R$.
  \begin{enumerate}[label=(\alph*)]
    \item Prove that the intersection $I\cap J$ is an ideal of $R$.
    \item Prove that the \textit{ideal sum}  \[
        I+J=\{a+b\mid a\in I,b\in J\} 
      \] is an ideal of $R$.
    \item The \textit{ideal product} of two ideals \[
        IJ = \{\sum_{i=1}^{n} a_ib_i\mid n\ge 1,\ a_i\in I,\ b_i\in J\} 
      .\] Prove that $IJ$ is an ideal of $R$.
      \item Let $R=\Z[x]$, and let $I=2\Z[x]+x\Z[x],\ J=3\Z[x]+x\Z[x]$. Prove that the set of
        products $\{ab\mid a\in I,b\in J\} $ is not an ideal.
      \item On the other hand, prove in general that if either $I$ or $J$ is a principal ideal, then
        the set of products $\{ab\mid a\in I,\ b\in J\} $ is an ideal.
  \end{enumerate}
\end{problem}

\begin{solution}
  \begin{enumerate}[label=(\alph*)]
    \item (Proven in 3.40b) Let $\phi:R\to R / I \times  R / J,\ \phi(a)=(a+I,a+J)$. From 3.40b, we
      get that $\ker(\phi)=I\cap J$, and since any kernel of a ring homomorphism $\phi:R\to R'$ is
      an ideal of $R$, $I\cap J$ is an ideal of $R$.

      Alternatively, let $a,b\in I\cap J$. Since $a,b\in I$, $a+b\in I$; similarly, $a,b\in J$
      implies $a+b\in J$. Hence $a+b\in I\cap J$. Additionally, $a\in I$ implies $ra\in I$ for any
      $r\in R$, and analogously for $J$; hence $ra\in I\cap J$, and so $I\cap J$ is an ideal of $R$.

    \item Let $\alpha=a_1+b_1,\ \beta=a_2+b_2\in I+J$. Thus by the commutativity of addition in $R$,
      \[
        \alpha+\beta=(a_1+b_1)+(a_2+b_2)=(a_1+a_2)+(b_1+b_2)
      ,\] and since $a_1+a_2\in I,\ b_1+b2\in J$, we get $(a_1+a_2)+(b_1+b_2)=\alpha+\beta\in I+J$.
      For $\alpha=a+b\in I+J$, $a\in I, b\in J$ implies $ra\in I,\ rb\in J$ for any $r\in R$. Thus \[
        r\alpha=r(a+b)=ra+rb\in I+J
      ,\] and so $I+J$ is an ideal.

    \item Let $\alpha,\beta\in IJ$, where $\alpha=\sum_{i=1}^{n} a_ib_i,\ \beta=\sum_{j=1}^{m}
      a_j'b_j'$. Then \[\alpha+\beta=\sum_{i=1}^{n} a_ib_i+\sum_{j=1}^{m}a_j'b_j'=\sum_{i=1}^{n+m}
      a_ib_i\in IJ,\] since each individual summand $a_ib_i$ has $a_i\in I,\ b_i\in J$ (pardon my
      slight abuse of notation). Moreover, for any $r\in R$, $ra\in I$; thus $r\alpha=r
      \sum_{i=1}^{n} a_ib_i=\sum_{i=1}^{n} (ra_i)b_i\in IJ$. Thus $IJ$ is an ideal of $R$.
    \item First, consider $2\in I,\ 3\in J$. Then $6\in \{ab\mid a\in I,b\in J\}$. Similarly, $x\in
      I,\ x\in J$, so $x^2\in \{ab\mid a\in I,b\in J\}$. But clearly $x^2+6$ has no integer (or, for
      that matter, real) roots, and so no such $a(x)\in \Z[x],\ b(x)\in \Z[x]$ satisfies
      $a(x)b(x)=x^2+6$; thus $x^2+6\not\in \{ab\mid a\in I,b\in J\} $, and so $\{ab\mid a\in I,b\in
      J\} $ is not an ideal.
    \item Suppose without loss of generality that $I$ is a principal ideal. Then for some $c\in R$,
      $I=cR$, and so any $a_\in I$ can be rewritten as $a=cr$ for some $r\in R$. Consider $\{ab\mid
      a\in I, b\in J\} $. For any $a_1,a_2\in I,\ b_1,b_2\in J$, we have \[
        a_1b_1+a_2b_2=cr_1b_1+cr_2b_2=c b_1'+c b_2'=c(b_1'+b_2')\in IJ
      ,\] where $b_i'=rb_i$, since $b_1',b_2'\in J$ and ideals are closed under addition. Moreover,
      for any $r\in R$, \[
        rab=r(r_1c)b=(rr_1c)b\in IJ
      ,\]  since $rr_1\in R$ by closure of multiplication, and for any $r\in R$, $rc\in I$; and so
      $rr_1c\in I,\ b\in J$. Hence $\{ab\mid a\in I,\ b\in J\} $ is an ideal (and commutativity
      implies $ab=ba$, so the proof with $J$ as the principal ideal follows analogously).
  \end{enumerate}
\end{solution}

\begin{problem}{\S 4}
  (3.45) Let $I=2\Z[x]+x\Z[x]$ be a subset of $\Z[x]$.
  \begin{enumerate}[label=(\alph*)]
    \item Prove that $I$ is an ideal of $\Z[x]$.
    \item Prove that $I\neq \Z[x]$.
    \item Prove that $I$ is not a principal ideal.
    \item Prove that $I$ is a maximal ideal of $\Z[x]$.
  \end{enumerate}
\end{problem}

\begin{solution}
  \begin{enumerate}[label=(\alph*)]
    \item Let $I=2\Z[x]+x\Z[x]$. Let $2a_1(x)+xb_1(x),\ 2a_2(x)+xb_2(x)\in I$. Then
      $2a_1(x)+xb_1(x)+2a_2(x)+xb_2(x)=2(a_1(x)+a_2(x))+x(b_1(x)+b_2(x))=2a'(x)+xb'(x)\in I$, since
      $a'(x),b'(x)\in \Z[x]$ by closure of addition.
      
      Let $c(x)\in \Z[x]$. Then $c(x)(2a(x)+xb(x))=2c(x)a(x)+xc(x)b(x)\in I$ by closure of
      multiplication. Thus $I$ is an ideal.
    \item Consider $1\in \Z[x]$. $1\not\in I$, since we need $a(x)=a_0,\ b(x)=0\in \Z[x]$; but all
      coefficients of $\Z[x]$ are integers, so no such $a_0\in \Z$ yields $2a_0=1$.
    \item Consider $x^2+2,x^2+4\in I$ (select $a_1(x)=1,\ a_2(x)=2; b(x)=x$ for
      $a_1(x),a_2(x),b(x)\in \Z[x]$). Suppose there exists some $c(x)\in \Z[x]$ such that
      $I=c(x)\Z[x]$; that is, for any $a(x)\in I$, $a(x)=c(x)d(x)$ for some $d(x)\in \Z[x]$. In
      particular, we have
      \begin{align*}
        x^2+2&= c(x)d(x) \\
        x^2+4&= c(x)d'(x)
      .\end{align*} Since $x^2+2$ has no real roots, we need either $c(x)=1, d(x)=x^2+2$ or
      $c(x)=x^2+2, d(x)=1$. Clearly, $c(x)\neq 1$, since $I\neq \Z[x]$, so suppose $c(x)=x^2+2$.
      Then for some $d'(x)\in \Z[x]$, $x^2+4=(x^2+2)d'(x)$. Clearly, no such $d'(x)$ exists (since
      both have same degrees, we need $d'(x)=a_0$ and $a_0x^2=x^2,\ 2a_0=4$; but no such $a_0$
      satisfies both $a_0=1$ and $a_0=2$). Hence no $c(x)\in \Z[x]$ successfully generates both
      $x^2+2$ and $xsr+4$, and so $I$ is not a principal ideal.

    \item Suppose $I\subsetneq J \subseteq R=\Z[x]$ for some ideal $J$ of $R$. Let $a(x)\in J,\
      a(x)\not\in I$.

      If $a(x)=a_0\in \Z[x]$ is a constant, then we must have $a_0\not\in 2\Z$ (if $a_0\in 2\Z$,
      then we can take $a'(x)=\frac{a_0}{2}$, and so $a(x)=\frac{2a_0}{2}=2a'(x)\in I$, a
      contradiction).  Thus $a_0$ must be odd; i.e. $a_0=2n+1\in 1+2\Z$. But then, note that $2n\in
      I$ (since we can set $a(x)=n,\ b(x)=0$ to get $2n\in I$); and since $J$ is an ideal, and
      $a_0=2n+1\in J$, $2n\in I\subset J$, we have $2n+1-2n=1\in J$. Hence $1\in J$, but then for
      every $r\in R$, we have $r=r\cdot 1\in J$ (by ideal properties); hence $J=R$.

      If $a(x)$ is a polynomial of degree $\ge 1$ (that is, $a(x)$ has at least $1$ ``$x$''
      variable), then $a(x)=a_0+a_1x+\ldots+a_nx^{n}$. But then
      $a(x)=a_0+x(a_1+\ldots+a_nx^{n-1})=a_0+xa'(x)$, where $a'(x)=a_1+\ldots+a_nx^{n-1}\in \Z[x]$.
      Thus, if $a(x)\not\in I$, we must have $a_0$ odd; but then, from above, if $a_0$ odd, then
      $J=R$.

      Thus, in either case, if $I\subsetneq J\subseteq R$, then $J=R$. Thus $I$ is a maximal ideal.

      \begin{remark}
        It seems that for any $a(x)\in J$, there are really only two cases: either $a_0$ even, or
        $a_0$ odd. $a(x)$ being a polynomial is pretty much irrelevant, since if its degree $\ge 1$,
        then we can always factor out $x$ to achieve a new polynomial of the form $a_0+xb(x)$; which
        again is dependent on the parity of $a_0$. Thus, it seems that $\Z[x] / (2\Z[x]+x\Z[x])$ is
        isomorphic to $\Z / 2\Z$. Indeed, since $\Z[x] / (2\Z[x]+x\Z[x])$ only has two elements (all
        elements of $J$ with $a_0$ the same parity are in the same congruence class, and there are
        only two parities), $0+I$ and $1+I$ depending on the parity of $a_0$, we can easily
        construct an isomorphism between $\Z[x] / (2\Z[x]+x\Z[x])$ and $\Z / 2\Z$.

        Moreover, I suspect this could be generalized to any $I = p\Z[x] + x\Z[x]$, where $p\in \Z$
        is a prime. Let $I\subsetneq J\subseteq R$, if $a(x)\in J,a(x)\not\in I$, then we need
        $a_0\not\equiv 0\mod{p}$. The $x\Z[x]$ essentially causes any $x^n$ to vanish, so we only
        really need to focus on $a_0$; and by Fermat's Little Theorem, since $p$ is prime, if
        $a_0\not\equiv 0\mod{p}$, then $a_0^{p-1}\equiv 1\mod{p}$, so $1\in J$, and so $J=R$.
        Moreover, any $a_0\equiv a_0'\mod{p}$ are in the same congruence class $\mod{I}$. Thus, any
        $I=p\Z[x]+x\Z[x]$ is a maximal ideal, and $\Z[x] / (p\Z[x] + x\Z[x])\cong \Z / p\Z$.
      \end{remark}
      
  \end{enumerate}
\end{solution}





\begin{problem}{\S 5}
  (3.46)
  \begin{enumerate}[label=(\alph*)]
    \item Let $m\neq 0$ be an integer. Prove that the ideal $m\Z$ is a maximal ideal (and hence also
      a prime ideal) if and only if $\left| m \right| $ is a prime number in the usual sense of
      primes in $\Z$.
    \item Let $F$ be a field, and let $a,b\in F$ with $a\neq 0$. Prove that the principal ideal
      $(ax+b)F[x]$ is a maximal ideal of the polynomial ring $F[x]$.
    \item Let $F$ be a field with characteristic not equal to $2$, and let $c\in F$ be an element
      with the property that $c$ is not the square of any element in $F$. Prove that the ideal
      $(x^2-c)F[x]$ is a maximal ideal of the polynomial ring $F[x]$.
  \end{enumerate}
\end{problem}



\begin{solution}
  \begin{enumerate}[label=(\alph*)]
    \item Suppose $\left| m \right| \in \Z$ is a prime. Then by Proposition 3.17, $\Z / m\Z$ is a
      field, and thus, by Theorem 3.40, $m\Z$ is a maximal (and prime) ideal.
      
      Conversely, suppose $m\Z$ is a maximal ideal, and suppose $m$ is composite. Then $ab=m$ for some
      $a,b\in \Z, 1<a,b<m$. But then $m\Z \subset a\Z \subset \Z$ (since $a\not\in m\Z$, but
      $m=ab\in a\Z$; moreover, $a-1\not\in a\Z$, but $a-1\in \Z$), and so $m\Z$ is not maximal, a
      contradiction. Thus $m$ must be prime.

      Alternatively, suppose $m\Z$ is a prime ideal. Then for any $a,b\in \Z$, if $ab\in m\Z$, then
      $a\in m\Z$ or $b\in m\Z$. Suppose $m$ is composite. Then $m=ab$ for some $a,b\in \Z$,
      $1<a,b<m$; but then $ab=m\in m\Z$, so either $a\in \Z$ or $b\in \Z$. But that's not possible,
      since $a\not\equiv 0\mod{m}$ and $b\not\equiv 0\mod{m}$ (neither $a$ nor $b$ are multiples of
      $m$, by construction); thus $m$ must be prime.

    \item Let $I=(ax+b)F[x]$, and let $a(x)\in F[x]$. If $a(x)\not\in I$, then $a(x)$ is not a
      multiple of $ax+b$; that is, \[
        a(x)=q(x)(ax+b)+r(x)
      ,\] where $q(x),\ r(x)\in F[x]$, and $0\le $ the degree of $r(x) < $ the degree of $ax+b$ by
      the Division Algorithm. But the degree of $ax+b=1$, so $r(x)$ has degree $0$. In other words,
      all cosets of $I$ in $F[x] / I$ are of the form $\{b_0+I\mid b_0\in F \} $. $b_0\neq 0$, since
      otherwise $b_0+I=I$, a contradiction of not being a multiple; but since $b_0\in F$, and $F$ is
      a field, all non-zero elements have a multiplicative inverse, so for some $b_0^{-1}\in F$, we
      have $(b_0+I)(b_0^{-1}+I)=1+I=1_{F[x] / I}$. Hence every non-zero coset has an inverse, and so
      $F[x] / (ax+b)F[x]$ is a field. By Theorem 3.40, $(ax+b)F[x]$ is therefore a maximal ideal.

    \item Let $I=(x^2-c)F[x]$. For some $a(x)\in F[x]$, let $a(x)+I\in F[x] / I$ be a non-zero coset
      of $I$; then $a(x)\not\in I$, so $a(x)$ is not a multiple of $x^2-c$; in other words, for any
      $q(x)\in F[x]$, $q(x)(x^2-c)\neq a(x)$.

      We then make one observation:
      \begin{itemize}
        \item If $p(x)$ divides $x^2-c$ for some $p(x)\in F[x]$, then either $p(x)=k$ or
          $p(x)=k(x^2-c)$ for some $k\in F$. Clearly, these two work: \[
            \frac{x^2-c}{k}=\frac{1}{k}(x^2-c)\in F[x], ~\text{and}~
            \frac{x^2-c}{k(x^2-c)}=\frac{1}{k}\in F[x]
          .\] We claim that only these two polynomials work. Clearly, any $p(x)$ with degree $\ge 2$
          doesn't work, so consider $a_1x+b_1\in F[x]$. Then \[
            \frac{x^2-c}{a_1x+b_1}
          .\] implies \[
            (a_1x+b_1)(a_2x+b_2)=a_1a_2x^2+(a_1b_2+a_2b_1)x+b_1b_2=x^2-c
          ;\] hence \[
            a_1a_2=1 \implies a_1=\frac{1}{a_2},\ b_1b_2=-c\implies b_1=-\frac{c}{b_2}
          .\] Moreover, we have $(a_1b_2+a_2b_1)=0$, so \[
            a_1b_2+a_2b_1=\frac{b_2}{a_2}-\frac{a_2c}{b_2}=\frac{b_2^2-a_2^2c}{a_2b_2}=0
          ,\] so $b_2^2=a_2^2c$, or $c=\left( \frac{b_2}{a_2} \right)^2$; but $c$ is not the
          square of any number, so this is a contradiction. Thus if $p(x)$ divides $x^2-c$, then
          either $p(x)=k$ or $p(x)=k(x^2-c)$ for some $k\in F$.
      \end{itemize}
      Let $\gcd{(a(x),x^2-c)}=r(x)$ for some $r(x)\in F[x]$. From the observation, $r(x)=k(x^2-c)$
      or $r(x)=k$. Suppose $r(x)=k(x^2-c)$. Since $r(x)$ is a divisor of $a(x)$, we have
      $a(x)=kq(x)(x^2-c)$ for some $q(x)\in F[x]$; but $a(x)\neq (kq(x))(x^2-c)$ for any $q(x)\in
      F[x]$, a contradiction. Hence $r(x)=k$ for some $k\in F$.

      By the Euclidean Algorithm, for some $u(x),\ v(x)\in F[x]$, we have \[
        u(x)a(x)+v(x)(x^2-c)=k=\gcd{(a(x),(x^2-c))}
      .\] Then $\frac{1}{k}u(x)a(x)+\frac{1}{k}v(x)(x^2-c)=1$. But $\frac{1}{k}v(x)(x^2-c)\equiv
      0\mod{(x^2-c)}$, so \[
        \frac{1}{k}u(x)a(x)+\frac{1}{k}v(x)(x^2-c)\equiv \frac{1}{k}u(x)a(x)\equiv 1 \mod{(x^2-c)}
      .\] Clearly, $\frac{1}{k}u(x)\in F[x]$, and so $(a(x)+I)(\frac{1}{k}u(x)+I)=1+I$; hence for
      any $a(x)+I\in F[x] / I$, we have $a^{-1}(x)+I\in F[x] / I$. Thus any non-zero coset $a(x)+I\in
      F[x] / I$ has an inverse, and so $F[x] / I$ is a field. By Theorem 3.40, $I=(x^2-c)F[x]$ is a
      maximal ideal.
  \end{enumerate}
\end{solution}




\end{document}
