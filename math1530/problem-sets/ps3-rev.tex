\documentclass{homework}
\homework{Week 4: Revisions}

\begin{document}
\begin{problem}{\S 4b}
  (I'm only including part b, since it was noted that parts a, c, and d had correct proofs) 

  Let $G$ be a group, and let the centralizer of $g\in G$ be denoted  \[
    Z_G(g) = \{g'\in G\mid gg'=g'g \} 
  .\] 
  Compute the centralizer $Z_G(g)$ for the following elements and groups:
  \begin{itemize}
    \item $G=\D_4$ and $g$ is rotation by $90^{\circ}$.
    \item $G=\D_4$ and $g$ is a flip fixing two vertices of a square.
    \item $G=\GL_2(\R)$ and $g=\begin{pmatrix} a&0\\0&d \end{pmatrix} $.
  \end{itemize}
\end{problem}

\begin{solution}
  Using the Dihedral group $\D_n$ and rotations as defined before (where $r_i(j)=j+i \mod{n}$,
  $f_i(j)=n-j+i \mod{n}$; compared to the book, $r_k=\rho_{k},\ f_k=\phi_{k+1}$), and noting that
  a rotation by $90^{\circ}$ is the same as $r_i$, we see that for any $r_k,\ k\in \{ 0,1,\ldots,n-1
  \}=V_n$, \[
    r_k\circ r_i(j)=r_k(j+i)=j+i+k=j+k+i=r_i(j+k)=r_i\circ r_k(j)
  ;\] equivalently, $r_1$ commutes with any other rotation (for $\D_4$, this means $\{
  \rho_0=e,\rho_1,\rho_2,\rho_3 \}$). Moreover, for any flip $f_k$, $k\in V_n$, we have
  \begin{align*}
    f_k\circ r_1(j)=f_k(j+1)=n-j-1+k && r_1\circ f_k(j)=r_1(n-j+k)=n-j+1+k
  .\end{align*}
  Clearly, these are not equivalent, so any flip $f_k$ does not commute with $ r_1$. Thus,
  $Z_{\D_4}(\rho_1)=\{ e,\rho_1,\rho_2,\rho_3 \}$.

  Now, consider flips that fix two vertices of a square; for this, we see that these flips are
  $f_0$ and $f_2$. Trivially, the identity commutes. Consider any rotation $f_i$, $i\in V_4$. Then
  \begin{align*}
    f_i\circ f_0(j)=f_i(n-j)=n-n+j+i=j+i && f_0\circ f_i(j)=f_0(n-j+i)=n-n+j-i=j-i\\
    f_i\circ f_2(j)=f_i(n-j+2)=n-n+j-2+i= j+i-2 && f_2\circ f_i(j)=f_2(n-j+i)=n-n+j-i+2
  .\end{align*}
  For $ f_0$, $j+i\equiv j-i\mod{4}$ only when $i=0,2$; thus only $f_{0,2}$ commutes with $ f_0$.
  The same is the case with $f_2$ (since $j+i-2\equiv j-i+2\mod{4}$). Thus, out of the flips, only
  $f_{0,2}$ commute with flips that fix two vertices.

  Finally, consider any rotation $r_k$, $k\in V_4$. Then
  \begin{align*}
    r_k\circ f_0(j)=r_k(n-j)=n-j+k && f_0\circ r_k(j)=f_0(j+k)=n-j-k \\
    r_k\circ f_2(j)=r_k(n-j+2)=n-j+k+2 && f_2\circ r_k(j)=f_2(j+k)=n-j-k+2
  .\end{align*}
  In both situations, $r_k\circ f_{0,2}\equiv f_{0,2}\circ r_k$ only if $k\equiv -k\mod{4}$;
  clearly, $k=2$ is the only valid solution, so $r_2=\rho_2$ is the only valid rotation. Thus
  $Z_{\D_4}(\phi_{1,3})=\{ e,\phi_1,\phi_3,\rho_2 \}$.



  Finally, consider elements $\alpha,\beta\in \GL_2(\R)$ where $a=\begin{pmatrix} a&0\\0&d
    \end{pmatrix} $ and $b=\begin{pmatrix} e&f\\g&h \end{pmatrix} $. Then
  \[
    \alpha\beta=\begin{pmatrix} a&0\\0&d \end{pmatrix} \begin{pmatrix} e&f\\g&h \end{pmatrix}
    =\begin{pmatrix} ae & af \\ dg & dh \end{pmatrix} 
  ,\] and \[
  \beta\alpha=\begin{pmatrix} e&f\\g&h \end{pmatrix} \begin{pmatrix} a&0\\0&d \end{pmatrix}
  =\begin{pmatrix} ae&df\\ ag & dh \end{pmatrix} 
.\] In other words, the two commute when $af=df$ and $ag=dg$; equivalently, whenever $a=d$ (or
  $f=g=0$, but that's included in $\GL_2(\R)$). Thus, if $a=d$, then $Z_{\GL_2(\R)}\begin{pmatrix}
    a&0\\0&d \end{pmatrix} =\GL_2(\R)$. If $a\neq d$, then $af=df$ and $ag=dg$ implies $f=g=0$.
    Thus, if $a\neq d$, then $Z_{\GL_2(\R)}\begin{pmatrix} a&0\\0&d
      \end{pmatrix}=\left\{\begin{pmatrix} e&0\\0&h \end{pmatrix} \mid e,h\in \R\setminus \{ 0
      \}\right\}$ (we exclude $0$ since otherwise the matrix would not be invertible, and thus would
      not be in $\GL_2(\R)$).
\end{solution}

\vspace{2mm}
\hrule
\vspace{2mm}

The main thing I was lacking in my original part b was no explanation; I had stated the centralizer
of the specified group element given a group, but I didn't provide justification for why the
centralizer was what it was. Thus, in this revision I sought to provide clear explanations behind
how I got my results.




\end{document}
