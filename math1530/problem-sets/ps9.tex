\documentclass{homework}
\homework{Week 12}

\begin{document}

\begin{problem}{\S 1}
  (6.21) Find all groups of order 15.
\end{problem}
\begin{solution}
  Let $G$ be a group of order 15. Then $G$ has $3$-Sylow subgroups and $5$-Sylow subgroups.
  Inspecting the $5$-Sylow subgroups, let $k$ be the number of $5$-Sylow subgroups of $G$. Sylow's
  theorem tells us that \[
    k|15 ~\text{and}~k\equiv 1(\mod{5})
  .\] This thus forces $k=1$; that is, $G$ has a unique $5$-Sylow subgroup, say $H_5$. $H_5$ is also
  normal, since for any $g\in G$ the conjugate subgroup $g^{-1}H_5g$ is also a subgroup of order
  $5$, and so equals $H_5$.

  Next, Sylow's theorem also tells us that there exists at least one $3$-Sylow subgroup, say $H_3$.
  From Remark 6.33, we have $H_3\cap H_5=\{ e \}$, so we can write \[
    H_3=\{ e,a,a^2 \},\ H_5=\{ e,b,b^2,b^3,b^4 \}
  \] (since all prime-order groups are cyclic); moreover, the only element in common is $e$.

  Consider $aba^{-1}\in H_5$ (since $H_5$ is normal); then \[
    aba^{-1}=b^{j}~\text{for some}~0\le j\le 4
  .\] We then get
  \begin{align*}
    b&=a^{-1}b^{j}a\\
     &=(a^{-1}ba)(a^{-1}ba)\ldots(a^{-1}ba)\\
     &= (a^{-1}ba)^j \\
     &= (a^{-1}(a^{-1}b^ja)a)^{j} \\
     &= (a^{-2}b^ja^2)^{j} \\
     &= ((a^{-2}ba^2)\ldots(a^{-2}ba^2))^{j} \\
     &= (a^{-2}ba^2)^{j^2}\\
     &= (a^{-3}b^{j}a^3)^{j^2} \\
     &= eb^{j^3}e
  .\end{align*}
  Thus $b=b^{j^3}$, so $b^{j^3-1}=e$. Since the order of $b$ is $5$, we need $j^3-1\equiv 0\mod{5}$,
  or $j^3\equiv 1\mod{5}$. Thus $j=1$; so $a^{-1}ba=b$, or $ab=ba$. Since every element of $G$ is a
  power of $a$ times a power of $b$, $G$ is thus Abelian. Moreover, the order of $ab$ is $15$: 
  \begin{align*}
    e=(ab)^k=a^kb^k&\implies a^{k}=b^{-k}\in H_3\cap H_5=\{ e \}\\
                   &\implies a^k=b^k=e\\
                   &\implies 3|k ~\text{and}~5|k\\
                   &\implies 15|k
  .\end{align*}
  Hence any group $G$ with $15$ elements is a cyclic group of order $15$.
\end{solution}

\begin{problem}{\S 2}
  (6.22) Let $G$ be a finite group, and let $H_1$ and $H_2$ be normal subgroups having the property
  that $\gcd{(\left| H_1 \right|, \left| H_2 \right|  )}=1$. Prove that the elements of $H_1$ and
  $H_2$ commute with one another. 
\end{problem}
\begin{solution}
  By Lagrange and since $H_1\cap H_2$ is a subgroup of both $H_1$ and $H_2$, we have \[
    \left| H_1\cap H_2 \right| \mid \gcd{(\left| H_1 \right|,\left| H_2 \right|  )}=1
  ;\] in other words, $H_1\cap H_2=\{ e \}$.

  Suppose $\alpha\in H_1,\ \beta\in H_2$. Since $H_1$ is normal, any $\alpha'\in H_1$ can be
  represented as $\alpha'=\beta^{-1}\alpha\beta$ for any $\beta\in H_2$; and similarly, any
  $\beta'\in H_2$ can be represented as $\beta'=\alpha\beta\alpha^{-1}$ for any $\alpha\in H_1$.

  Consider $\alpha\beta\alpha^{-1}\beta^{-1}\in G$. Clearly, \[
    \alpha(\beta\alpha^{-1}\beta^{-1})=(\alpha\beta\alpha^{-1})\beta^{-1}.
  \] Moreover, \[
  \alpha(\beta\alpha^{-1}\beta^{-1})=\alpha\alpha'\in H_1,
    ~\text{where}~\alpha'=\beta\alpha^{-1}\beta^{-1}\in H_1~(\text{since $H_1$ is normal})
  ,\] and \[
    (\alpha\beta\alpha^{-1}\beta^{-1})=\beta'\beta\in
    H_2,~\text{where}~\beta'=\alpha\beta\alpha^{-1}\in H_2
  .\] Thus \[
    \alpha\beta\alpha^{-1}\beta^{-1}\in H_1\cap H_2=\{ e \}
  ,\] and so \[
    \alpha\beta\alpha^{-1}\beta^{-1}=e \implies \alpha\beta=\beta\alpha
  .\] Thus the elements of $H_1$ and $H_2$ commute with each other.
\end{solution}

\begin{problem}{\S 3}
  (6.23) Let $G$ be a finite group of order $pq$, where $p$ and $q$ are primes satisfying $p>q$.
  Assume further that $p\not\equiv 1(\mod{q})$.
  \begin{enumerate}[label=(\alph*)]
    \item Prove that $G$ is an Abelian group.
    \item Prove that $G$ is cyclic.
  \end{enumerate}
\end{problem}

\begin{solution}
  \begin{enumerate}[label=(\alph*)]
    \item By Sylow's Theorem, $G$ has both a $p$-Sylow subgroup, say $H_p$, and a $q$-Sylow
      subgroup, say $H_q$. We first show that both subgroups are normal.

      Clearly, $H_p\subseteq N_G(H_p)$, and $N_G(H_p)\subseteq G$ is a subgroup of $G$; by Lagrange,
      $\left| H_p \right| =p$ thus divides $\left| N_G(H_p) \right| $, and furthermore $\left|
      N_G(H_p) \right| $ divides $pq=\left| G \right| $. Thus $\left| N_G(H_p) \right| $ is an
      integer that divides $pq$ and is divisible by $p$; so either \[
        \left| N_G(H_p) \right| =p, ~\text{or}~\left| N_G(H_p) \right| =pq
      .\] In the second case, $N_G(H_p)=G$, and so $H_p$ is a normal subgroup and we are done.
      Otherwise, if $\left| N_G(H_p) \right| =p$, then $N_G(H_p)=H_p$, so from Theorem 6.35(c) we
      get \[
        1\equiv k=\frac{\left| G \right| }{\left| N_G(H_p) \right|} = \frac{pq}{p}=q\mod{p}  
      .\] But this implies $p\mid q-1$, a contradiction of $p>q$. Thus $H_p$ is a normal subgroup.

      An analogous argument follows for $H_q$, except on the last step, we assume
      $p\not\equiv 1\mod{q}$ to derive a contradiction, rather than relying on $p>q$.

      Thus $H_p$ and $H_q$ are normal subgroups of $G$. Clearly, $\gcd{(p,q)}=1$, and since both are
      normal subgroups, Problem 6.22 tells us that the elements of $H_p$ and $H_q$ commute with each
      other. However, since $p\cdot q=pq=\left| G \right| $, and Remark 6.33 shows us that $H_p\cap
      H_q=\{ e \}$, every element in $G$ can be formed by multiplying some element $a\in H_p$ and
      $b\in H_q$; that is, for every $g\in G$, $g=a\cdot b$ for some $a\in H_p$, $b\in H_q$.
      Precisely, since $H_p$ and $H_q$ are both groups of prime order, they are cyclic, say
      generated by some $a\in H_p$ and $b\in H_q$ respectively, and every element in $G$ is a power
      of $a$ times a power of $b$. Then for any $g,g'\in G$, \[
        gg'=(a^ib^{i'})(a^jb^{j'})=(a^ia^j)(b^{i'}b^{j'})=(a^ja^i)(b^{j'}b^{i'})
        =(a^jb^{j'})(a^ib^{i'})=g'g
      .\] Thus $G$ is an Abelian group.
    \item Consider $ab\in G$ where $a\in H_p$ and $b\in H_q$ generate their subgroups respectively,
      and let $k$ be some integer such that $(ab)^k=e$. Then
      \begin{align*}
        (ab)^k=a^kb^k=e&\implies a^k=b^{-k}\in H_p\cap H_q=\{ e \}\\
                       &\implies a^k=b^k=e\\
                       &\implies p\mid k ~\text{and}~ q\mid k\\
                       &\implies pq\mid k
      .\end{align*} Thus $ab$ has order $pq$, and so $G$ is a cyclic group.
  \end{enumerate}
\end{solution}

\begin{problem}{\S 4}
  (6.24) Let $G$ be a group. An isomorphism from $G$ to itself is called an \textit{automorphism} of
  $G$. The set of automorphisms is denoted \[
    \Aut(G)=\{ ~\text{group isomorphisms}~ G\to G\}
  .\] We define a composition law on $\Aut(G)$ as follows: for $\alpha,\beta\in \Aut(G)$,
  $\alpha\beta$ is the map from $G$ to $G$ given by $(\alpha\beta)(g)=\alpha(\beta(g))$.
  \begin{enumerate}[label=(\alph*)]
    \item Prove that this composition law makes $G$ a group.
    \item Let $a\in G$. Define a map $\phi_a$ from $G$ to $G$ by \[
        \phi_a:G\longrightarrow G,\ \phi_a(g)=aga^{-1}
      .\] Prove that $\phi_a\in \Aut(G)$, and that the map \[
      G\longrightarrow \Aut(G),\ a\longmapsto \phi_a
      ,\] is a group homomorphism.
    \item Prove that the kernel of the above homomorphism is the center $Z(G)$ of $G$.
    \item Elements of $\Aut(G)$ in the form $\phi_a$, defined above for some $a\in G$, are called
      \textit{inner automorphisms}, and all other elements of $\Aut(G)$ are called \textit{outer
      automorphisms}. Prove that $G$ is Abelian if and only if its only inner automorphism is the
      identity map.
    \item More generally, if $H$ is a normal subgroup of $G$, prove that there is a well-defined
      group homomorphism \[
        G\longrightarrow \Aut(H),\ a\longmapsto \phi_a,\ ~\text{where}~\phi_a(h)=aha^{-1}
      ,\] and that the kernel of this homomorphism is the centralizer of $H$ in $G$.
  \end{enumerate}
\end{problem}
\begin{solution}
  \begin{enumerate}[label=(\alph*)]
    \item Compositions of isomorphisms give an isomorphism, so $G$ is closed under
      composition. Moreover, function composition is associative. Clearly, the identity map \[
        \phi_e:G\longrightarrow G,\ a\longmapsto a
      \] is an isomorphism. Finally, a map is bijective if and only if it has an inverse; thus for
      any $\alpha\in \Aut(G)$, there exists some inverse isomorphism $\alpha^{-1}\in \Aut(G)$ such
      that \[
        \alpha\alpha^{-1}=\alpha^{-1}\alpha=\phi_e
      .\] Thus $\Aut(G)$ is a group under composition.
    \item We first show $\phi_a$ is a group homomorphism for any $a\in G$. Let $g,g'\in G$. Then \[
        \phi_a(gg')=agg'a^{-1}=ageg'a^{-1}=ag(a^{-1}a)g'a^{-1}=(aga^{-1})(ag'a^{-1})=\phi_a(g)\phi_a(g')
      .\] Now, let suppose $g_1,g_2\in G$ and $\phi_a(g_1)=ag_1a^{-1}=ag_2a^{-1}=\phi_a(g_2)$. Then
      \[
        ag_1a^{-1}=ag_2a^{-1}\iff a^{-1}ag_1a^{-1}a=a^{-1}ag_2a^{-1}a\iff g_1=g_2
      .\] Thus $\phi_a$ is injective.

      Finally, for any $g\in G$, consider $a^{-1}ga\in G$ (since all of $a,a^{-1},g\in G$). Then \[
        \phi_a(a^{-1}ga)=a(a^{-1}ga)a^{-1}=g
      .\] Thus $\phi_a$ is an isomorphism.

      Let $\psi:G\to \Aut(G)$, where $\phi(a)=\phi_a$. Let $a_1,a_2\in G$. For any $g\in G$,
      \begin{align*}
        \psi(a_1a_2)(g)=\phi_{a_1a_2}(g)&= a_1a_2ga_2^{-1}a_1^{-1} \\
                                        &= a_1\phi_{a_2}(g)a_1^{-1}=\phi_{a_1}\circ \phi_{a_2}(g)\\
                                        &=\psi(a_1)\psi(a_2)(g)
      .\end{align*} 
      Hence $\psi:G\to \Aut(G)$ is a group homomorphism.
    \item Recall that $\ker{(\psi)}=\{a\in G\mid \phi_a=\phi_e\} $; that is, $\phi_a$
      must equal the identity isomorphism. Suppose $a\in \ker{(\psi)}$; then for any $g\in G$, \[
        \phi_a(g)=aga^{-1}=g=\phi_e(g)
      .\] But then $ag=ga$; in other words, $a\in Z(G)$. Thus $\ker{(\psi)}=Z(G)$.
    \item Suppose $G$ is Abelian. Then for any inner automorphism $\phi_a\in Aut(G)$, \[
        \phi_a(g)=aga^{-1}=aa^{-1}g=g=\phi_e(g)
    ;\] that is, any inner automorphism must be the identity map.

    Conversely, suppose that the only inner automorphism is the identity map. Suppose $G$ is not
    Abelian, and let $a\in G\setminus Z(G)$ be a non-trivial element in $G$ that does not commute
    with every element in $G$. Then for some $g\in G$, $ag\neq ga$. From (b), we know $\phi_a\in
    \Aut(G)$; moreover, \[
      \phi_a(g)=aga^{-1}\neq g=\phi_e(g)
    .\] Thus $\phi_a$ is a non-trivial inner automorphism; but this contradicts the only inner
    automorphism being the identity map. Thus $G$ must be Abelian.
  \item Let $\varkappa: G\to \Aut(H)$, $\varkappa(a)=\phi_a$ with $\phi_a(h)=aha^{-1}$. To show
    well-definedness, we need that for every $a\in G$, $\phi_a\in \Aut(H)$. Indeed, this follows
    trivially from $H$ being a normal subgroup, since $gHg^{-1}=H$ for any $g\in G$, so $ghg^{-1}\in
    H$; thus $\phi_a(H)=H$ and $\phi_a\in \Aut(H)$ for any $a\in G$, where an analogous argument
    from (b) can be used to show that $\phi_a$ is an isomorphism ($\varkappa$ is not necessarily
    well-defined if $H$ is not normal, since we could have $\phi_a(h)=aha^{-1}\not\in H$ for some
    $h\in H$). Consider $a_1,a_2\in G$; then for any $h\in H$, \[
      \varkappa(a_1a_2)(h)=a_1a_2ha_2^{-1}a_1^{-1}=a_1\varkappa(a_2)(h)a_1^{-1}=\varkappa(a_1)\varkappa(a_2)(h)
    .\] Thus $\varkappa$ is a well-defined group homomorphism. Any $a\in \ker{(\varkappa)}$ if
    $\phi_a(h)=aha^{-1}=h=\phi_e(h)$; this requires $ah=ha$, or equivalently, $a\in Z_G(H)$.
    Therefore $\ker{(\varkappa)}=Z_G(H)$.
  \end{enumerate}
\end{solution}

\begin{problem}{\S 5}
  (6.25) Let $\mc{C}_n$ be a cyclic subgroup of order $n$, and let $\Aut(\mc{C}_n)$ be the
  automorphism group of $\mc{C}_n$, as defined in Problem 6.24. Prove that $Aut(\mc{C}_n)$ is
  isomorphic to $(\Z / n\Z)^*$, the unit group in the ring $\Z / n\Z$.
\end{problem}
\begin{solution}
  We begin with a lemma:
  \begin{lemma}[]{}
    Let $\mc{C}_n$ be a cyclic group, with a generator $\left<g \right> $. The order of any element
    $g^k\in \mc{C}_n$ is equal to $\frac{n}{\gcd{(k,n)}}$.
  \end{lemma}
  \begin{proof}[Proof]
    We first show that $(g^k)^{\frac{n}{\gcd{(k,n)}}}=e$. Clearly, \[
      (g^k)^{\frac{n}{\gcd{(k,n)}}}=(g^n)^{\frac{k}{\gcd{(k,n)}}}=e
    .\] Then, we show that ${\frac{n}{\gcd{(k,n)}}}$ is the smallest positive integer $\alpha$ such
    that $(g^k)^\alpha=e$. Consider any $m$ that satisfies $(g^k)^m=e$. Since $\left| g \right| =n$,
    we have $n\mid km$. This then gives \[
      \frac{n}{\gcd{(k,n)}}\mid \frac{k}{\gcd{(k,n)}}m
    .\] Note that, if we decompose $k=p_1\cdot \ldots\cdot p_a$ and $n=q_1\cdot \ldots\cdot q_b$
    using the FToA and order them such that the first $c$ primes $p_{0\le i\le c}=q_{i}$ are equal
    and all $j>c$ have $p_j\neq q_j$, then $\gcd{(k,n)}=\prod_{i=0}^{c} p_i$, and
    $\gcd{(\frac{n}{\gcd{(k,n)}},\frac{k}{\gcd{(k,n)}})}=1$ (since none of the leftover primes are
    the same).

    In particular, this gives us \[
      \frac{n}{\gcd{(k,n)}}\mid m
    ,\] since $\frac{n}{\gcd{(k,n)}}$ and $\frac{k}{\gcd{(k,n)}}$ are relatively prime, and thus do
    not affect each other's divisibility. Thus for any $m$ such that $g^{km}=e$, we have
    $\frac{n}{\gcd{(k,n)}}\le m$, so the order of $g^k$ is $\frac{n}{\gcd{(k,n)}}$.
  \end{proof}
  
  
  Let $g$ be a generator of $\mc{C}_n$, and consider the map \[
    \varpi: (\Z / n\Z)^* \to \Aut(\mc{C}_n),\ k\mod{n}\mapsto \psi_k,~\text{where}~\psi_k(g)=g^k
  .\] We must first show that every $\psi_k$ is an automorphism of $\mc{C}_n$. Recall that any
  isomorphism from $\mc{C}_n$ to $\mc{C}_n$ must preserve the orders of elements; in particular,
  they must map generators to generators (since generators determine the entire cyclic group). For
  $\psi_k(g)=g^k$, since $k\in (\Z / n\Z)^*$, we have $\gcd{(k,n)}=1$, so any generator $g$ will
  still have order $\frac{n}{\gcd{(k,n)}}=n$ by Lemma 1.

  We then show that $\varpi$ is a group homomorphism. Let $a,b\in (\Z / n\Z)^*$, and recall that
  $ab\mod{n}\equiv (a\mod{n})(b\mod{n})$; then for any $g^i\in \mc{C}_n$,
  \begin{align*}
    \varpi(ab)(g^i)&=\psi_{ab\mod{n}}(g^i)\\
    &=(g^i)^{ab\mod{n}}\\
    &=(g^i)^{(b\mod{n})(a\mod{n})}\\
    &=\left( (g^i)^{b\mod{n}} \right) ^{a\mod{n}}\\
    &=\varpi(b)(g^i)^{a\mod{n}}\\
    &=\varpi(a)\varpi(b)(g^i)
  .\end{align*}
  Hence $\varpi$ is a group homomorphism.


  Suppose $a,b\in (\Z / n\Z)^*$ such that $\varpi(a)(g)=g^{a}=g^{b}=\varpi(b)(g)$ (we only inspect
  actions on $g$ here, since $\psi_i(g)$ completely determines the image $\psi_i(\mc{C}_n)$). Since
  $\psi_a$ and $\psi_b$ are isomorphisms and thus maintain the order, we need \[
    ak\equiv bk\mod{n}
  ,\] or equivalently $a\equiv b\mod{n}$. Thus $\varpi$ is injective.

  Consider any automorphism $\psi\in \Aut(\mc{C}_n)$. Recall again that $\psi$ is an automorphism on
  $\mc{C}_n$ if and only if it preserves the orders of every element, in particular the generators
  of $\mc{C}_n$. Equivalently, any $\psi$ must map $g\mapsto g^k$, where $\gcd{(k,n)}=1$, since by
  Lemma 1, $\left| g \right| =\left| g^k \right| =n$ only if $\gcd{(k,n)}=1$. But the set of all $k$
  that satisfy $\gcd{(k,n)}=1$ is exactly and entirely the unit group $(\Z / n\Z)^*$, by Proposition
  3.17. Thus any automorphism $\psi_k$ with $g\mapsto g^k$ has $k\in (\Z / n\Z)^*$, and so $\varpi$
  is surjective.

  Therefore $\varpi$ is a group isomorphism, and so $\mc{C}_n\cong (\Z / n\Z)^*$.
\end{solution}





\end{document}
