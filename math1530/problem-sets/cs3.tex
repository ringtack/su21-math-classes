\documentclass{homework}
\homework{--- Capstone 3}

\begin{document}
\begin{problem}{\S 1}
  \begin{enumerate}[label=(\alph*)]
    \item Let $N$ and $H$ be groups and suppose that $\phi:H\to \Aut(N)$ is a homomorphism from $H$
      to the group of automorphisms of $N$. For $a\in H$, we write $\phi_a$ to denote the
      corresponding automorphism, so that $\phi_a(n)\in N$ for each $n\in N$. Now let $G=N\times H$
      be the cartesian product of $N$ and $H$ and consider the following operation: \[
        (n_1,h_1)\cdot _{\phi}(n_2,h_2):=(n_1\phi_{h_1}(n_2),h_1h_2)
      .\] Verify the following statements:
      \begin{enumerate}
        \item The set $G$ with $\cdot _\phi$ is a group.
        \item The set of elements of the form $(n,e_H)$ is a \textbf{normal} subgroup of $G$
          isomorphic to $N$, and the set of elements of the form $(e_N,h)$ is a subgroup (not
          necessarily normal) of $G$ isomorphic to $H$.
      \end{enumerate}
      The group $G$ is called the \textbf{semi-direct product of $N$ and $H$} and is denoted
      $N\rtimes_\phi H$. Note that the special case where $\phi$ is the identity automorphism gives
      rise to the usual direct product.

    \item Suppose that $G$ is a group, $N$ is a normal subgroup and $H$ is a subgroup for with
      $G=NH$ and $N \cap H=\{ e \}$. Let $\phi:H\to \Aut(N)$ be given by $\phi_h(g)=hgh^{-1}$, as
      worked through in Exercise 6.24(e). Prove that the map \begin{align*}
        f: N\rtimes_\phi H &\longrightarrow G \\
        (n,h) &\longmapsto nh
      \end{align*} is an isomorphism. In this case, we say that $G$ decomposes as the semi-direct
      product of $N$ and $H$.
  \end{enumerate}
\end{problem}

\begin{solution}
  \begin{enumerate}[label=(\alph*)]
    \item
    \begin{enumerate}
      \item Since $\phi_{h_1}$ is an isomorphism on $N$, $\phi_{h_1}(n_2)\in N$, so
        $n_1\phi_{h_1}(n_2)\in N$ as well, and $\cdot_\phi$ is closed.

        Let $(n_1,h_1), (n_2,h_2), (n_3,h_3)\in G$. Then \[
          ((n_1,h_1)\cdot_\phi(n_2,h_2))\cdot_\phi(n_3,h_3)=(n_1\phi_{h_1}(n_2),h_1h_2)\cdot
          _\phi(n_3,h_3)=(n_1\phi_{h_1}(n_2)\phi_{h_1h_2}(n_3),h_1h_2h_3)
        ,\] and \[
          (n_1,h_1)\cdot_\phi((n_2,h_2)\cdot_\phi(n_3,h_3))=(n_1,h_1)\cdot_\phi(n_2\phi_{h_2}(n_3),
          h_2h_3)=(n_1\phi_{h_1}(n_2\phi_{h_2}(n_3)),h_1h_2h_3))
        .\] But we know $\phi_{h_i}$ is an isomorphism, so \[
          \phi_{h_i}(n_1n_2)=\phi_{h_i}(n_1)\phi_{h_i}(n_2)
        .\] Additionally, $\phi$ is a homomorphism, so \[
          \phi_{h_1h_2}=\phi(h_1h_2)=\phi(h_1)\phi(h_2)=\phi_{h_1}\phi_{h_2}
        .\] Together, we then get \[
          \phi_{h_1}(n_2\phi_{h_2}(n_3)))=\phi_{h_1}(n_2)\phi_{h_1h_2}(n_3)
        .\] Thus \[
          ((n_1,h_1)\cdot_\phi(n_2,h_2))\cdot_\phi(n_3,h_3)
          =(n_1,h_1)\cdot_\phi((n_2,h_2)\cdot_\phi(n_3,h_3))
        ,\] and so $\cdot_\phi$ is associative in $G$.

        Since $\phi$ is a homomorphism, $\phi(e_H)=\phi_{e_H}$, the identity isomorphism in
        $\Aut(N)$.  Then for any $(n,h)\in G$, \[
          (e_N,e_H)\cdot_\phi(n,h)=(e_N\phi_{e_H}(n),e_Hh)=(n,h)
        .\] Thus $G$ has an identity element, namely $(e_N,e_H)\in G$.

        Finally, for any $(n,h)\in G$, choose $(n^{-1},h^{-1})\in G$ (both of which exist since $N,H$
        are groups). Since $\phi$ is a homomorphism, \[
          \phi(h)\phi(h^{-1})=\phi(hh^{-1})=\phi(e_H)=\phi_{e_H}
        ,\] and similarly for $\phi(h^{-1})\phi(h)$. Thus \[
          (n,h)\cdot_\phi(n^{-1},h^{-1})=(n\phi_{hh^{-1}}(n^{-1}),hh^{-1})=(nn^{-1},e_H)=(e_N,e_H)
        ,\] and analogously for $(n^{-1},h^{-1})\cdot _\phi(n,h)$. Thus every element in $G$ has an
        identity.

        Therefore $G$ with $\cdot _\phi$ is a group.
      \item Consider the subset of $G$, \[
          G_N = \{(n,e_H)\in G\mid n\in N\} 
        .\] We first show this is a normal subgroup. Clearly, it is closed (since $n_1n_2\in N$ for
        any $n_1,n_2\in N$), and it has the identity element $(e_N,e_H)\in G_N\subseteq G$ (since
        $e_N\in N$); finally, every $(n,e_H)\in G_N$ has a corresponding inverse $(n^{-1},e_H)\in
        G_N$ (which is the inverse, as shown in part (a)(a) and since $e_H^{-1}=e_H$). To see why
        this is a normal subgroup, consider any $(n,h)\in G$, which has inverse $(n^{-1},h^{-1})\in
        G$. Then for any $(n',e_H)\in G_N$,
        \begin{align*}
          (n,h)\cdot_\phi(n',e_H)\cdot_\phi(n^{-1},h^{-1})&= (n\phi_h(n'),h)\cdot_\phi
          (n^{-1},h^{-1})\\
                                                          &= (n\phi_h(n')\phi_h(n^{-1}),hh^{-1}) \\
                                                          &= (n\phi_h(n'n^{-1}),e_H)\in G_N
        ,\end{align*}
        since $\phi_h(n'n^{-1})\in N$ and $n\in N$, so its product $n\phi_h(n'n^{-1})\in N$ as
        well. Hence $G_N$ is a normal subgroup of $G$. Isomorphism is simple; simply define a
        mapping \[
          \psi:G_N\longrightarrow N,\ (n,e_H)\longmapsto n
        .\] This is clearly an isomorphism (I spare the trivial demonstration of injectivity and
        surjectivity), so $G_N$ is isomorphic to $N$.\\

        Now, consider the subset of $G$, \[
          G_H = \{(e_N,h)\in G\mid h\in H\} 
        .\] Clearly, it is closed: for any $(e_N,h_1),(e_N,h_2)\in G_H$, \[
          (e_N,h_1)\cdot _\phi(e_N,h_2)=(e_N\phi_{h_1}(e_N), h_1h_2)=(e_N,h_1h_2)\in G_H
        \] (isomorphisms, even homomorphisms, preserve the identity). Moreover, $(e_N,e_H)\in G_H$,
        since $e_H\in H$. Finally, every $(e_N,h)\in G_H$ has a corresponding inverse
        $(e_N,h^{-1})\in G_H$. Hence $G_H$ is a subgroup of $G$ (note that it's not necessarily
        normal; for any non-identity $h'\in H$ and non-identity $(n,h)\in G$, the resulting first
        product from $(n,h)(e_N,h')(n^{-1},h^{-1})$ is $n\phi_{hh'}(n^{-1})$, which is not equal to
        $e_N$ unless $\phi_{hh'}=\phi_{e_H}$, the identity automorphism). Isomorphism is analogous;
        a mapping \[
          \psi:G_H\longrightarrow H,\ (e_N,h)\longmapsto h
        \] is clearly an isomorphism, so $G_H$ is isomorphic to $H$.
      \end{enumerate}

    \item Let $G$ be a group, $N\unlhd G$, and $H\le G$ such that $G=NH$ and $N \cap H=\{ e \}$. Let
      \begin{align*}
        \phi: H &\longrightarrow \Aut(N) \\
        h &\longmapsto \phi_h,\ \phi_h(g)=hgh^{-1}
        \end{align*} be a group homomorphism from $H$ to $\Aut(N)$ (since $N$ is a normal subgroup),
        and consider the map \begin{align*}
        f: N\rtimes_\phi H &\longrightarrow G \\
        (n,h) &\longmapsto nh
      .\end{align*} We first prove that $f$ is a homomorphism.

      Let $(n_1,h_1),(n_2,h_2)\in N\rtimes_\phi H$. Then \[
        (n_1,h_1)\cdot _\phi(n_2,h_2)=(n_1\phi_{h_1}(n_2),h_1h_2)
      .\] We thus have
      \begin{align*}
        f(n_1\phi_{h_1}(n_2),h_1h_2)&= n_1h_1n_2h_1^{-1}h_1h_2 \\
                                    &= n_1h_1n_2h_2 \\
                                    &= (n_1h_1)(n_2h_2)\\
                                    &= f(n_1,h_1)\cdot f(n_2,h_2)
      .\end{align*} Hence $f$ is a group homomorphism.

      For any $nh\in G$, simply choose $(n,h)\in N\rtimes_\phi H$; then \[
        f(n,h)=nh
      ,\] and so $f$ is surjective.

      Next, we show that in general, if the intersection of two groups $N$ and $H$, $N \cap H$, is
      trivial, then every element $nh\in NH$ is uniquely expressed by $n\in N$ and $h\in H$ (this
      directly shows injectivity of $f$, but is also applicable in future problems). Consider
      $n_1,n_2\in N$ and $h_1,h_2\in H$ such that $n_1h_1=n_2h_2$. Then \[
        n_2^{-1}n_1h_1h_1^{-1}=n_2^{-1}n_2h_2h_1^{-1}\implies n_2^{-1}n_1=h_2h_1^{-1}\in N \cap H=\{
        e\}
      ,\] so $n_1=n_2$ and $h_1=h_2$. Thus every element $nh\in NH=G$ is uniquely expressed by $n\in
      N$ and $h\in H$. This also proves injectivity of $f$, since if
      $f(n_1,h_1)=n_1h_1=n_2h_2=f(n_2,h_2)$, then we need $(n_1,h_1)=(n_2,h_2)$.

      Therefore $f$ is an isomorphism, and so $G$ decomposes as the semi-direct product of $N$ and
      $H$.
  \end{enumerate}
\end{solution}
\begin{problem}{\S 2}
  Let $G$ be a group of order $2p$ where $p$ is some odd prime number. Prove that $G$ is isomorphic
  to the cyclic group $\mc{C}_{2p}$ or to the dihedral group $\D_p$.
\end{problem}

\begin{solution}
  Suppose $G$ has order $2p$; then $G$ has $2$-Sylow subgroups and $p$-Sylow subgroups. Inspecting
  the $p$-Sylow subgroups, let $k$ be the number of $p$-Sylow subgroups of $G$. Sylow's theorem
  tells us that \[
    k\mid 2p ~\text{and}~k\equiv 1\mod{p}
  .\] Hence $k=1$; that is, $G$ has a unique $p$-Sylow subgroup, say $H_p$. $H_p$ is also normal,
  since for any $g\in G$ the conjugate subgroup $g^{-1}H_pg$ is also a subgroup of order $p$, and so
  equals $H_p$.

  Next, Sylow's theorem also says that there exists at least one $2$-Sylow subgroup, say $H_2$. From
  Remark 6.33, we have $H_2\cap H_p=\{ e \}$, so we can write \[
    H_2=\{ e,a \},\ H_p=\{ e,b,b^2,\ldots,b^{p-1} \}
  \] (since all prime-order groups are cyclic); moreover, the only shared element is $e$.

  Consider $aba^{-1}\in aH_pa^{-1}=H_p$; then \[
    aba^{-1}=b^{j}~\text{for some}~0\le j\le j-1
  .\] We then get
  \begin{align*}
    b&= a^{-1}b^ja \\
     &= (a^{-1}ba)^j \\
     &= (a^{-1}a^{-1}b^jaa)^j \\
     &= (a^{-2}b^ja^2)^j \\
     &= b^{j^2}
  .\end{align*}
  Hence $b^{j^2}=b$, so $b^{j^2-1}=e$. Since $b$ has order $j$, we get \[
    j^2\equiv 1\Mod{p}, ~\text{or}~j^2-1\equiv (j+1)(j-1)\equiv 0\Mod{p}
  .\] Thus $j=1$ or $p-1$ (since $\Z /p\Z$ is an integral domain, or since $d$-degree polynomials in
  polynomial rings $F[x]$---where $F$ is a field---have at most $d$ distinct roots, etc).

  If $j=1$, then $aba^{-1}=b$, or $ab=ba$. Since every element of $G$ is a power of $a$ times a
  power of $b$, and elements in $H_2$ commute with elements in $H_p$, $G$ is thus Abelian. Moreover,
  the element $ab$ has order $2p$:
  \begin{align*}
    e=(ab)^k=a^kb^k&\implies a^k=b^{-k}\in H_2\cap H_p=\{ e \}\\
                   &\implies a^k=b^k=e\\
                   &\implies 2\mid k~\text{and}~p\mid k\\
                   &\implies 2p\mid k
  ,\end{align*} where the last implication is true since $p$ odd, so $\gcd{(2,p)}=1$. Hence $ab$
  generates $G$, and so $G$ is a cyclic group of order $2p$; that is, $G$ is isomorphic to
  $\mc{C}_{2p}$.

  If $j=p-1$, then $aba^{-1}=b^{p-1}=b^{-1}$. Since again, powers of $a$ times powers of $b$
  completely form $G$, $G$ thus has $2p$ elements of the form $a^ib^j,\ 0\le i\le 1,\ 0\le j\le
  p-1$, which have the following properties: \[
    a^2=e,\ b^p=e,\ ab^{-1}=ba
  ,\] which exactly defines the dihedral group $\D_p$.

  Therefore, any group $G$ with order $2p$ is isomorphic to either the cyclic group of order $2p$,
  $\mc{C}_{2p}$, or the dihedral group $\D_p$.
\end{solution}

\begin{problem}{\S 3}
  \begin{enumerate}[label=(\alph*)]
    \item If $G$ is a group of order $60$ that has a normal $3$-Sylow subgroup, prove that $G$ also
      has a normal $5$-Sylow subgroup.
    \item If $G$ is a non-cylic group of order $21$, how many $3$-Sylow subgroups does $G$ have?
  \end{enumerate}
\end{problem}

\begin{solution}
  \begin{enumerate}[label=(\alph*)]
    \item 
      We start with three lemmas:
      \begin{lemma}{}
        A $p$-Sylow subgroup of a group $G$ is normal if and only if it is unique.
      \end{lemma}
      \begin{proof}[Proof]
        Let $G$ be a group, and suppose a $p$-Sylow subgroup $H$ of $G$ is normal. Suppose $H'$ is 
        another $p$-Sylow subgroup. By Sylow's theorem, any $p$-Sylow subgroups $H_1$ and $H_2$ are
        conjugates; that is, for some $g\in G$, \[
          H_2=g^{-1}H_1g
        .\] Thus \[
          H'=g^{-1}Hg
        \] for some $g\in G$; but $H$ is normal, so $g^{-1}Hg=H$ for any $g\in G$. Therefore $H=H'$, so
        $H$ is the unique $p$-Sylow subgroup of $G$.

        Conversely, suppose $H$ is the unique $p$-Sylow subgroup of $G$. For any $g\in G$, the conjugate
        group $g^{-1}Hg$ is also a $p$-Sylow subgroup (since it has the same order), so uniqueness of
        $p$-Sylow subgroups forces \[
          H=g^{-1}Hg
        \] for every $g\in G$. Thus $H$ is normal.
      \end{proof}
      \begin{lemma}{}
        Suppose $G$ is a group and $N$ is a normal subgroup of $G$ with order $k$. If the quotient group
        $G / N$ has a normal subgroup of order $m$, then $G$ has a normal subgroup of order $km$.
      \end{lemma}
      \begin{proof}[Proof]
        Let $H=\{ \mc{C}_1,\mc{C}_2,\ldots,\mc{C}_m \}$ be a normal subgroup of $G /N$ with order $m$,
        where each $\mc{C}_i$ represent a distinct coset. Recall the natural group homomorphism \[
          \phi:G\longrightarrow G / N,\ \phi(g)=gN
        .\] We claim that the pre-image of $H$, say ``$\phi^{-1}(H)$''$\subseteq G$, is a normal
        subgroup of $G$. Technically, this is an abuse of notation; there isn't a well-defined inverse
        function $\phi^{-1}$, since $\phi$ isn't isomorphic, so when we say $\phi^{-1}(H)$ we really
        mean \[
          \phi^{-1}(H)=\{g\in G\mid \phi(g)=gN\in H\} =\mc{C}_1\cup \mc{C}_2\cup \ldots\cup
          \mc{C}_m \subseteq G
        ;\] that is, the set of all $g\in G$ that are sent to some coset $\mc{C}_i$ in $H$. Since each
        coset has $k$ elements, $\phi^{-1}(H)$ has $km$ elements.

        We first show that $\phi^{-1}(H)$ is closed. Let $g_1,g_2\in \phi^{-1}(H)$, and consider
        $\phi(g_1g_2)=g_1g_2N$. This is exactly equal to $g_1N\cdot_{G /N}g_2N$, and since $g_1N,g_2N\in
        H$ and $H$ is a subgroup, we have $g_1g_2N\in H$ as well. Thus $g_1g_2\in \phi^{-1}(H)$.

        Every coset of $G / N$ has the identity element, including those in $H$, so $e\in \phi^{-1}(H)$.
        Since for any $gN\in H$ with $g\in \phi^{-1}(H)$, its inverse $g^{-1}N\in H$ as well (since
        $gNg^{-1}N=N=e_{G /H}$, and $H$ is a subgroup), we have $g^{-1}\in \phi^{-1}(H)$ for any $g\in
        \phi^{-1}(H)$. Therefore $\phi^{-1}(H)$, the set of all elements in $G$ are in some coset in
        $H$, is a subgroup of $G$ with order $km$.

        Since $H$ is normal in $G / N$, \[
          (aN)H(a^{-1}N)=H~\text{for any}~aN\in G /N
        .\] Additionally, recall that $G= \mc{C}_1\cup \mc{C}_2\cup \ldots\cup \mc{C}_{\left| G \right|
        /k}$, where $\mc{C}_i\in G /N$ are all elements of $G /N$; that is, $G$ is the (disjoint) union
        of all cosets of $N$. This means that any element $a\in G$ has a corresponding coset
        $aN=\mc{C}_j\in G /N$ for some $\mc{C}_j\in G /N$. Together, this means that for any $gN\in H$,
        $aNgNa^{-1}N=aga^{-1}N\in H$ for any $a\in G$.

        However, this means that for any $g\in \phi^{-1}(H)$, $aga^{-1}\in \phi^{-1}(H)$ for any $a\in
        G$ as well. Equivalently, $\phi^{-1}(H)$ is normal in $G$, as desired.

        Therefore, if $\left| N \right| =k$ and $G /N$ has a normal subgroup with order $m$, then $G$
        has a normal subgroup of order $km$.
      \end{proof}
      \begin{lemma}{}
        Let $H$ be a normal subgroup of a group $G$, and suppose $H_p$ is a normal $p$-Sylow subgroup of
        $H$. Then $H_p$ is a normal $p$-Sylow subgroup of $G$.
      \end{lemma}
      \begin{proof}[Proof]
        Since $H$ is normal in $G$, then $gHg^{-1}=H$ for every $g\in G$. Since $H_p$ is a subgroup of
        $H$, then $gH_pg^{-1}\subseteq gHg^{-1}=H$. From Proposition 6.10(b) and since any subgroup
        $H_p$ of a subgroup $H$ of $G$ is a subgroup of $G$ itself, the conjugate set $gH_pg^{-1}$ is a
        subgroup of $G$.

        Indeed, the conjugate set $gH_pg^{-1}$ is a subgroup of $H$, since $gH_pg^{-1}$ already contains
        an identity and inverses for every element, all of which are in $H$, due to $gH_pg^{-1}\subseteq
        H$ and $gH_pg^{-1}$ being a subgroup of $G$. Closure also holds: for any
        $gh_1g^{-1},gh_2g^{-1}\in gH_pg^{-1}$, the product \[
          gh_1g^{-1}gh_2g^{-1}=gh_1h_2g^{-1}=gh'g^{-1}\in gH_pg^{-1}
        ,\] where $h_1,h_2\in H$ and $h'=h_1h_2\in H$. 

        Thus, $gH_pg^{-1}$ is a subgroup of $H$ for any $g\in G$; moreover, $gH_pg^{-1}$ has the same
        order as $H_p$, namely $p$. However, since $H_p$ is a normal $p$-Sylow subgroup of $H$---and
        thus unique, by Lemma 1---we need $H_p=gH_pg^{-1}$. Since the choice of $g\in G$ was arbitrary,
        $H_p$ is thus a normal $p$-Sylow subgroup of $G$ as well.
      \end{proof}
      
       

      Now, suppose $G$ is a group of order $60$ with a normal $3$-Sylow subgroup, say $H_3$. Then the
      group $G /H_3$ is well-defined, and by Lagrange it has order $20$. Since $5\mid 20$, Sylow's
      theorem tells us that $G /H_3$ has a $5$-Sylow subgroup, say $H_5'$. By Sylow's theorem, if $k$
      represents the number of $5$-Sylow subgroups of $G /H_3$, we must also have \[
        k\mid 20~\text{and}~k\equiv 1\Mod{5}
      ;\] in other words, $k=1$, and $H_5'$ is unique and thus normal, by Lemma 1. But then Lemma 2
      tells us that $G$ has a normal subgroup of order $15$, say $H_{15}$; applying Sylow's again gives
      a normal $5$-Sylow subgroup $H_5$ of $H_{15}$. Lemma 3 finally tells us that $H_5$ is a normal
      $5$-Sylow subgroup in $G$, as desired.

      Therefore, if $G$ is a group of order $60$ that has a normal $3$-Sylow subgroup, then $G$ also has
      a normal $5$-Sylow subgroup.
    \item Suppose $G$ is a non-cyclic group of order $21$. Since $21=3\cdot 7$, $G$ has a $7$-Sylow
      subgroup, say $H_7$; moreover, $H_7$ is unique, since (letting $k_7$ represent the number of
      distinct $7$-Sylow subgroups in $G$) we need \[
        k_7\mid 21~\text{and}~k_7\equiv 1\Mod{7}
      ,\] or equivalently, $k_7=1$. Lemma 1 then tells us that $H_7$ is normal.

      Now, let $k_3$ represent the number of distinct $3$-Sylow subgroups of $G$. Sylow requires \[
        k_3\mid 21~\text{and}~k_3\equiv 1\Mod{3}
      ;\] in other words, $k_3=1$ or $7$.

      If $k_3=1$, then Lemma 1 says that the unique $3$-Sylow subgroup, say $H_3$, is normal. Since
      $\gcd{(3,7)}=1$ and both $H_3$ and $H_7$ are normal, Exercise 6.22 tells us that elements of
      $H_3$ and $H_7$ commute with each other. If we represent the subgroups as \[
        H_3=\{ e,a,a^2 \},H_7=\{ e,b,b^2,\ldots,b^7 \}, ~\text{and}~H_3\cap H_7=\{ e \}~\text{by
        Remark 6.33}~
      ,\] one can verify (following an identical procedure as Problem 2) that elements in $G$ are
      all powers of $a$ times powers of $b$; and since any $a^i$ commutes with any $b^j$ for $0\le
      i\le 2,\ 0\le j\le 6$, $G$ is thus Abelian. Moreover, $\left| ab \right| =21$ (again by
      proceeding analogously as Problem 2, or even Example 6.36; I won't regurgitate here), so $G$
      is cyclic, a contradiction of $G$ non-cyclic.

      Thus we need $k_3=7$; that is, if $G$ is a non-cyclic group of order $21$, then there are $7$
      distinct $3$-Sylow subgroups of $G$.
  \end{enumerate}
\end{solution}

\begin{problem}{\S 4}
  (8.9) Let $F$ be a finite field with $q$ elements, and let $m\mid q-1$.
  \begin{enumerate}[label=(\alph*)]
    \item Prove that $F^*$ has a unique subgroup of order $m$.
    \item Let $\alpha\in F^*$. Prove that the following are equivalent:
      \begin{itemize}
        \item $\alpha$ is an $m$-th power in $F$, i.e. $\alpha=\beta^m$ for some $\beta\in F^*$.
        \item $\alpha^{\frac{q-1}{m}}=1$.
      \end{itemize}
      This is known as \textit{Euler's criterion}.
    \item Suppose that $q$ is odd. Prove that \[
        -1~\text{is a square in}~F^*\iff q\equiv 1\Mod{4}
    .\] 
  \end{enumerate}
\end{problem}
\begin{solution}
  We begin with a lemma about cyclic groups.
  \begin{lemma}{}
    Let $G$ be a cyclic group with order $n$. For any divisor $k$ of $n$ ($k\mid n$), $G$ has a
    unique cyclic subgroup with order $k$.
  \end{lemma}
  \begin{proof}[Proof]
    Let $g\in G$ be a generator of $G$; that is, $\left<g \right> =G$. If $k\in \Z_{>0}$ is a
    divisor of $n$, then $kq=n$ for some $q\in \Z_{>0}$, $1\le q\le n$. Then the element $g^q\in G$
    has order $k$, and so forms a cyclic subgroup $\left<g^q \right> $ of $G$ with order $k$.

    To show uniqueness, suppose $\left<g^s \right> $ is another cyclic group with order $k$ (one can
    easily check that all subgroups of a cyclic group are cyclic). Then for any $g^{is}\in \left<g^s
    \right>$, by Corollary 2.42 we have $(g^{is})^k=g^{kis}=e$; further, $n\mid kis$, so $\alpha
    n=kis$ for some $\alpha\in \Z$. But $n=kq$, so \[
      \alpha n=kis \iff \alpha kq=kis \iff \alpha q=is
    .\] Thus $g^{is}=g^{\alpha q}$ for every $g^{is}\in \left<g^s \right>$; that is, every
    $g^{is}\in \left<g^s \right> $ is some power of $g^{q}$. Thus $\left<g^s \right> \subseteq
    \left<g^q \right> $; equality comes since they have the same order. Hence $\left<g^q \right>$ is
    the unique cyclic subgroup of $G$ with order $k$.
  \end{proof}
  
  \begin{enumerate}[label=(\alph*)]
    \item By Corollary 8.10 (and further Remark 8.12), the unit group $F^*$ is cyclic; from Lemma
      $1$, since $m\mid q-1=\left| F^* \right| $, $F^*$ has a unique (cyclic) subgroup of order $m$,
      as desired.
    \item Suppose $\alpha=\beta^m\in F^*$ and $m\mid q-1$. Then $\left<\alpha \right> =\left<\beta^m
      \right>$ forms a cyclic subgroup of $F^*$ with order $\frac{q-1}{m}$. In particular,
      $\alpha^{\frac{q-1}{m}}=e_{F^*}=1$.

      % For the other direction, $\alpha$ has order $\frac{q-1}{m}$ in the cyclic group $F^*$; thus, 
      % we can choose a generator $\beta\in F^*$ such that $\alpha=\beta^m$, since $\beta^m$ also
      % has order $\frac{q-1}{m}$, so $\left<\alpha \right> =\left<\beta^m \right>$.
      For the other direction, $\left<\alpha \right>$ forms a unique cyclic subgroup of $F^*$ with
      order $\frac{q-1}{m}$. Let $\beta\in F^*$ be a generator; then $\left<\beta^m \right> $ forms
      a cyclic subgroup of order $\frac{q-1}{m}$ as well. Uniqueness of $\left<\alpha \right>$ means
      that $\left<\alpha \right> =\left<\beta^m \right>$, so we can find some $\beta^{km}\in
      \left<\beta^m \right> $ with $\left| \beta^{km} \right|=\frac{q-1}{m} $ such that
      $\beta^{km}=\alpha$. Slightly abusing notation and relabeling $\beta^{k}=\beta$, we have
      $\alpha=\beta^m$ for some $\beta\in F^*$, as desired.

    \item Let $q$ be odd, and suppose that $-1$ is a square in $F^*$; that is, $-1=\beta^2$ for some
      $\beta\in F^*$. From (b), we get that $(-1)^{\frac{q-1}{2}}=1$; but $(-1)^2=1$, so $-1$ has
      order $2$ and $2\mid \frac{q-1}{2}$ by Lagrange. In other words, $2k=\frac{q-1}{2}$, or
      $4k=q-1$, or $q-1\equiv 0\Mod{4}$, or $q\equiv 1\Mod{4}$.
      
      Conversely, suppose $q\equiv 1\Mod{4}$. Then $q-1=4k$, or $\frac{q-1}{2}=2k$, or $2\mid
      \frac{q-1}{2}$.
      % Since $q$ is odd, $q-1$ is even (and thus divisible by 2), so from Lemma 1
      % $F^*$ has a unique cyclic subgroup of order $2$, with generator say $\alpha\in F^*$.
      By elementary properties of rings, $(-1)^2=1$; and since $2\mid \frac{q-1}{2}$,
      $(-1)^{\frac{q-1}{2}}=1$ as well. A direct application of (b) thus leaves us with our desired
      result: $-1$ is a square in $F$, i.e. $-1=\beta^2$ for some $\beta\in F^*$.
  \end{enumerate}
\end{solution}

\begin{problem}{\S 5}
  (8.11)
  \begin{enumerate}[label=(\alph*)]
    \item Let $f(x)=x^{4}-1\in \Q[x]$. Factor $f(x)$ into irreducible factors in $\Q[x]$, and then
      prove that $\Q(\sqrt{-1})$ is the splitting field of $f(x)$ over $\Q$.
    \item Let $f(x)=x^6-1\in \Q[x]$. Factor $f(x)$ into irreducible factors in $\Q[x]$, and then
      prove that $\Q(\sqrt{-3})$ is the splitting field of $f(x)$ over $\Q$.
  \end{enumerate}
\end{problem}
\begin{solution}
  \begin{enumerate}[label=(\alph*)]
    \item For $f(x)=x^4-1\in \Q[x]$, we factor into \[
        f(x)=x^4-1=(x^2+1)(x^2-1)=(x+1)(x-1)(x^2+1)
        ,\] where $x^2+1$ is irreducible in $\Q[x]$. The quadratic equation (or rudimentary
        algebraic experience) tells us we need \[
        \frac{0\pm \sqrt{0-4}}{2}=\frac{2\sqrt{-1}}{2}=\sqrt{-1}
      ,\] which would factor $x^2+1$ into $(x+\sqrt{-1})(x-\sqrt{-1})$. Thus, for any $F$ with
      $\sqrt{-1}\not\in F$, $f(x)$ will not split completely in $F$. Proposition 5.15 tells us that
      $\Q(\sqrt{-1})$ is the smallest extension field of $\Q$ that contains both $\Q$ and $\pm
      \sqrt{-1}$; thus $\Q(\sqrt{-1})$ is the splitting field of $f(x)$ over $\Q$.
    \item For $f(x)=x^6-1\in \Q[x]$, we factor into \[
        f(x)=(x^2-1)(x^4+x^2+1)=(x+1)(x-1)(x^2+x+1)(x^2-x+1)
      ,\] where $x^2\pm x+1$ is irreducible in $\Q[x]$. Using the quadratic formula, we need \[
        \frac{\mp 1\pm \sqrt{1-4}}{2}=\frac{\mp 1\pm \sqrt{-3}}{2}
      \] in order to factor $x^2\pm x+1$. Hence any splitting field must have both $\Q$ and
      $\sqrt{-3}$; Proposition 5.15 tells us that $\Q(\sqrt{-3})$ is the smallest such extension
      field that satisfies this. Thus $\Q(\sqrt{-3})$ is the splitting field of $f(x)$ over $\Q$.
  \end{enumerate}
\end{solution} 

\begin{problem}{\S 6}
  (8.17) Let $K$ be a field with $p^d$ elements, so in particular $K$ contains a copy of $\F_p$.
  \begin{enumerate}[label=(\alph*)]
    \item Prove that there exists an element $\gamma\in K$ so that the evaluation map \[
          E_\gamma:\F_p[x]\longrightarrow K
      \] is surjective.
    \item Prove that $FF_p[x]$ contains an irreducible polynomial of degree $d$. (Hint: take a
      generator for the kernel of the evaluation map in (a)).
  \end{enumerate}
  \hrule\vspace{1ex}
  (8.18) Let $K$ be a field with $p^d$ elements. Prove that the following are equivalent:
  \begin{enumerate}[label=(\alph*)]
    \item $K$ contains a subfield with $p^e$ elements.
    \item $e\mid d$.
  \end{enumerate}
\end{problem}

\begin{solution}
  Apologies, I have not the time to finish these problems :( If time permits, I will re-submit this
  PDF with solutions.
\end{solution}




\end{document}
