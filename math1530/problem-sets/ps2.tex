\documentclass{homework}
\homework{2}

\begin{document}

\begin{problem}{\S 1}
  (2.2) Let $G$ be the group of permutations on $S=\{1,2,\ldots,n\}$. Prove that $G$ is a finite group,
  and give a formula for the order of $G$.\\

  Then, let $P_n$ be a regular $n$-gon with $n$ vertices $1,2,\ldots,n$. Show that the map
  $\phi:\mc{D}_n\to \mc{S}_n$, that sends each element of the dihedral group $\mc{D}_n$ to the
  permutation of the corresponding vertices, is a homomorphism. Is $\phi$ injective? Surjective?
\end{problem}

\begin{solution}
  We first observe that $G$ is a group (by definition). A valid permutation of a set $S=\{
  1,2,\ldots,n \}$ is a bijective function $\pi:S\to S$ that assigns every $s \in S$ to another $s'
  \in S$ (not necessarily distinct). We observe that there are $n$ ways to assign one element (say,
  without loss of generality, $1\in S$): it can be assigned to some $i\in \{1,2,\ldots,n\}$. Then,
  there are $n-1$ ways to assign another element (say, without loss of generality again, $2\in S$);
  it can be assigned to some $j\in \{ 1,2,\ldots,n \}\setminus \{ i \}$. This process repeats until
  the last element (e.g.  $n$), which can only be assigned to one possible $k\in S\setminus
  \underbrace{\{i,\ldots,j\}}_{\text{$n-1$ elements of $S$}}$. In other words, there are \[
    n\cdot (n-1)\cdot \ldots\cdot 1 = n!
  \] possible unique permutations of $S$. Hence $G$ is a finite group of order $n!$.\\

  Now, let $P_n$ be the regular $n$-gon with vertices $N={1,2,\ldots,n}$. Let $\sigma\in \mc{D}_n$,
  and let $\phi(\sigma)=\pi$ map every $\sigma$ to its corresponding permutation $\pi\in \mc{S}_n$;
  that is, for every $i \in N$, we have $\sigma(i)=\pi(i)$. \\
  Let $\sigma_1,\sigma_2\in \mc{D}_n$. Then $\phi(\sigma_1\circ \sigma_2)=\pi$ for some $\pi\in
  \cS_n$ such that $\pi(i)=\sigma_1\circ \sigma_2(i)$ for every $i\in N$. But since every $
  \sigma_j \in \D_n$ corresponds to some $\phi(\sigma_j)=\pi_j\in \cS_n$ where $
  \sigma_j(i)=\pi_j(i),\ \forall i\in N$, we can deconstruct $\pi$ into $\pi_1\circ \pi_2$, where
  $\pi_1=\sigma_1$ and $\pi_2=\sigma_2$. Then \[
    \phi(\sigma_1\circ \sigma_2)=\pi=\pi_1\circ \pi_2=\phi(\sigma_1)\circ \phi(\sigma_2),
  .\] and so $\phi$ is a homomorphism.

  For all $n\in \N$, $\phi:\D_n\to \cS_n$ is  injective, since every unique permutation
  $\sigma$ on vertices $1,\ldots,n$ corresponds to only one unique permutation $\pi$ of $\{
  1,\ldots,n \}$; namely, $\sigma(i)=\pi(i)$ for every $i\in \{ 1,\ldots,n \}$. Formally, suppose
  $\pi_1=\phi(\sigma_1),\ \pi_2=\phi(\sigma_2)\in \cS_n$ and $\pi_1(i)=\pi_2(i),\ \forall i\in \{
  1,\ldots,n \}$. Then $\sigma(i)=\phi(\sigma)(i)=\pi(i)$ for any $\sigma\in \D_n$, so
  $\sigma_1(i)=\sigma_2(i)$, or equivalently, $\sigma_1=\sigma_2$. Hence $\phi$ is injective.\\
  $\phi$ is surjective only for $n\in \{ 1,2,3 \}$. From Exercise 1.16, we know that  $\phi$ 
  injective implies $\phi$ surjective if \[\left| \D_n \right| = \left| \cS_n \right|;\] and for
  $n=\{ 1,2,3 \}$, the above property holds ($\left| \D_n \right| =\left| \cS_n \right|=1,2,6$ for
  $1,2,3$ respectively; for $n=1,2$, the flips and rotations yield the same permutation). However,
  for any $n>3$, $\phi$ is not surjective. There does not exist a  $\sigma\in \D_n$ that fixes two
  vertices and rotates the rest, i.e.: \[
    \sigma(1)=1,\sigma(2)=2,\sigma(i)=i+1~\text{for}~2<i<n,\sigma(n)=3;
  \] hence $\left| \cS_n \right| > \left| \D_n \right|,$ and surjectivity fails.\\
  Thus $\phi$ is bijective for $n\in \{ 1,2,3 \}$, and injective only for all $n>3$.
\end{solution}

\begin{problem}{\S 2}
  (2.6) Let $G$ be a group, and let $g,h\in G$, and suppose $g$ has order $n$, and $h$ has order $m$.
  \begin{enumerate}[label=(\alph*)]
    \item If $G$ is an Abelian group and $\gcd{(m,n)}=1$, prove that the order of $gh$ is $mn$.
    \item Give an example showing (a) need not be true if  $\gcd{(m,n)}>1$.
    \item Give an example of a non-Abelian group showing (a) need not be true even if
      $\gcd{(m,n)}=1$.
  \end{enumerate}
\end{problem}
\begin{solution}
  \begin{enumerate}[label=(\alph*)]
    \item 
      We start with an observation, and a lemma.\\
      \textbf{Observation}: For any $a,b\in G$, $a\cdot b=e$ only when $b=a^{-1}$ or $a=b=e$.
      \begin{lemma}[Order of Inverse]{}
        Let $G$ be a group, and let $g\in G$. Then $\left| g \right| =\left| g^{-1} \right| $.
      \end{lemma}
      \begin{proof}[Proof]
        Let $\left| g \right| =n$; then $g^{n}=e$. From this, we get \[
          e = \left( g\cdot g^{-1} \right)^{n}=g^{n}\cdot (g^{-1})^{n}=e\cdot (g^{-1})^{n}
        ,\] and so $(g^{-1})^{n}=e$ ($g$ and $g^{-1}$ commute, even if $G$ is non-Abelian).

        Now, we show that $\left| g^{-1} \right| =n$. Suppose $\left| g^{-1} \right| =m$ ,and $m<n$.
        Then \[
          e = g^{n}\cdot (g^{-1})^{m}=(g\cdot g^{-1})^{m}\cdot g^{n-m}=g^{n-m}
        .\] But we know that $\left| g \right| =n$, or equivalently, $n$ is the smallest positive
        integer such that $g^{n}=e$; hence $g^{n-m}=e$ is a contradiction. Thus $m=n$, and so $\left|
        g^{-1} \right| =n$.
      \end{proof}
      
      Now, let $G$ be an Abelian group, and let $g,h\in G$, with $\left| g \right| $ and $\left| h
      \right| $ relatively prime, and let $\left| gh \right| =k$. By the lemma, we know that $h\neq
      g^{-1}$ (otherwise, the orders of $g$ and $h$ would not be relatively prime); hence $k\neq 1$. By
      the observation, $(gh)^{k}=g^{k}\cdot h^{k}=e$ only when $g^{k}=e,\ h^{k}=e$.

      Thus, we know that $n$ divides $k$, and $m$ divides $k$ (Proposition 2.9). Thus $k$ is the
      smallest positive integer such that $n|k$ and $m|k$; in other words, $k=\lcm{(m,n)}$. But by
      definition, $\lcm{(m,n)}=\frac{mn}{\gcd{(m,n)}}$, and so $k=\frac{mn}{\gcd{(m,n)}}=mn$.
    \item Consider the group $\Z / 3\Z$ with elements $\{ 0,1,2 \}$ under addition. Consider
      elements $1$ and $2$ ; $\left| 1 \right| =\left| 2 \right|$; so $\gcd{(2,2)}=2$. But
      $(1+2)^{1}=0=e$, so $\left| 1+2 \right| =1\neq 2\cdot 2$.
    \item Consider the group $\D_3$, and consider elements $r_1$ and $ f_2$ ($r_1$ rotates all
      vertices by $1$, and $f_2$ flips across the second vertex, i.e. $f(1)=3,\ f(2)=2,\ f(3)=1$.)
      $\left| r_1 \right| =3,\  \left| f_2 \right|=2 $; so $\gcd{(2,3)}=1$. But $(r_1\circ
      f_2)^2=e$: \[
        (r_1\circ f_2)(1)=1,\ (r_1\circ f_2)(2)=3,\ (r_1\circ f_2)(3)=2
      ;\]  so \[
      (r_1\circ f_2)^2(1)=1,\ (r_1\circ f_2)^2(2)=2,\ (r_1\circ f_2)^2(3)=3
      \] and thus $\left| r_1\circ f_2 \right| =2 \neq 2\cdot 3$.
  \end{enumerate}
\end{solution}



\begin{problem}{\S 3}
  (2.11) Prove that the dihedral group $\D_n$ has exactly $2n$ elements.
\end{problem}

\begin{solution}
  We start with a few notational adjustments. For this problem, we "zero-index" the set of $n$
  numbers $1,2,\ldots,n$; that is, instead of $\{ 1,2,\ldots,n \}$, we write $\{ 0,1,\ldots,n-1 \}$.
  We denote this \[ V_n=\{0,1,\ldots,n-1\} .\]
  Further, we use modular arithmetic: for $a,b\in V_n$, $a\pm b$ becomes $a\pm b\mod{n}$. \\
  Finally, we define $\mc{P}_n$ as the regular $n$-gon with vertices $(0,1,\ldots,n-1)$ [an ordered
  $n$-tuple].\\

  We define the set of all valid permutations on an $n$-gon $P_n$ as the \textbf{$n^{\text{th}}$ 
  dihedral group}, or $\D_n$. Roughly, we get the intuition that any permutation of vertices
  $\sigma\in \D_n$ is valid only if it ``preserves geometric structure;'' for example, given a
  square, rotating the square by $90^{\circ}$ or reflecting it horizontally preserves structure, but
  fixing two vertices and swapping the other two ``breaks'' the structure.

  Formally, we define a permutation $\sigma\in \D_n$ ($ \sigma $ is a valid permutation of $\mc{P}_n$)
  if, for any $i\in V_n$, \[
    \sigma(i)=j ~\text{implies}~\sigma(i\pm 1)=j\pm 1 ~\text{or}~j\mp 1~\text{for some}~j\in V_n
  .\] In other words, the permutation must maintain the adjacent vertices of any vertex, either in
  original or reverse order.

  Now, we define a rotation $r_i$, $i\in \Z_{\ge 0}$, as \[
    r_i(j)=j+i,\ \forall j\in V_n
  .\] So, for a square with vertices $\{ 0,1,2,3 \}$, $ r_1(0)=1,\ r_1(1)=2,\ r_1(2)=3,\ r_1(3)=0$
  (note the modulo). It is clear that $r_i\in \D_n$, as $\forall j\in V_n$, \[
    r_i(j-1)=(j+i)-1,\ r_i(j)=(j+i),\ r_i(j+1)=(j+i)+1
  .\]

  Additionally, we define a flip $f_i$, $i\in \Z_{\ge 0}$, as \[
    f_i(j)=n-j+i,\ \forall j\in V_n
  .\] So, for a square, $ f_0(0)=0 (n\mod{n}\equiv 0),\ f_0(1)=3,\ f_0(2)=2,\ f_0(3)=1$. Similarly,
  it is clear that $f_i\in \D_n$, as $\forall j\in V_n$, \[
    f_i(j-1)=(n-j+i)+1,\ f_i(j)=(n-j+i),\ f_i(j+1)=(n-j+i)-1
  .\] \\ \\
  Now, we make two observations about rotations and flips:
  \begin{enumerate}
    \item 
      For every $i\in V_n$, $r_i$ can be formed by raising $r_1$ to some power: \[
        r_i(j)= j+i=j+\underbrace{1+\ldots+1}_{\text{$i$
        times}}=r_1(j)+\underbrace{1+\ldots+1}_{\text{$i-1$ times}} = \ldots=(\underbrace{r_1\circ \ldots\circ
      r_1}_{\text{$i$ times}})(j) = r_1^{i}(j)
      .\] [for $i=0$, $ r_1^0(j)=r_0(j)=j$].

      Moreover, any $r_k$ for $k\ge n$ is identical to $r_{i}$, where $i=k\mod{n}=k-n\in V_n$: \[
        r_k(j)=j+k\equiv j+k\mod{n}=j+(k-n)=r_j(j)
      .\] Thus, there are \textbf{$n$ unique rotations in $\D_n$}.

    \item
      Similarly, for every $i\in V_n$, $f_i$ can be formed by composing $ f_0$ with some power
      (specifically, $i$) of $r_1$ (that is, $f_i=r_0^i\circ f_0$): \[
        f_i(j) = n-j+i=n-j+\underbrace{1+\ldots+1}_{\text{$i$ times}}=(\underbrace{r_1\circ
        \ldots\circ r_1}_{\text{$i$ times}})(f_0)(j)=r_1^i\circ f_0(j)
      ,\] and like rotations, any $f_k$ for $k\ge n$ is identical to $f_i$, where $i=k\mod{n}=k-n\in
      V_n$: \[
        f_k(j)=n-j+k\equiv n-j+k\mod{n}=n-j+(k-n)=n-j+i=f_i(j)
      .\] Thus, there are \textbf{$n$ unique flips in $\D_n$}.
  \end{enumerate}

  From this, we get that $\D_n$ has at least $2n$ elements: $n$ rotations and $n$ flips. Now, it
  remains to show that \[
    D_n = \{\sigma\mid \sigma=r_1^{i}\circ f_0^{j}, i\in V_n, j\in \{ 0,1 \}\} 
  ;\]  that is, the entire group $ \D_n$ consists of those $2n$ rotations and flips.

  Let $\sigma \in \D_n$. Then for $\sigma(i)=j$, where $i,j\in V_n$, \[
    \sigma(i\pm 1)=j\pm 1, ~\text{or}~\sigma(i\pm 1)=j\mp 1
  .\] If $\sigma(i\pm 1)=j\pm 1$, let $k=j-i\in V_n$ (if $i\ge j$, recall modular arithmetic;
  $k=j-i\mod{n}=n+j-i\in V_n$). Then \[
    r_1^{k}(i\pm 1)=r_k(i\pm 1)=(i\pm 1)+k=(i\pm 1)+j-i=j\pm 1=\sigma(i\pm 1)
  ,\]  and so $\sigma=r_k=r_1^k\circ f_0^{0}$ (no flip).\\
  Alternatively, if $\sigma(i\pm 1)=j\mp 1$, let $k=j+i\in V_n$ (again, if $j+i\ge n$, we have
  $k=j+i-n\in V_n$). Then \[
    r_1^k\circ f_0^{1}(i\pm 1)=r_k\circ f_0(i\pm 1)=r_k(n-(i\pm 1)+0)=r_k(n-i\mp 1)=(n-i\mp
    1)+k=(n-i\mp 1)+j+i=n+j\mp 1\equiv j\mp 1 = \sigma(i\pm 1)
  ,\]  and so $\sigma=r_k\circ f_0=r_1^k\circ f_0^1$.

  Thus, if $\sigma\in \D_n$, then $\sigma$ is either a rotation ($r_k=r_1^k=r_1^k\circ f_0^0$) or a
  flip ($f_k=r_1^k\circ f_0^1$). Hence, the entire group $ \D_n$ consists of only the unique rotations
  and flips, and so $\D_n$ has order $2n$.

\end{solution}

\begin{problem}{\S 4}
  (2.14)
  \begin{enumerate}[label=(\alph*)]
    \item Let \[
        \GL_2(\R) = \left\{\begin{pmatrix} a&b\\c&d \end{pmatrix} \mid a,b,c,d\in \R,ad-bc\neq
        0\right\} 
      ,\]  with composition law being matrix multiplication. Show that $\GL_2(\R)$ is a group.
    \item Let \[
        \SL_2(\R) = \left\{\begin{pmatrix} a&b\\c&d \end{pmatrix} \mid a,b,c,d\in \R,ad-bc=1\right\} 
      ,\]  with composition law being matrix multiplication again. Show that $\SL_2(\R)$ is a group.
    \item Fix an integer $n\ge 1$. Generalize (a) and (b) by proving that each of
      \begin{itemize}
        \item $\GL_n(\R)=\{$set of all $n$-by-$n$ matrices $A$ such that $\det(A)\neq 0\}$ 
        \item $\SL_n(\R)=\{$set of all $n$-by-$n$ matrices $A$ such that $\det(A)=1\}$
      \end{itemize}
      is a group under matrix multiplication.
  \end{enumerate}
\end{problem}

\begin{solution}
  For all parts, recall that $\det(AB)=\det(A)\det(B)$ for any $n$-by-$n$ matrices $A,B$; a useful
  corollary is that $\det(A^{-1})=\frac{1}{\det(A)}$.
  \begin{enumerate}[label=(\alph*)]
    \item 
      \begin{itemize}
        \item Associativity: Let $A,B,C\in \GL_2(\R)$. We know from standard linear algebra that
          matrix multiplication is associative; hence \[
            A(BC)=(AB)C
          .\] 
        \item Identity: Choose $I=\begin{pmatrix} 1&0\\0&1 \end{pmatrix} $. Then for every $A\in
          \GL_2(\R)$, we have $AI=IA=A$.
        \item Inverse:  For every $A\in \GL_2(\R)$, choose $A^{-1}=\frac{1}{\det(A)}\begin{pmatrix}
          d&-b\\-c&a\end{pmatrix}$. Then $AA^{-1}=A^{-1}A=I$. (We know $A^{-1}$ exists since
          $\det(A)\neq 0$ by definition).
      \end{itemize}
      Thus $\GL_2(\R)$ is a group.
    \item
      \begin{itemize}
        \item Associativity: again, we know from standard linear algebra that given any $n$-by-$n$ 
          matrices $A,B,C$, we have \[
            A(BC)=(AB)C
          .\] 
        \item Identity: we again choose $I=\begin{pmatrix} 1&0\\0&1 \end{pmatrix} $. Then for every
          $A\in \SL_2(\R)$, we have $AI=IA=A$.
        \item Inverse: For every $A\in \SL_2(\R)$, choose $A^{-1}=\begin{pmatrix} d&-b\\-c&a
          \end{pmatrix} $ (since $\det(A)=1$, $A$ is invertible and $\frac{1}{\det(A)}=1$). Then
          $AA^{-1}=A^{-1}A=I$.
      \end{itemize}
      Thus $SL_2(\R)$ is a group.

    \item Fix an integer $n\ge 1$. Define $I_n=\begin{pmatrix} 1&0&\cdots&0\\0&1&\cdots&0\\
      \vdots&\vdots&\ddots&\vdots\\0&0&\cdots&1 \end{pmatrix} $; that is, for any $a_{ij}$,
      $a_{ij}=1$ if $i=j$, $0$ otherwise. The proofs that $\GL_n(\R)$ and $\SL_n(\R)$ are groups are
      essentially identical:
      \begin{itemize}
        \item Associativity: from standard linear algebra, given any $n$-by-$n$ matrices $A,B,C$, we
          have \[
            A(BC)=(AB)C
          .\] 
        \item Identity: Choose $I_n$. Then for any $A\in \GL_n(\R)$ and $B\in \SL_n(\R)$, we have
          $AI=IA=A$ and $BI=IB=B$ from standard linear algebra.
        \item Inverse: For any $A\in \GL_n(\R)$ and $B\in \SL_n(\R)$, we know that $\det(A)\neq
          0$, and $\det(B)=1\neq 0$. Thus their inverses exist (from standard linear algebra, a
          matrix is invertible iff its determinant is non-zero), and we choose $A^{-1}\in
          \GL_n(\R),\ B^{-1}\in \SL_n(\R)$. Then $A A^{-1}=A^{-1}A=I_n$, and $BB^{-1}=B^{-1}B=I_n$.
      \end{itemize}
      Thus both $\GL_n(\R)$ and $\SL_n(\R)$ are groups.
  \end{enumerate}
\end{solution}

\begin{problem}{\S 5}
  (2.15) Prove or disprove whether each of the following subsets of $\GL_2(\R)$ is a group (and
  explain why).
  \begin{enumerate}[label=(\alph*)]
    \item $S = \left\{\begin{pmatrix} a&b\\c&d \end{pmatrix} \in \GL_2(\R)\mid ad-bc=2 \right\} $
    \item $S = \left\{\begin{pmatrix} a&b\\c&d \end{pmatrix} \in \GL_2(\R)\mid ad-bc\in \{ -1,1 \}\right\} $
    \item $S = \left\{\begin{pmatrix} a&b\\c&d \end{pmatrix} \in \GL_2(\R)\mid c=0\right\} $
    \item $S = \left\{\begin{pmatrix} a&b\\c&d \end{pmatrix} \in \GL_2(\R)\mid d=0\right\} $
    \item $S = \left\{\begin{pmatrix} a&b\\c&d \end{pmatrix} \in \GL_2(\R)\mid a=d=1,\ c=0\right\} $
  \end{enumerate}
\end{problem}
\begin{solution}
  \begin{enumerate}[label=(\alph*)]
    \item This is not a group.
      \begin{itemize}
        \item Closure does not hold: for $A,B\in S$, we have $\det(AB)=\det(A)\det(B)=2\cdot 2=4$,
          and so $AB\not\in S$.
        \item $\det(I)=1\neq 2$, so $I\not\in S$.
        \item Recall that $\det(A^{-1})=\frac{1}{\det(A)}$; thus given an $A\in S$,
          $\det(A^{-1})=\frac{1}{2}$, so $A^{-1}\not\in S$.
      \end{itemize}
    \item This is a group.
      \begin{itemize}
        \item Let $A,B\in S$. Then $\det(A),\det(B)\in \{ -1,1 \}$. So, $\det(AB)=\det(A)\det(B)\in
          \{ -1,1 \}$ (any combination of $1\cdot 1,\ 1\cdot -1, -1\cdot -1$ is in $\{ -1,1 \}$).
          Thus $AB\in S$.
        \item $det(I)=1\in \{ -1,1 \}$. Thus $I\in S$.
        \item Let $A\in S$. Then $\det(A)\in \{ -1,1 \}$. Recall that
          $\det(A^{-1})=\frac{1}{\det(A)}$; thus $\det(A^{-1})=\frac{1}{1~\text{or}~-1}\in \{ -1,1
          \}$, and so $A^{-1}\in S$.
      \end{itemize}
    \item This is a group.
      \begin{itemize}
        \item Let $A,B\in S$, where $A=\begin{pmatrix} a_1&b_1\\0&d_1 \end{pmatrix},\
          B=\begin{pmatrix} a_2&b_2\\0&d_2 \end{pmatrix}  $. Then \[
          A\cdot B = \begin{pmatrix} a_1&b_1\\0&d_1 \end{pmatrix} \begin{pmatrix} a_2&b_2\\0&d_2
          \end{pmatrix}=\begin{pmatrix} a_1a_2&a_1b_2+b_1d_2\\0&d_1d_2 \end{pmatrix}  \in S
          .\] Thus $AB\in S$.
        \item $I=\begin{pmatrix} 1&0\\0&1 \end{pmatrix} $ is clearly in $S$.
        \item Let $A\in S$, where $A=\begin{pmatrix} a&b\\0&d \end{pmatrix} $. Choose
          $A^{-1}=\frac{1}{ad}\begin{pmatrix} d&-b\\0&a \end{pmatrix} $. $A^{-1}$ is clearly the
          inverse of $A$: \[
          A\cdot A^{-1}=\frac{1}{ad}\begin{pmatrix} a&b\\0&d \end{pmatrix} \begin{pmatrix} d&-b\\0&a
          \end{pmatrix} = \frac{1}{ad}\begin{pmatrix} ad&ab-ab\\0&ad \end{pmatrix} =I
          ;\] Moreover, $A^{-1}\in S$, as $c=0$.
      \end{itemize}
    \item This is not a group.
      \begin{itemize}
        \item Closure does not hold: Let $A,B\in S$, where $A=\begin{pmatrix} a_1&b_1\\c_1&0
            \end{pmatrix} ,\ B=\begin{pmatrix} a_2&b_2\\c_2&0 \end{pmatrix} $. Then \[
              AB = \begin{pmatrix} a_1&b_1\\c_1&0\end{pmatrix}\begin{pmatrix} a_2&b_2\\c_2&0
              \end{pmatrix}=\begin{pmatrix} a_1a_2+b_1c_2&a_1b_2\\a_2c_1+c_1c_2&b_2c_1 \end{pmatrix}
              \not\in S
            .\] 
            \item $I=\begin{pmatrix} 1&0\\0&1 \end{pmatrix} \not\in S$, as $d\neq 0$.
            \item Let $A\in S$, where $A=\begin{pmatrix} a&b\\c&0 \end{pmatrix} $.
              $A^{-1}=\frac{1}{0-bc}\begin{pmatrix} 0&-b\\-c&a \end{pmatrix}$ is not in $S$, as
              $d=a\neq 0$ necessarily.
      \end{itemize}
    \item This is a group.
      \begin{itemize}
        \item Let $A,B\in S$, where $A=\begin{pmatrix} 1&b_1\\0&1 \end{pmatrix},\ B=\begin{pmatrix}
        1&b_2\\0&1 \end{pmatrix} $. Then \[
          A\cdot B= \begin{pmatrix} 1&b_1\\0&1 \end{pmatrix}\begin{pmatrix} 1&b_2\\0&1 \end{pmatrix}
          =\begin{pmatrix} 1&b_1+b_2\\0&1 \end{pmatrix} \in S  
        .\] Thus $AB\in S$.
        \item $I=\begin{pmatrix} 1&0\\0&1 \end{pmatrix} \in S$, as $a=d=1,\ c=0$.
        \item For any  $A\in S$, where $A=\begin{pmatrix} 1&b\\0&1 \end{pmatrix} $, we have
          $A^{-1}=\begin{pmatrix} 1&-b\\0&1 \end{pmatrix} $. Clearly $A\cdot A^{-1}=I$, and
          $A^{-1}\in S$.
      \end{itemize}
  \end{enumerate}
\end{solution}




\end{document}
