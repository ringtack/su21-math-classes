\documentclass{homework}
\homework{2}

\begin{document}

\begin{problem}{\S 1}
  (2.2) Let $G$ be the group of permutations on $S=\{1,2,\ldots,n\}$. Prove that $G$ is a finite group,
  and give a formula for the order of $G$.\\

  Then, let $P_n$ be a regular $n$-gon with $n$ vertices $1,2,\ldots,n$. Show that the map
  $\phi:\mc{D}_n\to \mc{S}_n$, that sends each element of the dihedral group $\mc{D}_n$ to the
  permutation of the corresponding vertices, is a homomorphism. Is $\phi$ injective? Surjective?
\end{problem}

\begin{solution}
  We first observe that $G$ is a group (by definition). A valid permutation of a set $S=\{
  1,2,\ldots,n \}$ is a bijective function $\pi:S\to S$ that assigns every $s \in S$ to another $s'
  \in S$ (not necessarily distinct). We observe that there are $n$ ways to assign one element (say,
  without loss of generality, $1\in S$): it can be assigned to some $i\in \{1,2,\ldots,n\}$. Then,
  there are $n-1$ ways to assign another element (say, without loss of generality again, $2\in S$);
  it can be assigned to some $j\in \{ 1,2,\ldots,n \}\setminus \{ i \}$. This process repeats until
  the last element (e.g.  $n$), which can only be assigned to one possible $k\in S\setminus
  \underbrace{\{i,\ldots,j\}}_{\text{$n-1$ elements of $S$}}$. In other words, there are \[
    n\cdot (n-1)\cdot \ldots\cdot 1 = n!
  \] possible unique permutations of $S$. Hence $G$ is a finite group of order $n!$.\\

  Now, let $P_n$ be the regular $n$-gon with vertices $N={1,2,\ldots,n}$. Let $\sigma\in \mc{D}_n$,
  and let $\phi(\sigma)=\pi$ map every $\sigma$ to its corresponding permutation $\pi\in \mc{S}_n$;
  that is, for every $i \in N$, we have $\sigma(i)=\pi(i)$. \\
  Let $\sigma_1,\sigma_2\in \mc{D}_n$. Then $\phi(\sigma_1\circ \sigma_2)=\pi$ for some $\pi\in
  \cS_n$ such that $\pi(i)=\sigma_1\circ \sigma_2(i)$ for every $i\in N$. But since every $
  \sigma_j \in \D_n$ corresponds to some $\phi(\sigma_j)=\pi_j\in \cS_n$ where $
  \sigma_j(i)=\pi_j(i),\ \forall i\in N$, we can deconstruct $\pi$ into $\pi_1\circ \pi_2$, where
  $\pi_1=\sigma_1$ and $\pi_2=\sigma_2$. Then \[
    \phi(\sigma_1\circ \sigma_2)=\pi=\pi_1\circ \pi_2=\phi(\sigma_1)\circ \phi(\sigma_2),
  .\] and so $\phi$ is a homomorphism.

  For all $n\in \N$, $\phi:\D_n\to \cS_n$ is  injective, since every unique permutation
  $\sigma$ on vertices $1,\ldots,n$ corresponds to only one unique permutation $\pi$ of $\{
  1,\ldots,n \}$; namely, $\sigma(i)=\pi(i)$ for every $i\in \{ 1,\ldots,n \}$. Formally, suppose
  $\pi_1=\phi(\sigma_1),\ \pi_2=\phi(\sigma_2)\in \cS_n$ and $\pi_1(i)=\pi_2(i),\ \forall i\in \{
  1,\ldots,n \}$. Then $\sigma(i)=\phi(\sigma)(i)=\pi(i)$ for any $\sigma\in \D_n$, so
  $\sigma_1(i)=\sigma_2(i)$, or equivalently, $\sigma_1=\sigma_2$. Hence $\phi$ is injective.\\
  $\phi$ is surjective only for $n\in \{ 1,2,3 \}$. From Exercise 1.16, we know that  $\phi$ 
  injective implies $\phi$ surjective if \[\left| \D_n \right| = \left| \cS_n \right|;\] and for
  $n=\{ 1,2,3 \}$, the above property holds ($\left| \D_n \right| =\left| \cS_n \right|=1,2,6$ for
  $1,2,3$ respectively; for $n=1,2$, the flips and rotations yield the same permutation). However,
  for any $n>3$, $\phi$ is not surjective. There does not exist a  $\sigma\in \D_n$ that fixes two
  vertices and rotates the rest, i.e.: \[
    \sigma(1)=1,\sigma(2)=2,\sigma(i)=i+1~\text{for}~2<i<n,\sigma(n)=3;
  \] hence $\left| \cS_n \right| > \left| \D_n \right|,$ and surjectivity fails.\\
  Thus $\phi$ is bijective for $n\in \{ 1,2,3 \}$, and injective only for all $n>3$.
\end{solution}

\begin{problem}{\S 2}
  WIP
\end{problem}
\begin{solution}
  WIP
\end{solution}



\begin{problem}{\S 3}
  (2.11) Prove that the dihedral group $\D_n$ has exactly $2n$ elements.
\end{problem}

\begin{solution}
  We start with a few notational adjustments. For this problem, we "zero-index" the set of $n$
  numbers $1,2,\ldots,n$; that is, instead of $\{ 1,2,\ldots,n \}$, we write $\{ 0,1,\ldots,n-1 \}$.
  We denote this \[ V_n=\{0,1,\ldots,n-1\} .\]
  Further, we use modular arithmetic: for $a,b\in V_n$, $a\pm b$ becomes $a\pm b\mod{n}$. \\
  Finally, we define $\mc{P}_n$ as the regular $n$-gon with vertices $(0,1,\ldots,n-1)$ [an ordered
  $n$-tuple].\\

  We define the set of all valid permutations on an $n$-gon $P_n$ as the \textbf{$n^{\text{th}}$ 
  dihedral group}, or $\D_n$. Roughly, we get the intuition that any permutation of vertices
  $\sigma\in \D_n$ is valid only if it ``preserves geometric structure;'' for example, given a
  square, rotating the square by $90^{\circ}$ or reflecting it horizontally preserves structure, but
  fixing two vertices and swapping the other two ``breaks'' the structure.

  Formally, we define a permutation $\sigma\in \D_n$ ($ \sigma $ is a valid permutation of $\mc{P}_n$)
  if, for any $i\in V_n$, \[
    \sigma(i)=j ~\text{implies}~\sigma(i\pm 1)=j\pm 1 ~\text{or}~j\mp 1~\text{for some}~j\in V_n
  .\] In other words, the permutation must maintain the adjacent vertices of any vertex, either in
  original or reverse order.

  Now, we define a rotation $r_i$, $i\in \Z_{\ge 0}$, as \[
    r_i(j)=j+i,\ \forall j\in V_n
  .\] So, for a square with vertices $\{ 0,1,2,3 \}$, $ r_1(0)=1,\ r_1(1)=2,\ r_1(2)=3,\ r_1(3)=0$
  (note the modulo). It is clear that $r_i\in \D_n$, as $\forall j\in V_n$, \[
    r_i(j-1)=(j+i)-1,\ r_i(j)=(j+i),\ r_i(j+1)=(j+i)+1
  .\]

  Additionally, we define a flip $f_i$, $i\in \Z_{\ge 0}$, as \[
    f_i(j)=n-j+i,\ \forall j\in V_n
  .\] So, for a square, $ f_0(0)=0 (n\mod{n}\equiv 0),\ f_0(1)=3,\ f_0(2)=2,\ f_0(3)=1$. Similarly,
  it is clear that $f_i\in \D_n$, as $\forall j\in V_n$, \[
    f_i(j-1)=(n-j+i)+1,\ f_i(j)=(n-j+i),\ f_i(j+1)=(n-j+i)-1
  .\] \\ \\
  Now, we make two observations about rotations and flips:
  \begin{enumerate}
    \item 
      For every $i\in V_n$, $r_i$ can be formed by raising $r_1$ to some power: \[
        r_i(j)= j+i=j+\underbrace{1+\ldots+1}_{\text{$i$
        times}}=r_1(j)+\underbrace{1+\ldots+1}_{\text{$i-1$ times}} = \ldots=(\underbrace{r_1\circ \ldots\circ
      r_1}_{\text{$i$ times}})(j) = r_1^{i}(j)
      .\] [for $i=0$, $ r_1^0(j)=r_0(j)=j$].

      Moreover, any $r_k$ for $k\ge n$ is identical to $r_{i}$, where $i=k\mod{n}=k-n\in V_n$: \[
        r_k(j)=j+k\equiv j+k\mod{n}=j+(k-n)=r_j(j)
      .\] Thus, there are \textbf{$n$ unique rotations in $\D_n$}.

    \item
      Similarly, for every $i\in V_n$, $f_i$ can be formed by composing $ f_0$ with some power
      (specifically, $i$) of $r_1$ (that is, $f_i=r_0^i\circ f_0$): \[
        f_i(j) = n-j+i=n-j+\underbrace{1+\ldots+1}_{\text{$i$ times}}=(\underbrace{r_1\circ
        \ldots\circ r_1}_{\text{$i$ times}})(f_0)(j)=r_1^i\circ f_0(j)
      ,\] and like rotations, any $f_k$ for $k\ge n$ is identical to $f_i$, where $i=k\mod{n}=k-n\in
      V_n$: \[
        f_k(j)=n-j+k\equiv n-j+k\mod{n}=n-j+(k-n)=n-j+i=f_i(j)
      .\] Thus, there are \textbf{$n$ unique flips in $\D_n$}.
  \end{enumerate}

  From this, we get that $\D_n$ has at least $2n$ elements: $n$ rotations and $n$ flips. Now, it
  remains to show that \[
    D_n = \{\sigma\mid \sigma=r_1^{i}\circ f_0^{j}, i\in V_n, j\in \{ 0,1 \}\} 
  ;\]  that is, the entire group $ \D_n$ consists of those $2n$ rotations and flips.

  Let $\sigma \in \D_n$. Then for $\sigma(i)=j$, where $i,j\in V_n$, \[
    \sigma(i\pm 1)=j\pm 1, ~\text{or}~\sigma(i\pm 1)=j\mp 1
  .\] If $\sigma(i\pm 1)=j\pm 1$, let $k=j-i\in V_n$ (if $i\ge j$, recall modular arithmetic;
  $k=j-i\mod{n}=n+j-i\in V_n$). Then \[
    r_1^{k}(i\pm 1)=r_k(i\pm 1)=(i\pm 1)+k=(i\pm 1)+j-i=j\pm 1=\sigma(i\pm 1)
  ,\]  and so $\sigma=r_k=r_1^k\circ f_0^{0}$ (no flip).\\
  Alternatively, if $\sigma(i\pm 1)=j\mp 1$, let $k=j+i\in V_n$ (again, if $j+i\ge n$, we have
  $k=j+i-n\in V_n$). Then \[
    r_1^k\circ f_0^{1}(i\pm 1)=r_k\circ f_0(i\pm 1)=r_k(n-(i\pm 1)+0)=r_k(n-i\mp 1)=(n-i\mp
    1)+k=(n-i\mp 1)+j+i=n+j\mp 1\equiv j\mp 1 = \sigma(i\pm 1)
  ,\]  and so $\sigma=r_k\circ f_0=r_1^k\circ f_0^1$.

  Thus, if $\sigma\in \D_n$, then $\sigma$ is either a rotation ($r_k=r_1^k=r_1^k\circ f_0^0$) or a
  flip ($f_k=r_1^k\circ f_0^1$). Hence, the entire group $ \D_n$ consists of only the unique rotations
  and flips, and so $\D_n$ has order $2n$.

\end{solution}




\end{document}
