\documentclass{homework}
\homework{Week 6}

\begin{document}

\begin{problem}{\S 1}
  (3.5) For any integer $D\in \Z$ that is not the square of an integer, we can form a ring \[
    \Z[\sqrt[]{D} ]=\{a+b\sqrt[]{D} \mid a,b\in \Z\} 
  .\] If $D>0$, then $\Z[\sqrt[]{D} ]$ is a subring of $\R$, while if $D<0$, then in any case it is
  a subring of $\C$.
  \begin{enumerate}[label=(\alph*)]
    \item Let $ \alpha=2+3\sqrt[]{5} $, $\beta=1-2\sqrt[]{5} $ be elements of $\Z[\sqrt[]{5} ]$.
      Compute the quantitties \[
        \alpha+\beta,\alpha\cdot \beta,\alpha^2
      .\] 
    \item Prove that the map \[
        \phi: \Z[\sqrt[]{D} ] \longrightarrow \Z[\sqrt[]{D} ],\ \phi(a+b\sqrt[]{D} )=a-b\sqrt[]{D} 
      \] is a ring homomorphism (where $\overline{\alpha}$ denotes the conjugate of $ \alpha$).
    \item With notation as in (b), prove that \[
        \alpha\cdot \overline{\alpha}\in \Z ~\text{for every}~ \alpha\in \Z[\sqrt[]{D} ]
    .\] 
  \end{enumerate}
\end{problem}

\begin{solution}
  \begin{enumerate}[label=(\alph*)]
    \item $\alpha=2+3\sqrt[]{5},\ \beta=1-2\sqrt{5},\ \alpha,\beta\in \Z[\sqrt{5} ]$.
      \begin{align*}
        \alpha+\beta&= 3+\sqrt{5}  \\
        \alpha\cdot \beta&= (2+3\sqrt{5} )(1-2\sqrt{5} )=2-4\sqrt{5} +3\sqrt{5} -6\cdot
        5=-28-\sqrt{5}  \\
        \alpha^2&= (2+3\sqrt{5} )^2=4+12\sqrt{5} +45=49+12\sqrt{5} 
      .\end{align*}
    \item $\phi:\Z[\sqrt{5} ]\to \Z[\sqrt{5} ]$, $\phi(\alpha)=\overline{\alpha}$.
      \begin{itemize}
        \item First, observe that $1_{\Z[\sqrt{D} ]}=1+0\sqrt{D} =1$. Then $\phi(1_{\Z[\sqrt{D}
          ]})=1-0\sqrt{D} =1_{\Z[\sqrt{D}]}$. Hence $\phi(1_{\Z[\sqrt{D}]})=1_{\Z[\sqrt{D}]}$.
        \item Let $\alpha=a+b\sqrt{D},\ \beta=c+d\sqrt{D}$. Then $\alpha+\beta=(a+c)+(b+d)\sqrt{D}$,
          so \[
            \phi(\alpha+\beta)=(a+c)-(b+d)\sqrt{D} =a-b\sqrt{D} +c-d\sqrt{D} =\phi(\alpha)+\phi(\beta)
          \] by additive associativity, so $\phi(\alpha+\beta)=\phi(\alpha)+\phi(\beta)$.
        \item Let $\alpha,\beta$ as before. Then $\alpha\cdot \beta=ac-ad\sqrt{D} -bc\sqrt{D}
          +bdD=(ac+bdD)+(ad+bc)\sqrt{D} $, and $\phi(\alpha)=a-b\sqrt{D} ,\
          \phi(\beta)=c-d\sqrt{D} $. Then
          \begin{align*}
            \phi(\alpha\cdot \beta)=ac+bdD-(ad+bc)\sqrt{D} && \phi(\alpha)\cdot
            \phi(\beta)=ac-ad\sqrt{D} -bc\sqrt{D} +bdD=(ac+bdD)-(ad+bc)\sqrt{D} 
          .\end{align*} Hence $\phi(\alpha\cdot \beta)=\phi(\alpha)\cdot \phi(\beta)$, and so
          $\phi$ is a ring homomorphism.
      \end{itemize}

    \item $\alpha=a+b\sqrt{D},\ \overline{\alpha}=\phi(\alpha)\cdot \phi(\beta)$. Then \[
      \alpha\cdot \overline{\alpha}=a-ab \sqrt{D} +ab \sqrt{D} -b^2D=a-b^2D\in \Z
    .\] (All of $a,b,D\in \Z$ and $\Z$ is closed under addition).
  \end{enumerate}
\end{solution}

\begin{problem}{\S 2}
  (3.15) For a quaternion $\alpha=a+bi+cj+dk\in \HH$, we let $\overline{\alpha}=a-bi-cj-dk$.
  \begin{enumerate}[label=(\alph*)]
    \item Prove that $\alpha\overline{\alpha}\in \R$.
    \item Prove that $\alpha\overline{\alpha}=0$ if and only if $\alpha=0$.
    \item Suppose that $\alpha,\beta\in \HH$ and that $ \alpha\beta=0$. Prove that either
      $\alpha=0$ or $\beta=0$.
    \item Let $\alpha,\beta\in \HH$. Prove that \[
      \overline{\alpha+\beta}=\overline{\alpha}+\overline{\beta} ~\text{and}~ \overline{\alpha\cdot
      \beta}=\overline{\beta}\cdot \overline{\alpha}
      .\] 
    \item Let $\alpha\in \HH$ with $ \alpha\neq 0$. Prove that there is a $\beta\in \HH$
      satisfying $\alpha\beta=\beta\alpha=1$, i.e. every non-zero element of $ \HH$ has a
      multiplicative inverse.
  \end{enumerate}
\end{problem}

\begin{solution}
  \begin{enumerate}[label=(\alph*)]
    \item For $\alpha=a+bi+cj+dk\in \HH$, $a,b,c,d\in \R$, we have \[
        \alpha\overline{\alpha}=(a+bi+cj+dk)(a-bi-cj-dk)=a^2+b^2+c^2+d^2\in \R
      .\] 
    \item Suppose $\alpha\overline{\alpha}=0$. Then $a^2+b^2+c^2+d^2=0$. Since $x^2\ge 0$ for all
      $x\in \R$, with $x^2=0$ only when $x=0$, then $a=b=c=d=0$, and so $\alpha=0$.

      Now, suppose $\alpha=0$. Then $a+bi+cj+dk=0$, and so $a=b=c=d=0$ (by definition, if any of
      $a,b,c,d\neq 0$, one can clearly see $\alpha\neq 0$, since $i,j,k$ dont cancel each other
      individually, i.e. $ai+yj=0$ only when $x,y=0$). Hence $
      \alpha\overline{\alpha}=0^2+0^2+0^2+0^2=0$.
    \item Suppose $ \alpha,\beta\in \HH$, with $\alpha=a+bi+cj+dk,\ \beta=w+xi+yj+zk$ and
      $\alpha\cdot \beta=0$. First, we
      note that $\alpha\overline{\alpha}=\overline{\alpha}\alpha=a^2=b^2+c^2+d^2$. Then
      \begin{align*}
        \alpha\beta&= 0 \\
        \overline{\alpha}\alpha\beta&=\overline{\alpha} \cdot 0\\
        (a^2+b^2+c^2+d^2)\beta\overline{\beta}&= 0\cdot\overline{\beta}  \\
        (a^2+b^2+c^2+d^2)(w^2+x^2+y^2+z^2)&=0
      .\end{align*} Since $x^2\ge 0$ for any $x\in \R$, and $x^2=0$ only when $x=0$, we see that
      either $a=b=c=d=0$ or $w=x=y=z=0$.
    \item Let $\alpha,\beta\in \HH$ as before. Then $\alpha+\beta=(a+w)+(b+x)i+(c+y)j+(d+z)k$. We
      have \[
        \overline{\alpha+\beta}=(a+w)-(b+x)i-(c+y)j-(d+z)k
      ,\] and \[
      \overline{\alpha}+\overline{\beta}=a-bi-cj-dk+w-xi-yj-zk=
        \overline{\alpha+\beta}=(a+w)-(b+x)i-(c+y)j-(d+z)k
      .\] Thus $ \overline{\alpha+\beta}=\overline{\alpha}+\overline{\beta}$.

      $\overline{\alpha\cdot \beta}=(aw-bx-cy-dz)-(ax+bw+cz-dy)i-(ay-bz+cw+dx)j-(az+by-cx+dw)k$, and
      $\overline{\beta}\cdot
      \overline{\alpha}=(aw-bx-cy-dz)-(ax+bw+cz-dy)i-(ay-bz+cw+dx)j-(az+by-cx+dw)k$, so
      $\overline{\alpha\cdot \beta}=\overline{\beta}\cdot \overline{\alpha}$.

    \item Note that $\alpha\overline{\alpha}=a^2+b^2+c^2+d^2$. For $\alpha\overline{\alpha}=1$,
      we need $\frac{\alpha\overline{\alpha}}{a^2+b^2+c^2+d^2}$. Thus, let
      $\beta=\frac{\overline{\alpha}}{a^2+b^2+c^2+d^2}\in \HH$ (since $\alpha\neq 0$, not all
      $a,b,c,d$ are zero). Then
      $\alpha\beta=\frac{\alpha\overline{\alpha}}{a^2+b^2+c^2+d^2}=\frac{a^2+b^2+c^2+d^2}{a^2+b^2+c^2+d^2}=1$.
      Hence any non-zero $\alpha$ has a multiplicative inverse.
  \end{enumerate}
\end{solution}

\begin{problem}{\S 3}
  \begin{itemize}
    \item (3.17) Let $R$ be a field. Prove that $R$ is an integral domain.
    \item (3.18) Let $R$ be a ring. Prove that $R$ is an integral domain if and only if $R$ has
      the cancellation property.
  \end{itemize}
\end{problem}

\begin{solution}
  \begin{itemize}
    \item Let $R$ be a field. Then $R$ is a commutative ring, and for any $a\in R$, $a\neq 0$,
      there exists a $b\in R$ such that $ab=1$.

      Let $a\in R,\ a\neq 0$, and suppose for some $b\in R$, $ab=0$. Since $R$ is a field, there
      exists some $c\in R$ such that $ac=ca=1$. Then
      \begin{align*}
        ab&= 0 \\
        cab&= c\cdot 0 \\
        1\cdot b&=0\\
        b&=0
      .\end{align*} Thus for any non-zero $a\in R$, if, for some $b\in R$, $ab=0$, then $b=0$; in
      other words, $R$ has no zero divisors, and thus is an integral domain.

    \item Let $R$ be a ring, and suppose $R$ has the cancellation property; that is, for every
      $a,b,c\in R$, if $ab=ac$ and $a\neq 0$, then $b=c$. Let $a,b\in R$ such that $a\neq 0$ and
      $ab=0$. Thus $ab=0$ implies \[
        ab=a\cdot 0
      ,\] and by the cancellation property, $b=0$. Thus if $a,b\in R$, $a\neq 0$, and $ab=0$,  then
      $b=0$. Hence $R$ has no zero divisors, and so $R$ is an integral domain.

      Conversely, suppose $R$ is an integral domain. Then for any $ab=0$, $a\neq 0$, then $b=0$.
      Let $a,b,c\in R$ with $a\neq 0$, and suppose $ab=ac$. Then \[
        ab-ac=a(b-c)=0
      .\] Since $R$ is an integral domain and $a\neq 0$, $b-c$ must be $0$. Thus $b=c$, and so
      $R$ has the cancellation property.
  \end{itemize}
\end{solution}

\begin{problem}{\S 4}
  (3.23 a-c) Let $R$ be a ring, and $a\in R$. $a$ is \textbf{nilpotent} if $a^{n}=0$ for some $n\ge
  1$. $a$ is \textbf{unipotent} if $a-1$ is nilpotent (e.g. $(a-1)^n=0$ for some $n\ge 1$). $a$ is
  \textbf{idempotent} if $a^2=a$.
  \begin{enumerate}[label=(\alph*)]
    \item If $R$ is an integral domain, describe all nil/uni/idempotent elements of $R$. How many
      are there of each?
    \item Let $p\in \Z$ and let $k\ge 1$. Describe all the nilpotent elements of $\Z / p^k\Z$. In
      particular, how many are there?
    \item Let $a\in R$ be unipotent. Prove that $a$ is a unit, i.e. it has a multiplicative inverse.
  \end{enumerate}
\end{problem}

\begin{solution}
  \begin{enumerate}[label=(\alph*)]
    \item Let a ring $R$ be an integral domain.
      \begin{itemize}
        \item \textbf{Nilpotent elements}: Clearly, $a=0$ is nilpotent, so suppose $a\neq 0$.
      Then $a(a^{n-1})=0$ implies $a^{n-1}=0$ by property of the integral domain. Repeating until
      $a\cdot a=0$, if $a\neq 0$, then $a=0$, a contradiction. Hence if $a$ is nilpotent, $a=0$.

        \item \textbf{Unipotent elements}: Clearly, $a=1$ is unipotent (since $a-1=0$ is nilpotent),
          so suppose $a\neq 1$. Like before, since $R$ is an integral domain, $(a-1)(a-1)^{n-1}=0$
          implies $(a-1)^{n-1}=0$, and repeating this process until $(a-1)(a-1)=0$, we get $a-1=0$,
          or $a=1$, a contradiction. Hence if $a$ is unipotent, then $a=1$.

        \item \textbf{Idempotent elements}: Clearly, $0$ and $1$ are idempotent elements. If
          $a^2=a$, then $a^2-a=a(a-1)=0$. If $a\neq 0$, then $a-1=0$, so $a=1$. If $a-1\neq 0$, then
          $a=0$. Hence $0$ and $1$ are the only idempotent elements.
      \end{itemize}

    \item Suppose $p\in \Z$ is a prime number, and $k\ge 1$. Consider $\Z / p^k\Z$. For any $a\in \Z
      / p^k \Z$, if $a=p^r$ for some $1\le r<k$, then $a^n=(p^r)^n=p^{\alpha k}$ for some
      $n,\alpha\in \Z$ (since for any $r\cdot n$, we can find $ \alpha,\ k$ such that $rn=\alpha
      k$); thus any $p^r$ is nilpotent.

      Now, we make an observation: given $a,p\in \Z$ and $p$ prime, if $p^k$ divides $a$, then $p$ 
      divides $a$. Equivalently, if $a\equiv 0\mod{p^k}$, then $a\equiv 0\mod{p}$ (since $p^k$
      divides $a$, any $p,p^2,p^3,\ldots,p^{k-1}$ divides $a$). Taking its contrapositive, if $p$
      does not divide $a$, then $p^k$ does not divide $a$.

      For any $a\in \Z / p^k\Z$, $a\neq p^r$, $0\le r<k$ (so $a\neq 1,p,p^2,\ldots,p^{k-1}$);
      $a^{p-1}\equiv 1\mod{p}$ by Fermat's Little Theorem. Clearly, $a^s\not\equiv 0\mod{p}$ for any
      $0\le s < p$ (since otherwise we would get $a^{p-1}\equiv a^{s+i}\equiv 0\mod{p}$ for some
      $i\in \Z$); and after $a^{p-1}$, for some $q,r\in \Z$, $0\le r<p-1$, any $a^i\equiv
      a^{q(p-1)+r}\equiv a^r\not\equiv 0\mod{p}$ for any $i\in \N$ (since $0\le r<p-1$, and any
      $a^r\not\equiv 0\mod{p}$ for $0\le 1<p$ from before). Thus $a^m\not\equiv 0\mod{p}$ for any
      $m\ge 1$, and so $a^m\not\equiv 0\mod{p^k}$. Thus $a\in \Z / p^k \Z$ is nilpotent only if
      $a=p^r$ for some $1\le r<k$, and so $\Z / p^k\Z$ has $k-1$ nilpotent elements.

    \item Suppose $a\in R$ is unipotent; then for some $n\in N$, $(a-1)^n=0$. By the Binomial
      Theorem, we have \[
        a^n+\binom{n}{n-1}(-1)a^{n-1}+\ldots+\binom{n}{1}(-1)^{n-1}a+(-1)^n=
        a\left(a^{n-1}+\binom{n}{n-1}(-1)a^{n-1}+\ldots+\binom{n}{1}(-1)^{n-1}\right) + (-1)^n=0
      ,\] so we have \[
      a(a^{n-1}+\binom{n}{n-1}(-1)a^{n-1}+\ldots+\binom{n}{1}(-1)^{n-1}) = (-1)^{n+1}
      .\] Let $b=a^{n-1}+\binom{n}{n-1}(-1)a^{n-1}+\ldots+\binom{n}{1}(-1)^{n-1}$. If $n$ even, then
      $a(-b)=-1$, so $ab=1$; and if $n$ odd, then $ab=1$. In either case, $\pm b\in R$ (by closure
      of ring addition and multiplication), so $a$ is a unit.
  \end{enumerate}
\end{solution}



\begin{problem}{\S 5}
  (3.25)
  \begin{enumerate}[label=(\alph*)]
    \item Compute the unit group $\Z^*$.
    \item Compute the unit group $\Q^*$.
    \item Compute the unit group $\Z[i]^*$.
    \item Consider the ring $\Z[\sqrt{2}]$. Prove that $1+\sqrt{2}\in \Z[\sqrt{2}]^*$. Prove that
      the powers of $1+\sqrt{2}$, or $(1+\sqrt{2})^n$ for $n=1,2,\ldots$ are all different, and use
      the fact to deduce that $\Z[\sqrt{2}]^*$ has infinitely many elements.
    \item Prove that $\R[x]^*=\R^*$.
    \item Prove that $1+2x$ is a unit in the ring $\Z / 4\Z[x]$.
  \end{enumerate}
\end{problem}

\begin{solution}
  \begin{enumerate}[label=(\alph*)]
    \item $\Z^*=\{\pm 1\}$, since $1\cdot 1=-1\cdot -1=1$, and for any $\left| x \right| >1$, $xy\neq
      1$ for any $y\in \Z$ (since $\frac{1}{x}\not\in \Z$ if $\left| x \right| >1$).
    \item $\Q^*=\{a\in \Q \mid a\neq 0\}$, since for any non-zero $a\in \Q$, we can take
      $a\cdot \frac{1}{a}=1$.
    \item Let $ \alpha,\beta\in \Z[i]$, $\alpha=a+bi,\ \beta=c+di$. Then
      $\alpha\beta=(a+bi)(c+di)=(ac-bd)+(ad+bc)i$. In order for $\alpha\beta=1$, we need both
      $ac-bd=1$ and $ad+bc=0$. Isolating for $d=-\frac{bc}{a}$ and plugging in, we get \[
        ac+\frac{b^2c}{a}=\frac{c}{a}(a^2+b^2)=1
      ,\] and so $c=\frac{a}{a^2+b^2}$. Plugging $c$ into $ad+bc=0$, we get \[
        ad+\frac{ab}{a^2+b^2}=0
      ,\] and so $d=-\frac{b}{a^2+b^2}$. Thus
      $\beta=\alpha^{-1}=(\frac{a}{a^2+b^2})-(\frac{b}{a^2+b^2})i$; but since $\frac{a}{a^2+b^2},
      \frac{b}{a^2+b^2}\in \Z$, we must have either $a=\pm 1,b=0$ or $a=0,b=\pm 1$ (since $a^2\ge
      a$, and $k(a^2+b^2)=\left| a \right|$ only when $k=1, a=\pm 1,b=0$). Thus $\Z[i]^*=\{ \pm
      1,\pm i \}$.

    \item Consider $(1+\sqrt{2})(a+b\sqrt{2})=(a+2b)+(a+b)\sqrt{2}$. For $a=-1,b=1$, we get
      $(-1+2)(-1+1)\sqrt{2}=1$; thus $(1+\sqrt{2})(-1+\sqrt{2})=1$, and so $1+\sqrt{2}\in
      \Z[\sqrt{2}]^*$.

      Now, we prove a lemma:
      \begin{lemma}[]{}
        For any $n\in \N$, $(1+\sqrt{2})^n=a+b\sqrt{2}$ for some $a,b\in \N$. Moreover, the sequence
        $s_n=(1+\sqrt{2})^n$ is strictly increasing.
      \end{lemma}
      \begin{proof}[Proof]
        We use induction: clearly, $a=b=1\in \N$, so the base case holds.

        Now, suppose $(1+\sqrt{2})^n=a+b\sqrt{2}$ for some $a,b\in \N$. Then
        $(1+\sqrt{2})^n(1+\sqrt{2})=(a+b\sqrt{2})(1+\sqrt{2})=(a+2b)+(a+b)\sqrt{2}$; and since
        $a,b\in \N$, we have $a+2b,a+b\in \N$ as well. Hence for any $n\in \N$, if
        $(1+\sqrt{2})^n=a+b\sqrt{2}$ for some positive $a,b\in \N$, we have
        $(1+\sqrt{2})^{n+1}=a'+b'\sqrt{2}$, $a,b\in \N$ as well; and since the base case holds, we
        have that $(1+\sqrt{2})^n=a+b\sqrt{2},\ a,b\in \N$ for any $n\in \N$.

        From this, we can also clearly see that $s_n$ is strictly increasing: for any $n\in \N$,
        $(1+\sqrt{2})^n=a+b\sqrt{2}<(a+2b)+(a+b)\sqrt{2}=(1+\sqrt{2})^{n+1}$, so $s_n < s_{n+1}$, as
        required\footnote{Many thanks to MATH1010 and Tamarkin Assistant Professor Huy Quang Nguyen
        for his illuminating insights into sequences.}.
      \end{proof}

      Now, since $(1+\sqrt{2})^n<(1+\sqrt{2})^{n+1}$, it naturally follows that for any $j,k\in \N$,
      $j\neq k$ (supposing without loss of generality that $j<k$), $(1+\sqrt{2})^j\neq
      (1+\sqrt{2})^k$, and so all $(1+\sqrt{2})^n$ are different for $n\in \N$ (in other words,
      there are infinitely many $\alpha\in \Z[\sqrt{2}]$). Moreover, for any $(1+\sqrt{2})^n$, we
      have $(1+\sqrt{2})^n(-1+\sqrt{2})^n=1$, since $(1+\sqrt{2})(-1+\sqrt{2})=1$. Thus any
      $(1+\sqrt{2})^n\in \Z[\sqrt{2}]^*$; and since there are infinitely many $(1+\sqrt{2})^n$,
      there are infinitely many elements in $\Z[\sqrt{2}]^*$.

    \item Clearly, $\R^*\subseteq \R[x]^*$ (since $\R$ is a field, and we can choose
      $a+0x+0x^2+\ldots\in \R[x]^*$ given an $a\in \R^*$). Suppose $\alpha,\ \beta\in \R[x]$ where
      $\alpha=a_0+a_1x+\ldots+a_nx^n,\ \beta=b_0+b_1x+\ldots+b_mx^m$, and $n,m$ respectively are the
      highest powers of $x$ with non-zero coefficients. In order for
      $\alpha\beta=c_0+\ldots+c_{m+n}x^{m+n}=1$, we need $ a_0b_0=1$, and $c_i=0$ for any $0<i\le m+n$.
      However, $a_nb_m\neq 0$ by definition, and so the highest power $c_{m+n}x^{m+n}=a_nb_mx^{m+n}$
      has a non-zero coefficient; thus $a_0b_0+\ldots+a_nb_mx^{m+n}\neq 1$ for any $m+n>0$. It
      naturally follows that we need $m=n=0$, and so $\alpha\beta=1$ only when $\alpha=a_0,\ \beta=b_0$,
      $a_0b_0=a_0\cdot \frac{1}{a_0}=1$. In other words, for any $\alpha\in \R[x]^*$, we have
      $\alpha\in \R^*$; thus $\R[x]^*\subseteq \R^*$, and so $\R^*=\R[x]^*$.
    \item Let $1+2x\in \Z / 4\Z$. Then  \[
      (1+2x)(1+2x)=1+4x+4x^2\equiv 1+0x+0x^2=1
    .\] Hence $1+2x\in (\Z / 4\Z)[x]^*$.
  \end{enumerate}
\end{solution}







\end{document}
