\documentclass[math1530-lecture-notes]{subfiles}
\begin{document}

\chapter{Groups: Part I}

Groups are a fundamental baseline for abstract algebra. We start with motivating examples, then move
on to a concrete definition.

\section{Motivation}
\subsection{Permutations}

\begin{definition}[Permutations]{}
  Let $X$ be a set. A \textbf{permutation} of $X$ is a bijective function \[\pi: X \to X\] with the
    property: $\forall x\in X, !\exists x'\in X$ such that $\pi(x')=x$. This allows us to define an
    inverse $\pi^{-1}$ to be the permutation \[
    \pi^{-1}:X\to X
  \] with the rule that $\pi^{-1}(x)=x'$, where $x'\in X$ is the unique element such that
  $\pi(x')=x$. \\
  The \textbf{identity permutation} of $X$ is the identity map \[
    e:X\to X, e(x)=x \forall x\in X
  .\] 
\end{definition}
In general, a \textit{permutation} of a set $X$ is a rule that ``mixes up'' the elements of $X$.

\begin{example}
  Let $X =\{1, 2, 3, 4 \} $. Then a permutation $\sigma: X \to X$ can be thought of as a shuffling of $X$ and
  visualized as follows:
  \begin{align*}
    1 &\Rightarrow 2 \\
    2 &\Rightarrow 3 \\
    3 &\Rightarrow 1 \\
    4 &\Rightarrow 4
  \end{align*}
  $\sigma^{-1}$ would be defined as
  \begin{align*}
    1 &\Rightarrow 3\\
    2 &\Rightarrow 1\\
    3 &\Rightarrow 2 \\
    4 &\Rightarrow 4 
  \end{align*}
  Now, suppose $\tau$ is defined as $1\Rightarrow 1, 2\Rightarrow 3, 3\Rightarrow 2, 4\Rightarrow
  4$. Then $\sigma\circ \tau$ is
  \begin{align*}
    1 &\Rightarrow 2 \\
    2 &\Rightarrow 1 \\
    3 &\Rightarrow 3 \\
    4 &\Rightarrow 4
  \end{align*}
  and $\tau\circ \sigma$ is 
  \begin{align*}
    1 &\Rightarrow 3 \\
    2 &\Rightarrow 2 \\
    3 &\Rightarrow 1 \\
    4 &\Rightarrow 4 
  \end{align*}
\end{example}

From this, we gather some observations.
\begin{itemize}
  \item Given any 2 permutations, we can compose to get a new one.
  \item There was a permutation that didn't do anything ($\sigma\circ \sigma^{-1}$).
  \item We can invert any permutation.
  \item If $\sigma,\tau$ are two permutations, then we don't necessarily have $\tau\circ
    \sigma=\sigma\circ \tau$ (in other words, the group of permutations with composition is not
    commutative).
\end{itemize}

\begin{definition}[Transformations]{}
  Let $X$ be a figure in $\R^2$. Then $Trafo(X)$ is the set of transformations on $X$.
\end{definition}

Consider the symmetries of a square (involving reflections/rotations on a square) as a motivating
example of transformations; are they invertible? commutative?

\begin{remark}
  Each transformation gives a permutation of the vertices $\{A,B,C,D\} $.
\end{remark}


\section{(Abstract) Groups}
\begin{definition}[Groups]{}
  A \textbf{group} $\{X, \cdot \}$ consists of a set $X$, together with a rule 
  \begin{align*}
    \cdot : G\times G &\to G\\
    (g_1,g_2) &\mapsto g_1\cdot g_2
  \end{align*}
  satisfying the following axioms:
  \begin{enumerate}
    \item (identity) there is an element $e\in G$ such that \[
        e\cdot g=g\cdot e=g
      .\] for all $g\in G$.
    \item (inverse) For all $g\in G$, there is an $h\in G$ such that \[
      g\cdot h=h\cdot g=e
      .\] The element $h$ is called $g^{-1}$, the inverse of $g$.
    \item (associativity) Given $ g_1, g_2, g_3$, we have \[
        g_1(g_2\cdot g_3) = (g_1\cdot g_2)g_3
    .\] \\
    If, in addition, the group satisfies
  \item (commutative) Given $ g_1,g_2\in G$, we have \[
    g_1\cdot g_2=g_2\cdot g_1
  .\] 
  \end{enumerate}
  then $G$ is an \textbf{Abelian} group.
\end{definition}







\end{document}
