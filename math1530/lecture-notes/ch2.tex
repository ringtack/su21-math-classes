\documentclass[math1530-lecture-notes]{subfiles}
\begin{document}

\chapter{Groups: Part I}

Groups are a fundamental baseline for abstract algebra. We start with motivating examples, then move
on to a concrete definition.

\section{Motivation}
\subsection{Permutations}

\begin{definition}[Permutations]{}
  Let $X$ be a set. A \textbf{permutation} of $X$ is a bijective function \[\pi: X \to X\] with the
    property: $\forall x\in X,\ !\exists x'\in X$ such that $\pi(x')=x$. This allows us to define an
    inverse $\pi^{-1}$ to be the permutation \[
    \pi^{-1}:X\to X
  \] with the rule that $\pi^{-1}(x)=x'$, where $x'\in X$ is the unique element such that
  $\pi(x')=x$. \\
  The \textbf{identity permutation} of $X$ is the identity map \[
    e:X\to X, e(x)=x,\ \forall x\in X
  .\] 
\end{definition}
In general, a \textit{permutation} of a set $X$ is a rule that ``mixes up'' the elements of $X$.

\begin{example}
  Let $X =\{1, 2, 3, 4 \} $. Then a permutation $\sigma: X \to X$ can be thought of as a shuffling of $X$ and
  visualized as follows:
  \begin{align*}
    1 &\Rightarrow 2 \\
    2 &\Rightarrow 3 \\
    3 &\Rightarrow 1 \\
    4 &\Rightarrow 4
  \end{align*}
  $\sigma^{-1}$ would be defined as
  \begin{align*}
    1 &\Rightarrow 3\\
    2 &\Rightarrow 1\\
    3 &\Rightarrow 2 \\
    4 &\Rightarrow 4 
  \end{align*}
  Now, suppose $\tau$ is defined as $1\Rightarrow 1, 2\Rightarrow 3, 3\Rightarrow 2, 4\Rightarrow
  4$. Then $\sigma\circ \tau$ is
  \begin{align*}
    1 &\Rightarrow 2 \\
    2 &\Rightarrow 1 \\
    3 &\Rightarrow 3 \\
    4 &\Rightarrow 4
  \end{align*}
  and $\tau\circ \sigma$ is 
  \begin{align*}
    1 &\Rightarrow 3 \\
    2 &\Rightarrow 2 \\
    3 &\Rightarrow 1 \\
    4 &\Rightarrow 4 
  \end{align*}
\end{example}

From this, we gather some observations.
\begin{itemize}
  \item Given any 2 permutations, we can compose to get a new one.
  \item There was a permutation that didn't do anything ($\sigma\circ \sigma^{-1}$).
  \item We can invert any permutation.
  \item If $\sigma,\tau$ are two permutations, then we don't necessarily have $\tau\circ
    \sigma=\sigma\circ \tau$ (in other words, the group of permutations with composition is not
    commutative).
\end{itemize}

\begin{definition}[Transformations]{}
  Let $X$ be a figure in $\R^2$. Then $Trafo(X)$ is the set of transformations on $X$.
\end{definition}

Consider the symmetries of a square (involving reflections/rotations on a square) as a motivating
example of transformations; are they invertible? commutative?

\begin{remark}
  Each transformation gives a permutation of the vertices $\{A,B,C,D\} $.
\end{remark}


\section{(Abstract) Groups}
We now formally define the notion of a \textbf{group}.
\begin{definition}[Groups]{}
  A \textbf{group} $\{X, \cdot \}$ consists of a set $X$, together with a group rule/law
  \begin{align*}
    \cdot : G\times G &\to G\\
    (g_1,g_2) &\mapsto g_1\cdot g_2
  \end{align*}
  satisfying the following axioms:
  \begin{enumerate}
    \item (identity) there is an element $e\in G$ such that \[
        e\cdot g=g\cdot e=g
      .\] for all $g\in G$.
    \item (inverse) For all $g\in G$, there is an $h\in G$ such that \[
      g\cdot h=h\cdot g=e
      .\] The element $h$ is called $g^{-1}$, the inverse of $g$.
    \item (associativity) Given $ g_1, g_2, g_3$, we have \[
        g_1(g_2\cdot g_3) = (g_1\cdot g_2)g_3
    .\] \\
    If, in addition, the group satisfies
  \item (commutative) Given $ g_1,g_2\in G$, we have \[
    g_1\cdot g_2=g_2\cdot g_1
  .\] 
  \end{enumerate}
  then $G$ is an \textbf{Abelian} group.
\end{definition}

Now, we observe some interesting properties that follow from the group axioms.
\begin{proposition}[Group Properties]{}
  Let $G$ be a group.
  \begin{enumerate}
    \item The identity element is unique.
    \item Each element of $G$ has only one inverse.
    \item If $g,h\in G$, then $(gh)^{-1}=h^{-1}g^{-1}$.
    \item Given $g\in G$, $(g^{-1})^{-1}=g$.
  \end{enumerate}
\end{proposition}
\begin{proof}[Proof of (b)]
  Suppose $g\in G$ and that both $h_1,h_2$ satisfy the inverse axiom. Then \[
  g\cdot h_1=e=g\cdot h_2
  .\]  By the inverse axiom, we multiply on the left by an inverse of $g $:
  \begin{align*}
    e\cdot h_1&= e\cdot h_2 \\
    h_1&=h_2
  .\end{align*} Thus the inverse is unique.
\end{proof}

\begin{definition}[Order]{}
  \begin{itemize}
    \item The \textbf{order} of a group $G$ is denoted  $\#G$ or  $\left| G \right| $ is the number of
          elements in $G$ if finite, and $\infty$ if infinite.
    \item If $G$ is a group and $g\in G$, the smallest $n$ in which $g^{n}=e$ is called \textbf{the
      order of $g$}. If no $n$ exists, we say $g$ has infinite order.
  \end{itemize}
\end{definition}

\begin{proposition}[Individual Order and Group Order]{}
  Suppose $G$ is a finite group and suppose $g^{n}=e$. Then the order of $g$ divides $n$.
\end{proposition}
\begin{proof}[Proof]
  Let $m$ be the order of $g\in G$; then $m$ is the smallest positive integer such that $g^{m}=e$.
  Dividing $n$ by $m$ yields \[
    n = mq+r,\ q,r\in \Z, 0\le r<m
  .\] In other words, dividing  $n$ by $m$ leaves a quotient $q$ and a remainder $r$.\\
  Using this equality together with $g^{n}=g^{m}=e$, we have \[
    e=g^{n}=g^{mq+r}=(g^{m})^{q}\cdot g^{r}=e^{q}\cdot g^{r}=g^{r}
  .\] Hence $g^{r}=e$, and $r\in [0,m)$. But by definition, $m$ is the smallest integer such that
  $g^{m}=e$. Therefore $r=0$, and $n=mq$, and so $m$, the order of $g$, divides $n$.
\end{proof}

\begin{proposition}[Order of Inverse]{}
  Let $G$ be a finite group, and $g\in G$. Then $\left| g \right| =\left| g^{-1} \right| $.
\end{proposition}
\begin{proof}[Proof]
  Let $\left| g \right| =n$; then $g^{n}=e$. From this, we get \[
    e = (g\cdot g^{-1})^{n}=g^{n}\cdot (g^{-1})^{n}=e\cdot (g^{-1})^{n}
  ,\] and so $(g^{-1})^{n}=e$. \\
  Now we show that $\left| g^{-1} \right| =n$. Suppose $\left| g^{-1} \right| =m$, and $m<n$.  Then
  \[
    e=g^{n}\cdot (g^{-1})^{m}=g^{n-m}
  .\] But we know that  $\left| g \right| =n$, or equivalently, $n$ is the smallest positive integer
  such that $g^{n}=e$; hence $g^{n-m}=e$ is a contradiction. Thus $m=n$, and so  $\left| g^{-1}
  \right| =n$.
\end{proof}



\subsection{Examples of Groups}
\begin{example}
  $\Z,\Q,\R$ and $\C$ are all Abelian groups with respect to addition. However, $\Z$ is not a group
  with respect to multiplication, as the multiplicative inverse does not exist. Additionally,
  $\Q,\R,$ and $\C$ are not groups with respect to multiplication, due to zero; but $\Q\setminus \{
  0\}, \R\setminus \{ 0 \}$, and $ \C\setminus \{ 0 \}$ are all groups under multiplication.
\end{example}

\begin{example}
  Let $\Z/m\Z$ be the set of integers modulo $m$. Then  $\Z / m\Z$ is a group under addition
  modulo $m$, $+_m$ ; $\Z / m\Z$ is finite with order $m$. We also observe that $\Z/m\Z$ is a cyclic
  group.
\end{example}

\begin{example}
  Let the set of $n\times n$ matrices be $M_n$. Then  $M_n$ is an Abelian group under addition, but
  not multiplication (since not all matrices have inverses).\\

  Let \[GL_n(\R)=\{M\in M_n\mid \det(M)\neq 0\}\] denote the \textbf{general linear group}. Then
  $GL_n(\R)$ is a non-Abelian group under matrix multiplication.
\end{example}



\subsection{Cyclic Groups}
\begin{definition}[Cyclic Groups]{}
  A group $G$ is \textbf{cyclic} if there is a $g\in G$ such that \[
    G=\{ \ldots,g^{-2},g^{-1} ,e\ (\text{or}~g^{0}),g,g^2,g^3,\ldots\}
  .\] We call $g$ a \textbf{generator}.\\
  \indent In general, for $n\ge 1$, the \textbf{abstract cyclic group order} $n$ is the set \[
    \mc{C}_n = \{ g_0,g_1,\ldots,g_{n-1} \}
  \] together with the composition rule \[
    g_i\cdot g_j = \left\{\begin{array}{lr} g_{i+j}, & i+j<n \\ g_{i+j-n}, & i+j\ge n\end{array}\right.
  \] The identity element of $\mc{C}_n$ is $g_0$, and the inverse of $g_i$ is $g_{n-i}$ (except
  $g_0$, whose inverse is $g_0$). Further, $\mc{C}_n$ is an Abelian group, as $g_{i+j}=g_{j+i}$.
\end{definition}

Some examples of cyclic groups are $\Z$ and $\Z / m\Z$; both have generators $1$. Another one is
the permutation group.
\begin{definition}[Permutation Groups]{}
  Given $X$ a set, let $S_X$ denote the  \textbf{symmetric group of $X$}, or the group of
  permutations of $X$. If  \[
    X=\{ 1,\ldots,n \}
  ,\] we use the notation $S_n$.

  Let $P_n$ be a regular  $n$-gon with vertices $1,\ldots,n$. The group of transformations of $D_n$
  (e.g. rotations, reflections, and compositions of such) is called the  \textbf{dihedral group}
  $D_n$. We will later prove that $D_n$ has order  $2n$.
\end{definition}


\section{Group Homomorphisms}

Suppose that $G,G'$ are groups, and suppose that $\phi$ is a function \[
  \phi:G \longrightarrow G'
\]  from elements of $G$ to elements of $G'$. Many functions exist, but we're interested in the ones
that preserve the "structure", or \textit{group-i-ness}, of $G$ and $G'$. But what makes a group a
group? Specifically, groups are \textbf{associative}, and have \textbf{identity and inverse
elements}. Thus, a function $\phi$ must preserve these qualities. We call such structure-preserving
functions \textbf{homomorphisms}.

\begin{definition}[Homomorphisms]{}
  Let $ G_1, G_2$ be groups. A \textbf{homomorphism} from $G_1$ to $G_2$ is a function \[
    \phi: G_1\to G_2\]
    satisfying: \[
    \phi(g_1\cdot g_2)=\phi(g_1)\cdot\phi(g_2)
  .\] In other words, the map $\phi$ preserves the group operations. (Note that the composition law
  is different on the left and right sides! The left is the composition law of $G_1$, while the
  right is the composition law of $ G_2$.)
\end{definition}

It turns out this property is enough to force the identities and inverses to exist under a function.

\begin{proposition}[]{}
  Let $\phi:G\to G'$ be a homomorphism of groups.
  \begin{enumerate}
    \item Let $e\in G$ be the identity element of $G$. Then $\phi(e)\in G'$ is the identity element
      of $G'$.
    \item Let $g\in G$, and let $g^{-1}\in G$ be its inverse. Then $ \phi(g^{-1})\in G$ is the
      inverse of $\phi(g)$.
  \end{enumerate}
\end{proposition}
\begin{proof}[Proof]
  \begin{enumerate}
    \item Observe that $e=e\cdot e$, and that $\phi$ is a homomorphism (and so $\phi(e)=\phi(e\cdot
      e)=\phi(e)\cdot \phi(e)$). Let $e'\in G'$ be the identity element of $G'$. Then
      \begin{align*}
        e' &= \phi(e)\cdot \phi(e)^{-1}\\
           &= (\phi(e)\cdot \phi(e))\cdot \phi(e)^{-1} \\
           &= \phi(e)\cdot (\phi(e)\cdot \phi(e)^{-1}) \\
           &= \phi(e)\cdot e' \\
           &= \phi(e)
      .\end{align*}
      Hence $e'=\phi(e)$.
    \item We have 
      \begin{align*}
        \phi(g^{-1})\cdot \phi(g) &= \phi(g^{-1}\cdot \phi(g)) \\
                                  &= \phi(e) \\
                                  &= e'
      .\end{align*}
      The proof that $\phi(g)\cdot \phi(g^{-1})=e'$ is similar. Hence $\phi(g^{-1})$ is the inverse of $\phi(g)$.
  \end{enumerate}
\end{proof}


\begin{example}
  Examples of homomorphisms:
  \begin{itemize}
    \item There exists a homomorphism from the dihedral group to the group $\pm 1$:  \[
        \phi: D_n \to \{ \pm 1 \}
      \], where $\phi(\sigma) = 1$ if rotation, $ \phi(\sigma)=-1$ if flip.
    \item For $n\ge m\ge 1$, there is an injective homomorphism \[
        f:S_m \to S_n
      .\] Note that this homomorphism is \textit{not} surjective. More generally, if $X_1\subseteq
      X_2$, then there is an injective homomorphism $f:S_{X_1} \to S_{X_2}$.
    \item There is a homomorphism \[
          \log: \left( \R,\times  \right) \to \left( \R,+ \right) 
      .\] 
    \item There is a homomorphism between the general linear group to the real numbers \begin{align*}
        det: GL_n(\R) &\longrightarrow \R \\
        AB &\longmapsto det(AB) = det(A)\cdot det(B)
    .\end{align*}
  \end{itemize}
\end{example}

\begin{definition}[Isomorphisms]{}
  Groups $  G_1,G_2$ are \textbf{isomorphic} if there exists a \textbf{bijective homomorphism} $f:
  G_1\to G_2$. In this case, $f$ is called an \textbf{isomorphism}.
\end{definition}
Interestingly, isomorphic groups are really the same group, but their elements are given different
names.


We've now seen two examples of cyclic groups of order $n $: $\Z/n\Z$ and $\mathcal{C}_n$. Naturally,
we wonder if these groups are actually different (from the perspective of group theory).
Equivalently,  \textbf{are these two groups isomorphic}?
\begin{example}
  $\Z / n\Z$ is isomorphic to $\mathcal{C}_n$ ($ \Z / n\Z\cong \mathcal{C}_n$). Consider the map
  \begin{align*}
    \phi: \Z / n\Z &\longrightarrow \mc{C}_n \\
    a &\longmapsto \phi(a) = g_a
  .\end{align*} Then $\phi(a+b)=\phi(a)\cdot \phi(b)$ by definition of group operations. So $\phi$ 
  is a homomorphism.\\
  $\phi$ is surjective since $i\in \{ 0,\ldots,n-1 \}$ maps to $g_i\in \{ g_0,\ldots,g_{n-1} \}$.
  \\Since $\Z / n\Z $ and $\mc{C}_n$ both have  $n$ elements, $\phi$ is injective as well. \\
  So, $\phi$ is an isomorphism and $\mc{C}_n \simeq \Z / n\Z$.
\end{example}

Note that if a group is isomorphic, there isn't necessarily a unique isomorphism. Consider the same
isomorphism as above, except map $a\mapsto g_{a+1}$. This is also an isomorphism.

\begin{example}
  Given any group $G$, and an element $g\in G$, then multiplication by $g$ permutes the elements of
  $G$. This gives rise to an injective homomorphism $\phi: G \to S_G$.\\

  This implies that by knowing every symmetric group, one knows much about every other group.
\end{example}

\section{Subgroups, Cosets, and Lagrange's Theorem}
In all mathematics, a three-step process exists for studying complicated objects. 
\begin{enumerate}
  \item Deconstruction: Break your object into smaller and simpler pieces.
  \item Analysis: Analyze the smaller, simpler pieces.
  \item Fit the pieces back together.
\end{enumerate}

For a group $G$, a natural way to form a smaller and simpler piece is by taking subsets 
$H\subseteq G$ that are themselves groups.

\begin{definition}[Subgroups]{}
  Let $G$ be a group. A \textbf{subgroup of $G$} is a subset $H\subset G$ that is itself a group
  under $G$'s group law. Explicitly, $H$ needs to satisfy
  \begin{enumerate}
    \item (Closure Under Composition) For every $h_1,h_2\in H$, $ h_1\cdot h2\in H$
    \item The identity element $e$ is in $H$.
    \item For every $h\in H$, its inverse $h^{-1}$ is in $H$.
  \end{enumerate}
  This is sometimes denoted $H < G$. \\

  Note that since $H$ uses $G$'s composition law, associativity is automatically satisfied. If $H$ 
  is finite, the \textbf{order}  of $H$ is the number of elements in $H$.
\end{definition}

\begin{proposition}[Easier Subgroup Checking]{}
  Let $G$ be a group, and $H\subseteq G$ a subset. If 
  \begin{itemize}
    \item $H\neq \varnothing$ 
    \item For every $h_1,h_2\in H$, the element $ h_1h_2^{-1}$ is in $H$
  \end{itemize}
  then $H$ is a subgroup of $G$.
\end{proposition}
\begin{proof}[Proof]
  Clearly, $H\neq \varnothing$ (otherwise the identity would not be in $H$). \\
  To show that $e\in H$, let $h_2=h_1$. Then \[
    h_1\cdot h_2^{-1}=h_1\cdot h_1^{-1}=e\in H
  .\] Thus the identity is in $H$.\\
  To show that $\forall h\in H,\ h^{-1}\in H$, let $h_1=e$. Then \[
    h_1\cdot h_2^{-1}=e\cdot h_2^{-1}=h_2^{-1}\in H
  .\] Thus for any $h\in H$, its inverse $h^{-1}$ is in H.\\
  To show closure, observe that for any $h\in H$, $h^{-1}\in H$ (from above), and that
  $(h^{-1})^{-1}=h$. Let $h_2=h^{-1}$. Then \[
    h_1\cdot h_2^{-1}=h_1\cdot (h^{-1})^{-1}=h_1\cdot h\in H
  .\] Thus for any $h1,h\in H$, we have $ h_1\cdot h\in H$. Thus $H$ is closed.\\

  Hence $H$ is a subgroup of $G$.
\end{proof}

\begin{example}
  Every group $G$ has at least two subgroups, the \textbf{trivial subgroup} $\{ e \}$ consisting of
  only the identity element, and the entire group $G$.
\end{example}
\begin{example}
  Let $G$ be a group, and let $g\in G$ be an element of order $n$. The \textbf{cyclic subgroup of
  $G$ generated by $g$}, denoted $\left< g \right>$, is the set \[
    \left<g \right> = \{ \ldots,g^{-2},g^{-1},e,g^{1},g^2,g^3,\ldots \}
  .\] 
  It is isomorphic to the cyclic group $\mc{C}_n$.\\
  
  If $g$ has infinite order, then $\left< g\right>\cong \Z$ ($\left<g \right>$ is isomorphic to
  $\Z$).
\end{example}

\begin{example}
  More examples of subgroups:
  \begin{itemize}
    \item Let $d\in \Z$; then we can form a subgroup of $\Z$ using multiples of $d$, or $d\Z$.
    \item The set of rotations in the dihedral group $\mc{D}_n$ is a subgroup of $\mc{D}_n$.
  \end{itemize}
\end{example}

Every group homomorphism has an associated subgroup, the \textbf{kernel}, which can be a convenient
check to see if the homomorphism is injective.
\begin{definition}[Kernel]{}
  Let $\phi:G\to G'$ be a group homomorphism. The \textbf{kernel of $\phi$}, denoted
  $\ker\left(\phi\right)$, is the set of elements of $G$ that are sent to the identity element of
  $G'$, \[
    \ker\left(\phi\right)=\{g\in G\mid \phi(g)=e'\} 
  .\] 
\end{definition}

\begin{example}
  The kernel of the determinant homomorphism \[
    \det:\GL_n(\R)\longrightarrow \R\setminus \{ 0 \}
  .\] 
   is \[
    \ker(\det) = \{A\in \GL_n(\R) \mid \det(A)=1\} 
  .\] 
\end{example}

We now observe two important properties of the kernel.
\begin{proposition}[Kernel Properties]{}
  Let $\phi: G \to G'$ be a group homomorphism.
  \begin{enumerate}
    \item $\ker(\phi)$ is a subgroup of $G$.
    \item $\phi$ is injective if and only if $\ker(\phi)=\{ e \}$.
  \end{enumerate}
\end{proposition}
\begin{proof}[Proof]
  We know that $\phi(e)=e'$, so $e\in \ker(\phi)$. Next, let $ g_1,g2\in \ker(\phi)$. By the
  homomorphism property, \[
    \phi(g_1\cdot g_2) = \phi(g_1) \cdot  \phi(g_2) = e'\cdot e'\in G'
  ,\] so $g_1,g_2\in \ker(\phi)$. Finally, for $g\in \ker(\phi)$, we know
  $\phi(g^{-1})=\phi(g)^{-1}=e'^{-1}=e$, so $g^{-1}\in \ker(\phi)$ Thus, $\ker(\phi)$ is a subgroup
  of $G$.\\

  Now, we know again that $e\in \ker(\phi)$ (since $\phi(e)=e'$). If $\phi$ is injective, by
  definition $\ker(\phi)=\{ e \}$ (at most one element $g\in G$ satisfies $\phi(g)=e'$).\\
  Now, suppose $\ker(\phi)=\{ e \}$. Let $\phi(g_1)=\phi(g_2)$ for some $g_1,g_2\in G$. Observe that
  $ g_2^{-1}\in G$, and $\phi(g_2^{-1})=\phi(g_2)^{-1}$. Then \[
    \phi(g_1)=\phi(g_2) \implies \phi(g_1)\cdot \phi(g_2)^{-1}=\phi(g_2)\cdot \phi(g_2)^{-1}=e'
  ,\] and so $\phi(g_1)\cdot \phi(g_2)^{-1}=\phi(g_1\cdot g_2^{-1})=e'$, which means $g_1\cdot
  g_2^{-1}\in \ker(\phi)=\{ e \}$. Hence $ g_1\cdot g_2^{-1}=e\implies g_1=g_2$, and so $\phi$ is
  injective.
\end{proof}

\subsection{Cosets}
We can use a subgroup $H$ of a group $G$ to break $G$ into pieces, called \textbf{cosets of $H$}.
\begin{definition}[Cosets]{}
  Let $G$ be a group, and let $H < G$ be a subgroup. For each $g\in G$, the (left) \textbf{coset of
  $H$ attached to $g$} is the set \[
    gH = \{gh\mid h\in H\} 
  .\] In other words, $gH$ is the resulting set we multiply $g$ by every element $h\in H$.\\

  Note that $gH$ is \textbf{not} necessarily a subgroup of $H $; sometimes $e\not\in gH$.
\end{definition}

We now prove several properties of cosets that help explain their importance.
\begin{proposition}[Properties of Cosets]{}
  Let $G$ be a finite group, and let $H<G$.
  \begin{enumerate}
    \item Every element in $G$ is in some coset of $H$.
    \item Every coset of $H$ has the same number of elements (namely, $\left| H \right| $).
    \item Let $g_1,g_2\in G$. Then the cosets $g_1H$ and $ g_2H$ satisfy either \[
      g_1H=g_2H ~\text{or}~g_1H\cap g_2H=\varnothing
    .\] In other words, $g_1H$ and $g_2H$ are either equal or disjoint.
  \end{enumerate}
\end{proposition}
\begin{proof}[Proof]
  \begin{enumerate}
    \item Let $g\in G$. Since $e\in H$ for any subgroup $H<G$, the coset $gH$ contains $g\cdot e=g$.
    \item Let $g\in G$. To prove that the cosets $gH$ and $H$ have the same number of elements, we
      show the map \begin{align*}
        F: H &\longrightarrow gH \\
        h &\longmapsto F(h) = gh
      \end{align*}
      is a bijective map from $H$ to $gH$.\\
      We first check that $F$ is injective. Suppose $h_1,h_2\in H$ satisfy $F(h_1)=F(h_2)$. Then
      $gh_1=gh_2$, and multiplying by $g^{-1}$, we get $h_1=h_2$. Hence $F$ is injective.\\
      For surjectivity, observe that every element of $gH$ looks like $gh$ for some $h\in H$, and
      $F(h)=gh$, so every element of $gH$ is the image of some element of $H$. Hence $F$ is
      surjective.\\
      Thus $F$ is bijective, so $H$ and $gH$ have the same number of elements. Since this is true
      for any $ginnG$, every coset of $H$ has the same number of elements.
    \item If $g_1H\cap g_2H=\varnothing$, we are done, so assume the two cosets are not disjoint.
      Then there are some elements $h_1,h_2\in H$ such that $g_1h_1=g_2h_2$. Since $h_1^{-1}\in H$,
      we rewrite this as $ g_1=g_2h_2h_1^{-1}$. Now, take any element $a\in g_1H$. $a$ is of the
      form $g_1h$ for some $h\in H$. Then \[
        a = g_1h = g_2h_2h_1^{-1}h \in g_2H
      ,\] as $H$ is a subgroup, so $h_2h_1^{-1}h\in H$. Hence $g_1H\subseteq g_2H$; and from above,
      every coset has the same number of elements, so $ g_1H\subseteq g_2H \implies g_1H=g_2H$.
  \end{enumerate}
\end{proof}

These properties lead to a fundamental divisibility property for the orders of subgroups.
\begin{theorem}[Lagranges Theorem]{}
  Let $G$ be a finite group, and let $H<G$. Then the order of $H$ divides the order of $G$; or,
  $\left| G \right| = k \left| H \right|,\ k\in \Z $.
\end{theorem}
\begin{proof}[Proof]
  We start by choosing $g_1,\ldots,g_k\in G$ so that $g_1H,\ldots,g_kH$ is a list of every different
  coset of $H$. Since every element of $G$ is in some coset of $H$, we have that $G$ is equal to the
  union of the cosets of $H$, namely \[
    G = g_1H \cup \ldots \cup g_kH
  .\] Additionally, we know that distinct cosets share no elements, so if $i\neq j$, then $g_iH\cap
  g_jH=\varnothing$. Thus the union of cosets is a disjoint union, so the number of elements in $G$ 
  is the sum of the number of elements in each coset: \[
    \left| G \right| = \left| g_1H \right| +\ldots+\left| g_kH \right| 
  .\] But we know that every coset of $H$ has the same number of elements, so $\left| g_iH \right|
  =\left| H \right| $. Thus, we get \[
  \left| G \right| = k\left| H \right| 
  .\] Thus the order of $G$ is a multiple of the order of $H$.
\end{proof}
\begin{definition}[Index]{}
  Let $G$ be a group, and $H<G$. The \textbf{index of $H$ in $G$}, denoted $(G : H)$, is the number
  of distinct cosets of $H$. In Lagrange's Theorem, the index $(G : H) = k$; so \[
    \left| G \right| = (G : H) \left| H \right| 
  .\] 
\end{definition}

\begin{corollary}[Extension of Lagrange's Theorem to Finite Groups]{}
  Let $G$ be a finite group, and let $g\in G$. Then the order of $g$ divides the order of $G$.
\end{corollary}
\begin{proof}[Proof]
  The order of the subgroup $\left<g \right>$ generated by $G$ is equal to the order of the element
  $g$, and Lagrange's Theorem tells us that the order of $\left<g \right>$ divides the order of $G$.
\end{proof}

We now give one application of Lagrange's Theorem, which marks the beginning of a long and ongoing
mathematical journey that strives to classify finite groups according to their orders.
\begin{proposition}[Prime-Ordered Groups]{}
  Let $p$ be a prime, and let $G$ be a finite group of order $p$. Then $G$ is isomorphic to the
  cyclic group $\mc{C}_p$.
\end{proposition}
\begin{proof}[Proof]
  Since $p\ge 2$, we know that $G$ contains more than just the identity element, so we choose some
  non-identity element $g\in G$. \\
  \indent By Lagrange's Theorem, we know that the order of the subgroup
  $\left<g \right>$ divides the order of $G$. But since $\left| G \right| = p$ is prime, the order of
  $\left<g \right>$ is either $1$ or $p$; and since $\left< g\right>$ contains both $e$ and $g$ (and
  so $\left<g \right> > 1$), we know $\left| \left<g \right> \right| =p=\left| G \right|$. Thus the
  subgroup $\left<g \right>$ has the same number of elements as the full group, so they are equal:
  $\left<g \right> = G$. \\
  Now, we denote the cyclic group $\mc{C}_p=\{ g_0,g_1,\ldots,g_{p-1} \}$. We obtain an isomorphism
  \begin{align*}
    C_p &\longrightarrow G \\
     g_i &\longmapsto g^{i}
  .\end{align*} Thus $G$ is isomorphic to $\mc{C}_p$.
\end{proof}

\begin{remark}
  The vast theory of finite groups has many fascinating (and frequently unexpected) results, with
  easy to understand statements, yet surprisingly intricate proofs. Two such theorems are stated.
\end{remark}

\begin{theorem}[]{}
  Let $p$ be a prime number, and let $G$ be a group of order $p^2$. Then $G$ is an Abelian group.
\end{theorem}

On the other hand, we know that there exist non-Abelian groups of order $p^3$. For instance,
$\mc{D}_4$ and the quaternion group $\mc{Q}$ are non-Abelian groups of order $8=2^3$.

The next result is a partial converse of  Lagrange's Theorem.
\begin{theorem}[Sylow's Theorem]{}
  Let $G$ be a finite group, let $p$ be a prime, and suppose $p^{n}$ divides $\left| G \right| $ for
  some power $n\ge 1$. Then $G$ has a subgroup or order $p^{n}$.
\end{theorem}

One might hope, more generally, that if $d$ is any number that divides the order of $G$, then $G$
has a subgroup of order $d$. Unfortunately, this is not true; however, we have not yet seen a
counterexample. 

Both theorems will be proved later.


\section{Products of Groups}
Subgroups provide a way to break complicated objects (groups) down into smaller, simpler pieces. We
now look at a way in which two smaller groups can be used to build a larger group.

\begin{definition}[Products of Groups]{}
  Let $G_1,G_2$ be groups. The \textbf{product} of $G_1$ and $G_2$ is the group whose elements
  consist of ordered pairs \[
    G_1 \times G_2 = \{(a,b)\mid a\in G_1 ~\text{and}~b\in G_2\} 
  ,\] and whose group operation is performed separately on each component. In other words, if the
  group operation of $G_1\times G_2$ is $*$, the group operation of $G_1$ is $\cdot $, and the 
  group operation of $ G_2$ is $\circ $, we have \[
    (a_1,b_1) * (a_2,b_2) = (a_1\cdot a_2,b_1\circ b_2)
  .\] It is clear that the identity element of $G_1\times G_2$ is $(e_1,e_2)$, and the inverse of an
  element $(a,b)\in G$ is given by \[
    (a^{-1},b^{-1}).
  \] More generally, we can take any list of groups $ G_1,\ldots,G_n$ and form the product group \[
  G_1 \times  \ldots \times G_n 
  .\] 
\end{definition}
\begin{remark}
  We observe that $G=G_1\times G_2$ has order $\left| G_1 \right| \cdot \left| G_2 \right| $.
\end{remark}

\begin{example}
  For any non-zero numbers $m,n$, there is a homomorphism
  \begin{align*}
    \Z / mn\Z &\longrightarrow \left( \Z / m\Z \right) \times \left( \Z / n\Z \right)  \\
    a\mod(mn) &\longmapsto \left( a\mod m,a\mod n \right) 
  .\end{align*}
  We claim that if $\gcd{\left(m,n\right)}=1$, then it is an isomorphism. To see this, suppose that
  $a\mod{mn}$ is in the kernel. Then \[
    a\equiv 0\mod{m} ~\text{and}~ a\equiv 0\mod{n}
  .\] In other words, $a$ is divisible by both $m$ and $n$, and then the assumption that
  $\gcd{(m,n)}=1$ implies that $a$ is divisible by $mn$. Thus $a\equiv 0\mod{mn}$, which proves that
  the kernel of the homomorphism is $\{ 0 \}$ (and thus the homomorphism is injective). Further,
  since the finite set $\Z / mn\Z$ has the same number of elements as $\Z / m\Z \times \Z / n\Z$
  ($\Z / mn\Z$ has $mn$ elements, while $\Z / m\Z \times \Z / n\Z$ has $m \cdot n = mn$ elements),
  so it is surjective, and thus the homomorphism is an isomorphism.
\end{example}

One interpretation of this example is that it tells us that if $\gcd{(m,n)}=1$, then the large group
$\Z / mn\Z$ may be broken down into the product of two smaller groups $\Z / m\Z$ and $\Z / n\Z$.
Repeated applications demonstrate that any finite cyclic group is isomorphic to the product of
cyclic groups of prime power order. The following theorem extends this to all finite Abelian groups.
\begin{theorem}[Structure Theorem for Finite Abelian Groups]{}
  Let $G$ be a finite Abelian group. Then there are integers $ m_1,\ldots,m_r)$ so that \[
    G \cong \left( \Z / m_1\Z \right) \times \left( \Z / m_2\Z \right) \times \ldots \times \left( \Z / m_r\Z \right) 
  .\] 
\end{theorem}


\begin{example}
  (\textbf{Projects and Inclusions}) Products of groups come with two natural \textbf{projection
  homomorphisms}
  \begin{align*}
    p_1: G_1\times G_2\longrightarrow G_1, && p_2: G_1\times G_2\longrightarrow G_2,\\
    (a,b)\longmapsto a, && (a,b)\longmapsto b
  ,\end{align*} and two natural \textbf{inclusion homomorphisms}
  \begin{align*}
    \iota_1: G_1\longrightarrow G_1\times G_2 && \iota_2: G_2 \longrightarrow G_1\times G_2\\
    a\longmapsto (a,e_2) && b\longmapsto(e_1,b)
  .\end{align*}
  The inclusion maps are clearly injective, but the projections have kernels \[
    \ker(p_1) = \{ e_1 \}\times G_2 ~\text{and}~ \ker(p_2) = G_1 \times \{ e_2 \}
  .\] 
\end{example}
















\end{document}
