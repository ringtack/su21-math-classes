\documentclass[math1530-lecture-notes]{subfiles}
\begin{document}

\chapter{Groups: Part I}

Groups are a fundamental baseline for abstract algebra. We start with motivating examples, then move
on to a concrete definition.

\section{Motivation}
\subsection{Permutations}

\begin{definition}[Permutations]{}
  Let $X$ be a set. A \textbf{permutation} of $X$ is a bijective function \[\pi: X \to X\] with the
    property: $\forall x\in X, !\exists x'\in X$ such that $\pi(x')=x$. This allows us to define an
    inverse $\pi^{-1}$ to be the permutation \[
    \pi^{-1}:X\to X
  \] with the rule that $\pi^{-1}(x)=x'$, where $x'\in X$ is the unique element such that
  $\pi(x')=x$. \\
  The \textbf{identity permutation} of $X$ is the identity map \[
    e:X\to X, e(x)=x \forall x\in X
  .\] 
\end{definition}
In general, a \textit{permutation} of a set $X$ is a rule that ``mixes up'' the elements of $X$.

\begin{example}
  Let $X =\{1, 2, 3, 4 \} $. Then a permutation $\sigma: X \to X$ can be thought of as a shuffling of $X$ and
  visualized as follows:
  \begin{align*}
    1 &\Rightarrow 2 \\
    2 &\Rightarrow 3 \\
    3 &\Rightarrow 1 \\
    4 &\Rightarrow 4
  \end{align*}
  $\sigma^{-1}$ would be defined as
  \begin{align*}
    1 &\Rightarrow 3\\
    2 &\Rightarrow 1\\
    3 &\Rightarrow 2 \\
    4 &\Rightarrow 4 
  \end{align*}
  Now, suppose $\tau$ is defined as $1\Rightarrow 1, 2\Rightarrow 3, 3\Rightarrow 2, 4\Rightarrow
  4$. Then $\sigma\circ \tau$ is
  \begin{align*}
    1 &\Rightarrow 2 \\
    2 &\Rightarrow 1 \\
    3 &\Rightarrow 3 \\
    4 &\Rightarrow 4
  \end{align*}
  and $\tau\circ \sigma$ is 
  \begin{align*}
    1 &\Rightarrow 3 \\
    2 &\Rightarrow 2 \\
    3 &\Rightarrow 1 \\
    4 &\Rightarrow 4 
  \end{align*}
\end{example}

From this, we gather some observations.
\begin{itemize}
  \item Given any 2 permutations, we can compose to get a new one.
  \item There was a permutation that didn't do anything ($\sigma\circ \sigma^{-1}$).
  \item We can invert any permutation.
  \item If $\sigma,\tau$ are two permutations, then we don't necessarily have $\tau\circ
    \sigma=\sigma\circ \tau$ (in other words, the group of permutations with composition is not
    commutative).
\end{itemize}

\begin{definition}[Transformations]{}
  Let $X$ be a figure in $\R^2$. Then $Trafo(X)$ is the set of transformations on $X$.
\end{definition}

Consider the symmetries of a square (involving reflections/rotations on a square) as a motivating
example of transformations; are they invertible? commutative?

\begin{remark}
  Each transformation gives a permutation of the vertices $\{A,B,C,D\} $.
\end{remark}


\section{(Abstract) Groups}
\begin{definition}[Groups]{}
  A \textbf{group} $\{X, \cdot \}$ consists of a set $X$, together with a rule 
  \begin{align*}
    \cdot : G\times G &\to G\\
    (g_1,g_2) &\mapsto g_1\cdot g_2
  \end{align*}
  satisfying the following axioms:
  \begin{enumerate}
    \item (identity) there is an element $e\in G$ such that \[
        e\cdot g=g\cdot e=g
      .\] for all $g\in G$.
    \item (inverse) For all $g\in G$, there is an $h\in G$ such that \[
      g\cdot h=h\cdot g=e
      .\] The element $h$ is called $g^{-1}$, the inverse of $g$.
    \item (associativity) Given $ g_1, g_2, g_3$, we have \[
        g_1(g_2\cdot g_3) = (g_1\cdot g_2)g_3
    .\] \\
    If, in addition, the group satisfies
  \item (commutative) Given $ g_1,g_2\in G$, we have \[
    g_1\cdot g_2=g_2\cdot g_1
  .\] 
  \end{enumerate}
  then $G$ is an \textbf{Abelian} group.
\end{definition}

Now, we observe some interesting properties that follow from the group axioms.
\begin{proposition}[Group Properties]{}
  Let $G$ be a group.
  \begin{enumerate}
    \item The identity element is unique.
    \item Each element of $G$ has only one inverse.
    \item If $g,h\in G$, then $(gh)^{-1}=h^{-1}g^{-1}$.
    \item Given $g\in G$, $(g^{-1})^{-1}=g$.
  \end{enumerate}
\end{proposition}
\begin{proof}[Proof of (b)]
  Suppose $g\in G$and that both $h_1,h_2$ satisfy the inverse axiom. Then \[
  g\cdot h_1=e=g\cdot h_2
  .\]  By the inverse axiom, we multiply on the left by an inverse of $g $:
  \begin{align*}
    e\cdot h_1&= e\cdot h_2 \\
    h_1&=h_2
  .\end{align*} Thus the inverse is unique.
\end{proof}

\begin{definition}[Order]{}
  \begin{itemize}
    \item The \textbf{order}of a group $G$ is denoted  $\#G$ or  $\left| G \right| $ is the number of
          elements in $G$ if finite, and $\infty$ if infinite.
    \item If $G$ is a group and $g\in G$, the smallest $n$ in which $g^{n}=e$ is called \textbf{the
      order of $g$}. If no $n$ exists, we say $g$ has infinite order.
  \end{itemize}
\end{definition}

\begin{proposition}[Individual Order and Group Order]{}
  Suppose $G$ is a finite group and suppose $g^{n}=e$. Then the order of $g$ divides $n$.
\end{proposition}
\begin{proof}[Proof]
  Let $n$ be the order of $g\in G$. Then, by long division, we can write \[
 m=n\cdot g+r,~0\le r<m
  .\]  Using this equality together with $g^{n}=e$ and $g^{m}=e$, we get \[
  e=g^{n}=g^{m\cdot q+r}=\left( g^{m} \right) ^{q}\cdot g^{r} =e^{q}\cdot g^{r}=e\cdot g^{r}=g^{r}
.\] We find that $g^{r}=e$; but $r<m$ and  $m$ is the order of $g$.

[TODO]: Finish this exercise with a well-defined proof, not this bullshit
\end{proof}

\subsection{Examples of Groups}
\begin{example}
  $\Z,\Q,\R$ and $\C$ are all Abelian groups with respect to addition. However, $\Z$ is not a group
  with respect to multiplication, as the multiplicative inverse does not exist. Additionally,
  $\Q,\R,$ and $\C$ are not groups with respect to multiplication, due to zero; but $\Q\setminus \{
  0\}, \R\setminus \{ 0 \}$, and $ \C\setminus \{ 0 \}$ are all groups under multiplication.
\end{example}

\begin{example}
  Let $\Z/m\Z$ be the set of integers modulo $m$. Then  $\Z / m\Z$ is a group under addition
  modulo $m$, $+_m$ ; $\Z / m\Z$ is finite with order $m$. We also observe that $\Z/m\Z$ is a cyclic
  group.
\end{example}

\begin{example}
  Let the set of $n\times n$ matrices be $M_n$. Then  $M_n$ is an Abelian group under addition, but
  not multiplication (since not all matrices have inverses).\\

  Let \[GL_n(\R)=\{M\in M_n\mid \det(M)\neq 0\}\], denote the \textbf{general linear group}. Then
  $GL_n(\R)$ is a group using matrix multiplication (not Abelian though).
\end{example}


\subsection{Cyclic Groups}
\begin{definition}[Cyclic Groups]{}
  A group $G$ is \textbf{cyclic} if there is a $g\in G$ such that \[
    G=\{ \ldots,g^{-2},g^{-1} ,e,g,g^2,g^3,\ldots\}
  .\] We call $g$ a \textbf{generator}.
\end{definition}

Some examples of cyclic groups are $\Z$ and $\Z / m\Z$; both have generators $1$. Another one is
the permutation group.
\begin{definition}[Permutation Groups]{}
  Given $X$ a set, let $S_X$ denote the  \textbf{symmetric group of $X$}, or the group of
  permutations of $X$. If  \[
    X=\{ 1,\ldots,n \}
  ,\] we use the notation $S_n$.

  Let $P_n$ be a regular  $n$-gon with vertices $1,\ldots,n$. The group of transformations of $P_n$
  (e.g. rotations, reflections, and compositions of such) is called the  \textbf{dihedral group}
  $D_n$. We will later prove that $D_n$ has order  $2n$.
\end{definition}












\end{document}
