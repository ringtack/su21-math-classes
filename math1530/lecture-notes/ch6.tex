\documentclass[math1530-lecture-notes]{subfiles}
\begin{document}

\chapter{Groups: Part II}

We now continue our study of group theory. POGGIES!!!

\section{Normal Subgroups and Quotient Groups}

Let $G$ be a group, and let $H$ be a subgroup. Recall that each $g\in G$ gives a left coset of $H$,
defined by \[
  gH = \{gh\mid h\in H\} 
.\] We now define the collection of cosets of $H$.

\begin{definition}[Set of Cosets]{}
  Let $G$ be a group, with $H<G$ a subgroup. We denote the set of (left) cosets of $H$ in $G$ by \[
    G / H = \{ ~\text{(left) cosets of}~ H\}
  .\] 
\end{definition}

Let $\mc{C}_1, \ldots,\mc{C}_k$ be distinct cosets of $H$. Proposition 2.39 tells us that $G$ is
equal to the disjoint union \[
  G = \mc{C}_1\cup \ldots\cup \mc{C}_k
\] We use the notation $\mc{C}_i$ to emphasize that for a given coset $\mc{C}$, many different
elements $g\in G$ will satisfy $\mc{C}=gH$. Indeed, (as one should check) if $\mc{C}$ is a coset of
$H$, then \[
  \mc{C}=gH \iff g\in \mc{C}
.\] 

Now, is there a way to turn this collection of cosets $\{ \mc{C}_1,\ldots,\ldots,\mc{C}_k \}$ into a
group? If so, how should we define the product of two cosets $\mc{C}_i$ and $\mc{C}_j$? One
intuitive method may be to take a product of two elements, one from $\mc{C}_i$ and one from
$\mc{C}_j$, and declare the resulting set $\mc{C}_i\cdot \mc{C}_j$ to be the coset of the product.

\begin{definition}[Proposed Coset Multiplication]{}
  Let $G$ be a group, and $H < G$ a subgroup. Given cosets $\mc{C}_i, \mc{C}_j$ of $H$, select one
  element from each ($g_1\in \mc{C}_i$, $g_2\in \mc{C}_j$), and define the resultant coset
  $g_1g_2H$.
\end{definition}

Is this a good definition? Where might this go wrong?

Well, if we have three cosets of a group $G$, and we select one element from each of two cosets,
there's a possibility that if we select different elements from each of the cosets, the first
product might not be in the same coset as the second product. Concretely, consider the dihedral
group of three vertices, $\D_3$. The subgroup $\{ e, \phi_1 \}$ (where $\phi_i$ are flips, and
$\rho_j$ are rotations) has three cosets, \[
  \mc{C}_1 = \{ e,\phi_1 \},\ \mc{C}_2=\{ \rho_1,\phi_2 \},\ \mc{C}_3=\{ \rho_2,\phi_3 \}
.\] If we select $\phi_1\in \mc{C}_2,\ \phi_2\in \mc{C}_3$, then we get $e\in \mc{C}_1$; but if we
select $\phi_2\in \mc{C}_2,\ \phi_3\in \mc{C}_3$, then we get $\rho_2\in \mc{C}_3$. The above
definition thus gives us two different results: either $\mc{C}_2\cdot \mc{C}_3=\mc{C}_1$, or
$\mc{C}_2\cdot \mc{C}_3=\mc{C}_3$!

However, the above definition \textit{does} work if we select our subgroup $H$ more carefully; for
instance, if we chose $H=\{ e,\rho_1,\rho_2 \}$, then the definition seems to work! So the
definition works for some, but not all, subgroups. How do we distinguish the ``good'' subgroups? We
want the subgroups to have the following property: \begin{center}
  For all cosets $\mc{C}_1, \mc{C}_2$ of $H$, for all pairs of elements $g_1,g_1'\in \mc{C}_1,\
  g_2,g_2'\in \mc{C}_2$, we wish for \[
    g_1g_2H=g_1'g_2'H
  .\] 
\end{center}

Let's see where this takes us. $g_1$ and $g_1'$ in the same coset $\mc{C}_1$ of $H$ means that there
is some $h_1\in H$ with $g_1'=g_1h_1$ (since all elements of a coset of $H$ can be represented as
$g_1h_1$, where $g_1$ is in the coset, and $h_1$ is in the subgroup), and similarly $g_2'=g_2h_2$
for some $h_2\in H$. So, we want \[
  g_1'=g_1h_1~\text{and}~g_2'=g_2h_2 \implies g_2^{-1}g_1^{-1}g_1'g_2'\in H
\] (since we want $g_1'g_2'=g_1g_2h_1h_2$ in order for $g_1'g_2'\in g_1g_2H$; or equivalently,
$g_2^{-1}g_1^{-1}g_1'g_2'\in H$). Substituting values of $g_1', g_2'$, we want \[
  g_2^{-1}g_1^{-1}g_1h_1g_2h_2\in H ~\text{for all}~g_1,g_2\in G,\ h_1,h_2\in H
.\] With cancellation, this becomes \[
  g_2^{-1}h_1g_2h_2\in H~\text{for all}~g_2\in G,\ h_1,h_2\in H
.\] But we know that $g_2h_2\in H$ if and only if $g_2\in H$, so we end up with \[
  g_2^{-1}h_1g_2\in H~\text{for all}~g_2\in G,\ h_1\in H
.\] Dropping subscripts, we now get the following definition:
\begin{definition}[Normal Subgroups]{}
  Let $G$ be a group, let $H < G$ be a subgroup, and let $g\in G$. The \textbf{$g$-conjugate of $H$}
  is the subgroup \[
    g^{-1}Hg=\{g^{-1}hg\mid g\in G\} 
  .\] We say that $H$ is a \textbf{normal subgroup} if it satisfies \[
    g^{-1}Hg=H~\text{for every }~g\in G
  .\] 
\end{definition}

\begin{example}
  If $G$ is an Abelian group, then every subgroup $H\subseteq G$ is normal, since \[
    g^{-1}hg=g^{-1}gh=h
  .\] 
\end{example}

\begin{definition}[Simple Groups]{}
  Every group $G$ has two normal subgroups, $\{ e \}$ and $G$. If these are the only two normal
  subgroups, then $G$ is called a \textbf{simple group}.
\end{definition}

\begin{example}
  As shown above, the subgroup $H=\{ e,\phi_1 \}$ is not a normal subgroup, since for example \[
    \phi_2^{-1}\{ e,\phi_1 \}\phi_2=\{ \phi_2^{-1}e\phi_2,\phi_2^{-1}\phi_1\phi_2 \}=\{ e,\phi_3 \}
  .\] However, the other subgroup $H=\{ e,\rho_1,\rho_2 \}$ is a normal subgroup. One could
  tediously check all conjugates, or realize that three rotations is a rotation, or a flip,
  rotation, flip is still a rotation. Similar results hold for $\D_n$.
\end{example}

Here is an important source of normal subgroups. Indeed, we will later show that every normal
subgroup of a group $G$ arises in this way.
\begin{proposition}
  Let $\phi:G\to G'$ be a homomorphism of groups. Then $\ker{(\phi)}$ is a normal subgroup of $G$.
\end{proposition}
\begin{proof}[Proof]
  We know from before that $\ker{(\phi)}$ is a subgroup of $G$. Let $h\in \ker{(\phi)}$, $g\in G$.
  Then
  \begin{align*}
    \phi(g^{-1}\cdot h\cdot g)&= \phi(g^{-1}) \cdot \phi(h)\cdot \phi(g)\\
                              &=\phi(g)^{-1}\cdot \phi(h)\cdot \phi(g)\\
                              &=\phi(g)^{-1}\cdot \phi(g) && [~\text{since $h\in \ker{(\phi)}$, so
                              $\phi(h)=e'$}~]\\
                              &= e'
  .\end{align*}
  Hence $g^{-1}\cdot h\cdot g\in \ker{(\phi)}$; since the choice of $g,h$ were arbitrary,
  $\ker{(\phi)}$ is thus a normal subgroup of $G$.
\end{proof}

Before turning the collection of cosets $G / H$ into a group, we now give three elementary
properties of normal subgroups. 
\begin{proposition}[Properties of Normal Subgroups]{pnsg}
  Let $G$ be a group, and $H < G$ a subgroup.
  \begin{enumerate}
    \item If $g^{-1}Hg\subseteq H$, then $H$ is a normal subgroup of $G$; in other words, it is
      enough only to check one inclusion.
    \item For all $g\in G$, the conjugate set $g^{-1}Hg$ is a subgroup of $G$.
    \item For all $g\in G$, the map $H\to g^{-1}Hg$ defined by $h\mapsto g^{-1}hg$ is a group
      isomorphism. In particular, if $H$ is finite, then $H$ and its conjugates have the same number
      of elements.
  \end{enumerate}
\end{proposition}
\begin{proof}[Proof]
  Left as an exercise.
\end{proof}

We now go back to our goal of turning the set of cosets $G / H$ into a group via the multiplication
rule \[
  g_1H\cdot g_2H=g_1g_2H
.\] Unfortunately, as we've seen before, choosing different elements $g_1'\in g_1H,\ g_2'\in g_2H$
may yield a different coset $g_1'g_2'H$. However, if $H$ is a \textit{normal} subgroup of $G$, then
we do actually get the same coset. Yay! Let us formally verify this.

\begin{lemma}{}
  Let $G$ be a group, and let $H<G$ be a normal subgroup. Let $g_1,g_1',g_2,g_2'\in G$ be elements
  of $G$ satisfying \[
    g_1H=g_1'H ~\text{and}~g_2H=g_2'H
  .\] Then \[
    g_1g_2H=g_1'g_2'H
  .\] 
\end{lemma}
\begin{proof}[Proof]
  The assumption that $g_1H=g_1'H$ implies in particular that \[
    g_1'\in g_1'H=g_1H, ~\text{so}~g_1'=g_1h_1 ~\text{for some}~h_1\in H
  .\] Similarly, \[
    g_2'\in g_2'H=g_2H \implies g_2'=g_2h_2~\text{for some}~h_2\in H
  .\] We now wish to prove the inclusion $g_1'g_2'H\subseteq g_1g_2H$, so let $g_1'g_2'h$ be an
  element of $g_1'g_2'H$. We want \[
     g_1'g_2'h\subseteq g_1g_2H
  .\] From above, we see that \[
     g_1'g_2'h=g_1h_1g_2h_2h
   .\] If we could flip $h_2$ and $g_2$, we are done (since we are left with $g_1g_2$ times an
   element of $H$); but alas, $G$ is not necessarily commutative, so we must look elsewhere.

   We want the leftmost $g_1$ to be followed by $g_2$; we can't just insert a $g_2$ there, but we
   can use the standard mathematical trick of putting the quantity where we want, and multiplying it
   to cancel it out: we do this by multiplying by $e=g_2g_2^{-1}$. So, we have \[
     g_1h_1g_2h_2h=g_1g_2g_2^{-1}h_1g_2h_2h=(g_1g_2)(g_2^{-1}h_1g_2)h_2h
   .\] But $g_2^{-1}h_1g_2\in H$, since $H$ is a normal subgroup; and closure implies $h_2h\in H$.
   Thus, we have \[
     g_1'g_2'h=g_1h_1g_2h_2h=g_1g_2(~\text{three elements in $H$}~)\in g_1g_2H
   ,\] and hence that \[
     g_1'g_2'H\subseteq g_1g_2H
   .\] Reversing the roles gives an analogous argument for $g_1g_2H\subseteq g_1'g_2'H$, and thus we
   get \[
     g_1g_2H=g_1'g_2'H
   .\] 
\end{proof}

Essentially, this lemma demonstrates that the multiplication rule for cosets, $g_1H\cdot
g_2H=g_1g_2H$, is well-defined if $H$ is a normal subgroup of $G$. We are now able to turn $G / H$
into a group!

\begin{theorem}[First Isomorphism Theorem For Groups]{}
  Let $G$ be a group, and let $H$ be a normal subgroup of $G$.
  \begin{enumerate}
    \item The collection of cosets $G / H$ is a group via the well-defined group operation \[
        g_1H\cdot g_2H=g_1g_2H
      .\] 
    \item The map \[
        \phi: G\longrightarrow G / H,\ \phi(g)=gH
      ,\] is a homomorphism whose kernel is $\ker{(\phi)}=H$.
    \item Let \[
      \psi:G\longrightarrow G'
      \] be a homomorphism with the property that $H\subseteq \ker{(\psi)}$. Then there is a unique
      homomorphism \[
        \lambda: G / H \longrightarrow G'~\text{satisfying}~\lambda(gH)=\psi(g)
      .\] 
    \item If we take $H=\ker{(\psi)}$ from above, then the homomorphism \[
        \lambda:G / \ker{(\psi)}\longrightarrow G'
      \] is injective. In particular, we get an isomorphism onto the image of $\lambda$, \[
        \lambda: G / \ker{(\psi)} \longrightarrow \lambda(G)\subseteq G'
      .\] 
  \end{enumerate}
\end{theorem}

\begin{proof}[Proof]
  \begin{enumerate}
    \item The above lemma proves that coset multiplication is well-defined (with a normal subgroup
      $H$). The other group properties follow directly from the corresponding properties of the
      group operation on $G$:
      \begin{gather}
        eH\cdot gH=gH\cdot eH=gH\\
        gH\cdot g^{-1}H=g^{-1}H\cdot gH=eH\\
        (g_1H\cdot g_2H)\cdot g_3H=g_1H\cdot (g_2H\cdot g_3H)
      .\end{gather} Thus $G / H$ is a group under coset multiplication.
    \item We first check that $\phi$ is a homomorphism. Note that \[
        \phi(g_1)\phi(g_2)=g_1H\cdot g_2H=g_1g_2H=\phi(g_1g_2)
      .\] Thus $\phi$ is a homomorphism. The kernel of $\phi$ is \[
        \ker{(\phi)}=\{g\in G\mid \phi(g)=H \} =\{g\in G\mid gH=H\} =H
      .\] 
    \item We want to define a homomorphism $\lambda:G / H \to G'$ by the following algorithm:
      \begin{enumerate}
        \item Let $\mc{C}\in G / H$ be a coset of $H$.
        \item Choose some $g\in \mc{C}$ with $\mc{C}=gH$.
        \item Define $\lambda(\mc{C})$ to be $\psi(g)=g'$; that is, $\psi(g)=\lambda(gH)=g'$.
      \end{enumerate}
      However, step 2 may pose a problem, since many possible $g\in \mc{C}$ can work. So, we need to
      prove that if $gH=g'H$, then $\psi(g)=\psi(g')$.

      $gH=g'H$ means that $g'=gh$ for some $h\in H$. Then
      \begin{align*}
        \psi(g')&= \psi(gh) \\
                &= \psi(g)\psi(h)\\
                &=\psi(g)e' && [~\text{since $H\subseteq \ker{(\psi)}$}~]\\
                &=\psi(g)
      ,\end{align*} as required. Thus our algorithm gives a well-defined map \[
        \lambda: G / H \longrightarrow G'
      .\] It's also easy to check that $\lambda$ is a homomorphism: for $g_1,g_2\in G$, \[
        \lambda(g_1g_2H)=\psi(g_1g_2)=\psi(g_1)\psi(g_2)=\lambda(g_1H)\lambda(g_2H)
      .\] Finally, for any given homomorphism $\psi:G\to G'$, there's only one map $\lambda$
      satisfying $\psi(g)=\lambda(gH)$, since this equality completely determines the values of
      $\lambda$ in terms of values of $\psi$; that is, for any $\lambda$ that maps $G / H$ to $G'$,
      every $gH\in G / H$ must be mapped to the same $\psi$.
    \item Let $H=\ker{(\psi)}$. From (c) we get that \[
        \lambda: G / H\longrightarrow G',\ \lambda(gH)=\psi(g)
    ;\] we just need to show that $\lambda$ is injective.

    Let $gH\in ker(\lambda)$. Then $\lambda(gH)=\psi(g)=e'$, so $g\in \ker{(\psi)}=H$. Therefore
    $gH=H$, which is the identity element of $G / H$. Thus, the kernel $\ker{(\lambda)}$ is trivial
    and consists of only the identity element, so $\lambda$ is injective. Surjectivity follows by
    definition of a function; any $\lambda$ surjects onto its image. Thus $\lambda: G/H\to G'$ is an
    is an isomorphism.
  \end{enumerate}
\end{proof}

\section{Groups Acting On Sets}

Among the first groups studied were the symmetric groups $\mc{S}_n$, and the dihedral groups $\D_n$.
The elements of $\mc{S}_n$ were permutations of $\{ 1,\ldots,n \}$, so one could view an element of
$\mc{S}_n$ as giving a rule that takes a number in $\{ 1,\ldots,n \}$ and assigns it to another
number in $\{ 1,\ldots,n \}$. Similarly, the elements of $\D_n$ are rigid re-arrangements of
vertices of an $n$-gon, so we can also view elements of $\D_n$ as assigning a rule that takes the
vertex of an $n$-gon to another vertex of the $n$-gon (with additional restrictions). We can
axiomatize these examples into arbitrary groups, by starting with a group $G$ and a set $X$ and
having each element in $G$ re-arrange elements in $X$ (with some restrictions).

\begin{definition}[Group Actions on Sets]{}
  Let $G$ be a group, and let $X$ be a set. An \textbf{action of $G$ on $X$} is a rule $\cdot$ that
  assigns to each element $g\in G$ and each element $x\in X$ another element $g\cdot x\in X$, so
  that the following two axioms hold:
  \begin{itemize}
    \item \textbf{Identity Axiom}: $e\cdot x=x$ for all $x\in X$.
    \item \textbf{Associative Axiom}: $(g_1g_2)\cdot x=g_1\cdot (g_2\cdot x)$ for all $g_1,g_2\in
      G$ and all $x\in X$.
  \end{itemize}
\end{definition}

\begin{remark}
  We can choose a fancier definition by showing that defining an action of $G$ on $X$ is the same as
  giving a group homomorphism \[
    \alpha: G\longrightarrow \mc{S}_X
  \] from the group $G$ to the symmetric group of $X$. Thus $\alpha$ sends each $g\in G$ to a
  permutation $\alpha(g):X\to X$ of the set $X$, and the group action is $g\cdot x=\alpha(g)(x)$.
\end{remark}

\begin{definition}[Orbits and Stabilizers]{}
  Given a group $G$ acting on a set $X$, each element $x\in X$ determines two natural objects of
  interest:
  \begin{itemize}
    \item What are the elements of $X$ to which $x$ is sent by the action of $G$? This set is the
      \textbf{orbit of $x$}, denoted \[
        Gx=\{ g\cdot x\mid g\in G \}
      .\] In other words, it is the all possible values that an $x\in X$ can take---its
      ``range''---when acted upon by the group $G$.
    \item What are the elements of $G$ that leave $x$ unchanged? This set is the \textbf{stabilizer
      of $x$}, denoted \[
        G_x=\{g\in G\mid g\cdot x=x\} 
      .\] In other words, it is all values in $G$ that preserve $x$; similar to the ``kernel'' of
      the action.
  \end{itemize}
\end{definition}
\begin{example}
  Consider the action of $G=\mc{S}_n$ on $X=\{ 1,\ldots,n \}$. Given any two elements $x,y\in X$,
  there is a permutation $\pi\in \mc{S}_n$ that sends $x$ to $y$ (e.g. $\pi(x)=y,\ \pi(y)=x$), and
  fixes all other elements of $X$. Thus, the orbit of $x$ is all of $X$, i.e. $Gx=X$, since there
  exists a $\pi\in G$ that sends $x$ to any other element in $X$. The stabilizer $G_x$ of $x$
  consists of all permutations of $X$ that fix $x$, so they are the permutations of the remaining
  $n-1$ elements of $X$. Thus $G_x$ is isomorphic to $\mc{S}_{n-1}$.
\end{example}
\begin{example}
  Consider the action of $G=\D_n$ on an $n$-gon whose vertices are labeled $X=\{ 1,\ldots,n \}$
  clockwise around the $n$-gon. The orbit of any vertex consists of every vertex ($X$), since
  there's a rotation that takes any vertex to any other vertex. With stabilizers, it's not quite as
  simple. Clearly, non-trivial rotations do not fix $x$; same with most flips. However, the flip
  about the axis that goes through $x$ fixes $x$ (it also fixes the opposite vertex if $n$ is even)
  Thus $G_x$ consists of two elements, the identity and the flip about the axis through $x$, so
  $G_x$ is a cyclic group of order $2$.
\end{example}
\begin{example}
  Let $G$ be the subgroup of $\mc{S}_5$ generated by the permutation \[
    \pi=(134)(25)
  .\] Clearly, $\pi$ has order $6$, so $G=\{ e,\pi,\pi^2,\pi^3,\pi^4,\pi^5 \}$. Then \[
    G\cdot 1=G\cdot 3=G\cdot 4=\{ 1,3,4 \},~\text{and}~G\cdot 2=G\cdot 5=\{ 2,5 \}
  .\] Thus $G$ has two distinct orbits.
\end{example}

\begin{proposition}{}
  Let $G$ be a group that acts on a set $X$.
  \begin{enumerate}
    \item Let $x\in X$. The stabilizer $G_x$ is a subgroup of $G$.
    \item Define a relation $\sim$ on $X$ by the following rule: \[
        x\sim y ~\text{if}~ y=gx ~\text{for some}~g\in G
      .\] Then $\sim$ is an equivalence relation, and \[
        (~\text{the equivalence class of $x$}~)=(~\text{the orbit $Gx$ of $x$}~)
      .\] 
    \item Let $x\in X$. There is a well-defined bijection \[
      \alpha:G / G_x \longrightarrow Gx
    \] defined by the following algorithm:
    \begin{itemize}
      \item \textbf{Input}: A coset $\mc{C}$ of the subgroup $G_x$.
      \item \textbf{Computation}: Choose an element $g\in \mc{C}$.
      \item \textbf{Output}: $\alpha(\mc{C})$ is the element $g\cdot x$ in the orbit $Gx$.
    \end{itemize}
    In particular, if $G$ is finite, then \[
      \left| Gx \right| = \frac{\left| G \right| }{\left| G_x \right| }
    .\] 
  \end{enumerate}
\end{proposition}

\begin{proof}[Proof]
  \begin{enumerate}
    \item First, we have $e\in G_x$, since the definition of a group action requires $ex=x$. Next, let
      $g,g'\in G_x$. Then since $g,g'$ both fix $x$ together with the associative law of group actions
      tell us that \[
        (gg')x=g(g'x)=gx=x
      ,\] so $gg'\in G_x$. Finally, applying $g^{-1}$ to both sides of $x=gx$ yields \[
        g^{-1}x=g^{-1}(gx)=(g^{-1}g)x=ex=x
      ,\] so $g^{-1}\in G_x$. Thus $G_x$ is a subgroup of $G$.
    \item First we have \[
        x=ex, ~\text{which shows that $x\sim x$}~
      ,\] and so $\sim$ is reflexive. Second, we note that
      \begin{align*}
        x\sim y &\implies y=gx ~\text{for some}~g\in G\\
                &\implies g^{-1}y=(g^{-1}g)x=x
      ,\end{align*} thus $x=g^{-1}y$, and so $y\sim x$, which shows that $\sim$ is symmetric. Finally,
      \begin{align*}
        x\sim y~\text{and}~y\sim z &\implies y=gx~\text{and}~z=g'y~\text{for some}~g,g'\in G\\
                                   &\implies z=g'(gx)=(g'g)x\\
                                   &\implies x\sim z
      .\end{align*} Thus $\sim$ is an equivalence relation.

      The equivalence class of an element $x\in X$ is \[
        \{y\in X\mid x\sim y\} =\{y\in X\mid y=gx~\text{for some}~g\in G\} =\{gx\mid g\in G\} =Gx
      .\] 

    \item We first must show that the output is well-defined, in that it doesn't depend on the
      choice of the group element in the coset $\mc{C}$. Suppose that $g'\in \mc{C}$ is some other
      element in the coset. This means that \[
        g'\in \mc{C}=gG_x,~\text{so}~g'=gh~\text{for some}~h\in G_x
      .\] It follows that \[
        g'x=(gh)x=g(hx)=gx
      ,\] since $h\in G_x$ means $hx=x$. This shows that $\alpha(\mc{C})=gx$ is well-defined, since
      it doesn't depend on choice of $g\in \mc{C}$ (explicitly, $\alpha(g'G_x)=g'x=gx=\alpha(gG_x)$
      for $g,g'\in \mc{C}$, a coset of $G_x$). It's easy to see that $\alpha$ is surjective, since
      every element of $Gx$ looks like $gx$ for some $g\in G$, and so we can simply choose
      $\alpha(gG_x)=gx$.

      Finally, we prove that $\alpha$ is injective, so suppose we have cosets $\mc{C}_1, \mc{C}_2$
      such that \[
        \alpha(\mc{C}_1)=\alpha(\mc{C}_2)
      .\] We wish to show that $\mc{C}_1=\mc{C}_2$. Write the cosets as \[
        \mc{C}_1=g_1G_x~\text{and}~\mc{C}_2=g_2G_x~\text{for some}~g_1,g_2\in G
      .\] Then
      \begin{align*}
        \alpha(\mc{C}_1)=\alpha(\mc{C}_2)&\implies \alpha(g_1G_x)=\alpha(g_2G_x)\\
                                         &\implies g_1x=g_2x\\
                                         &\implies x=g_1^{-1}g_2x\\
                                         &\implies g_1^{-1}g_2\in G_x&&[~\text{since $g_1^{-1}g_2$
                                         fixes x}~]\\
                                         &\implies g_1^{-1}g_2G_x=G_x\\
                                         &\implies g_2G_x=g_1G_x\\
                                         &\implies \mc{C}_1=\mc{C}_2
      .\end{align*} Thus $\alpha$ is a well-defined bijection.

      For the last part, we compute \[
        \left| Gx \right| =\left| (G / G_x) \right| =\left| G \right| /\left| G_x \right| 
      ,\] where the first equality is because they are isomorphic, and the second is due to
      Lagrange's Theorem.
  \end{enumerate}
\end{proof}

\begin{definition}[Transitivity]{}
  We say that $G$ \textbf{acts transitively on $X$} if $Gx=X$ for all $x\in X$.
\end{definition}

The dihedral group example demonstrates that $\D_n$ acts transitively on the vertices of an $n$-gon,
while the following example about the cyclic subgroup generated by $\pi$ gives an example of a group
and a set on which the group does not act transitively.


\section{The Orbit-Stabilizer Counting Theorem}

Let $G$ be a finite group that acts on a finite set $X$. We start with an important proof relating
the sizes of $G$, $X$, and the various orbits and stabilizers. It will later prove crucial in our
proof of Bylaw's theorem; and it has many other applications.

\begin{theorem}[Orbit-Stabilizer Counting Theorem]{}
  Let $G$ be a finite group that acts on a finite set $X$. Choose $x_1,\ldots,x_k\in X$ so that \[
    Gx_1,\ldots,Gx_k
  \] are the distinct orbits of elements of $X$. Then \[
    \left| X \right| =\sum_{i=1}^{n} Gx_i=\sum_{i=1}^{n} \frac{\left| G \right| }{\left| G_{x_i} \right| }
  .\] 
\end{theorem}
\begin{proof}[Proof]
  From above, we see that there is an equivalence relation on $X$ such that the equivalence class of
  $x$ is exactly its orbit $Gx$. Then, from general properties of equivalence classes, we have that
  $X$ is the disjoint union of the distinct equivalence classes: \[
    X=Gx_1\cup \ldots\cup Gx_k
  ,\] so in particular \[
    \left| X \right| =\left| Gx_1 \right| +\ldots+\left| Gx_k \right| 
  .\] This proves the first equality. We've also shown that for any orbit and stabilizer, $\left| Gx
  \right| =\frac{\left| G \right| }{\left| G_x \right| }$; this thus proves the second equality.
\end{proof}
\begin{example}
  Let $\D_6$ be the deferral group acting on a hexagon whose vertices are labeled $\{ A,B,C,D,E,F
  \}$. Let $r\in \D_6$ be a $60^{\circ}$ counterclockwise rotation, $f$ a flip on the axis between
  $A$ and $B$, and $f'$ a flip that fixes $C$ and $F$. 

  We consider the following six subgroups of $\D_6$, and their actions on the hexagon: \[
    \{ e,r,r^2,r^3,r^4,r^5 \},\ \{ e,r^2,r^4 \},\ \{ e,r^3 \},\ \{ e,f \},\ \{ e,f' \}, \{
    e,r^3,f,f' \}
  .\] For each, we compute their orbits and stabilizers:
  \begin{itemize}
    \item $\{ e,r,r^2,r^3,r^4,r^5 \}$:
      \begin{itemize}
        \item Orbit: $\{ A,B,C,D,E,F \}$
        \item Stabilizer: $\{ e \}$
      \end{itemize}
    \item $\{ e,r^2,r^4 \}$:
      \begin{itemize}
        \item Orbits: $\{ A,C,E \}$, $\{ B,D,F \}$
        \item Stabilizers: $\{ e \}, \{ e \}$
      \end{itemize}
    \item $\{ e,r^3 \}$:
      \begin{itemize}
        \item Orbits: $\{ A,D \},\{ B,E \},\{ C,F \}$
        \item Stabilizers: $\{ e \},\{ e \},\{ e \}$
      \end{itemize}
    \item $\{ e,f \}$:
      \begin{itemize}
        \item Orbits: $\{ A,B \},\{ C,F \},\{ D,E \}$
        \item Stabilizers: $\{ e \},\{ e \},\{ e \}$
      \end{itemize}
    \item $\{ e,f' \}$:
      \begin{itemize}
        \item Orbits: $\{ C\},\{F \},\{ A,E \},\{ B,D \}$
        \item Stabilizers: $\{ e,f' \},\{ e,f' \},\{ e \},\{ e \}$
      \end{itemize}
    \item $\{ e,r^3,f,f' \}$:
      \begin{itemize}
        \item Orbits: $\{ A,B,D,E \},\{ C,F \}$
        \item Stabilizers: $\{ e \},\{ e,f' \}$
      \end{itemize}
  \end{itemize}
  In every case, the orbits are disjoint and satisfy $\left| G_x \right| \cdot \left| Gx \right|
  =\left| G \right| $, as we've illustrated before.

  We can also check that the Orbit-Stabilizer Counting Theorem holds; the most interesting case is
  for $\{ e,r^3,f,f' \}$. The two orbits, \[
    G\cdot A=\{ A,B,D,E \} ~\text{and}~G\cdot C=\{ C,F \}
  ,\] have respective stabilizers $G_A=\{ e \}$ and $G_C=\{ e,f' \}$. Then \[
  \left| X \right| =\left| (G\cdot A) \right| +\left| (G\cdot C) \right| =4+2=6
  ,\] as required.
\end{example}

The Orbit-Stabilizer Counting Theorem is incredibly powerful for studying groups and the sets on
which they act. We illustrate this by proving a fundamental property of groups with prime power
order, and as an application show that every group with order $p^2$ is Abelian.

First, we start with a definition.
\begin{definition}[Center]{}
  Let $G$ be a group. The \textbf{center} of $G$, denoted $Z(G)$, is the set of elements of $G$ that
  commute with every element of $G$, \[
    Z(G)=\{g\in G\mid gg'=g'g ~\text{for every}~g'\in  \} 
  .\] We leave it as an exercise to prove that the center of $G$ is a normal subgroup of $G$.
\end{definition}

\begin{theorem}{}
  Let $p$ be a prime, and let $G$ be a finite group with $p^n$ elements for some $n\ge 1$. Then
  $Z(G)\neq \{ e \}$, that is, $G$ contains a non-identity element that commutes with every element
  of $G$.
\end{theorem}
\begin{proof}[Proof]
  Take the set $X$ to be a copy of $G$, and let $G$ act on $X$ with the formula \[
    g\in G ~\text{sends}~x\in X~\text{to}~gxg^{-1}\in X
  .\] This is called the \textbf{conjugation action} of $G$ on itself (check that this rule
  satisfies the group action axioms!). Let $x\in X$. What is the stabilizer $x\in X$? It is the set
  \[
    G_x=\{g\in G\mid gxg^{-1}=x\} =\{g\in G\mid gx=xg \} 
  .\] In other words, the stabilizer of $x$ is the set of elements in $G$ that commute with $x$. In
  particular, \[
    Gx=G \iff x~\text{commutes with every element of}~G\iff x\in Z(G)
  .\] Choose elements $x_1,\ldots,x_k\in X$ that give the distinct orbits. Then the Orbit-Stabilizer
  Counting Theorem tells us that \[
    \left| X \right| =\sum_{i=1}^{k} \frac{\left| G \right| }{\left| G_{x_i} \right| }
  .\] We have $\left| X \right| =\left| G \right| =p^n$, and each $G_{x_i}$ is a subgroup of $G$, so
  Lagrange's Theorem tells us that each $\left| G_{x_i} \right| $ divides $\left| G \right|
  $. Thus, for each $i$ we have \[
    \left| G_{x_i} \right| =p^{r_i},\ 0\le r_i\le n
  ,\] so the summation above becomes \[
    \left| X \right|=p^n =\sum_{i=1}^{k} \frac{p^n}{p^{r_i}}=\sum_{i=1}^{k} p^{n-r_i}
  .\] Terms with $r_i=n$ are especially interesting, since \[
    r_i=n\iff x_i~\text{commutes with every element of}~G\iff x_i\in G
  .\] So, if we separate out the summation above, we get the following formula: \[
    p^n=\underbrace{\sum_{i=0}^{k} 1}_\text{with $r_i=n$}+\underbrace{\sum_{i=0}^{k}
    p^{n-r_i}}_\text{with $r_i<n$}=\left| Z(G) \right| +\sum_{i=0}^{k} p^{n-r_i}
  .\] But we know that $\left| Z(G) \right| \ge 1$, since the identity element of $G$ is in $Z(G)$.
  Hence $\left| Z(G) \right| \ge p$ (indeed, the number of elements in $Z(G)$ is a power of
  $p$!).
\end{proof}

Using this, we can prove a surprising theorem about groups having $p^2$ elements.
\begin{corollary}{}
  Let $p$ be prime, and let $G$ be a group with $p^2$ elements. Then $G$ is an Abelian group.
\end{corollary}
\begin{proof}[Proof]
  For notational purposes, let \[
    Z=Z(G)
  .\] Lagrange tells us that the order of the center $Z$ divides $\left| G \right| =p^2$, so \[
    \left| Z \right| =1,\ p,~\text{or}~p^2
  .\] From above, we see that $ Z \neq \{ e \}$, so $\left| Z \right| \neq 1$.

  Now, assume $\left| Z \right| =p$. Since the center $Z$ is a normal subgroup of $G$, we can thus
  take the quotient group $G / Z$, and Lagrange tells us that \[
    \left| G / Z \right| = \frac{\left| G \right| }{\left| Z \right| }=\frac{p^2}{p}=p
  .\] Thus $G / Z$ has prime order, and so is cyclic. Let $hZ$ be a coset that generates $G / Z$, \[
    G / Z=\{ Z,hZ,h^2Z,\ldots,h^{p-1}Z \}
  .\] In particular, this implies that \[
    G=Z\cup hZ\cup \ldots\cup h^{p-1}Z
  ,\] since every element of $G$ is in some coset of $Z$.

  Now, let $g_1,g_2\in G$ be arbitrary elements. They are each in some coset of $Z$, so \[
    g_1=h^{i_1}z_1~\text{and}~g_2=h^{i_2}z_2~\text{for some}~z_1,z_2\in Z~\text{and}~0\le i_1,i_2\le
    p-1
  .\] Since $z_1,z_2\in Z$, they commute with every element of $G$; thus
  \begin{align*}
    g_1g_2=(h^{i_1}z_1)(h^{i_2}z_2)=(h^{i_1}h^{i_2})(z_1z_2)&= h^{i_1+i_2}z_2z_1 \\
    &= (h^{i_2}h^{i_1})(z_2z_1)=(h^{i_2}z_2)(h^{i_1}z_1)=g_2g_1
  .\end{align*}
  But then this shows that every element commutes with every other element; that is, $Z=G$; but we
  assumed that $\left| Z \right| =p\neq p^2=\left| G \right| $, a contradiction.

  Hence the order of $Z$ cannot be either $1$ or $p$, and so must be $p^2$. In other words,
  $Z=G$---every element commutes with every other element---and so $G$ is Abelian.
\end{proof}

\begin{remark}
  \textbf{Some notes on Commutativity}: Commutativity is King! While that doesn't mean we should
  only study Abelian groups, whenever we're given a group $G$, we should exploit whatever
  commutative Ty we can find. For example, given an element $h\in G$, it's worthwhile to inspect
  elements of $G$ that commute with $h$. These elements satisfy \[
    gh=hg,~\text{or equivalently}~, g^{-1}hg=h
  .\] More generally, we could take a subgroup $H\subset G$ and look at the elements $g\in G$ that
  commute with every element in $H$.

  Alternatively, we can take a weaker commutativity condition: instead of insisting that
  $g^{-1}hg=h$ for every $h\in H$, we could require only that $g^{-1}hg\in H$, not that it equals
  $h$. 
\end{remark}

The above remark provides the motivation for two important definitions.
\begin{definition}[Centralizer]{}
  Let $G$ be a group, and let $H\subseteq G$ be a subgroup of $G$. The \textbf{centralizer of $H$},
  denoted $Z(H)$ or $Z_G(H)$, is the set of elements of $G$ that commute with every element of $H$,
  \[
    Z_G(H)=\{g\in G\mid gh=hg~\text{for every}~h\in H\} 
  .\] For $h\in G$, we write $Z_G(h)$ for the centralizer of the cyclic subgroup $\left<h \right>$
  generated by $h$.
\end{definition}
\begin{definition}[Normalizer]{}
  Let $G$ be a group, and $H\subseteq G$ a subgroup of $G$. We define the \textbf{normalizer of $H$}
  to be \[
    N_G(H)=\{g\in G\mid g^{-1}Hg=H\} 
  .\] For example, the definition of normal subgroup says that \[
    N_G(H)=G \iff H ~\text{is a normal subgroup of}~G
  .\] Verifying that $Z_G(H)$ and $N_G(H)$ are subgroups is left as an exercise.
\end{definition}


\section{Sylow's Theorem, Part I}

We just proved that groups of $p$-power order have a non-trivial center. Now, we prove part of a
fundamental theorem regarding subgroups of $p$-power order, Sylow's Theorem. Sylow's Theorem is
incredibly important in the study of finite groups.

Let $G$ be a group, and $H\subseteq G$ a subgroup. Recall by Lagrange that the order of $H$ divides
the order of $G$. It would be nice if the converse were true; that is, if $m$ divides $\left| G
\right| $, then there exists a subgroup $H\subseteq G$ of order $m$. Unfortunately, this is not true
in general; however, it turns out that this holds if $m$ is of prime power.

\begin{theorem}[Sylow's Theorem: Part I]{}
  Let $G$ be a finite group, let $p$ be prime, and let $p^n$ be the largest power of $p$ that
  divides $\left| G \right| $. Then $G$ has a subgroup of order $p^n$.
\end{theorem}
\begin{proof}[Proof]
  We're given that $p^n$ is the largest power of $p$ that divides $\left| G \right| $, which means
  we can factor \[
    \left| G \right| =p^nm ~\text{with}~p\nmid m
  .\] We proceed with induction on $m$. If $m=1$, then $\left| G \right| =p^n$, so $G$ itself is the
  desired subgroup; so suppose $m\ge 2$, and that the theorem holds for groups of order $p^nm'$ with
  $m'<m$.

  How might we create a subgroup of $G$ with $p^n$ elements? Suppose we take a sub\underline{set}
  $A\subseteq G$ with $p^n$ elements. How might we tell that $A$ is a sub\underline{group}? We first
  observe that \[
    A~\text{is a subgroup}~\iff aA=A~\text{for every}~a\in A
  \] (verify this!). This suggests that we look at the collection of $p^n$-element subsets of $G$
  and let $G$ act on these subsets by left multiplication.

  In other words, we start with the set \[
    S=\{A\subseteq G\mid A~\text{is a subset of $G$ with}~\left| A \right| =p^n\} 
  .\] Note the structure of $S$; $S$ is a \textbf{set of sets}. For example, given a group $G=\{
  g_1,g_2,g_3,g_4 \}$ and $p^n=2$, then \[
    S=\{ \{ g_1,g_2 \}, \{ g_1,g_3 \}, \{ g_1,g_4 \}, \{ g_2,g_3 \}, \{ g_2,g_4 \}, \{ g_3,g_4 \} \}
  .\] How many subsets of $p^n$ are contained in $G$? Since we want to choose $p^n$ elements out of
  $p^nm$ total elements, the number of elements in $S$ is given by \[
    \left| S \right| =\binom{p^nm}{p^n}
  .\] Now, let $G$ act on $S$ by left-multiplication. Equivalently, for $g\in G$ and $A\in S$, we
  define \[
    gA=\{g\cdot a\mid a\in A\} \in S
  \] (since the resulting coset $gA$ also has $p^n$ elements, so must be in $S$; additionally, $gA$
  is still a subset of $G$). Choose elements $A_1,\ldots,A_r$ of $S$ so that $GA_1,\ldots,GA_r$ form
  distinct orbits. With the Orbit-Stabilizer Counting Theorem, we thus have \[
    \left| S \right| =\sum_{i=1}^{r} \frac{\left| G \right| }{\left| G_{A_i} \right| }
  ,\] where $G_{A_i}=$ (the stabilizer of $A_i$) = $\{g\in G\mid gA_i=A_i\} $. Substituting the
  sizes of $S$ and $G$, we then get \[
    \binom{p^nm}{p^n}=\sum_{i=1}^{r} \frac{p^nm}{\left| G_{A_i} \right| }
  .\] We'll now use a fact (proven later) that \[
    \binom{p^nm}{p^n}~\text{is not divisible by }~p
  .\] It follows that the sum $\sum_{i=1}^{r} \frac{p^nm}{\left| G_{A_i} \right| }$ is not divisible
  by $p$, and since each $\frac{p^nm}{\left| G_{A_i} \right| }$ is an integer, it must be the case
  that at least one of these integers is not divisible by $p$ (why? Because otherwise, every
  $\frac{p^nm}{\left| G_{A_i} \right| }$ would still have a $p$ term, and thus the entire sum would
  still be divisible by $p$). Hence there exists at least one $A_j\in S$ with the property that
  $\left| G_{A_i} \right| $ is a multiple of $p^n$, since $\left| G_{A_i} \right| $ needs to cancel
  out the entire $p^n$ in the numerator.

  \-\hspace{2em} In other words, we have shown that there exists a subset $A\subseteq G$ with
  $\left| A \right| =p^n$ (by construction of $A\in S$) such that the stabilizer $G_{A}$ of $A$ has
  order \[
    \left| G_A \right| =p^nm' ~\text{with}~m'|m
  .\] We consider two cases. If $m'<m$, then the induction hypothesis says that $G_A$ has a subgroup
  $H$ with order $p^n$. But $H$ is thus also a subgroup of $G$, so we're done.

  So, suppose $m'=m$. This assumption means that $G_A$ has the same number of elements as $G$, so
  $G_A=G$. By definition of a stabilizer $G_A$, this means that \[
    gA=A~\text{for every element}~g\in G
  .\] The set $A$ has $p^n$ elements, so it must be non-empty. Let $a\in A$ be some element of $A$.
  Then for every $g\in G$, we have \[
    g=(ga^{-1})a\in (ga^{-1})A=A
  \] (we use $ga^{-1}$ instead of just $g$). This shows that every element of $G$ is in $A$, so
  $G=A$, so $\left| G \right| =\left| A \right| =p^n$. But $G$ has order $p^nm$ with $m\ge 2$, a
  contradiction. 

  Therefore $m'<m$, and so $G$ has a subgroup of order $p^n$.
\end{proof}

We now prove a lemma about the divisibility of the binomial coefficient, that proved essential in
Sylow's Theorem.
\begin{lemma}{}
  Let $p$ be a prime, let $n\ge 0$, and let $m\ge 1$ with $p\nmid m$. Then the binomial coefficient
  $\binom{p^nm}{p^n}$ is not divisible by $p$.
\end{lemma}
\begin{proof}[Proof]
  The binomial coefficient is equal to \[
    \binom{p^nm}{p^n}=\frac{(p^nm)!}{p^n!(p^nm-p^n)!}=\frac{p^nm(p^nm-1)\ldots(p^nm-p^n+1)}{p^n(p^n-1)\ldots\cdot
    3\cdot 2\cdot 1}
  ,\] so it is equal to \[
    \binom{p^nm}{p^n}=\prod_{r=0}^{p^n-1} \frac{p^nm-r}{p^n-r} 
  .\] We claim that after simplifying any of the fractions $\frac{p^nm-r}{p^n-r}$, there are no $p$
  factors left in the numerator. With $r=0$, this is clear: since we're given that $p\nmid m$, and
  $\frac{p^nm}{p^n}=m$, that fraction is kosher.

  For the others, take any $r$ between $1$ and $p^n-1$, and factor it into \[
    r=p^is~\text{with}~0\le i<n~\text{and}~p\nmid s
  ;\] we can do this since if no power of $p$ divides $r$, then clearly for $r=s$, $p\nmid s$, and we
  simply choose $i=0$. Otherwise, we divide $r$ by the highest power of $p$ possible, and thus the
  remaining $\frac{r}{p^i}=s$ is not divisible by $p$.

  Then \[
    \frac{p^nm-r}{p^n-r}=\frac{p^nm-p^is}{p^n-p^is}=\frac{p^{n-i}m-s}{p^{n-i}-s}
  .\] Since $i<n$ and $p\nmid s$, we see that neither $p^{n-i}m-s$ nor $p^{n-i}-s$ is divisible by
  $p$. Thus, we can cancel out all of the $p$-factors in the numerator by all of the $p$-factors in
  the denominator, so for any $1\le r\le p^n-1$, the fraction $\frac{p^nm-r}{p^n-r}$ has no
  $p$-factors. Therefore, the product \[
    \binom{p^nm}{p^n}=\prod_{r=0}^{p^n-1} \frac{p^nm-r}{p^n-r} 
  \] has no factors of $p$, and thus is not divisible by $p$.
\end{proof}

Using Sylow's Theorem, we now classify subgroups further:
\begin{definition}[Sylow Subgroups]{}
  Let $G$ be a finite group, let $p$ be a prime, and let $p^n$ be the largest power of $p$ that
  divides $\left| G \right| $. A subgroup $H\subseteq G$ with $\left| H \right| =p^n$ is called a
  \textbf{$p$-Sylow subgroup of $G$}. Part I of Sylow's Theorem tells us that $G$ has at least one
  Sylow subgroup.
\end{definition}
\begin{remark}
  We will see later that it is true, more generally, that if $p^r$ divides $\left| G \right| $, then
  $G$ has a subgroup of size $p^r$, even if $p^r$ is not the largest power of $p$ dividing $\left| G
  \right| $. [TODO: import exercise solution here]
\end{remark}
\begin{remark}
  A useful observation is that if $p$ and $q$ are two distinct primes dividing $\left| G \right| $,
  and if we take a $p$-Sylow subgroup $H_p$ and a $q$-Sylow subgroup $H_q$, then $H_p\cap H_q=\{ e
  \}$. This actually follows immediately from Lagrange's Theorem, since if $H$ and $H'$ are any two
  subgroups of $G$, then $H\cap H'$ is a subgroup of both $H$ and $H'$, so Lagrange tells us that \[
    \left| H\cap H' \right| ~\text{divides}~ \gcd{(\left| H \right| ,\left| H' \right| )}
  .\] However, in our situation $gcd(\left| H \right|,\left| H' \right| )=\gcd{(p^n,q^m)}=1$, so
  $\left| H_p\cap H_q \right| =1$.
\end{remark}

\begin{remark}
  In general, if $p_1,\ldots,p_r$ are the distinct primes that divide the order of $G$, then Sylow's
  Theorem tells us that the $p$-Sylow subgroups of $G$ are large; indeed, they satisfy \[
    \left| G \right|=\left| H_{p_1} \right|  \cdot \ldots\cdot \left| H_{p_r} \right| 
  .\] This suggests that one way to understand a finite group $G$ is to start with its various
  $p$-Sylow subgroups, and then see how they might fit together. However, if you start with Sylow
  subgroups, there may be multiple ways to combine them. For example, the cyclic subgroups
  $\mc{C}_2$ and $\mc{C}_3$ are the Sylow subgroups of both $\mc{C}_6$ and $\mc{S}_3$.
\end{remark}

We now provide the complete statement of Sylow's Theorem; we postpone the proof until later, when
we've explored some additional results that will aid in our proof.
\begin{theorem}[Sylow's Theorem]{}
  Let $G$ be a finite group, and $p$ prime.
  \begin{enumerate}
    \item $G$ has at least one $p$-Sylow subgroup, i.e. there is a subgroup $H\subseteq G$ with
      $\left| H \right| =p^n$, where $p^n$ is the largest power of $p$ that divides $\left| G
      \right| $.
    \item Let $H_1$ and $H_2$ be $p$-Sylow subgroups of $G$. Then $H_1$ and $H_2$ are conjugate,
      i.e. for some $g\in G$, $H_2=g^{-1}H_1g$.
    \item Let $H$ be a $p$-Sylow subgroup of $G$, and let $k$ be the number of distinct $p$-Sylow
      subgroups of $G$. Then \[
        k|\left| G \right| ~\text{and}~k\equiv 1\mod{p}
      .\] (In fact, the divisibility follows a more precise formula: $k\left| N_G(H) \right| =\left|
      G\right| $.)
  \end{enumerate}
\end{theorem}

Let's see how Sylow's Theorem can be used to analyze or classify groups based on their order. In
particular, we can exploit the fact that the number of $p$-Sylow subgroups is simultaneously a
divisor of $\left| G \right| $ and is congruent to $1$ modulo $p$. This is quite a strict condition.
\begin{example}
  Let $G$ be a group of order $10$, so $G$ has $2$-Sylow subgroups and $5$-Sylow subgroups. Let's
  look first at the case of $5$-Sylow subgroups of $G$. Let $k$ be the number of distinct $5$-Sylow
  subgroups. Sylow's Theorem tells us that \[
    k | 10 ~\text{and}~k\equiv 1\mod{5}
  .\] This forces $k=1$; in other words, $G$ has a unique $5$-Sylow subgroup $H_5$. It follows that
  $H_5$ is normal, since for any $g\in G$, the conjugate $g^{-1}H_5g$ is also a subgroup (see
  \ref{pnsg}) of order $5$, so equals $H_5$.

  Next, we use the fact that there is also at least one $2$-Sylow subgroup, say $H_2$. As noted
  before, $H_2\cap H_5=\{ e \}$, so we can write \[
    H_2=\{ a,e \}~\text{and}~H_5=\{ e,b,b^2,b^3,b^4 \}
  \] (since prime order implies cyclic), where the only common element is $e$.

  Let's look at the extent to which $a$ and $b$ commute: $aba^{-1}$. We have \[
    aba^{-1}\in aH_5a^{-1}=H_5,~\text{since $H_5$ is normal}~
  .\] Hence $aba^{-1}=b_j$ for some $0\le j\le 4$.

  In order to determine the value of $j$, we compute
  \begin{align*}
    aba^{-1}=b^j \implies b&= a^{-1}b^ja \\
                           &=(a^{-1}ba)^j && [\text{since we can cancel out most of $a^{-1}a=e$
                           products}]\\
                           &= \left( a^{-1}(a^{-1}b^ja)a \right) ^j\\
                           &= (a^{-2}b^ja^2)^j\\
                           &= a^{-2}b^{j^2}a^2 &&[\text{again, we have a lot of $a^{-2}a^2=e$
                           cancellation}]\\
                           &=b^{j^2} &&[\text{since $a$ has order $2$, so $a^2=e$}]
  .\end{align*}
  In other words, $b=b^{j^2}$, so $b^{j^2-1}=e$. Since $b$ has order $5$, this means \[
    j^2-1\equiv 0\mod{5}\iff j^2\equiv 1\mod{5}
  .\] Inspection then reveals that either $j=1$ or $j=4$.

  Suppose first that $j=1$, and recall that $aba^{-1}=b^j$. This means that $ab=ba$, and since every
  element of $G$ is a power of $a$ times a power of $b$, this implies that $G$ is Abelian (to see
  why this is true, recall that $a\neq b^i$ for any power of $b$; that means $ab^i\not\in H_5$; thus
  the two cosets of $H_5$, which are composed of $b$ to some power times $a$ to some power, is the
  disjoint union of $G$)! It is then easy to check that $ab$ has order $10$:
  \begin{align*}
    e=(ab)^k=a^kb^k&\implies a^k=b^{-k}\in H_2\cap H_5=\{ e \}\\
                   &\implies a^k=b^k=e\\
                   &\implies 2|k~\text{and}~5|k\\
                   &\implies 10|k
  .\end{align*} Hence if $j=1$, then $G$ is a cyclic group of order $10$
  (we get the first inclusion statement since $b^{-k}\in H_5$, and if $b^{-k}=a^k\in H_2$, then
  $b^{-k}\in H_2$ as well).

  For the other case of $j=4$, that means that $ab=b^4a$. But $b^{5}=e$, so $b^4=b^{-1}$, which
  means that the group $G$ consists of the $10$ elements $a^ib^j$ with $0\le i\le 1$, $0\le j\le 4$
  (see above for why), and the group law is determined by the rules \[
    a^2=e,b^5=e,ba=ab^{-1}
  .\] Thus $G$ is simply the dihedral group $\D_5$ of order $10$, i.e. $G$ is isomorphic to the
  symmetry group of a regular pentagon. 

  In summary, there are only two groups of order $10$; specifically, if a group has order $10$, then
  it is either isomorphic to the cyclic group $\mc{C}_{10}$ or the dihedral group $\D_5$.
\end{example}

\begin{example}
  Let $G$ be a finite group of order $\left| G \right| =pq$, where $p$ and $q$ are distinct primes;
  say without loss of generality that $p>q$. Let $H$ be a $p$-Sylow subgroup of $G$. We claim that
  $H$ is a normal subgroup of $G$. We look at the normalizer $N_G(H)$ of $H$. It certainly contains
  $H$, which has $p$ elements, so $p|N_G(H)$ (by Lagrange). But $N_G(H)$ is a subgroup of $G$, so
  $\left| N_G(H) \right|$ divides $\left| G \right| =pq$. Thus $\left| N_G(H) \right| $ is an
  integer divisible by $p$ that divides $pq$, so either \[
    \left| N_G(H) \right| =p (\text{so}~N_G(H)=H),~\text{or}~\left| N_G(H) \right| =pq
    (\text{so}~N_G(H)=G)
  .\] If $N_G(H)=G$, then $H$ is a normal subgroup, and we are done (by definition of normalizer).
  On the other hand, if $N_G(H)=H$, then from Sylow's Theorem (c), we see that \[
    1\equiv\frac{\left| G \right| }{\left| N_G(H) \right| }=\frac{pq}{q}=q\mod{p}
  ,\] so $p|q-1$. But this contradicts the assertion that $p>q$, so this is not possible. Hence $H$
  is a normal subgroup of $G$.
\end{example}





































\end{document}
