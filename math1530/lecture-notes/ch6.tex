\documentclass[math1530-lecture-notes]{subfiles}
\begin{document}

\chapter{Groups: Part II}

We now continue our study of group theory. POGGIES!!!

\section{Normal Subgroups and Quotient Groups}

Let $G$ be a group, and let $H$ be a subgroup. Recall that each $g\in G$ gives a left coset of $H$,
defined by \[
  gH = \{gh\mid h\in H\} 
.\] We now define the collection of cosets of $H$.

\begin{definition}[Set of Cosets]{}
  Let $G$ be a group, with $H<G$ a subgroup. We denote the set of (left) cosets of $H$ in $G$ by \[
    G / H = \{ ~\text{(left) cosets of}~ H\}
  .\] 
\end{definition}

Let $\mc{C}_1, \ldots,\mc{C}_k$ be distinct cosets of $H$. Proposition 2.39 tells us that $G$ is
equal to the disjoint union \[
  G = \mc{C}_1\cup \ldots\cup \mc{C}_k
\] We use the notation $\mc{C}_i$ to emphasize that for a given coset $\mc{C}$, many different
elements $g\in G$ will satisfy $\mc{C}=gH$. Indeed, (as one should check) if $\mc{C}$ is a coset of
$H$, then \[
  \mc{C}=gH \iff g\in \mc{C}
.\] 

Now, is there a way to turn this collection of cosets $\{ \mc{C}_1,\ldots,\ldots,\mc{C}_k \}$ into a
group? If so, how should we define the product of two cosets $\mc{C}_i$ and $\mc{C}_j$? One
intuitive method may be to take a product of two elements, one from $\mc{C}_i$ and one from
$\mc{C}_j$, and declare the resulting set $\mc{C}_i\cdot \mc{C}_j$ to be the coset of the product.

\begin{definition}[Proposed Coset Multiplication]{}
  Let $G$ be a group, and $H < G$ a subgroup. Given cosets $\mc{C}_i, \mc{C}_j$ of $H$, select one
  element from each ($g_1\in \mc{C}_i$, $g_2\in \mc{C}_j$), and define the resultant coset
  $g_1g_2H$.
\end{definition}

Is this a good definition? Where might this go wrong?

Well, if we have three cosets of a group $G$, and we select one element from each of two cosets,
there's a possibility that if we select different elements from each of the cosets, the first
product might not be in the same coset as the second product. Concretely, consider the dihedral
group of three vertices, $\D_3$. The subgroup $\{ e, \phi_1 \}$ (where $\phi_i$ are flips, and
$\rho_j$ are rotations) has three cosets, \[
  \mc{C}_1 = \{ e,\phi_1 \},\ \mc{C}_2=\{ \rho_1,\phi_2 \},\ \mc{C}_3=\{ \rho_2,\phi_3 \}
.\] If we select $\phi_1\in \mc{C}_2,\ \phi_2\in \mc{C}_3$, then we get $e\in \mc{C}_1$; but if we
select $\phi_2\in \mc{C}_2,\ \phi_3\in \mc{C}_3$, then we get $\rho_2\in \mc{C}_3$. The above
definition thus gives us two different results: either $\mc{C}_2\cdot \mc{C}_3=\mc{C}_1$, or
$\mc{C}_2\cdot \mc{C}_3=\mc{C}_3$!

However, the above definition \textit{does} work if we select our subgroup $H$ more carefully; for
instance, if we chose $H=\{ e,\rho_1,\rho_2 \}$, then the definition seems to work! So the
definition works for some, but not all, subgroups. How do we distinguish the ``good'' subgroups? We
want the subgroups to have the following property: \begin{center}
  For all cosets $\mc{C}_1, \mc{C}_2$ of $H$, for all pairs of elements $g_1,g_1'\in \mc{C}_1,\
  g_2,g_2'\in \mc{C}_2$, we wish for \[
    g_1g_2H=g_1'g_2'H
  .\] 
\end{center}

Let's see where this takes us. $g_1$ and $g_1'$ in the same coset $\mc{C}_1$ of $H$ means that there
is some $h_1\in H$ with $g_1'=g_1h_1$ (since all elements of a coset of $H$ can be represented as
$g_1h_1$, where $g_1$ is in the coset, and $h_1$ is in the subgroup), and similarly $g_2'=g_2h_2$
for some $h_2\in H$. So, we want \[
  g_1'=g_1h_1~\text{and}~g_2'=g_2h_2 \implies g_2^{-1}g_1^{-1}g_1'g_2'\in H
\] (since we want $g_1'g_2'=g_1g_2h_1h_2$ in order for $g_1'g_2'\in g_1g_2H$; or equivalently,
$g_2^{-1}g_1^{-1}g_1'g_2'\in H$). Substituting values of $g_1', g_2'$, we want \[
  g_2^{-1}g_1^{-1}g_1h_1g_2h_2\in H ~\text{for all}~g_1,g_2\in G,\ h_1,h_2\in H
.\] With cancellation, this becomes \[
  g_2^{-1}h_1g_2h_2\in H~\text{for all}~g_2\in G,\ h_1,h_2\in H
.\] But we know that $g_2h_2\in H$ if and only if $g_2\in H$, so we end up with \[
  g_2^{-1}h_1g_2\in H~\text{for all}~g_2\in G,\ h_1\in H
.\] Dropping subscripts, we now get the following definition:
\begin{definition}[Normal Subgroups]{}
  Let $G$ be a group, let $H < G$ be a subgroup, and let $g\in G$. The \textbf{$g$-conjugate of $H$}
  is the subgroup \[
    g^{-1}Hg=\{g^{-1}hg\mid g\in G\} 
  .\] We say that $H$ is a \textbf{normal subgroup} if it satisfies \[
    g^{-1}Hg=H~\text{for every }~g\in G
  .\] 
\end{definition}

\begin{example}
  If $G$ is an Abelian group, then every subgroup $H\subseteq G$ is normal, since \[
    g^{-1}hg=g^{-1}gh=h
  .\] 
\end{example}

\begin{definition}[Simple Groups]{}
  Every group $G$ has two normal subgroups, $\{ e \}$ and $G$. If these are the only two normal
  subgroups, then $G$ is called a \textbf{simple group}.
\end{definition}

\begin{example}
  As shown above, the subgroup $H=\{ e,\phi_1 \}$ is not a normal subgroup, since for example \[
    \phi_2^{-1}\{ e,\phi_1 \}\phi_2=\{ \phi_2^{-1}e\phi_2,\phi_2^{-1}\phi_1\phi_2 \}=\{ e,\phi_3 \}
  .\] However, the other subgroup $H=\{ e,\rho_1,\rho_2 \}$ is a normal subgroup. One could
  tediously check all conjugates, or realize that three rotations is a rotation, or a flip,
  rotation, flip is still a rotation. Similar results hold for $\D_n$.
\end{example}

Here is an important source of normal subgroups. Indeed, we will later show that every normal
subgroup of a group $G$ arises in this way.
\begin{proposition}
  Let $\phi:G\to G'$ be a homomorphism of groups. Then $\ker{(\phi)}$ is a normal subgroup of $G$.
\end{proposition}
\begin{proof}[Proof]
  We know from before that $\ker{(\phi)}$ is a subgroup of $G$. Let $h\in \ker{(\phi)}$, $g\in G$.
  Then
  \begin{align*}
    \phi(g^{-1}\cdot h\cdot g)&= \phi(g^{-1}) \cdot \phi(h)\cdot \phi(g)\\
                              &=\phi(g)^{-1}\cdot \phi(h)\cdot \phi(g)\\
                              &=\phi(g)^{-1}\cdot \phi(g) && [~\text{since $h\in \ker{(\phi)}$, so
                              $\phi(h)=e'$}~]\\
                              &= e'
  .\end{align*}
  Hence $g^{-1}\cdot h\cdot g\in \ker{(\phi)}$; since the choice of $g,h$ were arbitrary,
  $\ker{(\phi)}$ is thus a normal subgroup of $G$.
\end{proof}

Before turning the collection of cosets $G / H$ into a group, we now give three elementary
properties of normal subgroups. 
\begin{proposition}[Properties of Normal Subgroups]{}
  Let $G$ be a group, and $H < G$ a subgroup.
  \begin{enumerate}
    \item If $g^{-1}Hg\subseteq H$, then $H$ is a normal subgroup of $G$; in other words, it is
      enough only to check one inclusion.
    \item For all $g\in G$, the conjugate set $g^{-1}Hg$ is a subgroup of $G$.
    \item For all $g\in G$, the map $H\to g^{-1}Hg$ defined by $h\mapsto g^{-1}hg$ is a group
      isomorphism. In particular, if $H$ is finite, then $H$ and its conjugates have the same number
      of elements.
  \end{enumerate}
\end{proposition}
\begin{proof}[Proof]
  Left as an exercise.
\end{proof}

We now go back to our goal of turning the set of cosets $G / H$ into a group via the multiplication
rule \[
  g_1H\cdot g_2H=g_1g_2H
.\] Unfortunately, as we've seen before, choosing different elements $g_1'\in g_1H,\ g_2'\in g_2H$
may yield a different coset $g_1'g_2'H$. However, if $H$ is a \textit{normal} subgroup of $G$, then
we do actually get the same coset. Yay! Let us formally verify this.

\begin{lemma}{}
  Let $G$ be a group, and let $H<G$ be a normal subgroup. Let $g_1,g_1',g_2,g_2'\in G$ be elements
  of $G$ satisfying \[
    g_1H=g_1'H ~\text{and}~g_2H=g_2'H
  .\] Then \[
    g_1g_2H=g_1'g_2'H
  .\] 
\end{lemma}
\begin{proof}[Proof]
  The assumption that $g_1H=g_1'H$ implies in particular that \[
    g_1'\in g_1'H=g_1H, ~\text{so}~g_1'=g_1h_1 ~\text{for some}~h_1\in H
  .\] Similarly, \[
    g_2'\in g_2'H=g_2H \implies g_2'=g_2h_2~\text{for some}~h_2\in H
  .\] We now wish to prove the inclusion $g_1'g_2'H\subseteq g_1g_2H$, so let $g_1'g_2'h$ be an
  element of $g_1'g_2'H$. We want \[
     g_1'g_2'h\subseteq g_1g_2H
  .\] From above, we see that \[
     g_1'g_2'h=g_1h_1g_2h_2h
   .\] If we could flip $h_2$ and $g_2$, we are done (since we are left with $g_1g_2$ times an
   element of $H$); but alas, $G$ is not necessarily commutative, so we must look elsewhere.

   We want the leftmost $g_1$ to be followed by $g_2$; we can't just insert a $g_2$ there, but we
   can use the standard mathematical trick of putting the quantity where we want, and multiplying it
   to cancel it out: we do this by multiplying by $e=g_2g_2^{-1}$. So, we have \[
     g_1h_1g_2h_2h=g_1g_2g_2^{-1}h_1g_2h_2h=(g_1g_2)(g_2^{-1}h_1g_2)h_2h
   .\] But $g_2^{-1}h_1g_2\in H$, since $H$ is a normal subgroup; and closure implies $h_2h\in H$.
   Thus, we have \[
     g_1'g_2'h=g_1h_1g_2h_2h=g_1g_2(~\text{three elements in $H$}~)\in g_1g_2H
   ,\] and hence that \[
     g_1'g_2'H\subseteq g_1g_2H
   .\] Reversing the roles gives an analogous argument for $g_1g_2H\subseteq g_1'g_2'H$, and thus we
   get \[
     g_1g_2H=g_1'g_2'H
   .\] 
\end{proof}

Essentially, this lemma demonstrates that the multiplication rule for cosets, $g_1H\cdot
g_2H=g_1g_2H$, is well-defined if $H$ is a normal subgroup of $G$. We are now able to turn $G / H$
into a group!

\begin{theorem}[First Isomorphism Theorem For Groups]{}
  Let $G$ be a group, and let $H$ be a normal subgroup of $G$.
  \begin{enumerate}
    \item The collection of cosets $G / H$ is a group via the well-defined group operation \[
        g_1H\cdot g_2H=g_1g_2H
      .\] 
    \item The map \[
        \phi: G\longrightarrow G / H,\ \phi(g)=gH
      ,\] is a homomorphism whose kernel is $\ker{(\phi)}=H$.
    \item Let \[
      \psi:G\longrightarrow G'
      \] be a homomorphism with the property that $H\subseteq \ker{(\psi)}$. Then there is a unique
      homomorphism \[
        \lambda: G / H \longrightarrow G'~\text{satisfying}~\lambda(gH)=\psi(g)
      .\] 
    \item If we take $H=\ker{(\psi)}$ from above, then the homomorphism \[
        \lambda:G / \ker{(\psi)}\longrightarrow G'
      \] is injective. In particular, we get an isomorphism onto the image of $\lambda$, \[
        \lambda: G / \ker{(\psi)} \longrightarrow \lambda(G)\subseteq G'
      .\] 
  \end{enumerate}
\end{theorem}

\begin{proof}[Proof]
  \begin{enumerate}
    \item The above lemma proves that coset multiplication is well-defined (with a normal subgroup
      $H$). The other group properties follow directly from the corresponding properties of the
      group operation on $G$:
      \begin{gather}
        eH\cdot gH=gH\cdot eH=gH\\
        gH\cdot g^{-1}H=g^{-1}H\cdot gH=eH\\
        (g_1H\cdot g_2H)\cdot g_3H=g_1H\cdot (g_2H\cdot g_3H)
      .\end{gather} Thus $G / H$ is a group under coset multiplication.
    \item We first check that $\phi$ is a homomorphism. Note that \[
        \phi(g_1)\phi(g_2)=g_1H\cdot g_2H=g_1g_2H=\phi(g_1g_2)
      .\] Thus $\phi$ is a homomorphism. The kernel of $\phi$ is \[
        \ker{(\phi)}=\{g\in G\mid \phi(g)=H \} =\{g\in G\mid gH=H\} =H
      .\] 
    \item We want to define a homomorphism $\lambda:G / H \to G'$ by the following algorithm:
      \begin{enumerate}
        \item Let $\mc{C}\in G / H$ be a coset of $H$.
        \item Choose some $g\in \mc{C}$ with $\mc{C}=gH$.
        \item Define $\lambda(\mc{C})$ to be $\psi(g)=g'$; that is, $\psi(g)=\lambda(gH)=g'$.
      \end{enumerate}
      However, step 2 may pose a problem, since many possible $g\in \mc{C}$ can work. So, we need to
      prove that if $gH=g'H$, then $\psi(g)=\psi(g')$.

      $gH=g'H$ means that $g'=gh$ for some $h\in H$. Then
      \begin{align*}
        \psi(g')&= \psi(gh) \\
                &= \psi(g)\psi(h)\\
                &=\psi(g)e' && [~\text{since $H\subseteq \ker{(\psi)}$}~]\\
                &=\psi(g)
      ,\end{align*} as required. Thus our algorithm gives a well-defined map \[
        \lambda: G / H \longrightarrow G'
      .\] It's also easy to check that $\lambda$ is a homomorphism: for $g_1,g_2\in G$, \[
        \lambda(g_1g_2H)=\psi(g_1g_2)=\psi(g_1)\psi(g_2)=\lambda(g_1H)\lambda(g_2H)
      .\] Finally, for any given homomorphism $\psi:G\to G'$, there's only one map $\lambda$
      satisfying $\psi(g)=\lambda(gH)$, since this equality completely determines the values of
      $\lambda$ in terms of values of $\psi$; that is, for any $\lambda$ that maps $G / H$ to $G'$,
      every $gH\in G / H$ must be mapped to the same $\psi$.
    \item Let $H=\ker{(\psi)}$. From (c) we get that \[
        \lambda: G / H\longrightarrow G',\ \lambda(gH)=\psi(g)
    ;\] we just need to show that $\lambda$ is injective.

    Let $gH\in ker(\lambda)$. Then $\lambda(gH)=\psi(g)=e'$, so $g\in \ker{(\psi)}=H$. Therefore
    $gH=H$, which is the identity element of $G / H$. Thus, the kernel $\ker{(\lambda)}$ is trivial
    and consists of only the identity element, so $\lambda$ is injective. Surjectivity follows by
    definition of a function; any $\lambda$ surjects onto its image. Thus $\lambda: G/H\to G'$ is an
    is an isomorphism.
  \end{enumerate}
\end{proof}

\section{Groups Acting On Sets}

Among the first groups studied were the symmetric groups $\mc{S}_n$, and the dihedral groups $\D_n$.
The elements of $\mc{S}_n$ were permutations of $\{ 1,\ldots,n \}$, so one could view an element of
$\mc{S}_n$ as giving a rule that takes a number in $\{ 1,\ldots,n \}$ and assigns it to another
number in $\{ 1,\ldots,n \}$. Similarly, the elements of $\D_n$ are rigid re-arrangements of
vertices of an $n$-gon, so we can also view elements of $\D_n$ as assigning a rule that takes the
vertex of an $n$-gon to another vertex of the $n$-gon (with additional restrictions). We can
axiomatize these examples into arbitrary groups, by starting with a group $G$ and a set $X$ and
having each element in $G$ re-arrange elements in $X$ (with some restrictions).

\begin{definition}[Group Actions on Sets]{}
  Let $G$ be a group, and let $X$ be a set. An \textbf{action of $G$ on $X$} is a rule $\cdot$ that
  assigns to each element $g\in G$ and each element $x\in X$ another element $g\cdot x\in X$, so
  that the following two axioms hold:
  \begin{itemize}
    \item \textbf{Identity Axiom}: $e\cdot x=x$ for all $x\in X$.
    \item \textbf{Associative Axiom}: $(g_1g_2)\cdot x=g_1\cdot (g_2\cdot x)$ for all $g_1,g_2\in
      G$ and all $x\in X$.
  \end{itemize}
\end{definition}

\begin{remark}
  We can choose a fancier definition by showing that defining an action of $G$ on $X$ is the same as
  giving a group homomorphism \[
    \alpha: G\longrightarrow \mc{S}_X
  \] from the group $G$ to the symmetric group of $X$. Thus $\alpha$ sends each $g\in G$ to a
  permutation $\alpha(g):X\to X$ of the set $X$, and the group action is $g\cdot x=\alpha(g)(x)$.
\end{remark}

\begin{definition}[Orbits and Stabilizers]{}
  Given a group $G$ acting on a set $X$, each element $x\in X$ determines two natural objects of
  interest:
  \begin{itemize}
    \item What are the elements of $X$ to which $x$ is sent by the action of $G$? This set is the
      \textbf{orbit of $x$}, denoted \[
        Gx=\{ g\cdot x\mid g\in G \}
      .\] In other words, it is the all possible values that an $x\in X$ can take---its
      ``range''---when acted upon by the group $G$.
    \item What are the elements of $G$ that leave $x$ unchanged? This set is the \textbf{stabilizer
      of $x$}, denoted \[
        G_x=\{g\in G\mid g\cdot x=x\} 
      .\] In other words, it is all values in $G$ that preserve $x$; similar to the ``kernel'' of
      the action.
  \end{itemize}
\end{definition}
\begin{example}
  Consider the action of $G=\mc{S}_n$ on $X=\{ 1,\ldots,n \}$. Given any two elements $x,y\in X$,
  there is a permutation $\pi\in \mc{S}_n$ that sends $x$ to $y$ (e.g. $\pi(x)=y,\ \pi(y)=x$), and
  fixes all other elements of $X$. Thus, the orbit of $x$ is all of $X$, i.e. $Gx=X$, since there
  exists a $\pi\in G$ that sends $x$ to any other element in $X$. The stabilizer $G_x$ of $x$
  consists of all permutations of $X$ that fix $x$, so they are the permutations of the remaining
  $n-1$ elements of $X$. Thus $G_x$ is isomorphic to $\mc{S}_{n-1}$.
\end{example}
\begin{example}
  Consider the action of $G=\D_n$ on an $n$-gon whose vertices are labeled $X=\{ 1,\ldots,n \}$
  clockwise around the $n$-gon. The orbit of any vertex consists of every vertex ($X$), since
  there's a rotation that takes any vertex to any other vertex. With stabilizers, it's not quite as
  simple. Clearly, non-trivial rotations do not fix $x$; same with most flips. However, the flip
  about the axis that goes through $x$ fixes $x$ (it also fixes the opposite vertex if $n$ is even)
  Thus $G_x$ consists of two elements, the identity and the flip about the axis through $x$, so
  $G_x$ is a cyclic group of order $2$.
\end{example}
\begin{example}
  Let $G$ be the subgroup of $\mc{S}_5$ generated by the permutation \[
    \pi=(134)(25)
  .\] Clearly, $\pi$ has order $6$, so $G=\{ e,\pi,\pi^2,\pi^3,\pi^4,\pi^5 \}$. Then \[
    G\cdot 1=G\cdot 3=G\cdot 4=\{ 1,3,4 \},~\text{and}~G\cdot 2=G\cdot 5=\{ 2,5 \}
  .\] Thus $G$ has two distinct orbits.
\end{example}

\begin{proposition}{}
  Let $G$ be a group that acts on a set $X$.
  \begin{enumerate}
    \item Let $x\in X$. The stabilizer $G_x$ is a subgroup of $G$.
    \item Define a relation $\sim$ on $X$ by the following rule: \[
        x\sim y ~\text{if}~ y=gx ~\text{for some}~g\in G
      .\] Then $\sim$ is an equivalence relation, and \[
        (~\text{the equivalence class of $x$}~)=(~\text{the orbit $Gx$ of $x$}~)
      .\] 
    \item Let $x\in X$. There is a well-defined bijection \[
      \alpha:G / G_x \longrightarrow Gx
    \] defined by the following algorithm:
    \begin{itemize}
      \item \textbf{Input}: A coset $\mc{C}$ of the subgroup $G_x$.
      \item \textbf{Computation}: Choose an element $g\in \mc{C}$.
      \item \textbf{Output}: $\alpha(\mc{C})$ is the element $g\cdot x$ in the orbit $Gx$.
    \end{itemize}
    In particular, if $G$ is finite, then \[
      \left| Gx \right| = \frac{\left| G \right| }{\left| G_x \right| }
    .\] 
  \end{enumerate}
\end{proposition}

\begin{proof}[Proof]
  \begin{enumerate}
    \item First, we have $e\in G_x$, since the definition of a group action requires $ex=x$. Next, let
      $g,g'\in G_x$. Then since $g,g'$ both fix $x$ together with the associative law of group actions
      tell us that \[
        (gg')x=g(g'x)=gx=x
      ,\] so $gg'\in G_x$. Finally, applying $g^{-1}$ to both sides of $x=gx$ yields \[
        g^{-1}x=g^{-1}(gx)=(g^{-1}g)x=ex=x
      ,\] so $g^{-1}\in G_x$. Thus $G_x$ is a subgroup of $G$.
    \item First we have \[
        x=ex, ~\text{which shows that $x\sim x$}~
      ,\] and so $\sim$ is reflexive. Second, we note that
      \begin{align*}
        x\sim y &\implies y=gx ~\text{for some}~g\in G\\
                &\implies g^{-1}y=(g^{-1}g)x=x
      ,\end{align*} thus $x=g^{-1}y$, and so $y\sim x$, which shows that $\sim$ is symmetric. Finally,
      \begin{align*}
        x\sim y~\text{and}~y\sim z &\implies y=gx~\text{and}~z=g'y~\text{for some}~g,g'\in G\\
                                   &\implies z=g'(gx)=(g'g)x\\
                                   &\implies x\sim z
      .\end{align*} Thus $\sim$ is an equivalence relation.

      The equivalence class of an element $x\in X$ is \[
        \{y\in X\mid x\sim y\} =\{y\in X\mid y=gx~\text{for some}~g\in G\} =\{gx\mid g\in G\} =Gx
      .\] 

    \item We first must show that the output is well-defined, in that it doesn't depend on the
      choice of the group element in the coset $\mc{C}$. Suppose that $g'\in \mc{C}$ is some other
      element in the coset. This means that \[
        g'\in \mc{C}=gG_x,~\text{so}~g'=gh~\text{for some}~h\in G_x
      .\] It follows that \[
        g'x=(gh)x=g(hx)=gx
      ,\] since $h\in G_x$ means $hx=x$. This shows that $\alpha(\mc{C})=gx$ is well-defined, since
      it doesn't depend on choice of $g\in \mc{C}$ (explicitly, $\alpha(g'G_x)=g'x=gx=\alpha(gG_x)$
      for $g,g'\in \mc{C}$, a coset of $G_x$). It's easy to see that $\alpha$ is surjective, since
      every element of $Gx$ looks like $gx$ for some $g\in G$, and so we can simply choose
      $\alpha(gG_x)=gx$.

      Finally, we prove that $\alpha$ is injective, so suppose we have cosets $\mc{C}_1, \mc{C}_2$
      such that \[
        \alpha(\mc{C}_1)=\alpha(\mc{C}_2)
      .\] We wish to show that $\mc{C}_1=\mc{C}_2$. Write the cosets as \[
        \mc{C}_1=g_1G_x~\text{and}~\mc{C}_2=g_2G_x~\text{for some}~g_1,g_2\in G
      .\] Then
      \begin{align*}
        \alpha(\mc{C}_1)=\alpha(\mc{C}_2)&\implies \alpha(g_1G_x)=\alpha(g_2G_x)\\
                                         &\implies g_1x=g_2x\\
                                         &\implies x=g_1^{-1}g_2x\\
                                         &\implies g_1^{-1}g_2\in G_x&&[~\text{since $g_1^{-1}g_2$
                                         fixes x}~]\\
                                         &\implies g_1^{-1}g_2G_x=G_x\\
                                         &\implies g_2G_x=g_1G_x\\
                                         &\implies \mc{C}_1=\mc{C}_2
      .\end{align*} Thus $\alpha$ is a well-defined bijection.

      For the last part, we compute \[
        \left| Gx \right| =\left| (G / G_x) \right| =\left| G \right| /\left| G_x \right| 
      ,\] where the first equality is because they are isomorphic, and the second is due to
      Lagrange's Theorem.
  \end{enumerate}
\end{proof}

\begin{definition}[Transitivity]{}
  We say that $G$ \textbf{acts transitively on $X$} if $Gx=X$ for all $x\in X$.
\end{definition}

The dihedral group example demonstrates that $\D_n$ acts transitively on the vertices of an $n$-gon,
while the following example about the cyclic subgroup generated by $\pi$ gives an example of a group
and a set on which the group does not act transitively.





















\end{document}
