\documentclass[math1530-lecture-notes]{subfiles}
\begin{document}

\chapter{Fields: Part II}

More field theory. POGGIES!!!

\section{Algebraic Numbers and Transcendental Numbers}
We are vaguely familiar with algebraic and transcendental numbers from high-school algebra. We now
formalize their definition.
\begin{definition}[Algebraic and Transcendental Numbers]{}
  Let $L / F$ be an extension of fields, and let $\alpha\in L$. $\alpha$ is \textbf{algebraic over
  $F$} if $\alpha$ is the root of a non-zero polynomial in $F[x]$. Otherwise, $\alpha$ is
  \textbf{transcendental over $F$}.
\end{definition}

The numbers $\sqrt{3}$ and $2+i$ are algebraic over $\Q$, since they are roots of, respectively, the
polynomials $x^2-3$ and $x^2-4x+5$. Other numbers may not seem algebraic, but really are: \[
  \sqrt{\sqrt{2}+1}+\sqrt[3]{5}
\] is the root of $x^{12}-
6x^{10}-20x^{9}+9x^{8}+154x^{6}-360x^{5}+441x^{4}-180x^3-456x^2-1680x+274$.

There are also many transcendental numbers, such as $e$, $\pi$, and the sum \[
  \sum_{n=1}^{\infty} \frac{1}{10^{n!}}=\frac{1}{10}+\frac{1}{10^2}+\frac{1}{10^6}+\ldots
,\] as proved by Louisville (although their proofs are highly trivial). Indeed, Cantor proved that
the set of algebraic numbers is countable, while the set of transcendental numbers is uncountable!




\end{document}
