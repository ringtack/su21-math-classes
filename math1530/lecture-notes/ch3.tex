\documentclass[math1530-lecture-notes]{subfiles}
\begin{document}

\chapter{Rings: Part I}

Unlike groups, which were completely new, the concept of a \textbf{ring} is mildly familiar! Some
examples of rings include:
\begin{itemize}
  \item $\Z,\ \Q,\ \R$, and $\C$ are rings ($\Q,\ \R$, and $\C$ are actually \textbf{fields}, a
    special type of ring; such is a discussion for later)
  \item The set of integers modulo $m$ is a ring
\end{itemize}

These examples all share something in common: they each have two operations, "addition" and
"multiplication", and each operation individually satisfies some axioms, along with the \~great and
powerful\~ distributive law.

In general, a ring is a set with two operations satisfying a bunch of axioms that are modeled after
the properties of addition and muliplication of integers. We will later formalize this; but first, a
little number theory.

\section{Review of Number Theory}

\subsection{Equivalence Relations}

We first introduce the notion of \textbf{equivalence relations}; while not strictly related to
number theory, equivalence relations will be significant for modular arithmetic.

\begin{definition}[Equivalence Relations]{}
  An \textbf{equivalence relation} on a set $S$ is a relation ``$\sim $'' satisfying
  \begin{enumerate}
    \item \textbf{Reflexivity}: For $a\in S$, $a\sim a$
    \item \textbf{Symmetry}: For $a,b\in S$, $a\sim b$ implies $b\sim a$ 
    \item \textbf{Transitivity}: For $a,b,c\in S$, if $a\sim b$ and $b\sim c$, then $a\sim c$.
  \end{enumerate}
  $a\sim b$ means $a$ is ``related'' to $b$; $a\not\sim b$ means $a$ is ``not related'' to $b$.

  Given an $a\in S$, the \textbf{equivalence class} of $a$ is \[
    S_a = \{b\in S\mid b\sim a\} 
  .\] Note that $S_a$ is never empty; it always contains $a$.
\end{definition}

Some examples of equivalence relations are equality ($=$) and congruence $\mod{m}$; on the other
hand, order (e.g. $\le $) is \textbf{not} an equivalence relation (symmetry does not hold).

We now look further into the congruence $\mod{m}$ equivalence relation.

\begin{example}
  Given $a \in \Z$, $b \equiv a\mod{m}$ iff \[
    n | b-a \iff b - a = kn,\ k\in \Z \iff b = a + kn
  .\] So $\Z_a$ actually forms a coset of $n\Z$:
  \begin{align*}
    \Z_a &= \{b\in \Z\mid b\equiv a\mod{m}\}  \\
    &= \{a+kn\mid k\in \Z\}  \\
    &= a+n\Z
  .\end{align*}
  That is, each equivalence class for congruence $\mod{m}$ is actually a coset of $m\Z$ in $\Z$: \[
    \Z / m\Z = ~\text{set of cosets of $m\Z$ in $\Z$}~
  .\] 
\end{example}

\begin{theorem}[]{}
  Let $S$ be a set with an equivalence relation $\sim $. Then
  \begin{enumerate}
    \item If $a,b\in S$, then either \[
        S_a \cap S_b = \varnothing ~\text{or}~ S_a = S_b
      .\] 
    \item Let $\{ C_i \}_{i\in I}$ be the disjoint equivalence classes of $S$. Then \[
        S = \coprod_{i\in I}C_i
      .\] In particular, if $S$ is finite, then \[
        \left| S \right| = \left| C_1 \right|+ \ldots+\left| C_n \right|  
      .\] 
  \end{enumerate}
\end{theorem}

\subsection{Modular Arithmetic}

We first observe an important characteristic of the $\gcd$:
\begin{proposition}[]{}
  Given integers $u,v$, there exists integers $x,y$ such that \[
    ux+vy = \gcd{(u, v)}
  .\] 
\end{proposition}





\section{Abstract Rings and Ring Homomorphisms}

\begin{definition}[Rings]{}
  A \textbf{ring} $R$ is a set with two operations, generally called \textbf{addition} and
  \textbf{multiplication} and written
  \begin{align*}
    \underbrace{a+b}_{\text{addition}} &&\text{and}&& \underbrace{a\cdot b
    ~\text{or}~ab}_{\text{multiplication}}
  \end{align*}
  satisfying the following axioms:
  \begin{enumerate}
    \item The set $R$ with addition law $+$ is an Abelian group, with identity $0$ (or $0_R$).
    \item The set $R$ with multiplication law $\cdot $ is a \textbf{monoid} (associative, identity,
      \textbf{but no inverse}), with identity $1$ (or $1_R$\footnote{to avoid the trivial ring, we
      require $1_R\neq 0_R$; however, this is not strictly required}).
    \item \textbf{Distributive Law}: For all $a,\ b,\ c\in R$, we have
      \begin{align*}
        a\cdot (b+c)=a\cdot b+a\cdot c && \text{and} && (b+c)\cdot a= b\cdot a+c\cdot a
      .\end{align*}
    \item If, in addition to these three properties, $a\cdot b=b\cdot a$ for all $a,\ b\in R$, then
      $R$ is a \textbf{commutative} ring.
  \end{enumerate}
\end{definition}

Experience with the integers seems to suggest that $0_R\cdot a=0_R$, and $(-a)\cdot (-b)=a\cdot b$;
yet why are these true? $0_R$ is the definition of the identity element for \textit{addition}, so
why should it say anything about \textit{multiplication}? Similarly, $-a$ relates to the definition
of \textit{additive} inverse, but what does that tell us about its product with other elements in
$R$? To show these intuitively obvious claims, we need the distributive law.

\begin{proposition}[]{}
  \begin{enumerate}
    \item $0_R\cdot a=0_R$ for all $a\in R$.
    \item $(-a)\cdot (-b)=a\cdot b$ for all $a,\ b\in R$. In particular, we have $(-1_R)\cdot a=-a$.
  \end{enumerate}
\end{proposition}
\begin{proof}[Proof]
  \begin{enumerate}
    \item Note that $1_R=0_R+0_R$. Then
      \begin{align*}
        a &= 1_R\cdot a && [1_R~\text{is the muliplicative identity}~]\\
          &= (1_R+0_R)\cdot a &&[~\text{from above}~]\\
          &= 1_R\cdot a+0_R\cdot a && [~\text{from distributivity}~] \\
          &= a+0_R\cdot a \\
      .\end{align*}
      Adding $-a$ to both sides, we get \[
        0_R = 0_R + 0_R\cdot a
      ,\] and so $0_R \cdot a=0_R$.

    \item First, we show that $(-1_R)\cdot a=-a$:
      \begin{align*}
        a + (-1_R)\cdot a&= 1_R\cdot a+(-1_R)\cdot a \\
                         &= (1_R+ -1_R)\cdot a \\
                         &= 0_R\cdot a \\
                         &= 0_R
       .\end{align*}
       Hence $(-1_R)\cdot a$ is the inverse of $a$, and so $-a=(-1_R)\cdot a$.

       Now, observe that $-ab = (-a)b$ (this proof is left as an exercise for the reader). Then
       \begin{align*}
         (-a)\cdot (-b) + -ab &= (-a)\cdot (-b) + (-a)\cdot b \\
                              &= (-a)\cdot (-b+b) \\
                              &= (-a)\cdot 0_R \\
                              &= 0_R
      .\end{align*}
      Thus $(-a)\cdot (-b)$ is the inverse of $-ab$, and so $(-a)\cdot (-b)=ab$.
  \end{enumerate}
\end{proof}

Just like groups, we want to investigate maps \[
  \phi:R\to R'
\] from one ring to another that respect the \textit{ring-i-ness} of $R$ and $R'$. Since rings are
characterized by their addition and multiplication properties, we get the following definition.

\begin{definition}[Ring Homomorphisms]{}
  Let $R,\ R'$ be rings. A \textbf{ring homomorphism} from $R$ to $R'$ is a function $\phi:R\to R'$ 
  satisfying\footnote{we have the first axiom to disallow the boring and trivial zero map $\phi:R\to
  R'$, $\phi(a)=0_{R'}$.}
  \begin{enumerate}
    \item $\phi(1_R)=1_{R'}$
    \item $\phi(a+b) = \phi(a) + \phi(b)$ for all $a,b\in R$
    \item $\phi(a\cdot b)=\phi(a)\cdot \phi(b)$ for all $a,b\in R$
  \end{enumerate}

  The \textbf{kernel} of $\phi$ is the set of elements that are sent to 0: \[
    \ker(\phi)=\{ a\in R \mid \phi(a)=0_{R'} \}
  .\] 

  As with groups, $R$ and $R'$ are \textbf{isomorphic} if here is a bijective ring homomorphism
  $\phi:R\to R'$, and we call such a map an \textbf{isomorphism}.
\end{definition}








\end{document}
