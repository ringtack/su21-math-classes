\documentclass{homework}
\homework{A}

\begin{document}
\begin{problem}{\S 1: Introduction}
\end{problem}

\begin{enumerate}[label=(\alph*)]
  \item \textit{Where are you?} I am on campus, living in Andrews Hall!
  \item \textit{Is there anything I should know about accessibility?} Everything should be good! I
    am experiencing chronic RSI (repetitive stress injury) in my wrist, which may hamper my ability
    to typeset everything in LaTeX, but if it becomes a problem, I'll let you know.
  \item \textit{Study Group Preferences?} Nothing in particular. I'd like to work with En-Hua,
    Raymond (Dai), Akash, Madhav (Ramesh), and Jack (Cheng) if possible! (We're all situated in
    Andrews and know each other decently well, which would hopefully facilitate a good group
    dynamic)
  \item \textit{What are your goals?} I've taken linear algebra before, but with a more
    computational focus, and not so much on proofs. Even so, I was very fascinated in the topics;
    through 0540, I hope to solidify my foundations of linear algebra. Moreover, linear algebra
    serves as the basis of numerous different fields, not only in math, but also in computer science
    (my intended concentration); and I'd like to strengthen my fundamentals in order to better
    approach these areas (specifically, machine learning for CS, and I've recently taken a course in
    applied algebraic topology, which employed a decent amount of linear algebra as well). Finally,
    I hope to pursue higher-level courses in mathematics, and developing a solid mathematical
    maturity is essential for success.
  \item \textit{Anything else?} Nope; only that I'm really excited to take 0540 over the summer!
\end{enumerate}

\begin{problem}{\S 2: Reflection}
  
\end{problem}

In my experience, especially in early years (e.g. elementary school), mathematics appears as
something highly quantifiable and measurable: it's pretty clear when someone knows this or that
formula, or can solve this equation, etc., which isn't always true with other fields (especially
English). While tests exist for other subjects, sometimes it's not as clear cut that someone has
expertise, or is proficient at a certain subject, compared to mathematics. For example, in my K-12
education, people rarely, if ever, skipped grades in English or Science or History; in contrast,
skipping grades in math was common, and sought after (yay competitiveness). I think this can have a
psychological effect, where upon viewing others' accomplishments and progress, and comparing to your
own (e.g. "X person can solve this problem, but I have no idea how to do it") can be quite
demoralizing, and falsely instill the belief that innate talent is the key indicator of mathematical
success; an effect that's only exacerbated by the structure of our K-12 math education, which often
becomes a laundry-list of equations and concepts to use that don't necessarily require the same work
to master as other math areas (e.g. higher-level mathematics and proof-writing, math competitions,
etc).

Additionally, math is a far more cumulative subject than any other; every grade and class builds
upon the previous ones, and mastery (or lack thereof) significantly affects your ability to succeed.
In history or English, for instance, subjects can be disjoint; my school hopped from ancient
Mediterranean history, to colonial America, to the Roman Empire, and then back to antebellum
America. Knowledge of one area won't necessarily affect your success in a future class. With math,
however, failing to master trigonometry or conics or etc. will negatively affect your success in a
future math class. If someone falls behind in a previous math class, they will only be punished
further in future math classes, and as their knowledge deficit accumulates, it might make them
falsely believe that they are not capable of math, and are stuck at their level of knowledge.

Thirdly, math can be hard, again in more obvious ways than other subjects. When faced with a
problem that you don't know how to do, it can be very discouraging, and finding the solution can
sometimes \textbf{appear to} take significantly more work or thought compared to another subject
(which isn't always true; analyzing a difficult English passage may be just as difficult, but it
\textbf{feels} more approachable). As a result, some people may more easily give up when faced with
a difficult math problem, and believe that it's impossible to improve their math skills.

Finally, while this isn't so much a cause, I think it could influence why sometimes people give up
more readily with math. Part of the reason why "innumeracy" exists may have roots in applications:
while English seem distinctly related and applicable to every-day life, the content taught in math
courses sometimes appear unrelated. While basic computation is obviously important, topics like
conics, trigonometry, geometry, etc. don't appear useful in the real world; for "when would I need
to find the roots of a quadratic equation?" When it's harder to find obvious motivations for math,
sometimes people would capitulate more easily, and decide early on that mathematical capability is
set by external factors (perhaps this could help explain why even though peoples' English skills
vary significantly, no one really suffers from illiteracy).


On the subject of a growth mindset, I think I unfortunately sometimes suffer from having a fixed
mindset toward math (and other subjects). I think my experiences relate pretty similarly to what was
explained in the article, where I was significantly ahead in early years and praised for my innate
talent, but as I faced more difficulties in higher-level math courses, I believed that I had peaked.
Partly due to lack of discipline, and partly because I believed that I was doomed to being mediocre
at math (failing math competitions when being 3 grades ahead in math is quite demoralizing), I
shifted my focus away from math and toward other subjects. In recent years, my love for mathematics
has mostly returned, and I am working toward adopting a growth mindset, but years-long habits are
difficult to break. I hope that the consistent and repeated focus on proofs will help me break out
of the mindset; but we'll just have to see.

\end{document}
