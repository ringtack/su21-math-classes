\documentclass{homework}
\homework{3}

\begin{document}

\begin{problem}{\S 1}
  Let $a\in \F$, and let $v\in V_\F$ be a non-zero vector. Prove that \[
    av=\textbf{0} ~\text{only if}~ a=0
  .\] 
\end{problem}

\begin{solution}
  Suppose $ a\neq 0$, yet $av=0$. Then
  \begin{align*}
    av &= \textbf{0} \\
       &= av+(-av) \\
       &= 0 + (-av) \\
       &= -av
  .\end{align*}
  Hence $av=-av \implies av\cdot v^{-1}=-av\cdot v^{-1}\implies a=-a$. But $a=-a$ only when $a=0$, a
  contradiction. Thus $av=0$ only when $a=0$.
\end{solution}


\begin{problem}{\S 2}
  Let $v,w\in V$, and suppose $v\neq 0$. Prove that there exists at most one  $a\in \F$ such that \[
    av = w
  .\] 
\end{problem}

\begin{solution}
  Suppose there exists a $b\in \F,\ b\neq a$ such that $bv=w$. Then \[
    av+bv = w + w = 2w = w + w = av + av
  .\] Adding $-av$ to both sides, \[
    bv = av \implies a = b
  ,\] a contradiction. \\
  Hence if $bv=w$ for some $b\in \F$, then $b=a$.
\end{solution}

\begin{problem}{\S 3}
  \begin{enumerate}[label=(\alph*)]
    \item (1.C.12) Prove that the union of two subspaces of $V$ is a subspace of $V$ if and only if
      one of the subspaces is contained in the other.
    \item (1.C.13, extra credit) Prove that the union of three subspaces of $V$ is a subspace of $V$ 
      if and only if one of the subspaces contains the other two.
  \end{enumerate}
\end{problem}
\begin{solution}
  \begin{enumerate}[label=(\alph*)]
    \item 
  Let $U_1,U_2\in V$ be two subspaces of $V$.\\

  Suppose that one of the subspaces is contained in the other; that is, suppose without loss of
  generality that $ U_1\subseteq U_2$. Then $ U_1\cup U_2=U_2\subseteq V$ is a subspace of $V$ 
  (since $ U_2$ is a subspace of $V$).\\

  Now, suppose that $ U_1\cup U_2\subseteq V$ is a subspace of $V$. Let $ v_1,v_2\in U_1\cup U_2$.
  Then \[
    v_1\in U_1 ~\text{or}~v_1\in U_2,~\text{and}~v_2\in U_1~\text{or}~v_2\in U_2
  .\] $ v_1,v_2\in U_1~\text{or}~v_1,v_2\in U_2$ tell us nothing new about the relationship between
  $ U_1$ and $U_2$ (we already know, by definition of a subspace, that $0,\ \lambda v_1,\ \lambda
  v_2,\ v_1+v_2\in U_i$ for some $\lambda\in \F$, $i\in \{ 1,2 \}$), so suppose without loss of
  generality that $v_1\in U_1,v_2\in U_2$. We know (since $ U_1\cup U_2$ is a subspace of $V$) that
  $v_1+v_2\in U_1\cup U_2$, so $ v_1+v_2\in U_1$ or $ v_1+v_2\in U_2$. If $ v_1+v_2\in U_1$, by
  closure of addition in $ U_1$, $ v_2\in U_1$ as well, and so since for any arbitrary $v_2\in U_2,\
  v_2\in U_1$, we have $ U_2\subseteq U_1$. Similarly, if $ v_1+v_2\in U_2$, by closure of addition, $v_1\in
  U_2$ as well, and so $ U_1\subseteq U_2$. \\
  Therefore, if $ U_1\cup U_2$ is a subspace of $V$, then one of the subspaces is contained in the
  other.\\

  Thus, the union of two subspaces of $V$ is a subspace of $V$ if and only if one of the subspaces
  is contained in the other.

    \item W.I.P.
  \end{enumerate}
\end{solution}

\begin{problem}{\S 4}
  Solved in Review Sheet 5.
\end{problem}

\begin{problem}{\S 5}
  Let $V,W$ be vector spaces over $\F$, and let $T:V\to W$ be a linear map. Suppose $V$ is
  finite-dimensional and $T$ is surjective. Prove that $W$ is finite-dimensional.
\end{problem}

\begin{solution}
  Since $T$ is surjective, we know that every $w\in W$ is mapped to by some $v\in V$. By the
  definition of a function, every $v\in V$ is mapped to \textbf{one} element $T(v)\in W$. Hence
  $T(V)$ (or the image of $V$ under $f$) has a maximum cardinality of $\left| V \right|
  $ (equivalently, $\left| T(V) \right| \le \left| V \right| $, since not every $T(v)$ is
  necessarily unique).  But since $T$ is surjective, we know that $T(V)=W$, and so $\left| V \right|
  > \left| T(v) \right| = \left| W \right|$ and hence $\left| V \right| \ge \left| W \right| $; and
  since $V$ is finite, $W$ is finite as well.
\end{solution}


\end{document}
