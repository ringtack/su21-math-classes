\documentclass{homework}
\homework{3}

\begin{document}

\begin{problem}{\S 1}
  Let $a\in \F$, and let $v\in V_\F$ be a non-zero vector. Prove that \[
    av=\textbf{0} ~\text{only if}~ a=0
  .\] 
\end{problem}

\begin{solution}
  Let $v\in V_\F$ be a non-zero vector, and let $a\in \F$ such that $av=\textbf{0}$. Suppose $ a\neq
  \textbf{0}$. Then
  \begin{align*}
    av &= \textbf{0} \\
    a^{-1}av&= a^{-1}\textbf{0} && [\text{since $a\in \F$ and $a\neq 0$, there exists $a^{-1}\in
    \F$ such that $aa^{-1}=1$}] \\
    v&= \textbf{0}
  .\end{align*} But this is a contradiction, as we assume $v$ non-zero. Hence $a$ must be zero.
\end{solution}


\begin{problem}{\S 2}
  Let $v,w\in V$, and suppose $v\neq 0$. Prove that there exists at most one  $a\in \F$ such that \[
    av = w
  .\] 
\end{problem}

\begin{solution}
  Let $v,w\in V_\F$, and let $a\in \F$ such that $av=w$. Suppose there exists a $b\in \F,\ b\neq a$
  such that $bv=w$. Then
  \begin{align*}
    av = w &= bv \\
    av &= bv \\
    av - bv &= bv - bv \\
    (a-b)v &= \textbf{0} && [\text{since $-bv=(-b)v$}]
  .\end{align*}
  From problem 1, we get that if $v$ is a non-zero vector, then $(a-b)v=0$ only when $a-b=0$. Hence
  $a=b$; but we assume $b\neq a$, a contradiction. Thus if $\exists b\in \F$ such that $bv=w$, then
  $b=a$.
\end{solution}

\begin{problem}{\S 3}
  \begin{enumerate}[label=(\alph*)]
    \item (1.C.12) Prove that the union of two subspaces of $V$ is a subspace of $V$ if and only if
      one of the subspaces is contained in the other.
    \item (1.C.13, extra credit) Prove that the union of three subspaces of $V$ is a subspace of $V$ 
      if and only if one of the subspaces contains the other two.
  \end{enumerate}
\end{problem}
\begin{solution}
  \begin{enumerate}[label=(\alph*)]
    \item 
  Let $U_1,U_2\in V$ be two subspaces of $V$.\\

  Suppose that one of the subspaces is contained in the other; that is, suppose without loss of
  generality that $ U_1\subseteq U_2$. Then $ U_1\cup U_2=U_2\subseteq V$ is a subspace of $V$ 
  (since $ U_2$ is a subspace of $V$).\\

  Now, suppose that $ U_1\cup U_2\subseteq V$ is a subspace of $V$. Let $ v_1,v_2\in U_1\cup U_2$.
  Then \[
    v_1\in U_1,\ v_1\in U_2,~\text{or}~v_1\in U_1\cap U_2,~\text{and}~v_2\in U_1,\ v_2\in
    U_2,~\text{or}~v_2\in U_1\cap U_2
  .\] $ v_1,v_2\in U_1~\text{or}~v_1,v_2\in U_2$ tell us nothing new about the relationship between
  $ U_1$ and $U_2$ (we already know, by definition of a subspace, that $0,\ \lambda v_1,\ \lambda
  v_2,\ v_1+v_2\in U_i$ for some $\lambda\in \F$, $i\in \{ 1,2 \}$; and if $v_i\in U_1\cap U_2$,
  then $v_i$ is in each individual subspace as well), so suppose without loss of
  generality that $v_1\in U_1,v_2\in U_2$ (if either are in the intersection of the two subspaces, then
  they are also in each individual subspace). We know (since $ U_1\cup U_2$ is a subspace of $V$) that
  $v_1+v_2\in U_1\cup U_2$, so $ v_1+v_2\in U_1$, $ v_1+v_2\in U_2$, or $ v_1+v_2\in U_1\cap U_2$.
  
  Suppose $ v_1+v_2\in U_1$. Then since  $v_1\in U_1$, we have $ v_1^{-1}\in U_1$, and by closure of
  addition $ v_1^{-1}+v_1+v_2\in U_1$; thus $v_2\in U_1$, and so since for any arbitrary $ v_2\in
  U_2,\ v_2\in U_1$, we have $ U_2\subseteq U_1$.

  Similarly, suppose $v_1+v_2\in U_2$. Then $v_2^{-1}\in U_2$, and by closure of addition
  $v_1+v_2+v_2^{-1}\in U_2$; thus $ v_1\in U_2$ as well, and so $U_1\subseteq U_2$.

  The case of $ v_1+v_2\in U_1\cap U_2$ follows the same logic; since being in the intersection
  implies being in each individual subspace, using the steps above, we have $ v_2\in U_1$, and $
  v_1\in U_2$. Thus, we have $U_1\subseteq U_2$, and $ U_2\subseteq U_1$, and so the subspaces are
  contained in each other (indeed, they are equivalent).

  Thus, the union of two subspaces of $V$ is a subspace of $V$ if and only if one of the subspaces
  is contained in the other.
  \end{enumerate}
\end{solution}

\begin{problem}{\S 4}
  Solved in Review Sheet 5.
\end{problem}

\begin{problem}{\S 5}
  Let $V,W$ be vector spaces over $\F$, and let $T:V\to W$ be a linear map. Suppose $V$ is
  finite-dimensional and $T$ is surjective. Prove that $W$ is finite-dimensional.
\end{problem}

\begin{solution}
  We begin with a lemma.
  \begin{lemma}[]{}
    Let $V,W$ be vector spaces over a field $\F$. If $V$ is an $n$-dimensional vector space, and
    $T:V\to W$ is a linear map, then $\dim(\range T) \le \dim V$.
  \end{lemma}
  \begin{proof}[Proof]
    If $\dim V=n$, then $V$ has a basis $B=\{ v_1,\ldots,v_n \}$ with $n$ linearly independent
    vectors in $V$ that span $V$. Then, for any $v\in V$,\[
      v=a_1v_1+\ldots+a_nv_n,\ ~\text{for}~a_i\in \F
    .\] Applying $T$ to both sides, we have \[
      T(v) = T(a_1v_1+\ldots+a_nv_n)= a_1T(v_1)+\ldots+a_nT(v_n)
    ,\] and since $v$ was an arbitrary $v\in V$, $T(v_1),\ldots,T(v_n)$ span $\range T$. Hence
    $\dim(\range T) \le n = \dim V$.\\\\
    (Note that there is no guarantee that $T(v_1),\ldots,T(v_n)$ are linearly independent and thus 
    form a basis for $\range T$; $\dim(\range T)$ could very much be less than $n$. For example,
    take $T=\textbf{0}$ the zero map. Then $T(v_i)=\textbf{0}$ for any $v_i\in B$, and so
    $\dim(\range T) < n$.)
  \end{proof}
  

  From this lemma, we get that if $V$ is a finite-dimensional vector space (say, with $\dim V = n$),
  and $T:V\to W$ a linear map, then $\dim(\range T) \le  \dim V$. Moreover, since $T$ is surjective,
  by definition we have $\range T = W$. Thus $\dim W \le \dim V$, and so $W$ is finite-dimensional
  as well.
\end{solution}


\end{document}
