\documentclass{homework}
\homework{B}

\begin{document}
\begin{problem}{\S 2}\\
  Fill in the tables for powers of $2$ in $\F_{13}$.
\end{problem}

\begin{solution}
  \begin{table}[htpb]
    \centering
    \caption{Problem 2}
    \begin{tabular}{c||c|c|c|c|c|c|c|c|c|c|c|c}
          $n$ & 0 & 1 & 2 & 3 & 4 & 5 & 6 & 7 & 8 & 9 & 10 & 11 \\
      \toprule \bottomrule
          $2^{n}$ & 1 & 2 & 4 & 8 & 3 & 6 & 12 & 11 & 9 & 5 & 10 & 7
    \end{tabular}
  \end{table}
\end{solution}

\begin{problem}{\S 2}
  Let \begin{align*}
    f: \R^2 &\longrightarrow \R^3 \\
    (x,y) &\longmapsto f((x,y)) = (x,x+y,y)
  .\end{align*}
  \begin{enumerate}[label=(\alph*)]
    \item Write down, using the definition of injectivity and preferably using universal
      quantifiers, the statement that $f$ is injective, and the statement that $f$ is not injective.
      Then prove the correct statement (that $f$ is injective).
    \item Write down, using universal quantifiers, the statement that $f$ is surjective, and the
      statement that $f$ is not surjective. Then prove the correct statement (that $f$ is not
      surjective).
  \end{enumerate}
\end{problem}

\begin{solution}
  \begin{enumerate}[label=(\alph*)]
    \item $f$ is injective: For any $a,b\in \R^2$, if $f(a)=f(b)$, then $a=b$.\\
      $f$ is not injective: There exists $a,b\in \R^2$ such that $f(a)=f(b)$ and $a\neq b$.\\
      Now, we prove that $f$ is injective.
      \begin{proof}[Proof]
        Choose $(x_0,y_0),(x_1,y_1)\in \R^2$ such that $(x_0,x_0+y_0,y_0)=(x_1,x_1+y_1,y_1)$. From
        this, we see that $ x_0=x_1, y_0=y_1$, so $(x_0,y_0)=(x_1,y_1)$ and thus $f$ is injective.
      \end{proof}

      \item $f$ is surjective: For any $v\in \R^3$, there exists an $u\in \R^2$ such that
        $f(u)=v$.\\
        $f$ is not surjective: There exists a $v\in \R^3$ such that for any $u\in \R^2$, $f(u)\neq
        v$.\\
        Now, we prove that $f$ is not surjective.
        \begin{proof}[Proof]
          Choose $(0,10,0)\in \R^3$. Clearly, for any $(x_0,y_0)\in \R^2$, if
          $f((x_0,y_0))=(0,10,0)$, then $ x_0=y_0=0$; but then $ x_0+y_0=0\neq 10$. Thus $f$ is not
          surjective.
      \end{proof}
  \end{enumerate}
\end{solution}


\begin{problem}{\S 3}
  Let $X$ be any set, and let $V$ be the set of all subsets of $X$. Define addition on $V$ as \[
  A+B=A\Delta B
\] for subsets $A,B\subseteq X$, and scalar multiplication on $V$ with scalars $ \F_2=\{ 0,1 \}$ as
\[
  0\cdot A=\varnothing, 1\cdot A=A
\] for any subset $A\subseteq X$.\\
Check that $V_{\F_2}$ is a vector space.
\end{problem}

\begin{solution}
  In order to be a vector space, $V_{\F_2}$ must satisfy 6 properties:
  \begin{itemize}
    \item \textbf{Associativity}: We start with additive associativity. Let $A,B,C\subseteq X$. Then 
      \begin{align*}
        A\Delta\left( B\Delta C \right) &= \left( A\cap \left( \left( B\cap C^{c} \right) \cup
        \left( C\cap B^{c} \right)  \right) ^{c} \right)\cup \left( \left( \left( B\cap C^{c} \right)
    \cup \left( C\cap B^{c} \right) \right)\cap  A^{c}\right)  \\
                                        &= \left( A\cap \left( \left( B^{c}\cup C \right) \cap
                                            \left( C^{c}\cup B \right)  \right)  \right)\cup
                                            \left(
                                      \left( A^{c}\cap \left( B\cap C^{c} \right)  \right) \cup
                                    \left( A^{c}\cap \left( C\cap B^{c} \right)  \right) \right)   \\
                                        &= \left( A\cap \left( \left( B^{c}\cap C^{c} \right)\cup
                                        \left( B^{c}\cap B \right) \right) \cup \left( \left(
                                    C\cap C^{c}\right) \cup \left( C\cap B \right)  \right)  \right)
                                    \cup \left( \left( A^{c}\cap \left( B\cap C^{c} \right)  \right)
                                    \cup \left( A^{c}\cap \left( C\cap B^{c} \right)  \right)
                                  \right) \\
                                        &= \left( A\cap \left( \left( B^{c}\cap C^{c} \right)\cup
                                        \left( B\cap C \right)   \right)  \right)  \cup \left(
                                      \left(  A^{c}\cap B\cap C^{c} \right) \cup \left( A^{c}\cap
                                    C\cap B^{c} \right) \right) \\
                                        &= \left( \left( A\cap B^{c}\cap C^{c} \right)\cup \left(
                                        A\cap B\cap C \right)   \right)\cup \left( \left( 
                                    B\cap A^{c}\cap C^{c} \right) \cup \left( C\cap A^{c}\cap B^{c} \right) \right) 
      .\end{align*} Conversely, 
      \begin{align*}
        \left( A\Delta B \right) \Delta C&= \left( \left( \left( A\cap B^{c} \right) \cup \left(
        B\cap A^{c} \right) \right) \cap C^{c} \right) \cup \left( C\cap \left( \left(      A\cap B^{c} \right) \cup \left( B\cap A^{c} \right) \right)^{c}  \right)  \\
                                         &= \left( \left( A\cap B^{c}\cap C^{c} \right) \cup \left(
                                         B\cap A^{c}\cap C^{c}\right)  \right) \cup \left( C\cap
                                       \left( \left( A^{c}\cup B \right) \cap \left( B^{c}\cup A \right)  \right)  \right)  \\
                                       &= \left( \left( A\cap B^{c}\cap C^{c} \right) \cup \left(
                                       B\cap A^{c}\cap C^{c}\right)  \right) \cup \left( C\cap
                                     \left( \left( A^{c}\cap B^{c} \right) \cup \left( B\cap B^{c}
                                         \right)  \right) \cup \left(\left( A^{c}\cap A \right) \cup \left(
                                   A\cap B \right)  \right) \right) \\
                                   &= \left( \left( A\cap B^{c}\cap C^{c} \right) \cup \left(
                                   B\cap A^{c}\cap C^{c}\right)  \right) \cup \left( \left( C\cap
                               A^{c}\cap B^{c} \right) \cup \left( C\cap A\cap B \right)  \right)
      .\end{align*} Due to commutativity of union and intersection of sets, we observe that
      $A\Delta\left( B\Delta C \right) = (A\Delta B)\Delta C$, and so it satisfies additive
      associativity.\\

      Now, we show scalar multiplicative associativity. Let $\alpha,\beta\in \F_2$. Then we show
      associativity holds for the four possible cases:
      \begin{itemize}
        \item $1\cdot \left( 1\cdot A \right) = 1\cdot A=A=1\cdot A=(1\cdot 1)\cdot A$
        \item $1\cdot \left( 0\cdot A \right) =1\cdot \varnothing=0\cdot A=(1\cdot 0)\cdot A$
        \item $0\cdot \left( 1\cdot A \right) =0\cdot A=\varnothing=0\cdot A=\left( 0\cdot 1 \right)
          \cdot A$
        \item $0\cdot \left( 0\cdot A \right) =0\cdot \varnothing=\varnothing=0\cdot
          \varnothing=\left( 0\cdot 0 \right) \cdot A$
      \end{itemize} Thus scalar multiplicative associativity holds as well, and so associativity
      holds.
      
    \item \textbf{Commutativity}: Let $A,B\subseteq X$. Then 
      \begin{align*}
        A\Delta B &= \left( A\setminus B \right) \cup \left( B\setminus A \right) \\
                  &= \left( B\setminus A \right) \cup \left( A\setminus B \right)  \\
                  &= B\Delta A
      ,\end{align*} by commutativity of set union. Thus commutativity holds.

    \item \textbf{Additive Identity}: Observe that for any $A\subseteq X$, \[
      A\Delta \varnothing=\varnothing\Delta A = A
      .\] Thus additive identity holds.
      
    \item \textbf{Additive Inverse}: For any $A\in X$, choose $A'=A^{c}\subseteq X$ (the complement
      of A). Then \[
        A\Delta A' = A' \Delta A = \varnothing
      .\] Thus additive inverse holds.

    \item \textbf{Multiplicative Identity}: Observe that for any $A\subseteq X$, \[
      1\cdot A=A
    \] by definition. Thus scalar multiplicative identity holds.

  \item \textbf{Distributive Properties}: First we show that for any $\lambda\in \F_2, A,B \in X$,
    $ \lambda\cdot (A+B)=\lambda\cdot A+\lambda\cdot B$:
    \begin{align*}
      \lambda\cdot \left( A+B \right) &= A+B \\
      &= \lambda\cdot A+\lambda\cdot B
    .\end{align*} The last statement holds true for both $\lambda=0$ and $\lambda=1$, and so the
    first distributive property holds.\\

    Now, we show that the second distributive property, $\left( \alpha+ \beta \right) A =
    \alpha\cdot A+\beta\cdot A$, is true for all $\alpha,\beta\in \F_2,A\subseteq X$; we do this by
    evaluating the three cases ($0+ 1=1+ 0=1$):
    \begin{itemize}
      \item If $\alpha=\beta=1$:
        \begin{align*}
          (1+ 1)\cdot A &= 0\cdot A \\
          &= \varnothing \\
          &= A\Delta A \\
          &= 1\cdot A+1\cdot A
        .\end{align*}
      \item If $\alpha=0,\beta=1$:
        \begin{align*}
          (1+0)\cdot A&= 1\cdot A \\
          &= A \\
          &= A\Delta\varnothing \\
          &= 1\cdot A+0\cdot A
        .\end{align*}
      \item If $\alpha=\beta=0$:
        \begin{align*}
          (0+0)\cdot A&= 0\cdot A \\
          &= \varnothing \\
          &= A\Delta A \\
          &= 0\cdot A+0\cdot A \\
        .\end{align*}
    \end{itemize}
    Thus the second distributive identity holds as well, and so distributive properties hold.

  \end{itemize}
\end{solution}




\end{document}
