\documentclass{homework}
\homework{E}

\begin{document}
\begin{problem}{\S 1}
  (Axler 3.B.30) Suppose $\varphi_1$ and $\varphi_2$ are linear maps from $V$ to $\F$ that have the same
  null space. Show that there exists a constant $c\in \F$ such that $\varphi_1=c\varphi_2$.
\end{problem}
\begin{solution}  
We start with a lemma (which is actually Axler 3.B.29, but proved to be very helpful):
\begin{lemma}[]{}
  Let $\varphi\in \mc{L}(V,\F)$, and $u\in V$ not in $\Null\varphi$. Then \[
    V = \Null\varphi \oplus \{au\mid a\in \F\} 
  .\] 
\end{lemma}
\begin{proof}[Proof]
  Suppose $au\in \Null\varphi$. Then \[
    0=\varphi(au)=a\varphi(u)
  .\] Since $u\not\in \Null\varphi$, $\varphi(u)\neq 0$, so $a=0$; hence $\Null\varphi\cap \{au\mid
  a\in \F\} =\{ 0 \}$, and by Proposition 1.45, $ \Null\varphi \oplus \{au\mid a\in \F\}$ is a
  direct sum.

  Now, for any $v\in V$, suppose $\varphi(v)=k,\ \varphi(u)=r$ for some $k,r\in \F$. Observe that
  \begin{align*}
    \varphi(v-\frac{k}{r}u)=\varphi(v)-\frac{k}{r}\varphi(u)=k-\frac{k}{r}\cdot r=0
  ,\end{align*} so $v-\frac{k}{r}u\in \Null\varphi$. Since $\frac{k}{r}\in \F$, $\frac{k}{r}u\in
  \{au\mid a\in \F\}$, and so any $v\in V$ can be written as \[
    v = (v-\frac{k}{r}u)+\frac{k}{r}u
  .\] Hence $V=\Null\varphi\oplus \{au\mid a\in \F\} $, as required.
\end{proof}

If $\Null\varphi_1=\Null\varphi_2=V$, then any $c\in \F$ works, so assume $ \Null\varphi\neq V$.

By the Lemma, any $v\in V$ can be represented as \[
  v=w+au,\ w\in \Null\varphi,\ u\not\in \Null\varphi,\ a\in \F
.\] Suppose $v\in V$, $v\not\in \Null\varphi$ (since $0=c0$ is unenlightening), and let
$\varphi_1(v)=k,\ \varphi_2(v)=r$. Then \[
  c\varphi_1(v)=c\varphi_1(w+au)=c(\varphi_1(w)+a\varphi_1(u))=cak
.\] Setting $c=\frac{r}{k}\in \F$, we get 
\begin{align*}
  \frac{r}{k}\varphi_1(v)=\frac{r}{k}ak=ar=0+a\varphi_2(u)&= \varphi_2(w)+\varphi_2(au) \\
                                                          &= \varphi_2(w+au)\\
                                                          &= \varphi_2(v)
.\end{align*}
Hence $c\varphi_1(v)=\varphi_2(v)$ for some $c\in \F$, as required.
\end{solution}


\begin{problem}{\S 2}
  A square matrix $A\in \F^{m,n}$ is called \textit{upper triangular} if $A_{j,k}=0$ whenever $j>k$.
  Let $A,C$ be $n\times n$ upper triangular matrices.
  \begin{enumerate}[label=(\alph*)]
    \item Prove that $AC$ is upper triangular.
    \item Also, prove that $(AC)_{j,j}=A_{j,j}C_{j,j}$ for each $j=1,\ldots,n$.
  \end{enumerate}
\end{problem}


\begin{solution}
  \begin{enumerate}[label=(\alph*)]
    \item Let $A,C\in \F^{m,n}$ be upper triangular matrices. Then for any $(AC)_{j,k}$ where $j>k$, we
      have \[
        (AC)_{j,k}=\sum_{i=1}^{n} A_{j,i}C_{i,k}=\sum_{i=1}^{j-1} A_{j,i}C_{i,k}+\sum_{i=j}^{n}
        A_{j,i}C_{i,k}
      .\] But for every $1\le i<j$, $A_{j,i}=0$, and for any $j\le i<n$, $j>k$, so $C_{i,k}=0$.
      Hence $(AC_{j,k})=0+0=0$, and so $AC$ is necessarily upper triangular as well.

    \item For any $(AC)_{j,j}$, we have \[
        (AC)_{j,j}=\left(\sum_{i=1}^{j-1} A_{j,i}C_{i,j}\right)+A_{j,j}C_{j,j}+\left( \sum_{i=j+1}^{n}
        A_{j,i}C_{i,j} \right) 
      .\] For any $1\le i<j$, $A_{j,i}=0$ (since $j>i$); and for any $j<i<n$, $C_{i,j}=0$ (since
      $i>j$). Hence $(AC)_{i,j}$ becomes \[
        (AC)_{i,j}=0 +A_{j,j}C_{j,j}+0=A_{j,j}C_{j,j}
      ,\] as required.
  \end{enumerate}
\end{solution}


\begin{problem}{\S 3}
  Players $1,2,3,4,5,6$ are seated around a circle, in that order. One of them is holding a hot
  potato. At each time $t=1,2,\ldots$ seconds from the beginning of the game, whoever is holding the
  potato passes it either to the person to their immediate right or to the person to their immediate
  left, with equal probability.
  \begin{enumerate}[label=(\alph*)]
    \item Let $A\in \R^{6,6}$ be the matrix in which $(A)_{j,k}$ is the probability that, supposing
      player $j$ has the potato at time $0$, player $k$ has the potato at time $1$. Write down $A$.
    \item Give an argument that $(A^2)_{j,k}$ is the probability that, supposing player $j$ has the
      potato at time $0$, player $k$ has the potato at time $2$.
    \item Guess the matrices $A^{10},\ A^{11}$. Once you have the best possible guess, compute the
      matrices and give an interpretation.
    \item Formulate a conjecture about locations of zero entries in $A^{n}$ for all integers $n>1$.
      Prove your conjecture by using the definition of matrix product, or by analyzing the rules of
      the game.
  \end{enumerate}
\end{problem}


\begin{solution}
  \begin{enumerate}[label=(\alph*)]
    \item Since every player passes it following a second, at time $t=1$, $A_{j,j}=0$. So \[
        A = \begin{pmatrix} 0&\frac{1}{2}&0&0&0&\frac{1}{2} \\ \frac{1}{2}&0&\frac{1}{2}&0&0&0 \\
        0&\frac{1}{2}&0&\frac{1}{2}&0&0 \\ 0&0&\frac{1}{2}&0&\frac{1}{2}&0 \\
      0&0&0&\frac{1}{2}&0&\frac{1}{2} \\ \frac{1}{2}&0&0&0&\frac{1}{2}&0 \end{pmatrix} 
    .\] 

    \item For each $(A^2)_{j,k}$, we have \[
      (A^2)_{j,k}=\sum_{i=1}^{6} A_{j,i}A_{i,k}=A_{j,1}A_{1,k}+A_{j,2}+A_{2,k}+\ldots+A_{j,6}A_{6,k}
      .\] For every multiple $A_{j,i}A_{i,k}$, it represents the probability of both:
      \begin{itemize}
        \item Given person $j$ started with the hot potato, person $i$ had the hot potato at time $1$.
          In other words, the probability that $j$ passed the hot potato to $i$.
        \item Given person $i$ \textbf{has} the hot potato, person $k$ received the hot potato.  
      \end{itemize}
      Note the important difference in wording for the second probability: instead of interpreting
      $A_{i,k}$ as person $i$ starting with the hot potato and then giving it to $k$ after one second,
      we can instead think of it as, \textbf{given that person $i$ had the hot potato at time $n$, what is
      the probability that person $k$ had the hot potato at time $n+1$?} 
      
      Thus, the sum $(A^2)_{j,k}$ is just the sum of all the probabilities of, with person $j$
      starting with the hot potato, person $j$ passing it to person $i$, and then person $i$ passing
      it to person $k$. In other words, $(A^2)_{j,k}$ represents the probability that, supposing $j$ 
      started with the potato, $k$ would have the potato after $2$ passes (or $2$ seconds).

    \item \[
        A^{10}=\frac{1}{1024}\begin{pmatrix} 342&0&341&0&341&0 \\ 0&342&0&341&0&341 \\
        341&0&342&0&341&0 \\ 0&341&0&342&0&341 \\ 341&0&341&0&342&0 \\ 0&341&0&341&0&342 \end{pmatrix} 
    .\] 
    \[
      A^{11}=\frac{1}{2048}\begin{pmatrix} 0&683&0&682&0&683 \\ 683&0&683&0&682&0 \\ 0&683&0&683&0&682 \\
      682&0&683&0&683&0 \\ 0&682&0&683&0&683 \\ 683&0&682&0&683&0\end{pmatrix} 
    .\] 

    From part (b), it makes sense to interpret $(A^{10})_{j,k}$ as the probability that, given
    person $j$ started with the hot potato, person $k$ has the hot potato at time $10$ seconds;
    similarly, $(A^{11})_{j,k}$ is the probability that, given person $j$ started with the hot
    potato, person $k$ has the hot potato at time $11$ seconds.

    \item Inspecting $n=2,3,\ldots,10,11$, we see that when $n$ is even, any $(A)_{j,k}$ where $j,k$ 
      are either both even or both odd are non-zero, while any $(A)_{j,k}$ where $j,k$ are of
      different parities are zero; and when $n$ is odd, the opposite is true. Thus, it seems that
      any $A^{n}$ with $n$ even has zero entries whenever $j,k$ have opposite parities, while any
      $A^n$ with $n$ odd has zero entries whenever $j,k$ have identical parities.

      From the rules, if any player $j$ has the hot potato at any time $n$, at time $n+1$, the
      player $j$ must pass it to either $j+1$ or $j-1$ (with modular arithmetic at the edges). Thus,
      the receiving player $k$ must have an opposite parity from $j$. Hence, if $n$ is even (e.g.
      the hot potato has been passed an even number of times), then the receiving player $k$ 
      \textbf{must} have the same parity as $j$, since the hot potato has swapped parities $n$ times
      (and since $n$ is even, the parity of $k$ is the same as the parity of $j$). Similarly, if
      $n$ is odd, then the receiving player $k$ \textbf{must} have a different parity from $j$,
      since the hot potato has swapped parities an odd number of times (so even $\to$ odd, odd $\to$
      even). Therefore, we see that if $n$ is even, then any $(A)_{j,k}$ with differing parities of
      $j,k$ is not possible (and therefore $0$), while if $n$ is odd, then any $(A)_{j,k}$ with
      identical parities of $j,k$ is not possible (and therefore $0$).
  \end{enumerate}
\end{solution}







\end{document}
