\documentclass{homework}
\homework{I}

\begin{document}
\begin{problem}{\S 1}
  (6.A.6) Suppose $u,v\in V$. Prove that $\left<u,v \right>=0$ if and only if \[
    \|u\|\le \|u+av\|
  \] for all $a\in \F$.
\end{problem}
\begin{solution}
  Suppose $\left<u,v \right>=0$. Observe that $\left<u,av \right> =0$ for all $a\in \F$; the
  Pythagorean Theorem then tells us that \[
    \|u+av\|^2=\|u\|^2+\|av\|^2
  .\] Since $\|av\|^2\ge 0$ for any $a\in F,\ v\in V$, we have \[
    \|u+av\|^2=\|u\|^2+\|av\|^2\ge \|u\|^2
  .\] Taking the square root of both sides, we get $\|u\|\le \|u+av\|$.

  Conversely, suppose $\|u\|\le \|u+av\|$. Then \[
    \|u\|^2\le \|u+av\|^2 \implies \|u+av\|^2-\|u\|^2\ge 0  
  .\] But $\|u+av\|^2=\left<u+av,u+av \right> $, and likewise for $\|u\|^2$, so \[
    \|u+av\|^2-\|u\|^2=\left<u,u \right> +\overline{a}\left<u,v \right> +a\left<v,u \right>
    +a\overline{a}\left<v,v \right> -\left<u,u \right> \ge 0
  .\] This then becomes \[
    \left| a \right| ^2\|v\|^2+\overline{a}\left<u,v \right> +a\left<v,u \right> \ge 0
  .\] Observe that $\left<u,v \right> \left<v,u \right> =\left| \left<u,v \right>  \right|
  ^2=\left<u,v \right> \left<u,v \right> $.  Since the above situation holds for any $a\in \F$, let
  \[
    a=-\frac{\left<u,v \right> }{\|v\|^2}=\overline{a}
  .\] Then
  \begin{align*}
    \left| a \right| ^2\|v\|^2+\overline{a}\left<u,v \right> +a\left<v,u \right> &= \frac{\left|
    \left<u,v \right>  \right| ^2\|v^2\|}{(\|v\|^2)^2}-\frac{\left<u,v \right> \left<u,v \right>
  }{\|v\|^2}-\frac{\left<u,v \right> \left<v,u \right> }{\|v\|^2}\\
  &= \frac{\left| \left<u,v \right>  \right|^2 }{\|v\|^2}-2 \frac{\left| \left<u,v \right>  \right|
  ^2}{\|v\|^2}\\
  &= -\frac{\left| \left<u,v \right>  \right| ^2}{\|v\|^2}\\
  &= -\left| \left<u,v \right>  \right| ^2 \\ 
  &\ge 0
  .\end{align*}
  However, $\left| \left<u,v \right>  \right| ^2\ge 0$ for any $u,v$, with equality occurring only
  when $\left<u,v \right> =0$. Thus $u$ and $v$ are orthogonal.
\end{solution}

\begin{problem}{\S 2}
  (6.A.13) Suppose $u,v$ are non-zero vectors in $\R^2$. Prove that \[
    \left<u,v \right> =\|u\|\|v\|\cos{\theta}
  \] where $\theta$ is the angle between $u$ and $v$.
\end{problem}
\begin{solution}
  Consider the triangle formed by $u,v,$ and $u-v$, with $\theta$ the angle between $u$ and $v$.
  By the law of cosines, \[
    \|u-v\|^2=\|u\|^2+\|v\|^2-2\|u\|\|v\|\cos{\theta}
  .\] Note that since $\R$ is a real inner product space, $\left<u,v \right> =\left<v,u \right> $;
  so \[
    \|u-v\|^2=\left<u-v,u-v \right> =\left<u,u \right> -2\left<u,v \right> +\left<v,v \right>
    =\|u\|^2-2\left<u,v \right> +\|v\|^2
  .\] Then
  \begin{align*}
    \|u\|^2-2\left<u,v \right> +\|v\|^2&= \|u\|^2+\|v\|^2-2\|u\|\|v\|\cos{\theta} \\
    -2\left<u,v \right> &= -2\|u\|\|v\|\cos{\theta} \\
    \left<u,v \right> &=\|u\|\|v\|\cos{\theta}
  ,\end{align*}as required.
\end{solution}


\begin{problem}{\S 3}
  (6.A.22) Show that the square of an average is less than or equal to the average of the squares;
  that is, for $a_1,\ldots,a_n\in \R$, then \[
    \left( \frac{a_1+\ldots+a_n}{n} \right) ^2\le \frac{a_1^2+\ldots+a_n^2}{n}
  .\] Also: for which choices of $a_1,\ldots,a_n\in \R$ does inequality become equality?
\end{problem}
\begin{solution}
  Let $u,v\in \R^n$, where $u=(a_1,\ldots,a_n)$ and $v=(1,\ldots,1)$. Then
  \begin{align*}
    \|u\|^2&=\left<u,u \right> =a_1^2+\ldots+a_n^2\\
    \|v\|^2&= \left<v,v \right> =1+\ldots+1=n \\
    \left<u,v \right> &= a_1+\ldots+a_n 
  .\end{align*}
  By the Cauchy-Schwarz Inequality, \[
    \left| \left<u,v \right>  \right| \le \|u\|\|v\| \implies \left| \left<u,v \right>  \right|
    ^2\le \|u\|^2\|v\|^2
  .\] Thus \[
  (a_1+\ldots+a_n)^2\le (a_1^2+\ldots+a_n^2)n \implies \frac{(a_1+\ldots+a_n)^2}{n} \le
  a_1^2+\ldots+a_n^2
  .\] Dividing by $n$ on both sides, we get \[
    \left( \frac{a_1+\ldots+a_n}{n} \right) ^2\le \frac{a_1^2+\ldots+a_n^2}{n} 
  .\] In other words, the square of the average is less than or equal to the average of the squares.

  By Cauchy-Schwarz, inequality holds only when one vector is a scalar multiple of the other. Thus
  \[
    \left( \frac{a_1+\ldots+a_n}{n} \right) ^2= \frac{a_1^2+\ldots+a_n^2}{n} 
  \] only when $a_1=\ldots=a_n$.
\end{solution}




\end{document}
