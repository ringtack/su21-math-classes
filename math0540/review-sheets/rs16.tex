\documentclass{review-sheet}
\homework{16}

\begin{document}
\begin{problem}{\S 1}
  (Treil 5.4) Use cofactor formula to compute the inverses of the following matrices:
  \begin{enumerate}[label=(\alph*)]
    \item $\begin{pmatrix} 1&2\\3&4 \end{pmatrix} $
    \item $\begin{pmatrix} 19&-17\\3&-2 \end{pmatrix} $
    \item $\begin{pmatrix} 1&0\\3&5 \end{pmatrix} $
    \item $\begin{pmatrix} 1&1&0\\2&1&2\\0&1&1 \end{pmatrix} $
  \end{enumerate}
\end{problem}

\begin{solution}
  \begin{enumerate}[label=(\alph*)]
    \item $A^{-1}=\frac{1}{4-6}\begin{pmatrix} 4&-3\\-2&1 \end{pmatrix}^T =\begin{pmatrix}
      -2&1\\ \frac{3}{2}&-\frac{1}{2} \end{pmatrix} $
    \item $A^{-1}=\frac{1}{-38-(-51)}\begin{pmatrix} -2&-3\\17&19 \end{pmatrix}^T =\begin{pmatrix}
      -\frac{2}{13}&\frac{17}{13}\\-\frac{3}{13}&\frac{19}{13} \end{pmatrix} $
    \item $A^{-1}=\frac{1}{5}\begin{pmatrix} 5&-3\\0&1 \end{pmatrix}^T =\begin{pmatrix}
      1&0\\-\frac{3}{5}&\frac{1}{5} \end{pmatrix} $
    \item $A^{-1}=\frac{1}{(1-2)-(2)}\begin{pmatrix} -1 & (-1)^3(2) & 2\\ (-1)^3(1)&1&(-1)^5(1)\\
        2&(-1)^5(2)&-1 \end{pmatrix}^T=\begin{pmatrix}
      \frac{1}{3}&\frac{1}{3}&-\frac{2}{3}\\\frac{2}{3}&-\frac{1}{3}
                 &\frac{2}{3}\\-\frac{2}{3}&\frac{1}{3}&\frac{1}{3} \end{pmatrix} $
  \end{enumerate}
\end{solution}

\begin{problem}{\S 2}
  (Axler Problem 5.C.16) The \textbf{Fibonacci Sequence} $F_1,F_2,\ldots$ is defined by \[
    F_1=1,\ F_2=1,\ F_n=F_{n-2}+F_{n-1},\ n\ge 3
  .\] Define $T\in \mc{L}(\R^2)$ by $T(x,y)=(y,x+y)$.
  \begin{enumerate}[label=(\alph*)]
    \item Show that $T^n(0,1)=(F_n,F_{n+1})$ for all positive integers $n$.
    \item Find the eigenvalues of $T$.
    \item Find a basis of $\R^2$ consisting of eigenvectors of $T$.
    \item Use the solution to part (c) to compute $T^n(0,1)$. Conclude that \[
        F_n=\frac{1}{\sqrt{5}}\left[ \left( \frac{1+\sqrt{5}}{2} \right) ^n-\left(
        \frac{1-\sqrt{5}}{2} \right)^n \right] 
    .\] 

  \end{enumerate}
\end{problem}

\begin{solution}
  \begin{enumerate}[label=(\alph*)]
    \item We begin by induction. Clearly, $T^1(0,1)=(1,1)=(F_1,F_2)$. Now, assume
      $T^n(0,1)=(F_n,F_{n+1})$. Then \[
        T^{n+1}(0,1)=T(T^n(0,1))=T(F_n,F_{n+1})=(F_{n+1},F_n+F_{n+1})=(F_{n+1},F_{n+2})
      ,\] as required. Thus \[
        T^n(0,1)=(F_{n},F_{n+1})
      \] for all positive integers $n$.
    \item Let $A=\mc{M}(T)$ with respect to the standard basis. Then \[
        A = \begin{pmatrix} 0 & 1\\ 1 & 1 \end{pmatrix} 
    .\] Then \[
    \det{(A-\lambda I)}=\begin{vmatrix} -\lambda & 1\\1 & 1-\lambda \end{vmatrix}
      =\lambda^2-\lambda-1
      .\] Thus \[
        \lambda_1=\frac{1+\sqrt{5}}{2},\ \lambda_2=\frac{1-\sqrt{5}}{2}
      ,\] since those are the roots of $\lambda^2-\lambda-1$.
    \item We now find the eigenspaces associated with $\lambda_1,\lambda_2$: \[
        E(\lambda_1,T)=\{ (x,y)\mid y=\left( \frac{1+\sqrt{5}}{2} \right)x,\ x+y=\left(
        \frac{1+\sqrt{5}}{2} \right)y \}
      .\] From the first condition, we get that any eigenvector corresponding to $\lambda_1$ is a
      scalar multiple of $\left( 1,\left( \frac{1+\sqrt{5}}{2} \right)  \right)$. Similarly, we get
      that any eigenvector corresponding to $\lambda_2$ is a scalar multiple of $\left( 1,\left(
      \frac{1-\sqrt{5}}{2} \right)  \right) $. Thus \[
        \{ \left( 1,\frac{1+\sqrt{5}}{2} \right) , \left( 1, \frac{1-\sqrt{5}}{2} \right)  \}
      \] is a basis of eigenvectors in $\R^2$ (since eigenvalues of distinct eigenvalues are
      linearly independent, and thus we have a list of $2$ linearly independent vectors in $\R^2$,
      thus forming a basis).
    \item Let $v_1=\left( 1,\frac{1+\sqrt{5}}{2} \right) $, $v_2=\left( 1, \frac{1-\sqrt{5}}{2}
      \right)$. Note that $(0,1)=\frac{v_1-v_2}{\sqrt{5}}$. Thus \[
        T^n(0,1)=\frac{1}{\sqrt{5}}T^n(v_1-v_2)=\frac{1}{\sqrt{5}}\left[
        \left(   \frac{1+\sqrt{5}}{2}\right)^nv_1-\left(\frac{1-\sqrt{5}}{2}\right)^nv_2 \right]
        .\] Since the first component is $F_n$, we get \[
        F_n = \frac{1}{\sqrt{5}}\left[
        \left(   \frac{1+\sqrt{5}}{2}\right)^n-\left(\frac{1-\sqrt{5}}{2}\right)^n\right]
      ,\] as required.
  \end{enumerate}
\end{solution}




\end{document}
