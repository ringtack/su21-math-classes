\documentclass{review-sheet}
\homework{15}

\begin{document}
\begin{problem}{\S 1}
  A square matrix is \textit{skew-symmetric} if $A^T=-A$. Prove that if $A$ is skew-symmetric and
  $n$ odd, then $\det{A}=0$. Is this true for even $n$?
\end{problem}
\begin{solution}
  Since $\det{A}=\det{A^T}$, $\det{A}=\det{-A}$. Since there are odd $-1$s, we get \[
    \det{A}=(-1)^n\det{A}=-\det{A}
  .\] Then $2\det{A}=0$, so $\det{A}=0$, as required.

  For even $n$, $(-1)^n=1$, so we don't know anything else about $\det{A}$.
\end{solution}

\begin{problem}{\S 2}
  Suppose the permutation of $\sigma$ takes $(1,2,3,4,5)$ to $(5,4,1,2,3)$.
  \begin{enumerate}[label=(\alph*)]
    \item Find the sign of $\sigma$.
    \item What does $\sigma^2$ do to $(1,2,3,4,5)$.
    \item What does the inverse permutation $\sigma^{-1}$ do to $(1,2,3,4,5)$?
    \item What is the sign of $\sigma^{-1}$.
  \end{enumerate}
\end{problem}
\begin{solution}
  \begin{enumerate}[label=(\alph*)]
    \item $\sgn{(\sigma)}=-1$
    \item $\sigma^2(1,2,3,4,5)=(3,2,5,4,1)$
    \item $\sigma^{-1}(1,2,3,4,5)=(3,4,5,2,1)$
    \item $\sgn{(\sigma^{-1})}=-1$.
  \end{enumerate}
\end{solution}

\begin{problem}{\S 3}
  Why is there an even number of permutations of $(1,2,\ldots,9)$ and why are exactly half of them
  old permutations?
\end{problem}
\begin{solution}
  The number of permutations of $(1,2,\ldots,9)$ is $9!=362880$; thus there are an even number of
  permutations. Now, recall that \[
    \det{(A)}=\sum_{\sigma\in ~\text{permutations}~}
    a_{\sigma(1),1}\ldots a_{\sigma(9),9}\sgn{(\sigma)}
  .\] Consider a matrix $A$ filled entirely with $1$s. Then the determinant becomes \[
  \det{A}=\sum_{\sigma\in ~\text{permutations}~} \sgn{(\sigma)}=0
\] (because if two rows are linearly dependent, then $\det{A}=0$). In other words, the determinant
is the number of even permutations, minus the number of odd permutations. Therefore, exactly half
are even, and half are odd.
\end{solution}

\begin{problem}{\S 4}
  Evaluate the determinants:
  \[
    \begin{vmatrix} 1&2&0\\1&1&5\\1&-3&0 \end{vmatrix}, \begin{vmatrix} 4&-6&-4&4\\2&1&0&0\\0&-3&1&3\\-2&2&-3&-5 \end{vmatrix} 
  .\] 
\end{problem}

\begin{solution}
  \begin{enumerate}[label=(\alph*)]
    \item $\begin{vmatrix} 1&2&0\\1&1&5\\1&-3&0 \end{vmatrix}=-5\begin{vmatrix} 1&2\\1&-3 \end{vmatrix} =25$
    \item 
      \begin{align*}
        \begin{vmatrix} 4&-6&-4&4\\2&1&0&0\\0&-3&1&3\\-2&2&-3&-5 \end{vmatrix} &= -2\begin{vmatrix}
        -6&4&4\\-3&1&3\\2&-3&-5\end{vmatrix} +1\begin{vmatrix} 4&-4&4\\0&1&3\\-22&-3&-5 \end{vmatrix} \\
          &= -2\left( -6\begin{vmatrix} 1&3\\-3&5 \end{vmatrix} +4\begin{vmatrix} -3&3\\2&-5
      \end{vmatrix} +4\begin{vmatrix} -3&1\\2&-3 \end{vmatrix}  \right) +\left(-1\begin{vmatrix}
    4&4\\-2&-5 \end{vmatrix} -3\begin{vmatrix} 4&-4\\-2&-3 \end{vmatrix}  \right) \\
     &= 12(4)-8(9)-8(7)-12-3(-20)\\
     &=-32
      .\end{align*}
  \end{enumerate}
\end{solution}






\end{document}
