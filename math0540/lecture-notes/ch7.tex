\documentclass[math0540-lecture-notes.tex]{subfiles}
\begin{document}

\chapter{Operators on Inner Product Spaces}

\section{Self-Adjoint and Normal Operators}

We start with a review of linear functional.

\begin{definition}[Linear Functionals]{}
  A linear map from a vector space $V$ to a field $\F$ is a \textbf{linear functional}.
\end{definition}
For example, if $V$ is an inner product space, then for a fixed $v\in V$, \[
  \left<\cdot , v \right> :V\longrightarrow \F
,\] is a linear functional on $V$. Indeed, this is a linear map, since inner products are linear in
the first slot.

\begin{theorem}[Reisz Representation Theorem]{}
  Let $V$ be a finite-dimensional inner product space. Then given a linear functional $\phi:V\to
  \F$, there exists a unique vector $u\in V$ such that \[
    \phi(v)=\left<v,u \right> 
  \] for every $v\in V$.
\end{theorem}

\begin{example}
  Let $\phi:\R^2\to \R$ (with the usual Euclidean inner product) be given by $\phi(x,y)=x+y$. Reisz
  then tells us that there exists some $v\in \R^2$ such that \[
    \phi(x,y)=\left<(x,y),v \right> ~\text{for every}~(x,y)\in \R^2
  .\] Indeed, let $v=(1,1)$. Then $\left<(x,y),(1,1) \right> =x+y$.
\end{example}

Reisz Representation Theorem actually gives us a bijective function \[
  R:\mc{L}(V,\F)\longrightarrow V
.\] In other words, the space of all linear functionals from $V$ to $\F$ is bijective to $V$!
Indeed, $R$ is given by \[
  R(\phi)=\overline{\phi(e_1)}e_1+\ldots+\overline{\phi(e_n)}e_n  
,\] where $\phi\in \mc{L}(V,\F)$ is a linear functional and $e_1,\ldots,e_n$ is an orthonormal basis
of $V$. $R$ is conjugate-linear, with linearity holding only when $\F$ is real.

Now, we move onto adjoints.
\begin{definition}[Adjoint, $T^*$]{}
  Suppose $T\in \mc{L}(V,W)$. The \textbf{adjoint} of $T$ is the function $T^*:W\to V$ such that \[
    \left<Tv,w \right> =\left<v,T^*w \right> 
  \] for every $v\in V$ and $w\in W$.

  Note that the left inner product is the \textbf{inner product on $W$}, while the right inner
  product is the \textbf{inner product on $V$}.
\end{definition}

But why does such a function $T^*$ actually exist, and why is it unique? Fix a $w\in W$, and
consider \[
  V\longrightarrow \F, v\mapsto \left<Tv,w \right> 
.\] This is a linear map from $V$ to $\F$, since it's just a linear map ($T$) composed with another
linear map ($\left<\cdot ,w \right> $). Then by the Reisz Representation Theorem, there is a unique
vector $T^*w\in V$ such that for all $v\in V$, \[
  \left<Tv,w \right> =\left<v,T^*w \right> 
.\] 

\begin{example}
  Define $T:\R^3\to \R^2$ by \[
    T(x_1,x_2,x_3)=(x_2+3x_3,2x_1)
  .\] Find a formula for $T^*$.
\end{example}
\begin{solution}
  Here $T^*\in \mc{L}(\R^2,\R^3)$. To compute $T^*$, fix a point $(y_1,y_2)\in \R^2$. Then for every
  $(x_1,x_2,x_3)\in \R^3$, we have
  \begin{align*}
    \left<(x_1,x_2,x_3),T^*(y_1,y_2) \right> &= \left<T(x_1,x_2,x_3),(y_1,y_2) \right>  \\ 
                                             &= \left<(x_2+3x_3,2x_1),(y_1,y_2) \right>  \\
                                             &= x_2y_1+3x_3y_1+2x_1y_2 \\
                                             &= \left<(x_1,x_2,x_3),(2y_2,y_1,3y_1) \right>
  .\end{align*} Thus \[
  T^*(y_1,y_2)=(2y_2,y_1,3y_1)
  .\] 
\end{solution}
\begin{example}
  Fix $u\in V$ and $x\in W$. Define $T\in \mc{L}(V,W)$ by \[
    Tv=\left<v,u \right> x
  \] for every $v\in V$. Find a formula for $T^*$.
\end{example}
\begin{solution}
  Fix $w\in W$. Then for every $v\in V$, we have
  \begin{align*}
    \left<v,T^*w \right> &= \left<Tv,w \right>  \\
    &= \left<\left<v,u \right> x,w \right>  \\
    &= \left<v,u \right> \left<x,w \right>  \\
    &= \left<v,\left<w,x \right> u \right> 
  .\end{align*} Thus \[
    T^*w=\left<w,x \right> u
  .\] 
\end{solution}

In the above examples, $T^*$ was not only a function, but indeed a linear map. This is true in
general.
\begin{proposition}[Adjoint is Linear]{}
  If $T\in \mc{L}(V,W)$, then $T^*\in \mc{L}(W,V)$ (equivalently, $T^*$ is a linear map).
\end{proposition}
\begin{proof}[Proof]
  Suppose $T\in \mc{L}(V,W)$. Fix $w_1,w_2\in W$. If $v\in V$, then
  \begin{align*}
    \left<v,T^*(w_1+w_2) \right>&= \left<Tv,w_1+w_2 \right>  \\
    &= \left<Tv,w_1 \right> +\left<Tv,w_2 \right>  \\
    &= \left<v,T^*w_1 \right> +\left<v,T^*w_2 \right>  \\
    &=\left<v,T^*w_1+T^*w_2 \right> 
  .\end{align*}
  Hence $T^*(w_1+w_2)=T^*w_1+T^*w_2$. A similar computation follows for $\lambda\in \F$; thus
  $T^*\in \mc{L}(W,V)$, as desired.
\end{proof}

Let's look at some properties of the adjoint.
\begin{proposition}[Properties of Adjoints]{}
  \begin{enumerate}
    \item $(S+T)^*=S^*+T^*$ for all $S,T\in \mc{L}(V,W)$;
    \item $\left( \lambda T \right)^*=\overline{\lambda}T^* $ for every $\lambda\in \F$, $T\in
      \mc{L}(V,W)$;
    \item $(T^*)^*=T$ for all $T\in \mc{L}(V,W)$;
    \item $I^*=I$, where $I$ is the identity operator on $V$;
    \item $(ST)^*=T^*S^*$ for all $T\in \mc{L}(V,W)$ and $S\in \mc{L}(W,U)$ (here $U$ is another
      inner product space over $\F$).
  \end{enumerate}
\end{proposition}
\begin{proof}[Proof]
  TODO
\end{proof}














\end{document}
