\documentclass[math0540-lecture-notes.tex]{subfiles}

\begin{document}

\chapter{Finite-Dimensional Vector Spaces}

\section{Span and Linear Independence}
Suppose a friend imagines a subspace $W\subseteq \R^3$. You know that $(1,0,0),(0,1,0)\in W$. What
else do you know must be in $W$? Well, first, $\textbf{0}=(0,0,0)\in W$ by definition. But moreover,
anything in the form $\{ (a,b,0) \mid a,b\in \R \}$ (the xy-plane) must be in $W$, since any point
on the plane can be made by $ \alpha \cdot a + \beta \cdot b$ (we will later see that $(1,0)$ and
$(0,1)$ are \textbf{basis vectors} of $\R^2$).

\begin{definition}[Linear Combination and Span]{}
  A \textbf{linear combination} of a list of vectors $ v_1,\ldots,v_n\in V$ is a vector of the form \[
      \lambda_1v_1+\ldots+\lambda_nv_n,~\text{where}~ \lambda_i\in \F
    .\]  The \textbf{span} (or \textbf{linear span}) of $ v_1,\ldots,v_n$, is the set of all linear
    combinations of $ v_1,\ldots,v_n$: \[
      ~\text{span}(v_1,\ldots,v_n) = \{ a_1v_1+\ldots+a_mv_m \mid a_i\in \F\}
    .\] 
  The span of no vectors is $\{ 0 \}$.
\end{definition}

\begin{proposition}[Span is Smallest Subspace]{}
  The span of $v_1,\ldots,v_m$ is the smallest subspace of $V$ containing $ v_1,\ldots,v_m$.
  Precisely:
  \begin{enumerate}
    \item $~\text{span}(v_1,\ldots,v_m)$ is a subspace of $V$.
    \item Any subspace $W$ of $V$ containing $v_1,\ldots,v_m$ also contains $\text{span}(v_1,\ldots,v_m)$.
  \end{enumerate}
\end{proposition}

\begin{proof}[Proof]
  Let $v_1,\ldots,v_m$ be a list of vectors in $V$.\\

  $\Span(v_1,\ldots,v_m)$ is clearly a subspace of $V$: achieve $\textbf{0}\in
  \Span(v_1,\ldots,v_m)$ by setting each $a_j=0$, and since $a_j+b_j,\lambda a_j\in \F$,
  $\Span(v_1,\ldots,v_m)$  is closed under addition and scalar multiplication.\\

  Now, we show that $\Span(v_1,\ldots,v_m)$ is the smallest subspace containing  $v_1,\ldots,v_m$.
  Every vector $v_j$ is a linear combination of $  v_1,\ldots,v_m$ (take, for $i\neq j$, $a_i=0$);
  thus $\Span(v_1,\ldots,v_m)$ contains each $v_j$. Additionally, every subspace $U$ of $V$ that
  contains each $v_j\in U$ is closed under addition and scalar multiplication, so $U$ contains every
  linear combination of $ v_1,\ldots,v_m$; thus $U$ contains $\Span(v_1,\ldots,v_m)$. So, since
  $\Span(v_1,\ldots,v_m)$ contains every vector $v_j$, and any subspace $U$ of $V$ that contains
  every vector $v_j$ also contains $\Span(v_1,\ldots,v_m)$, the span is the smallest subspace
  containing every $v_j$.
\end{proof}

\begin{definition}[Spanning a Vector Space]{}
  If $\Span(v_1,\ldots,v_m)=V$, then $ v_1,\ldots,v_m$ \textbf{spans} $V$, and $v_1,\ldots,v_m$ are
  a \textbf{spanning set}.
\end{definition}

We now make one of the key definitions of linear algebra.

\begin{definition}[Finite Dimensional Vector Spaces]{}
  If $V$ is spanned by a \textbf{finite} list of vectors $v_1,\ldots,v_m$ then $V$ is
  \textbf{finite-dimensional}.\\

  If $V$ is not finite-dimensional, then $V$ is \textbf{infinite-dimensional}.
\end{definition}

\begin{example}
  Let $\mc{P}(\F)$ be the set (indeed, vector space) of polynomials over a field $\F$. Show
  $\mc{P}(\F)$ is infinite-dimensional.
\end{example}
\begin{solution}
  Let $p\in \mc{P}(\F)$, and let $m$ denote the highest degree polynomial in $\mc{P}(\F)$. Then $p$ 
  has at most degree $m $; thus a polynomial $p^{m+1}$ is not spanned by any list of vectors in
  $\mc{P}(\F)$; thus $\mc{P}(\F)$ is finite-dimensional.
\end{solution}

\subsection{Linear Independence}

As with sums/direct sums, we are interested if a vector has a unique linear combination; that is,
given a list $v_1,\ldots,v_m\in V$, and $v\in \Span(v_1,\ldots,v_m)$, are there unique $a
1,\ldots,a_m\in \F$ such that \[
  v=a_1v_1+\ldots+a_mv_m
?\] In other words, is there only one way to create a certain vector given a span? Suppose there's
more than one way; then there exists $ b_1,\ldots,b_m\in \F$ such that \[
  v=b_1v_1+\ldots+b_mv_m
;\] then \[
  0 = (a_1-b_1)v_1+\ldots+(a_mb_m)v_m
.\] If the only way to do this is the obvious way, where $a_i-b_i=0$, then the representation is
unique. We call this \textbf{linear independence}.

\begin{definition}[Linear Independence]{}
  A list of vectors $ v_1,\ldots,v_m\in V$ is \textbf{linearly independent} if the only choice of $
  a_1,\ldots,a_m\in \F$ that makes $  a_1v_1+\ldots+a_mv_m$  equal $0 $ is $a_i=0$.\\

  A list of vectors in $V$ is \textbf{linearly dependent} if it is not linearly independent. That
  is, there exist non-zero $a_i \in \F$ such that \[
    0 = \sum_{i=1}^{m} a_iv_i
  .\]\\ 

  An empty list of vectors $()$ is linearly independent.
\end{definition}
\begin{example}
  \begin{enumerate}
    \item A list of one vector $v\in V$ is linearly independent if and only if $v$ is non-zero.
    \item A list of two vectors $v1,v_2\in V$ is linearly independent if and only if one vector is
      not a scalar combination of the other vector; that is, $ v_1\neq \lambda v_2$ for some $
      \lambda\in \F$.
    \item $(1,0,0),\ (0,1,0)\in \R^3$ is linearly independent.
    \item $(1,-1,0), (-1,0,1), (0,1,-1)\in \R^3$ is linearly dependent. In particular,
      $(1,-1,0)+(-1,0,1)+(0,1,-1)=\textbf{0}$. Alternatively, we can write $(-1,0,1)$ as a linear
      combination of the other two: \[
        (-1,0,1) = -1\cdot (1,-1,0) -(0,1,-1)
      .\] 
  \end{enumerate}
\end{example}

Intuitively, a list of vectors is linearly independent if none of its vectors are a linear
combination of the other vectors; each vector is "independent" of the other vectors. In other words,
a vector is linearly independent if it is not in the span of the other vectors. Its negation is also
important, and is arguably more intuitive: a list of vectors is linearly dependent \textbf{iff} it
is in the span of the other vectors. Formally, this gives rise to an important lemma, and theorem.

\begin{lemma}[Linear Dependence Lemma]{}
  Suppose that $v_1,\ldots,v_m\in V$ is a linearly dependent list of vectors. Then there exists some
  $j\in \{ 1,\ldots,m \}$ such that:
  \begin{enumerate}
    \item $v_j\in \Span(v_1,...,v_{j-1})$
    \item If the $j^{th}$ term is removed from the list, the span of the remaining vectors
      $v_1,\ldots,\hat{v_j}\footnote[1]{\text{here, hat means "with $v_j$ removed"}},\ldots,v_m$ equals $\Span(v_1,\ldots,v_m)$.
  \end{enumerate}
  In other words, removing the linearly dependent vector has no effect on the overall span of the
  vectors.
\end{lemma}
\begin{proof}[Proof]
  Because the list $v_1,\ldots,v_m$ is linearly dependent, there exist $a_1,\ldots,a_m\in \F$ not
  all $0$ such that \[
    a_1v_1+\ldots+a_mv_m=0
  .\] Let $j$ be the \textit{largest element} of $\{ 1,\ldots,m \}$ such that $a_j\neq 0$. Then \[
  v_j = -\frac{a_1}{a_j}v_1-\ldots-\frac{a_{j-1}}{a_j} \label{eq:lindep}
  ;\] hence $v_j$ is in the span of $v_1,\ldots,v_{j-1}$.\\

  Now, suppose $u\in \Span(v_1,\ldots,v_m)$. Then there exist $ b_1,\ldots,b_m\in \F$ such that \[
    u = b_1v_1+\ldots+b_mv_m
  .\] If we replace $v_j$ with \ref{eq:lindep}, the resulting list consists only of
  $v_1,\ldots,\hat{v_j},\ldots,v_m$; thus we see that $u$ is in the span of the list.
\end{proof}
\begin{theorem}[Length of Linearly Independent List and Span]{}
  In a finite-dimensional vector space, the length of every linearly independent list of vectors is
  less than or equal to the length of every spanning list of vectors.
\end{theorem}
\begin{proof}[Proof]
  Left as an exercise for the reader. Try starting with $u_1,\ldots,u_m\in V$ a list of linearly
  independent vectors, and $v_1,\ldots,v_n\in V$ a spanning list of $V$, and show that $m\le n$. Use
  the Linear Dependence Lemma to iteratively add $u_i$ and remove $w_j$; eventually, we are left
  with a list with all $u_i$, and optionally some $w_j$.
\end{proof}

Intuitively, every subspace of a finite-dimensional vector space is also finite-dimensional.
\begin{proposition}[Finite-Dimensional Subspaces]{}
  Every subspace of a finite-dimensional vector space is finite-dimensional.
\end{proposition}



\section{Bases}





  
\end{document}
